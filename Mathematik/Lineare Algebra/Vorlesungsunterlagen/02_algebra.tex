\chapter{Algebraische Strukturen}

\section{Mengen und Relationen}\label{section_mengen}

\setcounter{definition}{0}
\setcounter{beispiel}{0}
\setcounter{notiz}{0}

Mengen spielen eine zentrale Rolle in der Mathematik (und nicht nur dort). Wir folgen hier dem Zugang Georg 
Cantors (1845 - 1918) und verzichten auf eine axiomatische Einführung.

\begin{definition} Eine \textbf{Menge}\index{Menge} ist eine Zusammenfassung $M$ von bestimmten wohlunterschiedenen 
Objektem $m$, genannt die Elemente von $M$, unseres Anschauungsraums oder unseres Denkens zu einem Ganzen.
\end{definition}

Ist $m$ ein Element\index{Element} von $M$, so schreiben wir $m \in M$, andernfalls schreiben wir $m \notin M$. 
Für jedes Objekt $m$ unserer Anschauung und jede Menge $M$ gilt also genau entweder $m \in M$  oder 
$m \notin M$, nicht aber beides. 

\begin{beispiel}

\begin{itemize}
\item Die Menge, die keine Elemente enthält, bezeichnet man als \textbf{leere Menge}. Hierfür schreiben wir
$\emptyset$ oder $\{ \, \}$.
\item Eine Menge $M$, die nur endlich viele Elemente hat, lässt sich durch vollständige Auflistung aller 
ihrer Elemente zwischen Mengenklammern beschreiben, z.B. $M = \{ 1, 3, 5, 7, 9 \}$ ist die Menge, die als Elemente 
die Zahlen $1, 3, 5, 7, 9$ enthält, und $M = \{ \textrm{rot}, \textrm{blau}, \textrm{grün} \}$ ist die Menge, 
die als Elemente die Farben \textit{rot}, \textit{blau} und \textit{grün} enthält.
\item Auch unendliche Menge lassen sich oft in aufzählender Form darstellen. So wird die Menge $\mathbb N$ 
der natürlichen Zahlen oft in der Form 
  	$$ \mathbb N = \{ 0, 1, 2, 3, 4, \ldots \} $$
dargestellt, und die Menge $\mathbb Z$ der ganzen Zahlen als
  	$$ \mathbb Z = \{ \ldots -4, -3, -2, -1, 0, 1, 2, 3, 4, \ldots \} $$
\item Manche Menge lassen sich nicht oder nur schwer aufzählend darstellen. Häufig ist hierfür jedoch eine 
beschreibende Darstellung möglich. Die Menge $\mathbb P = \{ x \in \mathbb N \vert \, x 
\textrm{ ist eine Primzahl }\}$ 
etwa enthält als Elemente alle natürlichen Zahlen $x$, die Primzahlen sind, d.h. alle natürlichen Zahlen
$x \neq 1$, die nur durch 1 und sich selbst teilbar sind.  
\item Eine wichtige Rolle in diesem Skript spielt die Menge $\mathbb Q$ der rationalen Zahlen, 
  	$$ \mathbb Q = \left\{ \frac {m}{n} \, \big\vert \, m, n \in \mathbb Z \textrm{ und } n \neq 0 \right\}. $$ 
\end{itemize}
\end{beispiel}

\begin{definition} Eine Menge $N$ heißt \textbf{Teilmenge}\index{Teilmenge} eine Menge $M$, geschrieben 
$N \subseteq M$, wenn jedes Element von $N$ auch Element von $M$ ist, d.h. wenn die Implikation 
$x \in N \Longrightarrow x \in M$ gilt. In diesem Fall heißt $M$ auch \textbf{Obermenge}\index{Obermenge} 
von $N$.
\end{definition}

\begin{definition} Zwei Mengen $N$ und $M$ heißen gleich, geschrieben $M = N$, wenn ein Objekt $x$ genau 
dann Element von $N$ ist, wenn es Element von $M$ ist, d.h. wenn die Äquivalenz $x \in N \iff x \in M$ 
gilt.
\end{definition}

\begin{notiz} Eine Menge $N$ heißt echte \index{Teilmenge!echte}Teilmenge von $M$, geschrieben $N \subset 
N$, wenn $N$ Teilmenge von $M$ mit $N \neq M$ ist.
\end{notiz}

Die Gleichheit von Mengen lässt sich duch die Teilmengenbeziehung beschreiben.

\begin{satz} Für zwei Mengen $M$ und $N$ gilt genau dann $M = N$, wenn $M \subseteq N$ und $N \subseteq M$. 
\end{satz}

Für Teilmengen $A$ und $B$ einer gegebenen Menge $M$ lassen sich verschiedene Beziehungen definieren:

\begin{definition} Die \textbf{Schnittmenge}\index{Schnittmenge} von $A$ und $B$, geschrieben $A \cap B$ 
besteht aus den Elementen von $M$, die sowohl in $A$ als auch in $B$ sind:
  $$ A \cap B = \{ x \in M \vert x \in A \textrm{ und } x \in B \} $$
Falls $A \cap B = \emptyset$, so nennen wir $A$ und $B$ \index{Menge!disjunkt}\textbf{disjunkt}.
\end{definition}

	\begin{figure}[H]
	\vspace{-0.4cm}
		\begin{center}
		\begin{scaletikzpicturetowidth}{0.55\textwidth}
     			\input{12_mws_1000.tex} 
		\end{scaletikzpicturetowidth}
		\end{center}
	\vspace{-0.8cm}
	\end{figure}

\begin{definition} Die \textbf{Vereinigungsmenge}\index{Vereinigungsmenge} von $A$ und $B$, geschrieben $A \cup B$ 
besteht aus den Elementen von $M$, die entweder in $A$ oder in $B$ sind:
  $$ A \cup B = \{ x \in M \vert \, x \in A \textrm{ oder } x \in B \} $$
\end{definition}

	\begin{figure}[H]
	\vspace{-0.4cm}
		\begin{center}
		\begin{scaletikzpicturetowidth}{0.55\textwidth}
     			\input{12_mws_1100.tex} 
		\end{scaletikzpicturetowidth}
		\end{center}
	\vspace{-0.8cm}
	\end{figure}


Für die Bildung von Vereinigungs-- und Druchschnittsmengen gelten folgende wichtige Regeln

\begin{satz}\label{mengen_satz_teil_verein} Für Teilmengen $A, B, C \subseteq M$ gilt

  	\begin{tabular} {l l }
 	 Kommutativgesetz & $A \cup B = B \cup A$ \\
  	& $A \cap B = B \cap A$ \\
  	Assoziativgesetz & $( A \cup B) \cup C = A \cup ( B \cup C)$ \\
  	& $(A \cap B) \cap C = A  \cap (B \cap C)$ \\
  	Distributivgesetz & $(A \cup B) \cap C = (A \cap C) \cup (B \cap C)$ \\
  	& $(A \cap B) \cup C = (A \cup C) \cap (B \cup C)$ \\
  	Verschmelzungsgesetz & $A \cap (A \cup B) = A$ \\
  	& $A \cup (A \cap B) = A$   
  	\end{tabular}
\end{satz}

\beweis {Kommutativ-- und Assoziativgesetze sind klar. 

Wir zeigen zunächst das erste Distributivgesetz unter Ausnutzung des ersten Distributivgesetzes der Logik 
aus Satz~\ref{logik_regeln_kombi}:
  $$ \begin{array} {l c l}
  x \in (A \cup B) \cap C & \iff & \left(x \in (A \cup B)\right) \wedge \left(x \in C\right) \\
  & \iff & \left(\left(x \in A\right) \vee \left(x \in B\right) \right) \wedge \left(x \in C \right) \\
  & \iff & \left((x \in A) \wedge (x \in C)\right) \vee \left( (x \in B ) \wedge (x \in C)\right) \\
  & \iff & \left ( x \in A \cap C\right) \vee \left(x \in B \cap C\right) \\
  & \iff & x \in (A \cap C) \cup (B \cap C)
  \end{array} $$
Wir haben also gezeigt:
  $$ x \in (A \cup B) \cap C \, \iff \, x \in (A \cap C) \cup (B \cap C) $$
und damit
  $$ (A \cup B) \cap C = (A \cap C) \cup (B \cap C), $$
wie gewünscht. 

Das zweite Distributivgesetz zeigt man analog mit dem entsprechenden Distributivgesetz der Logik, siehe 
auch Aufgabe~\ref{mengen_aufg_distr}.

Den Beweis der Verschmelzungsgesetze überlassen wir dem Leser als \"Ubungsaufgabe, siehe 
Aufgabe~\ref{mengen_aufg_verschm}. 
}

\medbreak

\begin{definition} Die \textbf{Differenzmenge}\index{Differenzmenge} von $A$ und $B$, geschrieben $A \setminus B$ 
besteht aus den Elementen von $M$, die in $A$ aber nicht in $B$ sind:
  $$ A \setminus B = \{ x \in M \vert \, x \in A \textrm{ und } x \notin B \} $$
Falls $B \subseteq A$ nennen wir $A \setminus B$ auch das \textbf{Komplement}\index{Komplement} von $B$ in $A$ 
und schreiben 
hierfür $\overline{B}^A$. Falls $A = M$, schreiben wir hierfür auch kurz $\overline{B}$ 
und nennen es das Komplement von $B$. 
\end{definition}

	\begin{figure}[H]
	\vspace{-0.4cm}
		\begin{center}
		\begin{scaletikzpicturetowidth}{0.55\textwidth}
     			\input{12_mws_1200.tex} 
		\end{scaletikzpicturetowidth}
		\end{center}
	\vspace{-0.8cm}
	\end{figure}

Auch für die Komplementbildung gelten einige einfache Regeln:

\begin{regel}\label{mengen_regel_komplement} Für eine Teilmenge $A \subseteq M$ gilt:
\begin{enumerate}
\item $\overline{\overline{A}} = A$.
\item $\overline{A} \cup A = M$.
\item $\overline{A} \cap A = \emptyset$.
\item $\overline{\emptyset} = M$
\item $\overline{M} = \emptyset$.
\end{enumerate}
\end{regel}

\bigbreak

Auch zwei beliebige Mengen können miteinander verknüpft werden:

\begin{definition} Das \textbf{kartesische Produkt}\index{kartesisches Produkt} zweier Mengen $M$ und $N$ ist 
die Menge $M \times N$, deren Elemente die geordneten Paare $(m, n)$ sind, wobei $m \in M$ und $n \in N$, also
  	$$ M \times N = \{ (m, n) \vert \, m \in M \textrm{ und } n \in  N \} $$
\end{definition}

	\begin{figure}[H]
	\vspace{-0.4cm}
		\begin{center}
		\begin{scaletikzpicturetowidth}{0.55\textwidth}
     			\input{12_mws_1300.tex} 
		\end{scaletikzpicturetowidth}
		\end{center}
	\vspace{-0.8cm}
	\end{figure}

\begin{beispiel} Es seien $M = \{ 1, 2, 3, 4\}$ und $N = \{ 3, 5 \}$. Dann gilt:
 	 $$ M \times N = \{ (1, 3),  (1, 5), (2, 3),  (2, 5), (3, 3),  (3, 5), (4, 3), (4,5) \} $$
Beachten Sie, dass es dabei keine Rolle spielt, dass manche Elemente (hier etwa die 3) in beiden Mengen 
vorkommen.
\end{beispiel}

\begin{beispiel} Ist $N = \emptyset$, so gilt $M \times N = \emptyset$ (unabhängig von $M$).
\end{beispiel}

\bigbreak

\begin{definition} Eine \index{Relation}\textbf{Relation} $R$ zwischen zwei Mengen $M$ und $N$ ist eine Beziehung
zwischen Elementen von $M$ und $N$, dargestellt durch geordnete Paare $(m,n)$ mit $m \in M$ und $n \in N$. Wir 
schreiben hierfür $m \sim_R n$ oder $mRn$ und sagen \textit{$m$ steht in Relation mit $n$ (bezüglich $R$)}.

Ist $M = N$, so heißt $R$ auch Relation auf $M$. In diesem Fall nennen wir $R$ auch 
\index{Relation!homogen}\textbf{homogen}

\end{definition}

\begin{notiz} Eine Relation $R$ zwischen $M$ und $N$ ist ein Teilmenge $R \subseteq M \times N$ des kartesischen 
Produktes.
\end{notiz}

\begin{notiz}
Eine Funktion $f: D \longrightarrow \mathbb R$ (mit $D \subseteq \mathbb R$) ist eine Relation $R$ zwischen $D$ und 
$\mathbb R$ mit folgender besonderen Eigenschaft: 

$\qquad$ Für jedes $x \in D$ gibt es genau ein $y \in \mathbb R$ mit $(x,y) \in R$. 

Entsprechend ist auch jede Abbildung $f: M \longrightarrow N$ eine Relation $R$ zwischen $M$ und $N$ mit der 
Eigenschaft: 

$\qquad$ Für jedes $x \in M$ gibt es genau ein $y \in N$ mit $(x,y) \in R$. 

Umgekehrt definiert jede Relation $R$ zwischen $M$ und $N$ mit dieser Eigenschaft auch eine Abbildung 
$f: M \longrightarrow N$, und zwar wie folgt: 

Ist $x \in M$, so gibt es genau ein Tupel $(x,y) \in M \times N$ mit $(x,y) \in R$. Setzen wir 
	$$ f(x) = y $$ 
so wird dadurch eine (eindeutig bestimmte) Abbildung $f: M \longrightarrow N$ definiert. 
\end{notiz}

\begin{beispiel}
Die Funktion $f: \mathbb R \longrightarrow \mathbb R$ mit $f(x) = x^2 + x^3$ entspricht der Relation 
	$$ R = \{ (x, x^2+x^3) \, \vert \, x \in \mathbb R \} \subseteq \R \times \R $$ 
\end{beispiel}

\begin{beispiel}
Die Relation
	$$ R = \{ (x,x^3) \, \vert \, x \in \mathbb R \} \subseteq \R \times \R$$ 
beschreibt eine Funktion $f: \R \longrightarrow \R$, die explizit durch $f(x) = x^3$ beschrieben wird. 
Diese Relation (bzw. diese Funktion) kann auch dargestellt werden als 
	$$ R = \{ (\sqrt[3]{x},x) \, \vert \, x \in \mathbb R \} \subseteq \R \times \R $$ 
\end{beispiel}

\begin{beispiel}
Die Relation
	$$ R = \{ (x^2,x^3) \, \vert \, x \in \mathbb R \} \subseteq \R \times \R $$ 
beschreibt keine Funktion $f: \R \longrightarrow \R$, denn zu $-1$ gibt es kein $y$ mit 
$(-1,y) \in R$. Allerdings definiert die Einschränkung 
	$$ R = \{ (x^2,x^3) \, \vert \, x \in \mathbb R_+ \} \subseteq \R_+ \times \R_+ $$ 
eine Funktion $f: \R _+\longrightarrow \R_+$, die explizit durch $f(x) = \sqrt{x^3}$ beschrieben wird.
\end{beispiel}

\begin{beispiel}\label{relation_kind} ist $M$ die Menge aller Menschen, so definiert die Beziehung $R$: 
\textit{ist Kind von} eine Relation auf $M$.
\end{beispiel}

\begin{beispiel}\label{relation_geq} Ist $M = \mathbb Z$ die Menge der ganzen Zahlen, so definiert die Beziehung $R$: 
\textit{ist größer oder gleich} eine Relation auf $M$, die mit $\geq$ bezeichnet wird. Ein Zahlenpaar 
$(a,b)$ ist also genau dann in $R \subseteq \mathbb Z \times \mathbb Z$, wenn $a \geq b$.
\end{beispiel}

\begin{beispiel}\label{relation_eq} Ist $M$ die beliebige Menge, so defniert die Beziehung $R$: 
\textit{ist  gleich} eine Relation auf $M$, die mit $=$ bezeichnet wird. Ein Paar 
$(a,b) \in M \times M$ ist also genau dann in $R$, wenn $a = b$.
\end{beispiel}


\begin{beispiel}\label{relation_modn} Ist wieder $M = \mathbb Z$ die Menge der ganzen Zahlen und ist $n \in 
\mathbb Z$ eine vorgegebene Zahl, so definiert die Beziehung $R$: \textit{unterscheiden sich um ein 
Vielfaches von $n$} eine Relation of $\mathbb Z$. Ein Zahlenpaar $(a,b)$ ist also genau dann in $R$ wenn $a - b$ 
durch $n$ teilbar ist.
\end{beispiel}

Wir interessieren uns vor allem für Relation auf einer Menge $M$.

\begin{definition} Betrachte eine Relation $R$ auf einer Menge $M$.

\begin{itemize} 
\item $R$ heißt \index{Relation!reflexiv}\textbf{reflexiv}, wenn für alle $m$ aus $M$ gilt: $m \sim_R m$.
\item $R$ heißt \index{Relation!transitiv}\textbf{transitiv}, wenn gilt: 
   	$$ m_1 \sim_R m_2 \, \textrm{ und } m_2 \sim_R m_3 \, \Longrightarrow \, m_1 \sim_R m_3 $$
\item $R$ heißt \index{Relation!symmetrisch}\textbf{symmetrisch}, wenn gilt:
   	$$ m_1 \sim_R m_2 \, \Longrightarrow \, m_2 \sim_R m_1 $$
\item $R$ heißt \index{Relation!Äquivalenzrelation}\textbf{Äquivalenzrelation}, wenn $R$ reflexiv, 
transitiv und symmetrisch ist.
\item $R$ heißt \index{Relation!antisymmetrisch}\textbf{antisymmetrisch}, wenn gilt:
   	$$ m_1 \sim_R m_2 \, \textrm{ und } m_2 \sim_R m_1 \, \Longrightarrow \, m_1 = m_2 $$
\item $R$ heißt \index{Relation!asymmetrisch}\textbf{asymmetrisch}, wenn gilt:
   	$$ m_1 \sim_R m_2 \, \Longrightarrow \, \neg \left( m_2 \sim_R m_1 \right) $$
\end{itemize}
\end{definition}

\begin{beispiel} Die Relation $R$: \textit{ist Kind von} aus Beispiel~\ref{relation_kind} ist weder transitiv noch 
reflexiv noch symmetrisch. Betrachten wir stattdessen die Relation $R'$: \textit{ist Nachkomme von}, so ist $R'$ 
transitiv, aber nicht reflexiv (wenn wir eine Person nicht als ihren eigenen Nachkommen auffassen wollen) 
und nicht symmetrisch. Betrachten wir $R''$: \textit{ist verwandt mit}, so ist $R''$ reflexiv, transitiv und 
symmetrisch, also eine Äquivalenzrelation.

Die Relation $\geq$ aus Beispiel~\ref{relation_geq} ist reflexiv, transitiv und antisymmetrisch. Sie ist nicht 
symmetrisch.

Die Relation $=$ aus Beispiel~\ref{relation_eq} ist reflexiv, transitiv, symmetrisch und antisymmetrisch. 

Die Relation $R$: \textit{unterscheiden sich um ein Vielfaches von $n$} aus Beispiel~\ref{relation_modn} ist 
eine Äquivalenzrelation. Sie ist nicht antisymmetrisch
\end{beispiel}

Jede Äquivalenzrelation $\sim$ auf $M$ liefert uns eine natürliche Zerlegung von $M$ in disjunkte Teilmengen:

\begin{definition} Wir betrachten eine Äquivalenzrelation $\sim_R$ auf $M$.

Zwei Elemente $m, n \in M$ hei{ss}en \textbf{äquivalent} (bezüglich $\sim_R$), wenn $m \sim_R n$.

Eine Teilmenge $A \subseteq M$ heißt \index{Äquivalenzklasse}\textbf{Äquivalenzklasse}, wenn gilt

\begin{itemize}
\item Sind $m, n \in A$, so ist $m \sim_R n$.
\item Ist $m \in A$ und $n \in M$ mit $n \sim_R m$, so ist $n \in A$.
\end{itemize}

Ist $m \in M$, so heißt $[m]_R := \{ n \in M: n \sim_R m \}$ die \textbf{Äquivalenzklasse von $m$}.
\end{definition}

\begin{notiz} Ist $\sim_R$ eine Äuqivalenzrelation auf $M$ und sind $m, n \in M$, so gilt entweder  
$[m]_R = [n]_R$ oder $[m]_R$ und $[n]_R$ sind disjunkt. Das folgt sofort aus den Definitionen.

Eine Äquivalenzrelation $\sim_R$ auf $M$ zerlegt also $M$ in disjunkte Teilmengen, die Äquivalenzklassen. 
Wir schreiben $M/\sim_R$ für die Menge der Äquivalenzklassen, also 
  	$$ M/\sim_R = \{ [m]_R\, \vert \, m \in M \} $$
\end{notiz}

\begin{definition} Ist $R$ eine Äquivalenzrealtion auf $M$ und $A$ ein Äquivalenzklasse (bezüglich $R$), 
so heißt ein beliebiges Element $a \in A$ ein 
\index{Äquivalenzklasse!Repräsentant}\textbf{Repräsentant} der Äquivalenzklasse $A$.

Ein \index{Äquivalenzklasse!Repräsentantensystem}\textbf{Repräsentantensystem} der Äquivalenzrelation 
$R$ ist eine Teilmenge $N \subseteq M$ die genau einen Repräsentanten jeder Äquivlenzklasse enthält.
\end{definition}

\begin{beispiel}\label{rela_z_mod_n} 
Für die Äquivalenzrelation $\sim_R:$ \textit{unterscheiden sich um ein Vielfaches von $n$} aus 
Beispiel~\ref{relation_modn} gilt:

\begin{itemize}
\item Falls $n = 0$:
  	$$ \mathbb Z/\sim_R = \mathbb Z$$
\item Falls $n > 0$: 
  	$$ \mathbb Z/\sim_R = \{ [0]_R, [1]_R, \ldots , [n-1]_R \} $$
\item Falls $n < 0$:
  	$$ \mathbb Z/\sim_R = \{ [0]_R, [1]_R, \ldots , [-n-1]_R \} $$
\end{itemize}
Für $n \neq 0$ sind die angegebenen Äquivalenzklassen paarweise disjunkt. Wir schreiben in diesem Fall 
auch $\mathbb Z_n$, $\mathbb Z/(n)$ oder $\mathbb Z / n \mathbb Z$ für $\mathbb Z/\sim_R$.

Die Menge $\{0, 1, \, \ldots , n-1\}$ bildet also ein Repräsentantensystem der Relation $R$. Es gibt aber 
noch viele weitere Repräsentantensysteme, etwa $\{1, 2, \ldots, n \}$ oder $\{n, n+1, \ldots, 2n-1\}$ oder
$\{0, n+ 1, 2n+2, 3n+3 \ldots, n^2-1 \}$.
\end{beispiel}

\bigbreak

Neben Äquvalenzrelation eine besondere Rolle spielen Vergleichsrelationen, wie etwa die Relationen $\geq$ 
oder $>$ auf den reellen Zahlen. 

\begin{definition} Es sei $R$ eine Relation auf eine Menge $M$.

$R$ heißt \index{Relation!Ordnung}\textbf{Ordnungsrelation} oder \textbf{Ordnung} auf $M$, wenn sie 
reflexiv, transitiv und antisymmetrisch ist.

$R$ heißt \index{Relation!strikte Ordnung}\textbf{strikte Ordnungsrelation} oder \textbf{strikte Ordnung} 
auf $M$, wenn sie asymmetrisch und  transitiv ist.
\end{definition}

\begin{beispiel} Die Relation $\geq$ aus Beispiel~\ref{relation_geq} ist eine Ordnungsrelation aber keine 
strikte Ordnungsrelation.
\end{beispiel}

\begin{beispiel} Die Relation $>$ auf den ganzen Zahlen ist eine strikte Ordnungsrelation.
\end{beispiel}


\begin{beispiel} Wir betrachten eine Menge $M = \{m_1, m_2, \ldots , m_t\}$ von Menschen. Auf $M$ definieren wir die 
Relation $R$ durch $m_1Rm_2$ genau dann, wenn $m_1$ älter ist als $m_2$. Dann ist $R$ eine strikte Ordnung 
auf $M$. Die Relation $R'$ definiert durch $m_1 R' m_2$ genau dann, wenn $m_1$ mindestens so alt ist wie 
$m_2$ ist eine Ordnung auf $M$.
\end{beispiel}

\bigbreak

\begin{aufgabe} Bestimmen Sie jeweils die Mengen $A \cup B$, $A \cap B$, 
$A \setminus B$ und $(A \cup B) \setminus (A \cap B)$ möglichst explizit:
\begin{itemize}
\item[a)] $A = \{ x \in \mathbb Z: -5 \leq x \leq 10 \}, \quad  B = \{ x \in [-2,8]:
2x \in \mathbb N \}$.
\item[b)] $A = \{x \in \mathbb R: x^2 \leq 1 \}, \quad B = \{ x \in \mathbb R: (x - 1)^2 
\leq 1 \}$.
\end{itemize}
\end{aufgabe}

\begin{aufgabe}\label{mengen_aufg_distr} Zeigen Sie das zweite Distributivgesetz der Mengenlehre:
 $$(A \cap B) \cup C = (A \cup C) \cap (B \cup C)$$
\end{aufgabe}

\begin{aufgabe}\label{mengen_aufg_verschm} Zeigen Sie die Verschmelzungsgesetze der Mengenlehre:

Für Teilmengen $A, B \subseteq M$ gilt:
 	$$ \begin{array} {l c l}
  	A \cap (A \cup B) & = & A  \\
  	A \cup (A \cap B) & = & A 
 	\end{array} $$
\end{aufgabe}

\begin{aufgabe}\label{mengen_de_morgan} Zeigen Sie die \textbf{de Morganschen Regeln} der Mengenlehre:

Für Teilmengen $A, B \subseteq M$ gilt:
  	$$ \begin{array} {l c l}
   	\overline{A \cap B} & = & \overline{A} \cup \overline{B} \\
   	\overline{A \cup B} & = & \overline{A} \cap \overline{B}
  	\end{array} $$
\end{aufgabe}

\begin{aufgabe}\label{mengen_diff_komp} Zeigen Sie, dass für zwei Teilmengen $A, B \subseteq M$ gilt:
  	$$ A \setminus B = \overline{ \overline{A} \cup B} $$
\end{aufgabe}

\begin{aufgabe} Auf der Menge $\mathbb Z \times \mathbb Z$ definieren wir ein Relation $R$ durch 
$(a,b)R(c,d)$ genau dann, wenn $a+b = c+d$ ist. Zeigen sie, dass $R$ eine Äquivalenzrelation  
auf $\mathbb Z \times \mathbb Z$ ist und bestimmen Sie ein Repräsentantensystem von $R$. 
\end{aufgabe}

\begin{aufgabe}\label{relation_rat_zahlen} Auf der Menge $M := \{ (a,b) \vert \, a \in \mathbb Z 
\wedge b \in \mathbb Z \setminus \{ 0 \}$ definieren wir ein Relation $R$ durch 
$(a,b)R(c,d)$ genau dann, wenn $a \cdot d = c \cdot b$ ist. Zeigen sie, dass $R$ eine Äquivalenzrelation  
auf $M$ ist. 
\end{aufgabe}

\begin{notiz} Die Menge $M / \sim_R$ der Äquivalenzklassen der Relation $R$ aus Aufgabe~\ref{relation_rat_zahlen} 
ist die Menge der rationalen Zahlen.
\end{notiz}

\begin{aufgabe} Heute ist ein Donnerstag. Welcher Wochentag ist in 1000 Tagen? Welcher in 3572? Welcher 
Wochentag war vor 8000 Tagen?
\end{aufgabe}

\begin{aufgabe} Auf der Menge $\mathbb Z \times \mathbb Z$ definieren wir ein Relation $R$ durch 
$(a,b)R(c,d)$ genau dann, wenn $a < c$ oder $a = c$ und $b \leq d$ ist. Zeigen sie, dass $R$ 
eine Ordnungsrelation auf $\mathbb Z \times \mathbb Z$ ist.
\end{aufgabe}

\begin{aufgabe} Es sei $M$ eine Menge und $\mathfrak{P}(M)$ ihre Potenzmenge. Zeigen Sie, dass die 
Relation $R$ auf $\mathfrak{P}(M)$, gegeben durch $N_1 R N_2$ genau dann, wenn $N_1 \subseteq N_2$ eine 
Ordnung auf $\mathfrak{P}(M)$ definiert, und dass die Relation $R'$ auf $\mathfrak{P}(M)$, gegeben durch 
$N_1 R' N_2$ genau dann, wenn $N_1 \subset N_2$ eine strikte Ordnung auf $\mathfrak{P}(M)$ definiert.
\end{aufgabe}
  

\section{Gruppen}\label{section_gruppe}

\setcounter{definition}{0}
\setcounter{beispiel}{0}
\setcounter{notiz}{0}

Wir haben in den vorangegangenen Abschnitten verschiedene Mengen mit unterschiedlichen Strukturen kennengelernt. 
So haben wir etwa gesehen, dass wir auf den reellen Zahlen $\mathbb R$ eine Addition und eine Multiplikation 
haben , für die einige Gesetze erfüllt sind, nämlich die Gesetze $A1$ -- $A4$, $M1$ -- $M4$ und $D$. 
Ähnliches gilt für die rationalen Zahlen $\mathbb Q$. Wenn wir dagegen die ganzen Zahlen $\mathbb Z$ 
betrachten, so sehen wir, dass hier nicht mehr alle Gesetze gelten, denn nicht zu jeder ganzen Zahl $m \in 
\mathbb Z$ gibt es eine ganze Zahl $n \in \mathbb Z$ mit $m \cdot n = 1$. Also gilt $M4$ in $\mathbb Z$ 
nicht. Wenn wir uns noch weiter einschränken und nur noch die natürlichen Zahlen $\mathbb N$ betrachten, 
dann sehen wir, dass hier auch $A4$ nicht mehr gilt. Wir werden noch viele weitere Beispiele kennenlernen, 
in denen einige dieser Gesetze gelten, aber nicht alle. Daher wollen wir diese Gesetze und Gruppen dieser 
Gesetze in diesem Abschnitt isoliert voneinander betrachten.

\medbreak.

\begin{definition} Es sei $M$ eine Menge. Eine Abbildung 
  	$$ \circ : M \times M \longrightarrow M $$
also eine Abbildung mit Definitonsbereich $M \times M$ und Bildbereich $M$ heißt 
\index{Verknüpfung}\textbf{(innere) Verknüpfung} von $M$. 

Wir schreiben in diesem Fall $m \circ n$ für $\circ(m,n)$.
\end{definition}

\begin{beispiel} Es sei $M = \mathbb Z$ und 
  	$$ \circ = '+': \mathbb Z \times \mathbb Z \longrightarrow \mathbb Z, \quad (a,b) \longmapsto a + b $$
die Addition ganzer Zahlen. Dann ist $'+'$ eine innere Verknüpfung auf $\mathbb Z$.
\end{beispiel}

\begin{beispiel}\label{gruppe_monoid_abbild} 
Es sei $M = \{f: \mathbb R \longrightarrow \mathbb R \}$ die Menge aller Abbildungen 
$f: \mathbb R \longrightarrow \mathbb R$ und es sei  
  	$$ \circ : M \times M \longrightarrow M, \quad (f, g) \longmapsto g \circ f $$
die Komposition von zwei Abbildungen. Dann ist $\circ$ eine innere Verknüpfung von $M$.
\end{beispiel}

\begin{beispiel}\label{gruppe_operation_max} Es sei $M = \mathbb R$ und es sei
  	$$ \circ : M \times M \longrightarrow M, \quad (a,b) \longmapsto \textrm{max}\{a,b\} $$
die Zuordnung, die jedem Paar reeller Zahlen die größere von beiden zuordnet. Dann ist
$\circ$ eine innere Verknüpfung von $M$. (Analoges gilt für das Minimum).
\end{beispiel}

\begin{beispiel}\label{gruppe_operation_potenzm} Es sei $M$ eine beliebige Menge und es sei 
$\mathfrak{P}(M)$ ihre Potenzmenge. Dann wird durch
  	$$ \circ : \mathfrak{P}(M) \times \mathfrak{P}(M) \longrightarrow \mathfrak{P}(M), \quad
    	(A,B) \longmapsto A \cup B $$
also das Bilden der Vereinigungsmenge, eine innere Verknüpfung auf $\mathfrak{P}(M)$ definiert.
(Analoges gilt für die Durchschnittsbildung).
\end{beispiel}

\begin{beispiel}\label{gruppe_operation_non_ass} Auf der Menge $M = \mathbb R$ definieren wir
  	$$ \circ : M \times M \longrightarrow M $$
durch 
  	$$ \circ(a,b) = \left\{ \begin{array}{l c l} 1 & \quad & \textrm{ falls } a \geq b \\
	0 & & \textrm{ falls } a < b \end{array} \right. $$
Dann ist $\circ$ eine innere Operation auf $M$.
\end{beispiel}

\begin{beispiel}\label{gruppe_z_null} Es sei $M = \mathbb Z$ und 
  	$$ \circ = '0': \mathbb Z \times \mathbb Z \longrightarrow \mathbb Z, \quad (a,b) \longmapsto 0 $$
die die Nullabbildung. Dann ist $'0'$ eine innere Verknüpfung auf $\mathbb Z$.
\end{beispiel}

Für eine Menge $M$ mit einer Verknüpfung $\circ$ schreiben wir kurz $(M, \circ)$.

\begin{definition} Eine nichtleere Menge $(M, \circ)$ mit einer Verknüpfung $ \circ$ heißt 
\index{Halbgruppe}\textbf{Halbgruppe}, 
wenn $\circ$ das Assoziativgesetz erfüllt, also wenn gilt:
  	$$ (n \circ m) \circ l = n \circ (m \circ l) \qquad \textrm{ für alle } \, l, m, n \in M $$ 
\end{definition}

\begin{beispiel} $(\mathbb N, +)$ ist eine Halbgruppe. \end{beispiel}

\begin{beispiel} $(\mathbb N \setminus \{ 0 \}, +)$ ist eine Halbgruppe. \end{beispiel}

\begin{beispiel} $(\mathbb N \setminus \{ 0 \}, \cdot)$ ist eine Halbgruppe. \end{beispiel}

\begin{beispiel} $(\mathbb Z, \cdot)$ ist eine Halbgruppe \end{beispiel}

\begin{beispiel} Ist $M = \{f: \mathbb R \longrightarrow \mathbb R \}$ die Menge aller Abbildungen aus 
Beispiel~\ref{gruppe_monoid_abbild} mit Verknüpfung $\circ$, so ist $(M \circ)$ eine Halbgruppe.
\end{beispiel}

\begin{beispiel} $(\mathbb Z, '0')$ aus Beispiel~\ref{gruppe_z_null} ist eine Halbgruppe \end{beispiel}

\begin{beispiel} $(\mathbb R, \circ)$ aus Beispiel~\ref{gruppe_operation_max} ist eine Halbgruppe.
\end{beispiel}

\begin{beispiel} $(\mathfrak{P}(M), \circ)$ aus Beispiel~\ref{gruppe_operation_potenzm} ist eine 
Halbgruppe.
\end{beispiel}

\begin{beispiel} $(\mathbb R, \circ)$ aus Beispiel~\ref{gruppe_operation_non_ass} ist keine Halbgruppe. 
Hierfür gilt nämlich
  	$$ ((-1) \circ (-2)) \circ(-3) = 1 \neq 0 = (-1) \circ ((-2) \circ (-3)) $$
\end{beispiel}

\begin{definition} Ein Element $e$ einer Halbgruppe $(M, \circ)$ heißt 
\index{neutrales Element}\textbf{neutrales Element} der Halbgruppe, wenn gilt 
  	$$ m \circ e = m, \quad e \circ m = m \qquad \textrm{ für alle }\, m \in M $$
Eine Halbgruppe $(M, \circ)$ mt neutralem Element $e$ heiß \index{Monoid}\textbf{Monoid}. Wir 
schreiben hierfür auch $(M, \circ, e)$
\end{definition}

\begin{notiz} Das neutrale Element eines Monoids $(M,\circ)$ ist eindeutig. Sind nämlich $e$ und $e'$ zwei 
Elemente aus $M$ mit de Eigenschaft des neutralen Elements, so folgt aus ebendieser Eigenschaft
  	$$ e' = e \circ e' = e $$
\end{notiz}

\begin{beispiel} $(\mathbb N, +, 0)$ ist ein Monoid mit neutralem Element 0. \end{beispiel}

\begin{beispiel} $(\mathbb N \setminus \{ 0 \}, +)$ ist kein Monoid. \end{beispiel}

\begin{beispiel} $(\mathbb N \setminus \{ 0 \}, \cdot)$ ist ein Monoid mit neutralem Element 1.
\end{beispiel}

\begin{beispiel} $(\mathbb Z, +, 0)$ ist ein Monoid mit neutralem Element 0. 
$(\mathbb Z, \cdot)$ ist ein Monoid mit neutralem Element 1. \end{beispiel}

\begin{beispiel} Ist $M = \{f: \mathbb R \longrightarrow \mathbb R \}$ die Menge aller Abbildungen aus 
Beispiel~\ref{gruppe_monoid_abbild} mit Verknüpfung $\circ$, so ist $(M \circ)$ ein Monoid mit 
neutralem Element 
 	$$ e: \mathbb R \longrightarrow \mathbb R, \quad x \longmapsto x \, . $$
\end{beispiel}

\begin{beispiel} $(\mathbb Z, '0')$ aus Beispiel~\ref{gruppe_z_null} ist kein Monoid. \end{beispiel}

\begin{definition} Ist $(M, \circ)$ ein Monoid mit neutralem Element $e$ und ist $m \in M$, so heißt 
ein Element $n \in M$ \index{inverses Element}\textbf{inverses Element} zu $m$ wenn gilt
  	$$ m \circ n  = e, \qquad n \circ m = e $$
In diesem Fall schreiben wir $m^{-1}$ für $n$.
Ein Monoid $(G, \circ)$ heißt \index{Gruppe}\textbf{Gruppe}, wenn es zu jedem Element $m \in G$ eine 
inverses Element in $G$ gibt.
\end{definition}

\begin{beispiel} $(\mathbb N, +)$ ist keine Gruppe. So gibt es etwa keine natürliche Zahl $n$ mit
$1 + n = 0$. \end{beispiel}

\begin{beispiel} $(\mathbb Z, +)$ ist eine Gruppe. Für jede Zahl $z$ gilt nämlich 
  	$$ z + (-z) = 0 $$
und mit $z$ ist auch $-z$ in $\mathbb Z$. 
 $(\mathbb Z, \cdot)$ ist keine Gruppe. So gibt es etwa zu $0$ keine ganze Zahl $z$ mit $0 \cdot z = 1$.
\end{beispiel}

\begin{beispiel} $(\mathbb R, +), (\mathbb Q, +)$ sind Gruppen, aber $(\mathbb R, \cdot), 
(\mathbb Q, \cdot)$ sind keine Gruppen, da es bezüglich $\cdot$ kein inverses Element zu $0$ gibt.

$(\mathbb R \setminus \{0\}, \cdot), (\mathbb Q  \setminus \{0\}, \cdot)$ sind Gruppen, da es zu 
jeder von Null verschiedenen reellen oder rationalen Zahl ein multiplikatives Inverses in $\mathbb R$ 
oder $\mathbb Q$ gibt, aber $(\mathbb Z \setminus \{0\}, \cdot)$ ist keine Gruppe, denn $2$ hat kein 
multiplikatives Inverses in $\mathbb Z$.
\end{beispiel}

\begin{beispiel} 
Ist $M = \{f: \mathbb R \longrightarrow \mathbb R \}$ die Menge aller Abbildungen aus 
Beispiel~\ref{gruppe_monoid_abbild} mit Verknüpfung $\circ$, so ist $(M \circ)$ keine Gruppe. Ist 
nämlich $f \in M$ die Nullabbildung, $f(x) = 0$, so gibt es kein inverses Element zu $f$.

Ist aber $M' = \{f: \mathbb R \longrightarrow \mathbb R \}$ die Menge aller bijektiven Abbildungen aus 
$M$ mit Verknüpfung $\circ$, so ist $(M \circ)$ eine Gruppe. Ist 
nämlich $f \in M'$, so existiert eine inverse Abbildung $f^{-1} \in M'$ mit $f \circ f^{-1} = e$ und 
$f^{-1} \circ f = e$, wie wir im Abschnitt über Funktionen gesehen haben.

Ganz allgemein gilt: 

Ist $A$ eine beliebige Menge, so ist 
	$$ \mathrm{Aut}(A) = \{ f: A \longrightarrow A \, \vert \,\, f \text{ ist bijektiv} \} $$
zusammen mit der Komposition von Abbildungen eine Gruppe. 
\end{beispiel}

\begin{beispiel}\label{gruppe_permutation} 
Ist $M_n = \{1, 2, \, \ldots , n \}$ die Menge der Zahlen $1, 2, \ldots, n$, und ist $S_n$ die Menge der 
bijektiven Abbildungen auf $M_n$, so heißt 
$S_n$ \index{Permutationsgruppe}\textbf{Permutationsgruppe} der Zahlen $1,  
\ldots, n$ und ihre Elemente heißen Permutationen von $1, \ldots, n$. 

Ein Element $\sigma \in S_n$ lässt sich am besten tabellarisch darstellen

\medbreak

	$$ \begin{array}{ | c | c | c | c | }
	\hline
	1 & 2 & \quad \ldots \quad & n  \\
	\hline
	\sigma(1) & \sigma(2) & \quad \ldots  \quad & \sigma(n)  \\
	\hline
	\end{array} $$

%\begin{tabular}{ | c | c | c | c | }
%\hline
%1 & 2 & $\quad \ldots \quad$ & $n$  \\
%\hline
%$\sigma(1)$ & $\sigma(2)$ & $\quad \ldots  \quad$ & $\sigma(n)$  \\
%\hline
%\end{tabular}

\medbreak

Hierfür schreiben wir auch 
  	$$ \sigma = \left( \begin{matrix} 1 & 2 & \ldots & n \\ \sigma(1) & \sigma(2) & \ldots & \sigma(n) 
   	\end{matrix} \right) $$
In der zweiten Zeile dieser Darstellung tauchen wieder alle Zahlen $1, \ldots, n$ auf, allerdings in 
geänderter, also permutierter Reihenfolge. Darum nennen wir die Elemente $\sigma \in S_n$ auch 
\textbf{Permutationen} der Zahlen $1, \ldots, n$.

So ist etwa 
  	$$ \sigma = \left(\begin{matrix} 1 & 2 & 3 & 4 \\ 4 & 2 & 3 & 1 \end{matrix} \right) $$
die Permutation der Zahlen $1, 2, 3, 4$, die die Zahlen 1 und 4 vertauscht (und alle anderen festlässt. 
Hierfür schreiben wir auch kurz $\langle 1 \, 4 \rangle$. Ganz allgemein schreiben wir für eine 
Permutation $\sigma \in S_n$. die nur die Zahlen $i$ und $j$ ($i \neq j$) vertauscht kurz
$\langle i \, j \rangle$ und nennen sie die \index{Permutationsgruppe!Transposition}\textbf{Transposition} 
der Zahlen 
$i$ und $j$. Wegen $\langle i \, j \rangle =  \langle j \, i \rangle$ können wir dabei stets annehmen, 
dass $i < j$.

Jede Permuation lässt sich (in nicht--eindeutiger Weise) aus Transpositionen zusammensetzen, d.h. ist 
$\sigma \in S_n$ so gibt es Zahlenpaare $(i_1, j_1), (i_2, j_2), \ldots$, $(i_t, j_t)$ mit $1 \leq i_{\tau}
< j_{\tau} \leq n$, so dass
  	$$ \sigma = \langle i_1 \, j_1 \rangle \circ \langle i_2 \, j_2 \rangle \circ \cdots \circ 
	\langle i_t \, j_t \rangle $$

Mit $\mathrm{id} = \mathrm{id}_n$ bezeichnen wir das Element von $S_n$, das die Reihenfolge der Zahlen nicht 
verändert, also 
  	$$ \textrm{id} = \left( \begin{matrix} 1 & 2 & \ldots & n \\  1 & 2 & \ldots & n \end{matrix} \right) $$

%Ferner definieren wir eine Verknüpfung $\circ$ auf $S_n$ durch die Komposition von Abbildungen, also 
%  	$$ \sigma \circ \tau = \left( \begin{matrix} 1 & 2 & \ldots & n \\ \sigma(\tau(1)) & \sigma(\tau(2)) & 
%	\ldots & \sigma(\tau(n))  \end{matrix} \right) $$
%Dann ist $(S_n, \circ)$ eine Gruppe mit neutralem Element $\mathrm{id}$, die 
%\index{Gruppe!Permutationsgruppe}\index{Permutation}\textbf{Permutationsgruppe} von $\{1, \ldots , n \}$.

\medbreak

Eine Permutation entspricht einer Anordnung der Zahlen $1, \ldots, n$, also einem geordnete $n$--Tupel 
$(a_1, \ldots, a_n)$ mit $a_i  \in \{1, \ldots, n \}$ und $a_i \neq a_j$ für $i \neq j$. 
Eine einfache kombinatorische Übung zeigt, dass 
  	$$ \vert S_n \vert \, = n ! $$
\medbreak

Permuationen werden uns noch öfter begegnen, etwa bei den Determinanten. Daher 
wollen wir sie noch etwas genauer untersuchen.

Ist $\tau = \langle i, j \rangle$ die Transposition der Zahlen $i,j$, so gilt hierfür
  	$$ \tau \circ \tau = \textrm{id} $$
und damit gilt allgemein für eine beliebige Permutation $\sigma$ mit Darstellung
  	$$ \sigma  = \langle i_1 \, j_1 \rangle \circ \langle i_2 \, j_2 \rangle \circ \cdots \circ 
	\langle i_t \, j_t \rangle $$
die Beziehung
  	$$ \sigma \circ \langle i_t \, j_t \rangle \circ \cdots \circ \langle i_1 \, j_1 \rangle = \mathrm{id} $$
so dass wir also auch für $\sigma^{-1}$ eine Darstellung mit Transpositionen gefunden haben:
  	$$ \sigma^{-1} = \langle i_t \, j_t \rangle \circ \cdots \circ \langle i_1 \, j_1 \rangle  $$

Eine wichtige Größe bei der Betrachtung von Permutationen ist die Anzahl der Fehlstände. Dabei 
heißt ein Paar $i, j \in \{ 1, \ldots , n\}$ eine Fehlstand von $\sigma$, wenn $i < j$ aber
$\sigma(i) > \sigma(j)$. Wir definieren die \index{Permutationsgruppe!Signatur}\textbf{Signatur} 
$\mathrm{sign}(\sigma)$ von $\sigma$ durch
  	$$ \mathrm{sign}(\sigma) = \left\{ \begin{array} {l c l}
		+1 & \quad &\textrm{ falls die Anzahl der Fehlstände gerade ist} \\
		-1 & \quad &\textrm{ falls die Anzahl der Fehlstände ungerade ist} 
	\end{array} \right. $$
Wir nennen eine Permuation $\sigma$ \textbf{gerade}, wenn $\mathrm{sign}(\sigma) = +1$ und \textbf{ungerade}, 
wenn $\mathrm{sign}(\sigma) = -1$.

Die Signatur hat auch folgende Beschreibung
  	$$ \mathrm{sign}(\sigma) = \prod_{i<j} \frac {\sigma(j) - \sigma(i)}{j - i} $$
wie man sehr leicht nachrechnet (siehe auch Aufgabe~\ref{gruppe_aufg_permut_1}). Dabei steht 
$\prod$ für das Produktzeichen, also
  	$$ \prod_{i=1}^n a_i = a_1 \cdot a_2 \cdots a_n $$

Wir notieren die folgenden Eigenschaften der Signatur:

\begin{itemize}
\item Ist $\tau = \langle i \, j \rangle$ eine Transposition, so gilt $\textrm{sign}(\tau) = -1$
\item Sind $\sigma, \tau$ zwei Permuationen, so gilt $\textrm{sign}(\sigma \circ \tau) = 
\textrm{sign}(\sigma) \cdot \textrm{sign}(\tau)$ (vergleiche Aufgabe~\ref{gruppe_aufg_permut_2}).
\item $\textrm{sign}(\textrm{id}) = 1$ und $\textrm{sign}(\sigma^{-1}) = \textrm{sign}(\sigma)$.
\item Ist $\sigma = \langle i_1 \, j_1 \rangle \circ \langle i_2 \, j_2 \rangle \circ \cdots \circ 
           \langle i_t \, j_t \rangle$, so ist $\textrm{sign}(\sigma) = (-1)^t$.
\item Für $n \geq 2$ gibt es genauso viele gerade Permutionen wie ungerade Permuationen.
\end{itemize}
\end{beispiel} 

\bigbreak

\begin{beispiel}\label{gruppe_pm} 
Sei $\mathcal{PM} = \{1, -1\}$ mit der Komposition $a \circ b = a \cdot b$. 
Dann ist $\mathcal{PM}$ eine Gruppe.
\end{beispiel}


\begin{definition} Eine Gruppe $(G, \circ)$ heißt \index{Gruppe!kommutative}\textbf{kommutativ} oder
\textbf{abelsch} wenn für je zwei Elemente $g, h \in G$ gilt:
  $$ g \circ h = h \circ g $$
\end{definition}

\begin{beispiel} $(\mathbb Z, +), (\mathbb R, +)$ und $(\mathbb Q, +)$ sind kommutative Gruppen.

$(\mathbb R \setminus \{0\}, \cdot)$ und $(\mathbb Q  \setminus \{0\}, \cdot)$ sind kommutative Gruppen
\end{beispiel}

\begin{beispiel} Ist $G$ die Gruppe der bijektiven Abbildungen von $\mathbb R$ in sich, 
so ist $G$ nicht kommutativ, wie wir schon im Beispiel~\ref{function_komp_non_komm} 
in Abschnitt~\ref{section_function} gesehen haben.
\end{beispiel}

\begin{beispiel} Für $n \geq 3$ ist die Gruppe $S_n$ der Permuationen 
aus Beispiel~\ref{gruppe_permutation} nicht 
kommutativ. So ist etwa $\langle 1 \, 2 \rangle \circ \langle 1 \, 3 \rangle 
\neq \langle 1 \, 3 \rangle \circ \langle 1 \, 2 \rangle$.
\end{beispiel}

\bigbreak

\begin{definition} Eine Gruppe $(G, \circ)$ heißt 
\index{Gruppe!endliche Gruppe}\textbf{endlich}, wenn $\vert G \vert < \infty$, also 
wenn $G$ nur endlich viele Elemente hat. 

In diesem Fall heißt $\textrm{ord}(G) = \vert G \vert$ die 
\index{Gruppe!Ordnung einer Gruppe}\textbf{Ordnung} der Gruppe $G$.
\end{definition}

\begin{notiz} Eine endliche Gruppe $G$ kann durch ihre Verknüpfungstafel beschrieben werden. Dazu 
schreiben wir $G = \{ g_1, g_2, \ldots , g_n \}$ und erstellen eine Tabelle der Form
  $$ \begin{array} { c | c c c c }
  \circ \, & \, g_1 \, & \, g_2 \,  & \, \ldots \, & \, g_n \, \\ \hline
  g_1\, \, & g_{1,1} & g_{1,2} & \ldots & g_{1,n} \\
  g_2\, \, & g_{2,1} & g_{2,2} & \ldots & g_{2,n} \\
  \vdots\, \, & \vdots & \vdots & \ddots & \vdots \\
  g_n\, \, & g_{n,1} & g_{n,2} & \ldots & g_{n,n} 
  \end{array} $$
wobei $g_{i,j}$ kurz für das Element $g_i \circ g_j$ steht. Genau dann ist die Gruppe kommutativ, wenn 
die Verknüpfungstafel symmetrisch (bezüglich der Diagonale) ist.
\end{notiz}

\begin{beispiel} Die Gruppe $\mathcal{PM}$ aus Beispiel~\ref{gruppe_pm} hat genau zwei Elemente, ist also 
endlich. Ihre Verknüpfungstafel ist
  $$ \begin{array} { c | c c }
  \circ \, & \, -1 \, & \, 1 \,  \\ \hline
  -1\, \, & 1 & -1  \\
  1\, \, & -1 & 1  \\
  \end{array} $$
\end{beispiel}

\begin{beispiel} Wir betrachten die Menge $G = \{0, 1\}$ mit der durch die folgende Tafel gegebenen 
Verknüpfung
  $$ \begin{array} { c | c c }
  \circ \, & \, 0 \, & \, 1 \,   \\ \hline
  0\, \, & 0 & 1  \\
  1\, \, & 1 & 0  \\
  \end{array} $$
Hierdurch wird eine kommutative Gruppe definiert (die Gruppe $\mathbb Z/ 2 \mathbb Z$).
\end{beispiel}

\begin{beispiel} Wir betrachten die Menge $G = \{0, 1, 2\}$ mit der durch die folgende Tafel gegebenen 
Verknüpfung
  $$ \begin{array} { c | c c c c }
  \circ \, & \, 0 \, & \, 1 \,  & \, 3 \,  \\ \hline
  0\, \, & 0 & 1 & 2  \\
  1\, \, & 1 & 2 & 0  \\
  2\, \, & 2 & 0 & 1 \\
  \end{array} $$
Hierdurch wird eine kommutative Gruppe definiert (die Gruppe $\mathbb Z/ 3 \mathbb Z$).
\end{beispiel}

\begin{beispiel}\label{gruppe_z_mod_n_plus} Für $n \geq 1$ betrachten wir $\mathbb Z/n \mathbb Z$, 
die Menge der Äquivalenzklassen der Relation 
''unterscheiden sich um ein Vielfaches von $n$'' aus Beispiel~\ref{rela_z_mod_n}. Hierauf definieren wir 
eine Verknüpfung ''+'' durch 
  $$ [a] + [b] = [a+b] $$
Mit dieser Verknüpfung wird $\mathbb Z/n \mathbb Z$ eine kommutative Gruppe (siehe auch 
Aufgabe~\ref{gruppe_aufg_z_mod_n_plus}).
\end{beispiel}

\begin{beispiel} Die Permuationsgruppe $S_n$ aus Beispiel~\ref{gruppe_permutation} ist endlich. 
\end{beispiel}

\medbreak

Für ein Element $g \in (G, \circ)$ mit neutralem Element $e$ setzen wir $g^0 = e$, $g^2 = g \circ g$, und 
allgemein für $n \geq 1$:
  $$ \begin{array} {l c l}
  g^n & = & \underbrace{g \circ g \circ \cdots \circ g}_{n--\textrm{mal}} \\
  g^{-n} & = & \underbrace{g^{-1} \circ g^{-1} \circ \cdots \circ g^{-1}}_{n--\textrm{mal}}
  \end{array} $$
(so dass also $g^n \circ g^m = g^{n+m}$ für alle $n, m \in \mathbb Z$). 

\begin{lemma}\label{gruppe_endl_ordnung1} Ist $G$ eine endliche Gruppe und $g \in G$, so gibt es ein $n \geq 1$ 
mit $g^n = e$.
\end{lemma}

\beweis{ Wir betrachten die Menge $B(g) := \{ g = g^1, g^2, \ldots, g^n, \ldots \} \subseteq G$. Da $G$ selbst 
endlich ist, ist es sicherlich auch $B(g)$, und daher gibt es $l, k \geq 1$, $l < k$, mit $g^l = g^k$. Dann gilt
  $$e = g^{l} \circ g^{-l} = g^{k} \circ g^{-l} = g^{k-l}$$
wir können also $n = k-l$ setzen.
}

\medbreak

\begin{definition} Ist $G$ eine endliche Gruppe und $g \in G$, so heißt 
  $$ \textrm{ord}(g) := \textrm{min} \{ n \geq 1: g^n = e \} $$
die \index{Gruppe!Ordnung eines Elements}\textbf{Ordnung} von $g$.
\end{definition}

\begin{notiz}\label{gruppe_ordnung_distinkt} Wir betrachten eine endliche Gruppe $G$. 

\begin{enumerate}
\item Genau dann gilt $\textrm{ord}(g) = 1$, wenn $g = e$.
\item Für alle $1 \leq i < j \leq \textrm{ord}(g)$ gilt: $g^i \neq g^j$.
\end{enumerate}
\end{notiz} 

Um den Ordnungsbegriff besser studieren zu können, benötigen wir eine Eigenschaft, die in den ganzen Zahlen 
bzw\. im Monoid $(\mathbb N \setminus \{0 \}, \cdot )$ definiert ist:

\begin{definition} Sind $m,n \in \mathbb N \setminus \{0 \}$, so heißt $m$ \index{Teiler}\textbf{Teiler} 
von $n$, wenn 
es ein $a \in \mathbb N$ gibt mit $a \cdot m = n$, und $m$ heißt \textbf{echter Teiler} von $n$, wenn 
es ein $a \in \mathbb N \setminus \{1, n\}$ gibt mit $a \cdot m = n$.

Ist $m$ ein Teiler von $n$ so schreiben wir hierfür $m \vert n$. Entsprechend schreiben wir $m \not\vert n$, 
falls $m$ kein Teiler von $n$ ist. 
\end{definition}

\begin{beispiel} Ist $n = 6$, so sind $m = 1, 2, 3, 6$ die Teiler von $n$ und $m = 2,3$ die echten Teiler 
von $n$. 
\end{beispiel} 


\begin{lemma}\label{gruppe_endl_ordnung2} Ist $G$ eine endliche Gruppe und $g \in G$, so gilt
  $$ \textrm{ord}(g) \, \vert \, \textrm{ord}(G) $$ 
also die Ordnung des Elements $g$ teilt die Ordnung der Gruppe $G$. Speziell gilt also 
  $$ g^{\textrm{ord}(G)} = e $$
\end{lemma}

\bemerkung{Beweisidee.} 
Es sei $n = \textrm{ord}(g)$. Wie wir schon im Beweis von Lemma~\ref{gruppe_endl_ordnung1} gesehen haben, 
ist die Menge $B(g) = \{ g = g^1, g^2, \ldots \}$ endlich, und zwar gilt genauer
  $$ B(g) = \{ g, g^2, \ldots , g^{n-1}, g^n = e \} $$
und $B(g)$ hat genau $n  = \textrm{ord}(g)$ verschiedene Elemente. 
(Wären nämlich zwei Elemente von in dieser Auflistung gleich, so könnten 
wir den Beweis von Lemma~\ref{gruppe_endl_ordnung1} imitieren und fänden ein $l < n$ 
mit $g^l = e$, im Widerspruch zur Definition der Ordnung des Elements). 

Wir überdecken nun $G$ mit den \textbf{Bahnen} von $g$, dh. wir definieren für $h \in G$ beliebig
  	$$ B_{h}(g) = \{h \circ g, h \circ g^2 , \ldots , h \circ g^{n-1}, h \circ e = h\} $$
Hierfür rechnet man leicht nach

\begin{itemize}
\item $B_h(g)$ hat genau $n$ Elemente.
\item Für $h, h' \in G$ gilt entweder $B_h(g) = B_{h'}(g)$ oder $B_h(g) \cap B_{h'}(g) = \emptyset$.
\end{itemize}

Da $G$ endlich ist, gibt es auch nur endlich viele verschiedene Bahnen, d.h. wir finden $h_1 = e, h_2, \ldots, 
h_t$ mit $B_{h_i} (g) \cap B_{h_j}(g ) = \emptyset$ für $i \neq j$ und $G = B_{h_1}(g) \cup \cdots \cup 
B_{h_t}(g)$. 

Damit gilt aber 
  $$ \begin{array} {l c l}
     \textrm{ord}(G) & = & \vert B_{h_1}(g) \cup \cdots \cup B_{h_t}(g) \vert \\
   & = & \vert B_{h_1}(g) \vert +  \cdots + \vert B_{h_t}(g) \vert \\
   & = & t \cdot n 
  \end{array} $$
 und es folgt $n \vert \textrm{ord}(G)$ und 
  $$ g^{\textrm{ord}(G)} = g^{nt} = \left(g^n\right)^t = e^t = e,$$ 
also die Behauptung.

\medbreak

\bemerkung{Bezeichung.} Ist $G$ eine endliche Gruppe und $g \in G$ so schreiben wir
  	$$ \langle g \rangle := B(g) = \{ g = g^1, g^2, g^3, \ldots \} $$
für die Teilmenge von $G$, bestehend aus allen Potenzen von $g$.

\medbreak

\begin{notiz} Die Menge $\langle g \rangle$ ist selbst wieder eine (endliche) Gruppe der Ordnung
$\textrm{ord}(g)$.
\end{notiz}

\begin{notiz}\label{gruppe_endl_ordn_ele_pot} Ist $g$ ein Element einer endlichen Gruppe $G$ und ist $n \geq 1$, 
so gilt
  $$ \textrm{ord}(g^n) \vert  \textrm{ord}(g) $$
\end{notiz}

\beweis{ Das Element $g^n$ ist Element der endlichen Gruppe $\langle g \rangle$, und damit folgt die Behauptung 
aus Lemma~\ref{gruppe_endl_ordnung2}.
}

\bigbreak

\begin{definition} Eine endliche Gruppe $G$ heißt 
\index{Gruppe!zyklisch}\textbf{zyklisch}, wenn es ein $g \in G$ gibt mit 
  	$$ \langle g \rangle = G $$
In diesem Fall heißt $g$ ein \textbf{Erzeuger} von $G$.
\end{definition}

\begin{beispiel} Die Gruppe $\mathbb Z/(n) \mathbb Z$ ist eine zyklische Gruppe mit Erzeuger $[1]$.
\end{beispiel}

\begin{notiz} Ist $G$ eine endliche Gruppe mit $\textrm{ord}(G) = p$, wobei $p$ eine Primzahl ist, 
so ist $G$ zyklisch. Dazu wählen wir ein beliebige $g \neq e$ in $G$. Da es nur ein neutrales 
Element $e$ in $G$ gibt, existiert ein solches $g$. Gemäß Lemma~\ref{gruppe_endl_ordnung2} gilt 
dann $\textrm{ord}(g) \vert \, p$, also, da $p$ eine Primzahl ist, $\textrm{ord}(g) = 1$ oder 
$\textrm{ord}(g) = p$. Da aber $g \neq e$ muss gelten $\textrm{ord}(g) \neq 1$, und damit $\textrm{ord}(g) 
= p$, also $\langle g \rangle = G$.
\end{notiz}

\section{Ringe}\label{section_gruppe}

Besonders interessant für uns sind Strukturen wie $\mathbb Z$ oder $\mathbb R$, die zwei Verknüpfungen 
tragen. 

\begin{definition} Ein \index{Ring}\textbf{Ring} $(R, + , \cdot)$ ist eine nichtleere Menge $R$ mit 
zwei Verknüpfungen $+$ und 
$\cdot$, die zwei Elemente $0, 1$ enthält, wobei $0 \neq 1$ ist und wobei gilt

  \begin{itemize}
  \item $(R, +, 0)$ ist eine kommutative Gruppe.
  \item $(R, \cdot, 1)$ ist ein Monoid.
  \item Es gelten die Distributivgesetze
  $$ \begin{array} {l c l c l}
    (r + s) \cdot t & = & r \cdot t + s \cdot t & \quad & \textrm{ für alle } r, s, t \in R \\
    r \cdot (s + t) & = & r \cdot s + r \cdot t & \quad & \textrm{ für alle } r, s, t \in R
  \end{array} $$
 \end{itemize}
  
Ist $(R, \cdot, 1)$ ein kommutatives Monoid, so nennen wir $(R, + , \cdot)$ einen kommutativen Ring.
\end{definition}

\begin{notiz} Wenn die Operationen klar sind, werden wir kurz $R$ für $(R, + , \cdot)$ schreiben.
\end{notiz}

\begin{notiz} Wir setzen nicht voraus, dass $(R, \cdot, 1)$ kommutativ ist, da wir mit den Matrizen 
auch nicht--kommutative Ringe kennenlernen werden.
\end{notiz}

\begin{beispiel} $(\mathbb Z, +, \cdot), (\mathbb R, +, \cdot)$ und $(\mathbb Q, +, \cdot)$ sind kommutative 
Ringe.
\end{beispiel}

\begin{beispiel}\label{gruppe_z_mod_n} 
Es sei $n > 1$ eine ganze Zahl $\mathbb Z/ (n)$ die Menge der Äquivalenzklassen der Relation 
''unterscheiden sich um ein Vielfaches von $n$'' aus Beispiel~\ref{rela_z_mod_n} in 
Abschnitt ~\ref{section_mengen}, also 
  	$$ \mathbb Z/ (n) = \{ [0], [1], \ldots, [n-1] \}. $$
Wir definieren auf $\mathbb Z/ (n) $ Verknüpfungen $+$ und $\cdot$ durch
  	$$ \begin{array} {l c l c l}
  	\, [r] + [s] & := & [r \cdot s] & \quad & \textrm{ für alle } \, [r], [s] \in \mathbb Z/ (n) \\
  	\, [r] \cdot [s] & := & [r \cdot s ] & \quad & \textrm{ für alle } \, [r], [s] \in \mathbb Z/ (n)
  	\end{array} $$
Die Operation ''+'' haben wir ja schon in Beispiel~\ref{gruppe_z_mod_n_plus} betrachtet.

Zunächst müssen wir uns überzeugen, dass diese Operationen sinnvoll und wohldefiniert sind. Wir haben 
also zu zeigen: 

Sind $r' \sim_R r$ und $s' \sim_R s$ so gilt
  	$$ [r' + s'] = [r + s], \qquad [r' \cdot s'] = [r \cdot s] $$
Das ist aber leicht einzusehen. Ist etwa $r' = r + an$ und $s' = s + bn$, so gilt
  	$$ r' \cdot s' = r \cdot s + (as + rb + abn) \cdot n $$
also unterscheidet sich $ r' \cdot s'$ nur um ein Vielfaches von $n$ von $r \cdot s$. Der Nachweis für 
die Summation ist noch einfacher. 

Mit diesen Operationen gilt:

$(\mathbb Z/ (n), +, \cdot )$ ist ein kommutativer Ring mit Nullelement $[0]$ und Einselement $[1]$. Die 
Ringeigenschaften ergeben sich dabei leicht aus den entsprechenden Eigenschaften von $(\mathbb Z, +, \cdot )$, 
siehe etwa Aufgabe~\ref{gruppe_aufg_z_mod_n_ring}.
\end{beispiel}

\begin{notiz} Die Ringoperationen in $\mathbb Z/ (n)$ können auch direkt mir Hilfe des Repräsentantensystems 
$0, 1, \ldots, n-1$ erklärt werden: Sind $r, s \in \{0, 1, \ldots , n-1 \}$, so können wir 
Divison durch $n$ mit Rest anwenden und erhalten
  	$$ \begin{array} {l c l}
  	r + s & = & a \cdot n + d \\
  	r \cdot s & = & \alpha \cdot n + \delta 
  	\end{array} $$
für geeignete $a, \alpha \in \mathbb Z$ und $d , \delta \in \{0, 1, \ldots, n-1 \}$, und es gilt
  	$$ \begin{array} {l c l}
  	\, [r] + [s] & = &  [d] \\
  	\, [r] \cdot [s] & = & [delta]
  	\end{array} $$
Das ist unmittelbar klar nach Definition, da ja $[a \cdot n] = [\alpha \cdot n] = [0]$. 
\end{notiz}

\begin{beispiel} Es sei jetzt speziell $n = 4$. Dann gilt in $\mathbb Z/ (4)$:
  	$$ [2] \cdot [2] = [4] = [0] $$
Wir haben also zwei von Null verschiedene Elemente in $\mathbb Z/ (4)$ deren Produkt $[0]$ ergibt. Diese 
Situation tritt in $\mathbb Z$ oder $\mathbb R$ nicht auf.
\end{beispiel}

\begin{beispiel}  Ist $R$ ein kmmutativer Ring, so betrachten wir die Menge
  	$$ R[X] := \{ r_n X^n + r_{n-1} X^{n-1} + \cdots + r_1 X + r_0 \vert \, n \in \mathbb N, r_0, \ldots, r_n \in R \} $$
aller formalen Linearkombinationen von $1, X, X^2, \ldots$ über $R$. Sind
  	$$ f(X) =  r_n X^n + r_{n-1} X^{n-1} + \cdots + r_1 X + r_0, \quad 
	g(X) =  s_m X^m + s_{m-1} X^{m-1} + \cdots + s_1 X + s_0 $$
So definieren wir Summe und Produkt von $f(X)$ und $g(X)$ wie folgt:

Falls $n \geq m$, so setzen wir 
  	$$ f(X) + g(X) = t_n X^n + t_{n-1} X^{n-1} + \cdots + t_1 X + t_0 $$
wobei 
  	$$ t_i = \left\{ \begin{array}{l c l} r_i & \quad & \textrm{ für } i > m \\ r_i + s_i & & 
	\textrm{ für } i \leq m \end{array} \right. $$
und falls $n < m$, so setzen wir 
  	$$ f(X) + g(X) = t_m X^m + t_{m-1} X^{m-1} + \cdots + t_1 X + t_0 $$
wobei 
  	$$ t_i = \left\{ \begin{array}{l c l} s_i & \quad & \textrm{ für } i > n \\ r_i + s_i & & 
	\textrm{ für } i \leq n \end{array} \right. $$

Mit $l = n \cdot m$ definieren wir 
  	$$ f(X) \cdot g(X) =  t_l X^l + t_{l-1} X^{l-1} + \cdots + t_1 X + t_0 $$
wobei
  	$$ t_i = \sum\limits_{k = 0}^{i} r_k \cdot s_{i-k} $$
(mit $r_k = 0$ für $k > n$ und $s_k = 0$ für $k > m$).

Dann ist $R[X]$, zusammen mit dieser Addition und Multiplikation ein kommutativer Ring. Das Einselement ist 
das Element $1(X) = 1$ und das Nullelement ist das Element $0(X) = 0$. Wir nennen $R[X]$ den 
\index{Polynom!Polynomring}\textbf{Polynomring} (in der Variable $X$) über $R$ und seine Elemente $f(X)$ 
\index{Polynom}\textbf{Polynome} mit Koeffizienten aus $R$. 

Ist $ f(X) =  r_n X^n + r_{n-1} X^{n-1} + \cdots + r_1 X + r_0$ ein Polynom mit $r_n \neq 0$, so nennen wir 
$n$ den \index{Polynom!Grad}\textbf{Grad} des Polynoms $f(X)$ und wir nennen $f(X)$ ein Polynom vom 
Grad $n$. Wir schreiben auch $\textrm{deg}(f(X))$ für den Grad des Polynoms $f(X)$.

Jedes vom Nullpolynom verschiedene Polynom hat also einen eindeutigen Grad. Das Nullpolynom bildet eine 
Ausnahme, denn ihm kann kein Grad in vernünftiger Weise zugeordnet werden. Daher lässt man für das 
Nullpolynom jeden Grad zu, und damit ist es das einzige Polynom, das mehr als einen Grad hat.
\end{beispiel}

\begin{aufgabe} Zeigen Sie, dass $R[X]$ ein Ring ist. \end{aufgabe}

\begin{aufgabe} Die Polynome $f(X) \in R[X]$ vom Grad 0 entsprechen genau den Elemente von $R$.
\end{aufgabe}

\begin{definition} Es sei $(R, + , \cdot)$ ein kommutativer Ring. Ein Element $r \in R \setminus \{0\}$ heißt 
\index{Ring!Nullteiler}\textbf{Nullteiler} von $R$ wenn es ein $s \in R \setminus \{0\}$ gibt mit $r \cdot s = 0$.

Ein kommutativer Ring, in dem es keine Nullteiler gibt, heißt \textbf{nullteilerfrei} oder \textbf{Integritätsbereich}
\end{definition}

\begin{beispiel} Die Ringe $\mathbb R$, $\mathbb Q$ und $\mathbb Z$ sind nullteilerfrei.
\end{beispiel}

\begin{beispiel} Die Ringe $\mathbb Z/ (2)$ und $\mathbb Z/ (3)$ sind nullteilerfrei. Das sieht man sofort indem 
man die wenigen Elemente dieser Ringe einzeln durchgeht. 
\end{beispiel}

\begin{beispiel} Die Ringe $\mathbb Z/ (4)$, $\mathbb Z/ (6)$ und $\mathbb Z/ (15)$ sind nicht nullteilerfrei.
\end{beispiel}

\begin{aufgabe}\label{aufgabe_polynom_nzd} Wir betrachten einen nullteilerfreien Ring $R$. Zeigen Sie
\begin{itemize}
\item[a)] Der Polynomring $R[X]$ ist nullteilerfrei.
\item[b)] Sind $f(X), g(X) \in R[X]$ Polynome, die verschieden vom Nullpolynom sind, so gilt 
$\textrm{deg}(f(X) \cdot g(X)) = \textrm{deg}(f(X)) + \textrm{deg}(g(X))$.
\end{itemize} 
\end{aufgabe}

\medbreak

Nullteilerfrei Ringe haben eine für uns interessante Kürzungseigenschaft:

\begin{satz}\label{gruppe_kurzungsregel} 
Ist $R$ ein nullteilerfreier Ring und $a \in R \setminus \{0\}$, so gilt für alle $x, y \in R$.
  	$$ ax = ay \Longrightarrow x = y $$
\end{satz}

\beweis{ Wir haben die Beziehungskette
  	$$ ax = ay \Longrightarrow ax - ay = 0 \Longrightarrow a \cdot (x-y) = 0 \longrightarrow x-y = 0 $$
wobei wir für die  letzte Implikation die Nullteilerfreiheit von $R$ ausgenutzt haben (und die Tatsache, dass
$a \neq 0$) . Aber $x-y = 0$ ist äquivalent zu $x = y$ und es folgt die Behauptung.
}
\medbreak

\begin{beispiel} In $\mathbb Z/ (8)$ gilt
  	$$ [2] \cdot [7] = [2] \cdot [3] $$
aber es gilt 
  	$$ [7] \neq [3] $$
Die Nullteilerfreiheit ist also wichtig für die Kürzungsregel.
\end{beispiel}

\section{Körper}\label{section_fields}

Die Ringe $\mathbb Z$ und  $\mathbb Z/ (n)$ spielen eine wichtige Rolle in der Informatik, und die 
Frage nach ihrer Nullteilerfreiheit bzw. ihren Nullteilern ist dabei von entscheidender Bedeutung.

Wir haben also jetzt unter den kommutativen Ringen eine Klasse mit einer besonderen Eigenschaft, nämlich der
Nullteilerfreiheit, isoliert und gesehen, dass nicht alle Ringe diese Eigenschaft haben. Einige Beispiele 
nullteilerfreier Ringe sind $\mathbb F_p$ für Primzahlen $p$, $\mathbb Z$, $\mathbb Q$ und $\mathbb R$. 
Trotzdem gibt es noch einen entscheidenden Unterschied zwischen $\mathbb Z$ und $\mathbb Q$ oder $\mathbb R$, 
nämlich die Tatsache, dass wir in $\mathbb Q$ oder $\mathbb R$ multiplikative Inverse bilden können. Das 
Führt uns zur letzten Definition in diesem Abschnitt

\begin{definition}  Ein \index{Körper}\textbf{Körper} $(K, + , \cdot)$ ist eine nichtleere Menge $K$ mit 
zwei Verknüpfungen $+$ und 
$\cdot$, die zwei Elemente $0, 1$ enthält, wobei $0 \neq 1$ ist und wobei gilt

\begin{itemize}
  	\item $(K, +, 0)$ ist eine kommutative Gruppe.
  	\item $(K \setminus \{0\}, \cdot, 1)$ ist eine kommutative Gruppe.
  	\item Es gilt das Distributivgesetz
  		$$ \begin{array} {l c l c l}
    		(r + s) \cdot t & = & r \cdot t + s \cdot t & \quad & \textrm{ für alle } r, s, t \in R 
  		\end{array} $$
 \end{itemize}
\end{definition}

\begin{notiz} Da wir voraussetzen, dass $(R \setminus \{0\}, \cdot, 1)$ kommutativ ist, brauchen wir nur 
ein Distributivgesetz zu fordern. Das zweite Distributivgesetz ($t \cdot (r+s) = tr + ts$) folgt sofort 
aus der Kommutativität.
\end{notiz}

\begin{notiz} Jeder Körper ist ein nullteilerfreier kommutativer Ring. Die Umkehrung gilt nicht.
\end{notiz}

\begin{notiz} Schreiben wir alle Bedingungen aus, so sehen wir, dass $(K, +, \cdot)$ genau dann ein Körper 
ist, wenn die Gesetze $A1$ - $A4$, $M1$ - $M4$ und $D$, die wir bei der Einführung der reellen Zahlen 
notiert haben, für $K$  gelten.
\end{notiz} 

\begin{beispiel} $\mathbb Q$ und $\mathbb R$ sind Körper. \end{beispiel}

\begin{beispiel} $\mathbb Z$ ist kein Körper. \end{beispiel}

\medbreak

\begin{notiz}\label{gruppe_einheitengruppe} Ist $(R, +, \cdot)$ ein kommutativer Ring, so heißt ein 
$r \in R$  \textbf{Einheit}, wenn es ein $s \in R$ gibt mit $r \cdot s = 1$. Die Einheiten von $R$ bilden 
eine Gruppe (bezüglich $\cdot$), die \textbf{Einheitengruppe} $E(R)$ von $R$. 
\end{notiz}


\bigbreak

Die Unterscheidung zwischen Körpern und nullteilerfreien Ringen wird durch den folgenden Satz gegeben, 
der trivialerweise aus der Definition folgt:

\begin{satz}\label{gruppe_ring_korper} 
Genau dann ist eine nullteilerfreier Ring $R$ ein Körper wenn jedes Element $x \in R \setminus 
\{0\}$ ein multiplikatives Inverses hat, wenn also $E(R) = R \setminus \{ 0 \}$.
\end{satz}

Ist $K$ ein Körper, so ist der Polynomring $K[X]$ kein Körper, da das Element $X \in K[X]$ kein Inverses besitzt. 
Trotzdem ist der Polynomring von großem Interesse für das Studium von Körpern. Eine wichtige Eigenschaft 
von Polynomen ist nämlich, dass wir Elemente aus $K$ einsetzen können: Ist 
  	$$ f(X) = r_n X^n + r_{n-1} X^{n-1} + \ldots + r_1 X + r_0 \, \in K[X] $$
ein Polynom, und ist $a \in K$, so ist 
  	$$ f(a) := r_n a^n + r_{n-1} a^{n-1} + \ldots + r_1 a + r_0  $$
eine wohldefiniertes Element von $K$, so dass also $f(X)$ eine Abbildung
  	$$ f: K \longrightarrow K, \quad a \mapsto f(a) $$
definiert. 

\begin{definition} Ein $a \in K$ heißt Nullstelle eines Polynoms $f(X) \in K[X]$, wenn $f(a) = 0$
\end{definition}

Eine wichtige Eigenschaft von Polynomringen über Körpern ist

\begin{satz}\label{polynom_koerper_nullstellen} Ist $K$ ein Körper und $f(X)$ ein vom Nullpolynom verschiedenes 
Polynom vom Grad $n$, so hat $f(X)$ höchstens $n$ Nullstellen.
\end{satz}

\beweis{ Wir beweisen die Aussage durch Induktion nach $n$.

Die Aussage ist klar für ein Polynom vom Grad 0. Da das Polynom $f(X)$ nach Voraussetzung nicht das 
Nullpolynom ist, ist $f(X) = r$ mit einem Element $r \in K \setminus \{ 0 \} $. Hierfür gilt offensichtlich
  	$$ f(a) = r \neq 0 $$
für jedes $a \in K$, so dass also $f(X)$ hier 0 Nullstellen hat.

Der Induktionsschritt beruht darauf, dass wir Polynomdivision über beliebigen Körpern durchführen können. Sind 
also $f(X)$, $g(X)$ zwei Polynome mit Koeffizienten aus $K$, und hat $g(X)$ den Grad $n$, so können wir immer
schreiben
  	$$ f(X) = q(X) \cdot g(X) +  r(X) $$
mit einem Polynom $r(X)$ mit $\textrm{deg}(r(X)) < n$ (also $f(X) = g(X) = q(X)$ Rest $r(X)$). Das kann man genauso 
einsehen wie die Polynomdivision über $\mathbb Q$ oder $\mathbb R$.

Ist also jetzt $f(X)$ ein vom Nullpolynom verschiedenes Polynom vom Grad $n$ und $a$ eine Nullstelle von $f(X)$, so 
können wir schreiben
  	$$ f(X) = q(X) \cdot (X-a) + r(X) $$
wobei $r(X) = r$ ein Polynom vom Grad $0$, also eine Element aus $K$ ist.  Setzen wir $a$ auf beiden Seiten 
ein, so erhalten wir $f(a) = 0$ (da ja $a$ eine Nullstelle von $f(X)$ und $q(a) \cdot (a-a) +  r = r$, und damit
  	$$ 0 = r $$
Das bedeutet $f(X) = g(X) \cdot (X-a) $. Notwendigerweise ist $g(X)$ nicht das Nullpolynom, und aus 
Aufgabe~\ref{aufgabe_polynom_nzd} folgt, dass $\textrm{deg}(q(X)) = n-1$. Daher wissen wir 
gemäß Induktionsvoraussetzung bereits, dass $q(X)$ höchstens $n-1$ Nullstellen hat, und hieraus folgt 
leicht, dass $f(X)$ höchstens $n$ Nullstellen hat.  
}

\bigbreak

Die Körper $\R$ und $\Q$ sind aus der Schule bekannt, im Abschnitt~\ref{section_ganze_zahl} werden wir auch noch 
einige Körper mit nur endliche vielen Elementen kennenlernen. 

Ein weiterer wichtiger Körper ist der Körper $\C$ der komplexen Zahlen\index{komplexe Zahlen}. 
Wie die reellen Zahlen als Erweiterung der 
rationalen Zahlen konstruiert werden, um Wurzeln aus (positiven) Zahlen zu ziehen und besondere Zahlen wie $\ee$ 
oder $\pi$ definieren zu können, werden die komplexen Zahlen als Erweiterung der reellen Zahlen betrachtet um einen 
Mangel der reellen Zahlen zu beheben, nämlich die Tatsache, dass in $\R$ nicht aus allen Zahlen eine Quadratwurzel 
gezogen werden kann. Die Konstruktion dieses Körpers ist sehr explizit: 

Wir definieren $\C$ als die Menge alle Zahlenpaare $(x,y) \\in \mathbb R^2$ und mit der folgenden Addtion 
und Multiplikation.

\begin{itemize}
\item $(x_1, y_1) + (x_2, y_2) = (x_1+x_2, y_1+y_2)$. 
\item $(x_1,y_1) \cdot (x_2,y_2) = (x_1 \cdot x_2 - y_1 \cdot y_2, x_1 \cdot y_2 + x_2 \cdot y_1)$. 
\end{itemize}

Zur Vereinfachung der Schreibweise führen wir die folgenden Notationen ein: 

\begin{itemize}
\item[] $0 = (0,0)$ (das Nullelement).
\item[] $1 = (1,0)$ (das Einselement). 
\item[] $\ii = (0,1)$. 
\item[] $(x,y) = x+\ii \cdot y$. 
\end{itemize}

Für eine komplexe Zahl $z = x+\ii \cdot y$ heißt dann $x$ der \textbf{Realteil}\index{Realteil} und $y$ der 
\textbf{Imaginärteil}\index{Imaginärteil} von $z$. 

Addition und Multiplikation der komplexen Zahlen schreiben sich damit wie folgt

\begin{itemize}
\item $(x_1 + \ii \cdot y_1) + (x_2+ \ii \cdot y_2) =(x_1+x_2 + \ii \cdot (y_1+y_2)$. 
\item $(x_1 + \ii \cdot y_1) \cdot (x_2 + \ii \cdot y_2) 
= x_1 \cdot x_2 - y_1 \cdot y_2 + \ii \cdot (x_1 \cdot y_2 + x_2 \cdot y_1)$. 
\end{itemize}

\medbreak

\begin{satz}[Der Körper der komplexen Zahlen]
Die komplexen Zahlen $\C$ zusammen mit dieser Addition und Multiplikation bilden einen Körper. 

Das inverse Element $\frac {1}{z}$ zu einer komplexen Zahl $z = x + \ii \cdot y \neq 0$ ist gegeben durch 
	$$ \frac {1}{z} = \frac {x - \ii \cdot y}{x^2+y^2} $$
das Nullelement bezüglich der Addition ist $0 = 0 + \ii \cdot 0$, das Einselemnet bezüglich der Multiplikation 
ist $1 = 1 + \ii \cdot 0$.  
\end{satz}

\beweis{ 
Die Körperaxiome können direkt nachgerechnet werden.
}

\begin{notiz}
Sofort aus der Definition der Multiplikation erhalten wir 
	$$ \ii^2 = (0 + \ii \cdot 1( \cdot (0 + \ii \cdot 1) = 0 \cdot 0 - 1 \cdot 1 + \ii \cdot (0 \cdot 1 + 1 \cdot 0) = - 1 $$
also ist $\ii$ eine Quadratwurzel aus $-1$. 

Allgemeiner gilt für jede positive Zahl $ r \in \R$ mit (positiver) Quadratwurzel $\sqrt{r}$: 
	$$ \left(\ii \cdot \sqrt{r} \right)^2 = \ii^2 \cdot \sqrt{r}^2 = (-1) \cdot r = - r $$
dh. $\ii \cdot \sqrt{r}$ ist eine Quadratwurzel von $-r$, und damit haben in $\C$ auch alle negativen Zahlen 
eine Quadratwurzel  
\end{notiz}

Es gilt sogar allgemeiner

\begin{satz}[Fundamentalsatz der Algebra] 
Der Körper $\C$ ist algebraisch abgeschlossen, dh. jedes Polynom 
	$$ f(x) = a_0 + a_1\cdot x + \cdots + a_n \cdot x^n \quad \in \C[X] $$
vom Grad $\geq 1$ hat eine Wurzel, d.h. es gibt ein $z_0 \in \C$ mit $f(z_0) = 0$.
\end{satz}

\bigbreak 

\bigbreak

\begin{aufgabe}\label{gruppe_aufg_z_mod_n_plus}
Zeigen Sie, dass $(\mathbb Z/ (n), +)$ mit den in Beispiel~\ref{gruppe_z_mod_n} definierten Operationen 
tatsächlich eine kommutative Gruppe ist.
\end{aufgabe}

\begin{aufgabe}\label{gruppe_aufg_z_mod_n_ring}
Zeigen Sie, dass $(\mathbb Z/ (n), +, \cdot )$ mit den in Beispiel~\ref{gruppe_z_mod_n} definierten Operationen 
tatsächlich ein Ring ist.
\end{aufgabe}

\begin{aufgabe} In einem Spiel für zwei Spieler werden zu Begin $n$ Spielsteine auf einem Spielplan 
aufgestellt. Jeder Spieler kann in seinem Zug 1 - 4 Steine vom Spielplan entfernen. Gewonnen hat der 
Spieler, der den letzten Stein vom Brett nimmt. Zeigen Sie: Ist $n$ durch 5 teilbar, so kann der nachziehende 
Spieler den Sieg erzwingen. In allen anderen Situationen gibt es für den anziehenden Spieler einen 
Gewinnweg.
\end{aufgabe}

\begin{aufgabe} Wir betrachten die Menge $G = \{ a, b ,c , d \} $ mit der Verknüpfung
  	$$ \begin{array} { c | c c c c }
  	\circ \, & \, a \, & \, b \,  & \, c \, & \, d \, \\ \hline
  	a\, \, & a & b & c & d \\
  	b\, \, & b & a & d & c \\
  	c\, \, & c & d & a & b \\
  	d\, \, & d & c & b & a 
  	\end{array} $$
Zeigen Sie, dass $G$ eine kommutative Gruppe der Ordnung 4 ist. Was ist das neutrale Element von $G$? 
Was ist der wesentliche Unterschied zwischen dieser 
Gruppe und der Gruppe $\mathbb Z / 4 \mathbb Z$?
\end{aufgabe}

\begin{aufgabe} Zeigen Sie: Ist $G$ eine endliche zyklische Gruppe, so ist $G$ kommutativ.
\end{aufgabe}

\begin{aufgabe} Schreiben Sie die Permutation
  	$$ \sigma = \left( \begin{matrix} 1 & 2 & 3 & 4 & 5 \\ 2 & 3 & 5 & 1 & 4 \end{matrix} \right) $$
als Komposition von Vertauschungen $\langle i \, j \rangle$.
\end{aufgabe}


\begin{aufgabe}\label{gruppe_aufg_permut_1} 
Es sei $G$ die Permuationsgruppe $S_n$. Definiert man $\textrm{sign}(\sigma)$ als 
  	$$ \textrm{sign}(\sigma) = \left\{ \begin{array} {l c l}
	+1 & \quad &\textrm{ falls die Anzahl der Fehlstände gerade ist} \\
	-1 & \quad &\textrm{ falls die Anzahl der Fehlstände ungerade ist} 
 	\end{array} \right. $$
so gilt: 
  	$$ \textrm{sign}(\sigma) = \prod_{i<j} \frac {\sigma(j) - \sigma(i)}{j - i} $$
\end{aufgabe}

\begin{aufgabe}\label{gruppe_aufg_permut_2} 
Es sei $G$ die Permuationsgruppe $S_n$. Dann gilt:
  	$$ \textrm{sign}(\sigma \circ \tau) = \textrm{sign}(\sigma) \circ \textrm{sign}(\tau) $$
\end{aufgabe}

\begin{aufgabe} Bestimmen Sei alle Nullteiler des Rings $\mathbb Z/ (24)$. \end{aufgabe}

\begin{aufgabe} Zeigen Sie, dass $\mathbb Z/ 5 \mathbb Z$ ein Körper ist.
\end{aufgabe}


\newpage

\section{Ganze Zahlen und ihre Restklassen}\label{section_ganze_zahl}

\setcounter{definition}{0}
\setcounter{beispiel}{0}
\setcounter{notiz}{0}


Die ganzen Zahlen und ihre Restklassenringe sind von fundamentaler Bedeutung für die Mathematik, aber 
auch weit darüberhinaus, etwa in der Informatik. Daher wollen wir diesen Ringen einen eigenen Abschnitt 
widmen. Ein fundamentaler Begriff in diesem Zusammenhang ist der Begriff der Teilbarkeit, und wesentlich 
dabei wiederum ist der Begriff der Primzahl:

\begin{definition} Eine Zahl $p \in \mathbb N \setminus \{0, 1\}$ heißt \index{Primzahl}\textbf{Primzahl}, 
wenn $p$ keine echten Teiler hat.
\end{definition}

\begin{beispiel} $2, 3, 5, 7$ und $11$ sind Primzahlen, $9$ ist keine.
\end{beispiel}

Wir notieren hier zwei wichtige Eigenschaften von Primzahlen und ganzen Zahlen, deren Nachweis sich 
in jedem Buch zur Algebra oder zur elmentaren Zahlentheorie findet (etwa in \textit{Gerhard Frey}: 
Elementare Zahlentheorie; Vieweg Verlag 1984, und dort speziell die Aussagen 2.2,  3.1, 3.2 und 3.4)).

\begin{notiz} Eine Zahl $p \in \mathbb N \setminus \{0, 1\}$ ist genau dann eine Primzahl, wenn 
für alle ganzen Zahlen $a, b \in \mathbb Z$ gilt
  	$$ p \vert \, a \cdot b \iff p \vert a \textrm{ oder } p \vert b $$
also $p$ teilt genau dann ein Produkt von zwei Zahlen, wenn es schon eine der beiden Zahlen teilt.
\end{notiz}

\begin{notiz}\label{gruppe_primfaktorzerlegung_eindeutig} Jede ganze Zahle $z \in \mathbb Z 
\setminus \{ 0 \}$ schreibt sich in eindeutiger Weise als
  	$$ z = \varepsilon \cdot p_1^{n_1} \cdot p_2^{n_2} \cdots p_t^{n_t} $$
mit $\varepsilon \in \{-1,1\}$, Primzahlen $p_1 < p_2 < \ldots < p_t$ und positiven natürlichen Zahlen 
$n_1, \ldots , n_t$. (Dabei lassen wir den Spezialfall $t = 0$ für $z = \pm 1$ zu).
\end{notiz}

\begin{definition} Sind $m,n \in \mathbb N \setminus \{0 \}$, so heißt eine Zahl $g \in \mathbb N$ der 
\textbf{größte gemeinsame Teiler} von $m$ und $n$, wenn gilt:
\begin{itemize}
\item $g$ ist ein Teiler von $m$ und ein Teiler von $n$.
\item Ist $h$ ein weiterer Teiler von $m$ und $n$, so ist $h$ auch ein Teiler von $g$.
\end{itemize}
Wir schreiben 
	$$ \mathrm{ggT}(m,n) := g $$ 
für den größten gemeinsame Teiler von $m$ und $n$.

Zwei Zahlen $m,n \in \mathbb N \setminus \{0 \}$ heißen \index{teilerfremd}\textbf{teilerfremd}, wenn 
$\textrm{ggT}(m,n) = 1$, wenn sie also keinen echten gemeinsamen Teiler besitzen.
\end{definition}

\begin{notiz}
Ist die Primfaktorzerlegung von $m$ und $n$ bekannt, so lässt sich daraus der größte gemeinsame Teiler von 
$m$ und $n$ sofort ablesen. Sind nämlich $p_1, \ldots, p_t$ die Primzahlen, die entweder $m$ oder $n$ teilen, und 
schreiben wir 
	$$ m = p_1^{a_1} \cdots p_t^{a_t}, \qquad n = p_1^{b_1} \cdots p_t^{a_t} $$
mit $b_i, a_i \geq 0$, so gilt 
	$$ \mathrm{ggT}(m,n) = p_1^{\mathrm{min}\{ a_1, b_1\}} \cdots p_t^{\mathrm{min}\{ a_t, b_t\}} $$
dh. im größten gemeinsamen Teiler von $m$ und $n$ kommen alle Primzahlen vor, die sowohl $m$ als auch $n$ 
teilen, und zwar so oft, wie sie mindestens beide teilen. 
\end{notiz}

Eine unmittelbare Konsequenz der Definition ist

\begin{satz}\label{gruppe_prim_teilerfremd} 
Ist $p$ eine Primzahl und $m \in \mathbb N \setminus \{0 \}$ kein Vielfaches von $p$, so 
sind $m$ und $p$ schon teilerfremd.
\end{satz}


\medbreak

\begin{notiz} Wir hätten die Begriffe \textit{Teiler} und \textit{Primzahlen} auch für ganze Zahlen 
definieren können. In diesem Fall wäre etwa $m$ ein echter Teiler von $n$ wenn es ein $a \in \mathbb Z 
\setminus \{\pm 1, \pm n\}$ gibt mit $a \cdot m = n$. 
Das ändert wenig an den grundsätzlichen Eigenschaften, es sind lediglich Vorzeichen 
zu berücksichtigen. So ist etwa mit $p$ auch $-p$ eine Primzahl, und mit $m$ ist auch $-m$ ein Teiler von 
$n$. Wir erhalten aber weder grundsätzlich neue Teiler noch einen grundsätzlich anderen Begriff von 
Primzahlen. So ist etwa $p$ genau dann eine Primzahl in $\mathbb Z$ wenn entweder $p$ oder $-p$ eine Primzahl 
im Sinne unserer Definition ist. Für uns sollen aber etwa $2$ und $-2$ - lax gesprochen - die gleiche Primzahl 
darstellen, während $2$ und $3$ für uns grundsätzlich verschiedene Primzahlen sind. 

Um daher lästige Vorzeichenbetrachtungen zu vermeiden, verzichten wir auf diesen 
augenscheinlich allgemeineren Begriff. 
\end{notiz}

\medbreak

Der größte gemeinsame Teiler zweier natürlicher Zahlen $m,n\in \mathbb N \setminus \{0 \}$ 
lässt sich mit Hilfe des \index{euklidischer Algorithmus}\textbf{euklidischen Algorithmus} 
schnell berechnen. 

\bemerkung{\textbf{\, \, Der euklidische Algorithmus}}

\vspace{-0.4cm}
\begin{itemize} 
\item \textbf{Vorbereitungsschritt:\,} Wir ordnen $m$ und $n$ so an, dass $m \geq n$. Gegebenenfalls vertauschen 
wir dazu die Rollen von $m$ und $n$, denn $\textrm{ggT}(m,n) = \textrm{ggT}(n,m)$. Wir setzen $i = 0$ und 
$r_0 = m$, $r_1 = n$.
\item \textbf{Verarbeitungsschritt:\,} Wir dividieren $r_i$ durch $r_{i+1}$ mit Rest:
  	$$ r_i = q \cdot r_{i+1} + b $$
mit einer natürlichen Zahl $q$ und einem Rest $b \in \{0, 1, \ldots, r_{i+1} - 1 \}$.
\begin{itemize}
\item[-] Falls $b = 0$ (dh. die Division geht ohne Rest auf) $\, \longrightarrow \textbf{ STOPP}$.
\item[-] Falls $b \neq 0$ setze $r_{i+2} = b$ und $i = i+1$. Wiederhole den Verarbeitungsschritt.
\end{itemize}
\item \textbf{Ergebnisschritt:} Nach endlich vielen Verarbeitungsschritten (höchstens $m$ vielen) geht die 
Division erstmals ohne Rest auf, d.h.
  	$$ r_i = q \cdot r_{i+1} + 0 $$
mit $r_{i+1} \neq 0$, Das STOPP--Kriterium wird also immer erreicht. 
Dann ist $r_{i+1}$ der größte gemeinsame Teiler von $m$ und $n$, $r_{i+1} = \textrm{ggT}(m,n)$.
\end{itemize}

Der Nachweis, dass dieser Algorithmus tatsächlich zum größten gemeinsame Teiler führt, kann etwa mit 
Aufgabe~\ref{gruppe_aufg_euklid} geführt werden.

\bigbreak

\begin{beispiel}\label{LA_alg_bsp_euklid1} Wir betrachten die Zahlen $m = 222$ und $n = 156$. Hier gilt bereits 
$m \geq n$, und wir setzen $r_0 = 234$ und $r_1 = 156$.

\begin{itemize}
\item[ ]$i = 0$: $\quad 222 = 1 \cdot 156 + 66$. Wir setzen $r_2 = 66$.
\item[ ]$i = 1$: $\quad 156 = 2 \cdot 66 + 24$. Wir setzen $r_3 = 24$.
\item[ ]$i = 2$: $\quad 66 = 2 \cdot 24 + 18$. Wir setzen $r_4 = 18$.
\item[ ]$i = 3$: $\quad 24 = 1 \cdot 18 + 6$. Wir setzen $r_5 = 6$.
\item[ ]$i = 4$: $\quad 18 = 3 \cdot 6 + 0$. $\, \longrightarrow \textbf{ STOPP}$.
\end{itemize}

\textbf{Ergebnis:} $\textrm{ggT}(222, 156) = 6$.
\end{beispiel}

\begin{beispiel}\label{gruppe_bsp_teilerfremd_1} 
Wir betrachten die Zahlen $m = 19$ und $n = 234$. Hier müssen wir zuerst die Rollen 
von $m$ und $n$ vertauschen, setzen also $r_0 = 234$ und $r_1 = 19$. 

\begin{itemize}
\item[ ]$i = 0$: $\quad 234 = 12 \cdot 19 + 6$. Wir setzen $r_2 = 6$.
\item[ ]$i = 1$: $\quad 19 = 3 \cdot 6 + 1$. Wir setzen $r_3 = 1$.
\item[ ]$i = 2$: $\quad 6 = 6 \cdot 1 + 0$.  $\, \longrightarrow \textbf{ STOPP}$.
\end{itemize}

\textbf{Ergebnis:} $\textrm{ggT}(234, 19) = 1$. Die Zahlen 234 und 19 sind also teilerfremd.
\end{beispiel}

\begin{notiz}
Der euklidische Algorithmus liefert ein effizientes Verfahren zur Berechnung des grlßten gemeinsamen Teilers, 
das für große Zahlen sehr viel schneller ist als der Weg über die Primfaktorzerlegung und die gemeinsamen 
Primfaktoren. 

Zunächst scheint es ja ein Problem zu sein, dass wir nicht wissen, wie viele Schritte der Algorithmus braucht, 
dh. wie viele $r_i$ berechnet werden müssen (und damit auch, wie viel Speicher bereitgestellt werden soll). 
Allerdings sieht man sofort, dass in jedem Verarbeitungsschritt aus dem vorherigen Teil der Verarbeitung 
nur die beiden Zahlen, die mit Rest durcheinander dividiert werden müssen, benötigt werden, daher brauchen 
wir immer nur diese beiden Zahlen (und den neuen Rest). Damit kann die Bestimmung des größten gemeinsamen 
Teilers effizient umgesetzt werden: 

\makebox[\textwidth]{\hrulefill}  
\lstinputlisting{la_euklid01.m}
\makebox[\textwidth]{\hrulefill} 

\end{notiz}

\medbreak

Eine interessante  und wichtige Folgerung aus dem euklidischen Algorithmus ist

\begin{satz}\label{gruppe_erw_euklid} Sind $m,n\in \mathbb N \setminus \{0 \}$ mit 
$\textrm{ggT}(m,n) = g$, so gibt es ganze Zahlen $a,b$ mit 
  	$$ a \cdot m + b \cdot n = g $$
\end{satz}

Der Beweis des Satzes besteht darin, dass wir aus dem vorletzten Schritt des euklidischen Algorithmus 
rückwärtsrechnen, also die Beziehung $r_{i-1} = q \cdot r_i + r_{i+1}$, in der $r_{i+1}$ der größte 
gemeinsame Teiler von $m$ und $n$ ist, nach $r_{i+1}$ auflösen, $r_{i+1} = r_{i-1} - q \cdot r_i$ und dann 
sukkzessive die verschiedenen Schritte des Algorithmus zurückgehen. Für die Details siehe 
Aufgabe~\ref{gruppe_aufg_erw_euklid}

\begin{beispiel}
In Beispiel~\ref{LA_alg_bsp_euklid1} haben wir gesehen, dass $\mathrm{ggT}(234,156) = 6$. Um $6$ mit 
$234$ und $156$ zu beschreiben, gehen wir vor wie folgt: 

Aus dem Schritt $i = 3$ erhalten wir 
	$$ 6 = 24 - 1 \cdot 18 $$
Aus $i =2$ folgt $18 = 66 - 2 \cdot 24$. Setzen wir das ein, so ergibt sich 
	$$ 6 = 24 - 1 \cdot 18 = 24 - 1 \cdot (66 - 2 \cdot 24) = 3 \cdot 24 - 1 \cdot 66 $$
  Aus $i =1$ erhalten wir $24 = 156 - 2 \cdot 66$. Setzen wir das ein, so ergibt sich 
	$$ 6 = 3 \cdot 24 - 1 \cdot 66 = 3 \cdot (156 - 2 \cdot 66) - 1 \cdot 66 = 3 \cdot 156 - 7 \cdot 66 $$
Aus $i =0$ folgt schließlich  $66 = 222 - 1 \cdot 156$. Setzen wir das ein, so ergibt sich 
	$$ 6 = 3 \cdot 156 - 7 \cdot 66 = 3 \cdot 156 - 7 \cdot (222 - 156) = 10 \cdot 156 - 7 \cdot 222 $$
Damit haben wir eine Darstellung 
	$$ 6 = 10 \cdot 156 - 7 \cdot 222 $$
 wie gewünscht gefunden. 
\end{beispiel}

\begin{beispiel} In Beispiel~\ref{gruppe_bsp_teilerfremd_1} haben wir gesehen, dass $\textrm{ggT}(234, 19) = 1$. 
Wir wollen nun $1$ mit $19$ und $234$ darstellen.

\begin{enumerate}
\item Aus Schritt $i=1$ erhalte: $1 = 19 - 3 \cdot 6$.
\item Aus Schritt $i=0$ erhalte: 
  	$$1 = 19 - 3 \cdot 6 = 19 - 3 \cdot \left(234 - 12 \cdot 19\right) =
      37 \cdot 19 - 3 \cdot 234$$
\end{enumerate}
\end{beispiel}

\begin{notiz}
Die Bestimmung der beiden Zahlen $a$ und $b$, für die gilt 
	$$ a \cdot m + b \cdot n = \mathrm{ggT}(m,n) $$
ist sehr wichtig für Anwendungen in der Kryptographie oder der Codierungstheorie. Einer effizienten 
Umsetzung des oben beschreibenen Vorgehens scheint die Rückwärtsrechnung, die dabei durchgeführt 
wurde, entgegenzustehen, denn diese kann offensichtlich nicht effizient umgesetzt werden. Bei genauerer 
Betrachtung sehen wir, dass dieser Rückwärtsrechenschritt allerdings vermieden werden kann und alle 
notwendige Information beim Vorwärtsrechnen ihm Rahmen des eigentlichen euklidischen Algorithmus 
mit übergeben werden kann. Das soll nun am Beispiel~\ref{LA_alg_bsp_euklid1} gezeigt werden. 
\end{notiz}

\begin{beispiel}
Wir greifen noch einmal Beispiel~\ref{LA_alg_bsp_euklid1} mit $m = 222$ und $n = 156$ auf. 

\textbf{Vorbereitungsschritt:} Es gilt $m \geq n$, also setzen wir 
	$$ r_0 = 234 \quad \text{ und } \, \, r_1 = 156 $$

Außerdem setzen wir $a_0 = 1$, $b_0 = 0$ und $a_1 = 0$, $b_1 = 1$. Dann gilt hierfür 
	$$ \begin{array} {l c r c r}
	r_0 & = & a_0 \cdot m & + & b_0 \cdot n \\
	r_1 & = & a_1 \cdot m & + & b_1 \cdot n 
	\end{array} $$

\textbf{Verarbeitung:}

\begin{itemize}
\item[ ]$i = 0$: $\quad 222 = 1 \cdot 156 + 66$, also $q = 1$, $ r = 66$. Wir setzen $r_2 = 66$ und wir beachten, 
dass
	$$ r_2 = r_0 - q \cdot r_1 = (a_0 \cdot m + b_0 \cdot n) - q \cdot (a_1 \cdot m + b_1 \cdot n) 
	= (a_0 - q \cdot a_1) \cdot m + (b_0 - q \cdot b_1) \cdot n $$
Setzen wir also 
	$$ a_2 = a_0 - q \cdot a_1 = 1, \quad b_2 = b_0 - q \cdot b_1 = - 1 $$ 
so erhalten wir 
	$$ r_2 = a_2 \cdot m + b_2 \cdot n = 1 \cdot 222 + (-1) \cdot 156 $$
\item[ ]$i = 1$: $\quad 156 = 2 \cdot 66 + 24$, also $q = 2$, $r = 24$. Wir setzen $r_3 = 24$ und wir beachten, 
dass
	$$ r_3 = r_1 - q \cdot r_2 = (a_1 \cdot m + b_1 \cdot n) - q \cdot (a_2 \cdot m + b_2 \cdot n) 
	= (a_1 - q \cdot a_2) \cdot m + (b_1 - q \cdot b_2) \cdot n $$
Setzen wir also 
	$$ a_3 = a_1 - q \cdot a_2 = -2, \quad b_3 = b_1 - q \cdot b_2 = 3 $$ 
so erhalten wir 
	$$ r_3 = a_3 \cdot m + b_3 \cdot n = (-2) \cdot 222 + 3 \cdot 156 $$
\item[ ]$i = 2$: $\quad 66 = 2 \cdot 24 + 18$, also $q = 2$, $r = 18$. Wir setzen $r_4 = 18$ und (wie oben) 
	$$ a_4 = a_2 - q \cdot a_3 = 5, \quad b_4 = b_2 - q \cdot b_3 = -7 $$ 
Damit erhalten wir 
	$$ r_4 = a_4 \cdot m + b_4 \cdot n = 5 \cdot 222 + (-7) \cdot 156 $$
\item[ ]$i = 3$: $\quad 24 = 1 \cdot 18 + 6$, also $q = 1$, $r = 6$. Wir setzen $r_5 = 6$ und (wie oben) 
	$$ a_5 = a_3 - q \cdot a_4 = -7, \quad b_5 = b_3 - q \cdot b_4 = 10 $$ 
Damit erhalten wir 
	$$ r_5 = a_5 \cdot m + b_5 \cdot n = (-7) \cdot 222 + 10 \cdot 156 $$
\item[ ]$i = 4$: $\quad 18 = 3 \cdot 6 + 0$. $\, \longrightarrow \textbf{ STOPP}$.
\end{itemize}

\textbf{Ergebnis:} $\textrm{ggT}(222, 156) = r_5 = 6$. und 
	$$ 6 = (-7) \cdot 222 + 10 \cdot 156 $$
\end{beispiel}

\begin{notiz} Wie bei der Umsetzung des euklidischen Algorithmus kann auch hier die Abhängigkeit von 
dem Schrittzähler $i$  noch eliminiert werden, so dass jeder Schritt für seine Verarbeitung nur die beiden 
Zahlen, die mit Rest durcheinander dividiert werden, und die Darstellung dieser beiden Zahlen mithilfe 
von $m$ und $n$ enthält. Daraus ergibt sich dann insgesamt der folgende Algorithmus

\makebox[\textwidth]{\hrulefill}  
\lstinputlisting{la_euklid02.m}
\makebox[\textwidth]{\hrulefill} 

\end{notiz}

\bigbreak

Nach diesen Überlegungen können wir wieder zur Betrachtung der Ringe $\mathbb Z/(n)$ zurückkehren:

\begin{satz}\label{gruppe_nullteiler_mod_n} 
Für Zahlen $m,n\in \mathbb N \setminus \{0 \}$, für die $m$ kein 
Vielfaches von $n$ ist, sind die folgenden Aussagen äquivalent:
\begin{itemize}
\item[i)] $[m]$ ist eine Nullteiler in $\mathbb Z/(n)$.
\item[ii)] $\textrm{ggT}(m,n) \neq 1$.
\end{itemize}
\end{satz}

\beweis{ Ist $[m]$ ein Nullteiler von $\mathbb Z/(n)$, so gibt es ein $a \in \{1, \ldots, n-1\}$ mit
  	$$ [m] \cdot [a] = [0] $$
Das bedeutet aber, $m \cdot a$ ist ein Vielfaches von $n$, also $m \cdot a = l \cdot n$. Nehmen wir an, 
dass ii) nicht gilt, dass also $m$ und $n$ teilerfremd sind, so finden wir nach Satz~\ref{gruppe_erw_euklid}
Zahlen $r$ und $s$ mit $r \cdot m + s \cdot n = 1$. Damit gilt aber
  	$$ r \cdot l \cdot n = a \cdot r \cdot m = a \cdot \left( 1 - s \cdot n \right) = a - a \cdot s \cdot n $$
also $a = n \cdot \left(r \cdot l + a \cdot s\right)$ ist ein Vielfaches von $n$, im Widerspruch zur Wahl von 
$a$.

Ist umgekehrt $g = \textrm{ggT}(m,n)$, so können wir $n = g \cdot n_1$ und $m = g \cdot m_1$ schreiben, wobei 
notwendig $n_1 \in \{2, \ldots, n-1\}$. Dann gilt aber
  	$$ m \cdot n_1 = g \cdot m_1 \cdot n_1 = g \cdot n_1 \cdot m_1 = n \cdot m_1 $$
ist ein Vielfaches von $n$, also $[m] \cdot [n_1] = [0]$. Da $[n_1] \neq [0]$ ist $[m]$ ein Nullteiler. 
} 

\medbreak

\begin{korollar}\label{gruppe_fp_nullt} Genau dann ist $\mathbb Z/(n)$ nullteilerfrei, wenn $n$ eine 
Primzahl ist. 
\end{korollar}

\beweis{ Dazu reicht es nach Satz~\ref{gruppe_nullteiler_mod_n} zu zeigen, dass genau dann jede Zahl $m \in 
\{1, \ldots , n-1\}$ teilerfremd zu $n$ ist, wenn $n$ eine Primzahl ist. 

Ist dazu zunächst $n$ eine Primzahl, so folgt aus Satz~\ref{gruppe_prim_teilerfremd}, dass $m$ und $n$ 
teilerfremd sind. Ist umgekehrt jedes $m \in \{1, \ldots , n-1\}$ teilerfremd zu $n$, so kann $n$ insbesondere 
keine echten Teiler haben und ist also ein Primzahl.
}
\medbreak

\begin{notiz}
Um die Notation zu vereinfachen, schreiben wir häufig für die Restklasse eine Zahl $m \in \Z$ in $\Z/(n)$ ebenfalls 
$m$ anstelle von $[m]$. Aus dem Kontext wird immer klar sein, um was es sich handelt. 

Ferner haben sich für $\Z/(n)$ auch die Notationen $\Z/n\Z$ und $\Z_n$ eingebürgert. Auch diese werden häufig verwendet. 
\end{notiz}

\medbreak

\bemerkung{Bezeichnung:} Ist $p$ eine Primzahl, so schreiben wir auch $\mathbb F_p$ für $\mathbb Z/(p)$.

In Abschnitt~\ref{section_gruppe} haben wir neben dem Begriff des Rings auch den des Körpers 
studiert. Neben den bekannten Beispielen $\mathbb Q$ und $\mathbb R$ haben wir nun einen weiteren 
Körper kennengelernt ohne das bis jetzt zu wissen:

\begin{satz} Für jede Primzahl $p$ ist $\mathbb F_p$ ein Körper.
\end{satz}

\beweis{ Nach Korollar~\ref{gruppe_fp_nullt} ist $\mathbb F_p$ nullteilerfrei, und
wegen Satz~\ref{gruppe_ring_korper} brauchen wir daher nur noch zu zeigen, dass jedes $x \neq 0$ in  
$\mathbb F_p$ ein multiplikatives Inverses hat. Sei dazu $x = [m]$ mit $m \in \mathbb Z$. Da $[m] \neq [0]$ 
sind $m$ und $p$ teilerfremd. Deshalb finden wir Zahlen $r, s$ mit $r \cdot m + s \cdot p = 1$, und daraus 
folgt
  	$$ [r][m] = [1 - s \cdot p] = [1] - [s]\cdot [p] = [1] $$
also das Gewünschte.
}
\medbreak

\begin{beispiel}
Die Ermittlung des inversen Elements zu einem Element $m \in \F_p$ (mit Repräsentanten $m \in \Z$) erfolgt 
am besten mithilfe des erweiterten euklicdischen Algorithmus: 

Wir betrachten den endlichen Körper $k = \mathbb F_{71}$ mit $71$  
Elementen und wollen die folgenden Elemente
	$$ a = \frac {1}{24} \, , \quad b = \frac {17}{30} $$
von $k$ mit den Repräsentanten $0, \, 1, \, \ldots , 70$ darstellen.

Zur Bestimmung von $\frac {1}{24}$ in $\mathbb F_{71}$ gehen wir vor wie folgt:
 	$$ \begin{array} {l c l}
	71 &=& 2 \cdot 24 + 23 \\
	24 &=& 1 \cdot 23 + 1 \\
	23 &=& 23 \cdot 1 + 0 \\
	\end{array} $$
Damit ergibt Rückwärtsrechnen
	$$ \begin{array} {l c l}
	1 &=& 24 - 1 \cdot 23 \\
	&=& 24 - 1 \cdot (71 - 2 \cdot 24) \\
	&=& 3 \cdot 24 - 1 \cdot 79 \\
	\end{array} $$
und damit gilt in $\mathbb F_{59}$: 
	$$ 1 = 3 \cdot 24  $$
also 
	$$ \frac {1}{24} = 3 $$
Zur Bestimmung von $b = \frac {17}{30}$. Berechnen wir zunächst $\frac {1}{30}$.  Hier ist 
(wieder nach dem erweiterten euklidischen Algorithmus)
	$$ (-26) \cdot 30 + 11 \cdot 71 = 1 $$
also in $\mathbb F_{71}$:
	$$ \frac {1}{30} = -26 = 45 $$
 und damit 
	$$ \frac {17}{30} = 17 \cdot \frac {1}{30} = 17 \cdot 45 = 55 $$
\end{beispiel}

Diese Körper $\mathbb F_p$ haben eine für die Anwendungen sehr interessante Eigenschaft.

\begin{satz} \index{Satz!kleiner Fermatscher Satz}\textbf{(Kleiner Satz von Fermat)} Für jedes $a \in 
\mathbb F_p \setminus \{ [0] \}$ gilt
  	$$ a^{p-1} = 1 \qquad \textrm{ in } \mathbb F_p $$
und für jedes $a \in \mathbb F_p$ gilt
  	$$ a^{p} = a \qquad \textrm{ in } \mathbb F_p $$
\end{satz}

\beweis{ Wir wissen, dass $(\mathbb F_p \setminus \{[0] \}, \cdot)$ eine Gruppe ist, und zwar eine Gruppe der 
Ordnung $p-1$. Damit folgt aus Lemma~\ref{gruppe_endl_ordnung2} schon $a^{p-1} = 1$.

Damit gilt auch $a^p = a^{p-1} \cdot a = a$. Da sicherlich $[0]^p = [0]$, folgt auch der Zusatz.
}
\medbreak

Die additive Gruppe des Körpers $\mathbb F_p$ ist $(\mathbb Z/p \mathbb Z, +)$, also eine sehr einfache 
zyklsiche Gruppe. Die Einheitengruppe $E(\mathbb F_p)$ hat ebenfalls eine sehr einfache Struktur

\begin{satz}\label{fp_mult_zyklisch}
Für jede Primzahl $p$ ist die Einheitengruppe $E(\mathbb F_p)$ von  $\mathbb F_p$ eine zyklische Gruppe.
\end{satz} 

\beweis{ Im Abschnitt~\ref{section_gruppe} haben wir schon den Begriff der Ordnung eines Elementes $g$ 
eine endlichen Gruppe $G$ kennengelernt
  	$$ \textrm{ord}(g) = \textrm{min}\{ n \geq 1 \vert \, g^n = e \} $$ 
und aus Lemma~\ref{gruppe_endl_ordnung2} wissen wir schon, dass 
  	$$ \textrm{ord}(g) \, \vert \, \textrm{ord}(G) $$
Wenden wir das jetzt speziell auf die Einheitengruppe $E(\mathbb F_p)$ von  $\mathbb F_p$ an, so erhalten wir, 
dass für jedes $x \in E(\mathbb F_p)$ gilt
  	$$  \textrm{ord}(x) \, \vert \, p-1 $$
Wenn wir nun ein $x \in E(\mathbb F_p)$ finden mit $\textrm{ord}(x) =  p-1$, so muss dieses $x$ notwendigerweise 
ein Erzeuger der gesamten Einheitengruppe sein (und diese damit zyklisch), denn in 
Bemerkung~\ref{gruppe_ordnung_distinkt} haben wir gesehen, dass $x^l \neq x^k$ für alle $1 \leq l < k 
\leq  \textrm{ord}(x)$ und daher sind die Elemente der Menge
  	$$ <x> = \{ x, x^2, x^3, \ldots x^{p-1} \} $$
alle paarweise verschieden, wenn $\textrm{ord}(x) =  p-1$, müssen also schon die ganze Gruppe 
$E(\mathbb F_p)$ sein.

Um eine solches Element zu finden, setzen wir 
  	$$ M := \textrm{max} \{ n \in \mathbb N \, \vert \, \textrm{ es gibt ein } a \in E(\mathbb F_p) \textrm{ mit }
	\textrm{ord}(a) = n \} $$
und wir wählen ein $x \in E(\mathbb F_p)$ mit $ \textrm{ord}(x) = M$. 

Zunächst behaupten wir: Ist $y \in E(\mathbb F_p)$ beliebig, und ist $\textrm{ord}(y) = m$, so gilt $m \vert M$.
Dabei gilt sicherlich $m \leq M$. 

Nehmen wir an, dass $m \not\vert M$, so gilt 
  	$$ g := \textrm{ggT}(m,M) < m $$
also $m = m' \cdot g$ mit $m' > 1$ und $m'$ teilerfremd zu $M$. Setzen wir $y' = y^g$, so hat $y'$ die Ordnung 
$m'$, denn 
  	$$\left(y'\right)^{m'} = y^{gm'} = y^m = 1 $$
und für $n\geq 1$, $n<m'$ gilt
  	$$ \left(y'\right)^{n'} = y^{gn} \neq 1 $$
da $gn < m$. Setzen wir nun $a = \frac {x}{y'}$ und $n =  \textrm{ord}(a)$, so gilt $a^n = 1$, also 
  	$$ x^n = \left(y'\right)^n $$
und damit insbesondere $ \textrm{ord}(x^n) =  \textrm{ord}(\left(y'\right)^n)$. Nun gilt aber
  	$$  \textrm{ord}(\left(y'\right)^n) \, \vert \,  m', \quad \textrm{ord}(x^n) \, \vert \, M $$
Da $m'$ und $M$ teilerfremd sind, muss notwendig
   	$$  \textrm{ord}(x^n) =  \textrm{ord}(\left(y'\right)^n) = 1 $$
gelten, also $x^n = \left(y'\right)^n = 1$. 
Da  $\textrm{ord}(x) = M$, muss also gelten $n \geq M$, und da  $\textrm{ord}(y') = m'$  teilerfremd zu $M$ ist, 
muss sogar gelten $n > M$, im Widerspruch zur Wahl von $M$. 

Wir haben also jetzt gezeigt, dass $\textrm{ord}(y) \, \vert M$ für alle $y \in  E(\mathbb F_p)$, also insbesondere, 
dass $y^M = 1$ für alle $y \in  E(\mathbb F_p)$. Damit ist jedes $y \in E(\mathbb F_p)$ eine Nullstelle von 
$X^M-1$. In Satz~\ref{polynom_koerper_nullstellen} haben wir gesehen, dass $X^M-1$ höchstens $M$ Nullstellen 
hat. Da jedes Element $y \in  E(\mathbb F_p)$ aber Nullstelle von $X^M-1$ ist, muss also $M \geq p-1$ gelten. 
Da aber andererseits $M \, \vert \, (p-1)$, folgt hieraus $M = p-1$, und damit ist unser Satz gezeigt.
}
\bigbreak

\begin{aufgabe} Unter $n+1$ ganzen Zahlen $\{ a_1, \ldots, a_{n+1}\}$ gibt es mindestens zwei, deren 
Differenz durch $n$ teilbar ist.
\end{aufgabe}

\begin{aufgabe}\label{gruppe_aufg_euklid} Sind $m,n,l \in  \mathbb N \setminus \{0 \}$, so gilt
  	$$ \textrm{ggT}(m,n) = \textrm{ggT}(m,n+km) $$    
\end{aufgabe}

\begin{aufgabe} Bestimmen Sie mit Hilfe des euklidischen Algorithmus den größten gemeinsamen Teiler von 
  	$$  a) \,\, 1008 \textrm{ und } 840, \qquad b) \,\, 481 \textrm{ und } 1755, \qquad c) \,\, 
     	2940 \textrm{ und } 1617   $$
\end{aufgabe}

\begin{aufgabe}\label{gruppe_aufg_erw_euklid} 
Benutzen Sie den euklidischen Algorithmus, um folgende Aussage zu zeigen: 

Sind $m,n\in \mathbb N \setminus \{0 \}$ mit $\textrm{ggT}(m,n) = g$, so gibt es ganze Zahlen $a,b$ mit 
  	$$ a \cdot m + b \cdot n = g $$
\end{aufgabe}

\begin{aufgabe} Berechnen Sie den größten gemeinsamen Teiler $g$ der Zahlen $318$ und $99$ 
und stellen Sie diesen in der Form
  	$$ g = a \cdot 318 + b \cdot  99 $$
mit ganzen Zahlen $a$, $b$ dar. 
\end{aufgabe}


\begin{aufgabe} Für welche natürlichen Zahlen $n$ ist $n^2 + n + 1$ durch 3 teilbar?
\end{aufgabe}

\begin{aufgabe} 
\begin{itemize}
\item[a)] Zeigen Sie: Ist die natürliche Zahl $m \geq 1$ ein Teiler von $a$ und von $a+b$, so teilt $m$ auch $b$.
\item[b)]Für welche natürlichen Zahlen $ n \geq 1$ teilt $n+1$ die Zahl $n^2 +1$?
\end{itemize}
\end{aufgabe}

\begin{aufgabe} Bestimmen Sie (in Abhängigkeit von $n$) 
  	$$ a) \,\, \textrm{ggT}(2^n+1, 3), \qquad b) \,\, \textrm{ggT}(2^n+1, 9) $$
\end{aufgabe}

\begin{aufgabe} Ist $p \geq 3$ eine Primzahl und $n \geq 2$ eine beliebige natürliche Zahl, so ist $2 \cdot p$ 
ein Teiler von $n^p-n$.
\end{aufgabe}

\begin{aufgabe} Ist $R = \mathbb Z/(n)$ mit $n = p \cdot q$ wobei $p \neq q$ zwei Primzahlen sind, so hat die 
Einheitengruppe von $R$ die Ordnung $(p-1) \cdot (q-1)$. Speziell gilt also für jede Zahl $a \in \mathbb Z$, 
die teilerfremd zu $n$ ist: 
  	$$ [a^{(p-1)(q-1)}] = [1] $$
und 
  	$$ [a^{(p-1)(q-1)+1}] = [a] $$
\end{aufgabe}



\bigbreak

Für die Anwendung in der Kryptographie, speziell die sogenannten \textit{public key Kryptosysteme} wichtiger 
als die Körper $\mathbb F_p$ sind die Ringe $\mathbb Z/(n)$ wobei $n = p \cdot q$ mit zwei Primzahlen 
$p \neq q$. Die Mathematik, die dem sogenannten RSA--Algorithmus zugrunde liegt, wird in den folgenden
Aufgaben beschrieben:

\begin{aufgabe} Ist $R = \mathbb Z/(n)$, so ist ein $[a]$ genau dann eine Einheit in $R$, wenn $a$ und $n$ 
teilerfremd sind.
\end{aufgabe}

\begin{aufgabe} Ist $R = \mathbb Z/(n)$ mit $n = p \cdot q$ wobei $p \neq q$ zwei Primzahlen sind, so hat die 
Einheitengruppe von $R$ die Ordnung $(p-1) \cdot (q-1)$. Speziell gilt also für jede Zahl $a \in \mathbb Z$, 
die teilerfremd zu $n$ ist: 
  	$$ [a^{(p-1)(q-1)}] = [1] $$
und 
  	$$ [a^{(p-1)(q-1)+1}] = [a] $$
\end{aufgabe}

Damit sind wir in der Lage, den RSA--Algorithmus zu beschreiben und zu erklären:

Es seien $p$ und $q$ zwei voneinander verschiedene Primzahlen, und es sei $N = p \cdot q$. 
wir wählen eine zu $(p-1) \cdot (q-1)$ teilerfremde ganze Zahl $e \in \mathbb Z$. Mit Hilfe des 
euklidischen Algorithmus ermitteln wir   $d, t \in \mathbb Z$ mit 
 	$$ d \cdot e + t \cdot (p-1) \cdot (q-1) = 1. $$

\begin{aufgabe}\label{gruppe_rsa_dekrypt} Für jedes $a \in \mathbb Z$ gilt in $\mathbb Z/(N)$:
  	$$ [a^{e \cdot d}] = [a] $$
\end{aufgabe}

\index{RSA--Kryptosystem}Damit nennt man $(N, e)$ den \textbf{öffentlichen Schlüssel} und $(N,d)$ den 
\textbf{privaten Schlüssel} 
des Verfahrens. Wir veröffentlichen unseren öffentlichen Schlüssel $(N, e)$ und halten die Zahl $d$ 
geheim. Will uns eine andere Partei eine Nachricht zukommen lassen, die sich durch eine Zahl $m$ aus 
$\mathbb Z/(N)$ bzw. eine Zahl aus $\{0, 1, \ldots, N-1\}$ darstellen lässt (z.B. eine Kontonummer), 
so kann diese Partei wie folgt vorgehen:

\begin{itemize}
\item Sie benutzt unseren öffentlichen Schlüssel $(N,e)$ und berechnet 
    	$$ [a] = [m^e] \, \textrm{ in } \mathbb Z/(N).$$
\item Sie schickt uns die Zahl $a$ auf einem öffentlichen Kanal, etwa per Post.
\end{itemize}

Wir gehen dann wie folgt vor

\begin{itemize} 
\item Wir benutzen unseren privaten Schlüssel $d$ und berechnen $[b] = [a^d]$ in  $\mathbb Z/(N)$.
\item Wir benutzen Aufgabe~\ref{gruppe_rsa_dekrypt} und schließen, dass 
   	$$[b] = [\left(m^e\right)^d] = [m^{e \cdot d}] = [m] $$
erhalten also das gewünschte $m$.
\end{itemize}

Die Sicherheit des RSA--Kryptosystems besteht darin, dass es im allgemeinen sehr schwierig ist, alle 
Primteiler einer sehr großen Zahl zu ermitteln. Sind etwa $p$ und $q$ Primzahlen mit mindestens 100 Stellen, so ist 
gegenwärtig kein effizientes Verfahren bekannt, dass aus der Kenntnis von $N = p \cdot q$ die Zahlen 
$p$ und $q$ ermittelt. Damit kann aber auch $(p-1) \cdot (q-1)$ nicht effizient berechnet werde, und ohne Kenntnis 
dieser Zahl wiederum lässt sich $d$ nicht ermitteln. Wichtig ist also, dass wir nur $N$ veröffentlichen 
und nicht seine Teiler $p$ und $q$. 

\begin{beispiel} Wir wählen die Primzahlen $p = 7$ und $q = 13$, also $N = 91$. Dann ist $(p-1) \cdot (q-1) 
= 72$, und wir können $e = 5$ wählen. Hierfür gilt
  	$$ 5 \cdot 29 + (-2) \cdot 72 = 1 $$
so dass also $d = 29$ unser privater Schlüssel ist.

Wir veröffentlichen also das Zahlenpaar $(91, 5)$.

Ein Freund will uns eine Nachricht zukommen lassen, die durch die Zahl $m = 20$ dargestellt wird. 
Er berechnet
  	$$ [a] = [20^5] = [76] $$
und schickt uns die Zahl 76 auf dem Postweg. Wir ermitteln 
  	$$ [76^{29}] = [20] $$
und haben also tatsächlich die Nachricht entschlüsselt.
\end{beispiel}

\begin{notiz} die Zahl $N = 91$, die wir in unserem Beispiel gewählt haben, bietet natürlich keinerlei 
Sicherheit, da jeder aus ihr die Primfaktoren $p = 7$ und $q = 13$ ablesen kann. Trotzdem ist die direkte 
Berechnung hier schon kompliziert und kein Taschenrechner ist in der Lage, $76^{29}$ korrekt anzugeben. Selbst 
Computer kommen beim Potenzieren schnell an die Grenzen ihrer Rechengenauigkeit. Daher behilft man sich hier 
mit der \textbf{Methode des iterierten Quadrierens}. In unserem Beispiel etwa schreiben wir 
  	$$ 29 = 16 + 8 + 4 + 1 = 2^4 + 2^3 + 2^2 + 1 $$
und erhalten damit 
  	$$ 76^{29} = \left(\left(\left( 76^2 \right)^2\right)^2\right)^2 \cdot \left(\left( 76^2 \right)^2\right)^2
	\cdot \left( 76^2 \right)^2 \cdot 76 $$
Wir berechnen nun jedes Quadrat einzeln und reduzieren nach dem Quadrieren wieder modulo 91. Dadurch bleiben 
die Zahlen überschaubar. In unserem Beispiel etwa erhalten wir dadurch (immer modulo $91$): 
  	$$ \begin{array} {l c l c l}
	76^2 & & & = & 43, \\ 
	\left( 76^2 \right)^2 & = & 43^2 & = &  29, \\ 
	\left(\left( 76^2 \right)^2\right)^2 & = & 29^2 & = & 22, \\ 
  	\left(\left(\left( 76^2 \right)^2\right)^2\right)^2 & = & 22^2 & = & 29 
	\end{array} $$
Damit ergibt sich
	$$ \begin{array} {l c l c l c l c l }
	& & 76^{16} & = & \left(\left(\left( 76^2 \right)^2\right)^2\right)^2 & & & = & 29 \\
	& & 76^{16} \cdot 76^{8} & = & 29 \cdot \left(\left( 76^2 \right)^2\right)^2 & = & 29 \cdot 22 & = & 1 \\
	& & 76^{16} \cdot 76^{8} \cdot 76^{4} &=& 1 \cdot 76^4 & = & 1 \cdot 29 & = & 29 \\
	76^{29} & = &  76^{16} \cdot 76^{8} \cdot 76^{4} \cdot 76 & = & 29 \cdot 76^1 & = & 29 \cdot 76 &  = & 20 \\
	\end{array} $$ 
Mit diesen Techniken bleiben die Zahlen, seobst bei sehr hohen Potenzen, immer in einem Bereich, den die 
Prozessorarithmetik verarbeiten kann. Genauere Analysen führen noch zu einer weiteren Verbesserung dieser 
Methode, die in den Übungen behandelt wird. 
\end{notiz}

\bigbreak
\bigbreak

Wir wollen hier noch eine weitere kryptographische Anwendung der elementaren Zahlentheorie untersuchen, nämlich 
das sogenannte \textit{Diffie--Hellman--Protokoll}, das auf Satz~\ref{fp_mult_zyklisch} beruht. Dieses Verfahren wird 
in erster Linie zum Austausch von Schlüsseln verwendet. Dabei werden also nicht die Nachrichten selbst ausgetauscht. 
Dei beiden Kommunikationspartner einigen sich vielmehr vorab auf eine Liste von (klassischen) 
Verschlüsselungsverfahren (z.B. Verfahren die von Parametern abhängen). In einer gegebenen Situation müssen 
sich die beiden dann nur noch auf ein Verfahren einigen. Dadurch kann das Verfahren bei jeder Kommunikation 
geänder werden, was einen Angriff sehr viel schwieriger macht als bei einem immer wieder verwendeten Verfahren.

Wir geben uns dazu eine Primzahl $p$ vor (die in der praktischen Anwendung mindestens 100 Stellen haben sollte), 
und wir nehmen an, dass die Liste unserer Schlüssel durch die Elemente der 
Einheitengruppe $E(\mathbb F_p)$ von $\mathbb F_p$gegeben ist.
Die Einheitengruppe $E(\mathbb F_p)$ ist gemäß Satz~\ref{fp_mult_zyklisch} zyklisch, wird also von einem 
Element erzeugt, und wir wählen einen Erzeuger $g$ dieser Gruppe. Diese Daten (also $p$ und $g$) sind 
öffentlich und bekannt. Wollen nun zwei Kommunikationspartner $A$ und $B$ einen Schlüssel austauschen, 
so wählen $A$ und $B$ jeweils eine beliebige Zahl $a, b \in \{2, \ldots , p-2\}$. Dann berechnet $A$ die Zahl
$\alpha = g^a \textrm{ mod } p$ und $B$ die Zahl $\beta = g^b \textrm{ mod } p$. Diese beiden Zahlen 
$\alpha$ und $\beta$ werden nun ausgetauscht. Die Zahlen $a$ und $b$ halten die beiden geheim. Nun berechnet 
$B$ die Zahle $S_1 = \alpha^{b}  \textrm{ mod } p$ und $A$ die Zahl $S_2 = \beta^{a}  \textrm{ mod } p$. 
Da (in $\mathbb F_p$) gilt
  	$$ S_1 = \alpha^b = \left(g^a\right)^b =  g^{ab} = \left(g^b\right)^a = \beta^a = S_2 $$
haben also die beiden auf diese Art und Weise denselben Schlüssel $S$ berechnet, den sie jetzt für 
ihre weitere Kommunikation benutzen. 

Die Sicherheit des Verfahrens beruht darauf, dass es kein effizientes Verfahren gibt, aus der Kenntnis von 
$p$, $g$ und $\alpha = g^a$ die Zahl $a$ zu berechnen (\textit{Problem des diskreten Logarithmus}). Die Kenntnis 
von $g^a$ und $g^b$ reicht aber nicht aus, um $g^{ab}$ zu berechnen.  Bei hinreichend 
großen Primzahlen kann also der Schlüssel nicht in akzeptabler Zeit berechnet werden und die Kommunikation 
von $A$ und $B$ kann daher ungestört erfolgen.

\begin{beispiel} Wir erläutern hier das Diffie--Hellman--Protokoll mit Hilfe der Zahlen $p = 13$ und $g = 2$.

\begin{enumerate}
\item $A$ wählt $a = 5$ und $B$ wählt $b = 7$.
\item $A$ berechnet $\alpha = 2^5 \textrm{ mod } 13 = 32 \textrm{ mod } 13  = 6 \textrm{ mod } 13 $ und $B$ 
berechnet $\beta = 2^7 \textrm{ mod } 13 = 128 \textrm{ mod } 13 = 11 \textrm{ mod } 13$.
\item $A$ und $B$ tauschen die Zahlen $6$ und $11$ aus.
\item $A$ berechnt $S = 11^5 \textrm{ mod } 13 = 7 \textrm{ mod } 13$ und $B$ berechnet 
$S = 6^7 \textrm{ mod } 13 = 7 \textrm{ mod } 13$.
\item $A$ und $B$ benutzen den Schlüssel $7$ für ihre weitere Kommunikation.
\end{enumerate}
\end{beispiel}