
\chapter{Determinanten}

\section{Definition der Determinanten}

Von besonderem Interesse für uns und für viele Anwendungen sind quadratische Matrizen bzw. Gleichungssysteme 
mit genausovielen Unbekannten wie Gleichungen. Wie wir bereits in~\ref{gls_rangsatz} gesehen haben, muss eine 
Matrix $A$ quadratisch sein, damit alle zugehörigen Gleichungssysteme $A \cdot \vektor{x} = 
\vektor{b}$ eine eindeutige Lösung besitzen. In diesem Abschnitt werden wir jeder quadratischen 
Matrix in subtiler Weise einen Skalar, ihre Determinanete, zuordnen, der uns erlaubt, die Frage nach der 
eindeutigen Lösbarkeit allgemein zu entscheiden, und die uns auch hilft, diese Lösung gegebenvalls zu 
finden.

Determinanten waren wohl schon G. Leibniz (1646 - 1716) bekannt. Wiederentdeckt und weiterentwickelt wurde 
sie vom Mathematiker G. Cramer (1704 - 1752). Beiträge zur Determinantentheorie stammen auch von 
dem englischen Mathematiker C. L. Dodgson (1832 - 1898), der allerdings besser als Lewis Carroll und für 
seine Kinderliteratur (etwa \textit{Alice im Wunderland}) bekannt ist.


\subsection{Zweidimensionale Determinanten}\label{section_det_2}

\setcounter{definition}{0}
\setcounter{beispiel}{0}
\setcounter{notiz}{0}


Der einfachste Fall eines linearen Gleichungssystems ist natürlich der einer einzigen Gleichung in einer 
Unbekannten, 
  \begin{equation}\label{det_lin_gls_1} a \cdot x = b \end{equation}
In dieser Situation ist die Lage offensichtlich: 

\begin{regel}
Genau dann hat (~\ref{det_lin_gls_1}) für jede Wahl von $b$ eine eindeutige Lösung, wenn $a \neq 0$; 
in diesem Fall ist $x = \frac {b}{a}$.
\end{regel}

Unser Ziel ist es, diese Aussage in möglichst großen Umfang auf $n \times n$--Matrizen zu verallgemeinern. 

Wir wollen uns in diesem Abschnitt mit dem Spezialfall $n = 2$ beschäftigen, den 
nach (~\ref{det_lin_gls_1}) einfachsten Fall. Ein solches Gleichungssystem hat die allgemeine Gestalt
  	\begin{equation}\label{det_lin_gls_2_1} \begin{array} {l c l c l c l}
  	a_{1,1} x_1 & + & a_{1,2} x_2 & = & b_1 \\
  	a_{2,1} x_1 & + & a_{2,2} x_2 & = & b_2
  	\end{array} \end{equation}
Wenn wir versuchen, dieses Gleichungssystem nach den Eliminations\-me\-tho\-den von Gauß zu lösen, kommen 
wir schnell auf Fallunterscheidungen (der Fall $a_{1,1} = 0$ wird anders behandelt als der Fall $a_{1,1} \neq 0$). 
Wir wollen auf diese Fallunterscheidung verzichten und multiplizieren die erste Gleichung mit $a_{2,2}$ und 
die zweite mit $a_{1,2}$. Damit erhalten wir 
  	\begin{equation}\label{det_lin_gls_2_2} \begin{array} {l c l c l c l}
  	a_{2,2} a_{1,1} x_1 & + & a_{2,2} a_{1,2} x_2 & = & a_{2,2} b_1 \\
  	a_{1,2} a_{2,1} x_1 & + & a_{1,2} a_{2,2} x_2 & = & a_{1,2} b_2
  	\end{array} \end{equation}
wobei wir aber beachten müssen, dass das erhaltene Gleichungssystem nicht mehr äquivalent zum Ausgangssystem 
ist, wenn $a_{2,2} = 0$ oder $a_{1,2} = 0$. Addieren wir in (~\ref{det_lin_gls_2_2}) die erste Gleichung zum  
Negativen der zweiten, so erhalten wir 
  	\begin{equation}\label{det_lin_gls_2_3} \begin{array} {l c l c l c l}
  	a_{2,2} a_{1,1} x_1 & + & a_{2,2} a_{1,2} x_2 & = & a_{2,2} b_1 \\
  	\left(a_{1,1} a_{2,2} - a_{1,2} a_{2,1} \right) \cdot x_1 &  &  & = & a_{2,2} b_1 - a_{1,2} b_2
  	\end{array} \end{equation}
Falls jetzt $\left(a_{1,2} a_{2,1} - a_{1,1} a_{2,2} \right) \neq 0$, so hat die zweite Zeile, wie wir 
schon im Spezialfall $n = 1$ gesehen haben, die eindeutige Lösung
  	$$ x_1 = \frac {a_{2,2} b_1 - a_{1,2} b_2}{a_{1,1} a_{2,2} - a_{1,2} a_{2,1}} $$
und hieraus ergibt sich aus der ersten Gleichung
  	$$ x_2 = \frac {a_{1,1} b_2 - a_{2,1} b_1}{a_{1,1} a_{2,2} - a_{1,2} a_{2,1}} $$
als eindeutiger Wert für $x_2$. Damit haben wir gesehen, dass (~\ref{det_lin_gls_2_3}) für jede Wahl von 
$b_1$ und $b_2$ genau eine Lösung hat, wenn $a_{1,1} a_{2,2} - a_{1,2} a_{2,1} \neq 0$. Setzen wir diese 
Lösung in das ursprüngliche Gleichungssystem (~\ref{det_lin_gls_2_1}) ein, so rechnen wir leicht nach, dass
auch dieses gelöst wird. Damit haben wir also eine Lösung von (~\ref{det_lin_gls_2_1}) gefunden, falls 
$a_{1,1} a_{2,2} - a_{1,2} a_{2,1} \neq 0$. Wenn aber $a_{2,2} = 0$ oder $a_{1,2} = 0$, so ist das transformierte 
Gleichungssystem nicht äquivalent zum ursprünglichen. Allerdings kann sich durch unsere Transformationen 
die Anzahl der Lösungen höchstens vergrö{ss}ern, jede Lösung von (~\ref{det_lin_gls_2_1}) ist auch eine 
Lösung von (~\ref{det_lin_gls_2_3}). Damit haben wir schon gesehen:

Falls $a_{1,2} a_{2,1} - a_{1,1} a_{2,2} \neq 0$ so hat (~\ref{det_lin_gls_2_1}) für jede Wahl von $b_1$ 
und $b_2$ eine eindeutig bestimmte Lösung, die gegeben wird durch
  	$$ \begin{array} {l c l}
   	x_1 & = & \dfrac {a_{2,2} b_1 - a_{1,2} b_2}{a_{1,1} a_{2,2} - a_{1,2} a_{2,1}} \\[1.0em]
   	x_2 & = & \dfrac {a_{1,1} b_2 - a_{2,1} b_1}{a_{1,1} a_{2,2} - a_{1,2} a_{2,1}}
  	\end{array} $$

Es bleibt noch der Fall $a_{1,1} a_{2,2} - a_{1,2} a_{2,1} = 0$ zu betrachten. In diesem Fall hat des 
Gleichungssystem (\ref{det_lin_gls_2_3}) keine Lösung, falls $a_{1,2} b_2 - a_{2,2} b_1 \neq 0$, oder 
falls $a_{2,2}a_{1,1} = 0$ und $a_{2,2} a_{1,2} = 0$ und $a_{2,2} b_1 \neq 0$. Andernfalls 
hat (\ref{det_lin_gls_2_3}) unendlich viele Lösungen. 

Aber was ist aber in diesem Fall mit der ursprünglichen Gleichung (\ref{det_lin_gls_2_1})? 
Wir betrachten hierzu alle in Frage kommenden Fälle. Klar ist der triviale Fall $a_{1,1} = 
a_{1,2} = a_{2,1} = a_{2,2} = 0$: In diesem Fall gibt es für $b_1 = b_2 = 0$ unendlich viele Lösungen und
in allen anderen Fällen keine. Wir können uns daher auf die Situation beschränken, dass mindestens ein 
Koeffizient $a_{i,j} \neq 0$. Ist aber etwa $a_{1,1} \neq 0$, so muss wegen $a_{1,1} a_{2,2} - 
a_{1,2} a_{2,1} = 0$ schon $a_{2,2} = \frac {a_{1,2} a_{2,1}}{a_{1,1}}$ gelten, und daher ist $x_1 = a_{1,2}$ 
und $x_2 = - a_{1,1} $ eine nichttriviale Lösung von 
  	\begin{equation}\label{det_lin_gls_2_4} \begin{array} {l c l c l c l}
  	a_{1,1} x_1 & + & a_{1,2} x_2 & = & 0 \\
  	a_{2,1} x_1 & + & a_{2,2} x_2 & = & 0
  	\end{array} \end{equation}
wie wir sofort nachrechnen. Entsprechend finden wir im Fall $a_{2,2} \neq 0$ die nichttriviale Lösung 
$x_1 = a_{2,2}$ und $x_2 = -a_{2,1}$, ist $a_{2,1} \neq 0$, so löst ebenfalls $x_1 = a_{2,2}$ 
und $x_2 = -a_{2,1}$ das Gleichungssystem (\ref{det_lin_gls_2_4}) und ist nicht--trivial, und 
ist $a_{1,2} \neq 0$, so liefert wiederum $x_1 = a_{1,2}$ 
und $x_2 = -a_{1,1}$ eine nicht--triviale Lösung von (\ref{det_lin_gls_2_4}). 
Da es in diesem 
Fall auch noch die triviale Lösung gibt, hat also das homogene System (\ref{det_lin_gls_2_4}) mehrere 
Lösungen und entsprechend hat das System (\ref{det_lin_gls_2_1}) wegen Regel~\ref{gls_summary} entweder 
gar keine oder unendlich viele Lösungen. 

\begin{definition} Die \index{Determinante!zweidimensional}\textbf{Determinante} $\det{A}$ einer 
$2 \times 2$--Matrix $A = \left( \begin{smallmatrix} a_{1,1} & a_{1,2} \\ a_{2,1} & a_{2,2} 
\end{smallmatrix} \right) $ ist der Ausdruck
  	$$ \det{A} = a_{1,1} \cdot a_{2,2} - a_{2,1} \cdot a_{1,2} $$
\end{definition}

\bigbreak

Wir haben in unseren Vorüberlegungen schon das folgende Resultat abgeleitet:

\begin{satz}\label{det_eindeutig_2} Ist $A$ eine $2 \times 2$--Matrix, so hat das Gleichungssystem
  	$$ A  \cdot \vektor{x} = \vektor{b} $$
genau dann eine eindeutige Lösung für jedes $\vektor{b} \in \mathbb R^2$, wenn $\det{A} \neq 0$.
\end{satz}

Darüberhinaus haben wir sogar eine Formel für die eindeutig bestimmte Lösung gefunden, die sogenannte 
\index{Cramersche Regel!zweidimensional}\textbf{Cramersche Regel} (für $n = 2$).

\begin{satz}\label{cramer_regel_2} Ist $A$ eine $2 \times 2$--Matrix mit $\det{A} \neq 0$, so ist 
die eindeutige Lösung von 
  	$$ A  \cdot \vektor{x} = \vektor{b} $$
für einen beliebigen Vektor $ \vektor{b} = \left( \begin{smallmatrix} b_1 \\ b_2 
\end{smallmatrix} \right)$ gegeben durch 
  	$$ \begin{array} {l c l}
   	x_1 & = & \dfrac {a_{2,2} \cdot b_1 - a_{1,2} \cdot b_2}{\det{A}} \\[1.0em]
   	x_2 & = & \dfrac {a_{1,1} \cdot b_2 - a_{2,1} \cdot b_1}{\det{A}}
  	\end{array} $$
\end{satz}

\bigbreak

\begin{beispiel} Wir betrachten die Matrix
  	$$ A = \left( \begin{matrix} 1 & 2 \\ 3 & 4  \end{matrix} \right) $$
Hierfür gilt
  	$$ \det{A} = 1 \cdot 4 - 2 \cdot 3 = 4 - 6 = -2 $$
Insbesondere ist also $\det{A} \neq 0$ und damit hat jedes Gleichungssystem 
  	$$  A  \cdot \vektor{x} = \vektor{b} $$ 
eine eindeutige Lösung. Betrachten wir etwa
  	$$ \begin{array} {r c l c l}
   	x_1 & + & 2 x_2 & = & 2 \\
   	3 x_1 & + & 4 x_2 & = & 4 
   	\end{array} $$
so gilt
  	$$ \begin{array} {l c l c l}
   	x_1 & = & \frac {4 \cdot 2 - 2 \cdot 4}{-2} & = & 0 \\
   	x_2 & = & \frac {1 \cdot 4 - 3 \cdot 2}{-2} & = & 1
  	\end{array} $$
Natürlich führt auch der Gaußalgorithmus zu diesem Ergebnis.
\end{beispiel}

\begin{beispiel} Wir betrachten die Matrix
  	$$ A = \left( \begin{matrix} 1 & -1 \\ 1 & 1  \end{matrix} \right) $$
Hierfür gilt
  	$$ \det{A} = 1 \cdot 1 - 1 \cdot (-1) = 1 - (-1) = 2 $$
Insbesondere ist also $\det{A} \neq 0$ und damit hat jedes Gleichungssystem 
  	$$  A  \cdot \vektor{x} = \vektor{b} $$ 
eine eindeutige Lösung. Betrachten wir etwa
  	$$ \begin{array} {r c l c l}
   	x_1 & - &  x_2 & = & 4 \\
   	x_1 & + &  x_2 & = & 2 
   	\end{array} $$
so gilt
  	$$ \begin{array} {l c l c l}
   	x_1 & = & \frac {1 \cdot 4 - (-1) \cdot 2}{2} & = & 3 \\
   	x_2 & = & \frac {1 \cdot 2 - 1 \cdot 4}{2} & = & -1
  	\end{array} $$
\end{beispiel}

\begin{beispiel} Wir betrachten die Matrix
  	$$ A = \left( \begin{matrix} 1 & 2 \\ 2 & 4  \end{matrix} \right) $$
Hierfür gilt
  	$$ \det{A} = 1 \cdot 4 - 2 \cdot 2 = 4 - 4 = 0 $$
Insbesondere ist also $\det{A} = 0$ und damit gibt es kein Gleichungssystem 
  	$$  A  \cdot \vektor{x} = \vektor{b} $$ 
mit einer eindeutigen Lösung. Betrachten wir das zugehörige homogene System
  	$$ \begin{array} {r c l c l}
   	x_1 & + &  2 x_2 & = & 0 \\
   	2x_1 & + &  4 x_2 & = & 0 
   	\end{array} $$
so hat dieses die Normalform 
  	$$ \begin{array} {r c l c l}
   	x_1 & + & 2 x_2 & = & 0 \\
     	&   &  0 & = & 0 
   	\end{array} $$
und wir sehen, dass $x_1 = -2$ und $x_2 = 1$ eine nichttriviale Lösung ist. Aus der Normalform 
lesen wir auch ab, dass $\mathrm{rg}(A) = 1$, also wegen Satz~\ref{gls_rangsatz} auch $\mathrm{nul}(A) = 1$. 
Also ist $\mathrm{Ker}(A)$ eine Vektorraum der Dimension 1, und $\left( \begin{smallmatrix} -2 \\ 1 
\end{smallmatrix} \right)$ ist eine Basis.
\end{beispiel}

\bigbreak

In Satz~\ref{cramer_regel_2} haben wir eine Formel für die Lösungen eines Gleichungssystems kennengelernt, 
falls die Determinante der Koeffizientenmatrix von Null verschieden ist. In dieser Formel haben auch die 
Zähler eine Gestalt, die an eine Determinante erinnert. Dazu betrachten wir wieder ein Gleichungssystem
  	$$ A  \cdot \vektor{x} = \vektor{b} $$ 
mit Koeffizientenmatrix $A = \left( \begin{smallmatrix} a_{1,1} & a_{1,2} \\ a_{2,1} & a_{2,2}
\end{smallmatrix} \right)$ und $\vektor{b} = \left( \begin{smallmatrix} b_1 \\ b_2
\end{smallmatrix} \right)$ und wir bezeichnen mit $A_1(\vektor{b})$ bzw. $A_2(\vektor{b})$ 
die $2 \times 2$--Matrix, die wir aus $A$ erhalten indem wir die erste bzw. zweite Spalte von $A$ durch den 
Vektor $\vektor{b}$ ersetzen, also
  	$$ A_1(\vektor{b}) =  \left( \begin{matrix} b_1 & a_{1,2} \\ b_2 & a_{2,2} \end{matrix} \right), \qquad 
	A_2(\vektor{b}) =  \left( \begin{matrix} a_{1,1} & b_1 \\ a_{2,1} & b_2 \end{matrix} \right) $$
Dann folgt sofort aus Satz~\ref{cramer_regel_2}, dass auch die folgende Form der Cramerschen Regel gilt:

\begin{satz}\label{cramer_regel_2_1} Ist $A$ eine $2 \times 2$--Matrix mit $\det{A} \neq 0$, so ist 
die eindeutige Lösung von 
  	$$ A  \cdot \vektor{x} = \vektor{b} $$
für einen beliebigen Vektor $ \vektor{b}$ gegeben durch 
  	$$ \begin{array} {l c l}
   	x_1 & = & \dfrac {\det{A_1(\vektor{b})}}{\det{A}} \\[1.0em]
   	x_2 & = & \dfrac {\det{A_2(\vektor{b})}}{\det{A}}
  	\end{array} $$
\end{satz}

\bigbreak 

Die Determinante einer $2 \times 2$--Matrix ist sehr einfach zu berechnen. Im Hinblick auf spätere 
Verallgemeinerungen wollen wir trotzdem schon in diesem Fall Regeln für die Berechnung von Determinanten 
aufstellen, die die Berechnungen weiter vereinfachen.

\begin{regel}\label{det_norm_2} Für die $2 \times 2$--Einheitsmatrix
$E_2 = \left( \begin{smallmatrix} 1 & 0 \\ 0 & 1 \end{smallmatrix} \right)$ gilt
  	$$ \det{E_2} = 1 $$
\end{regel}

\medbreak

\begin{regel}\label{det_transpon_2} Für eine $2 \times 2$--Matrix $A$ gilt $\det{A} = \det{A^T}$.
\end{regel}

\beweis{ Ist $A = \left( \begin{smallmatrix} a_{1,1} & a_{1,2} \\ a_{2,1} & a_{2,2}
\end{smallmatrix} \right)$, so ist $A^T = \left( \begin{smallmatrix} a_{1,1} & a_{2,1} \\ a_{1,2} & a_{2,2}
\end{smallmatrix} \right)$. Jetzt folgt die Aussage aus der Definition.
}

\begin{beispiel}
  	$$ \det{ \begin{matrix} 1 & 2 \\ 3 & 0 \end{matrix} } = -6 = 
   	\det{ \begin{matrix} 1 & 3 \\ 2 & 0 \end{matrix} } $$
\end{beispiel} 

\medbreak

\begin{regel}\label{det_rule_2_2} Für eine $2 \times 2$--Matrix $A$ gilt $\det{A} = 0$, wenn 

\begin{enumerate}
\item Eine Zeile von $A$ die Nullzeile ist.
\item Eine Spalte von $A$ die Nullspalte ist.
\item Eine Zeile von $A$ ein Vielfaches der anderen Zeile ist.
\item Eine Spalte von $A$ ein Vielfaches der anderen Spalte ist. 
\end{enumerate}
\end{regel}

\begin{beispiel} Ist $A = \left( \begin{smallmatrix} 1 & 0 \\ 0 & 0 \end{smallmatrix} \right)$ so gilt
  	$$ \det{A} = 0 $$
\end{beispiel}

\begin{beispiel} Ist $A = \left( \begin{smallmatrix} 1 & 2 \\ 4 & 8 \end{smallmatrix} \right)$ so gilt
  	$$ \det{A} = 0 $$
\end{beispiel}

\beweis{ Die ersten beiden Formeln sind klar. Für die dritte betrachten wir den Fall $a_{2,1} = \lambda 
a_{1,1}, \, a_{2,2} = \lambda a_{1,2}$ und erhalten
  	$$ \det{A} = a_{1,1} \cdot \lambda \cdot a_{1,2} - a_{1,2} \cdot \lambda \cdot a_{1,1} = 0 $$
Genause behandeln wir den Fall, dass die erste Zeile ein Vielfaches der zweiten ist. 

Die vierte Regel kann analog bewiesen werden. Sie folgt aber wegen Regel~\ref{det_transpon_2} auch sofort 
aus der dritten!
}

\medbreak

\begin{regel}\label{det_rul_2_trig} Ist $A$ eine obere oder untere Dreiecksmatrix (d.h. gilt $a_{1,2} = 0$ 
oder $a_{2,1} = 0$), so gilt
  	$$ \det{A} = a_{1,1} \cdot a_{2,2} $$
\end{regel}

\beweis{ Allgemein gilt
  	$$ \det{A} = a_{1,1} \cdot a_{2,2} - a_{1,2} \cdot a_{2,1} $$
Ist $A$ eine Dreiecksmatrix, so ist $a_{1,2} = 0$ oder $a_{2,1} = 0$, und unsere Formel folgt.
}

\begin{beispiel}
	$$ \det{ \begin{matrix} 2 & 2 \\ 0 & 3 \end{matrix} } = 2 \cdot 3 = 6 $$
\end{beispiel}

\begin{regel}\label{det_rule_2_3} Entsteht die $2 \times 2$--Matrix $A'$ aus $A$ durch Vertauschen der 
beiden Zeilen, so gilt
  	$$ \det{A'} = - \det{A} $$
Das Gleiche gilt, wenn $A'$ durch Vertauschung der Spalten aus $A$ entsteht.
\end{regel}

\beweis{ Ist $A = \left( \begin{smallmatrix} a_{1,1} & a_{1,2} \\ a_{2,1} & a_{2,2}
\end{smallmatrix} \right)$ so ist $A' = \left( \begin{smallmatrix} a_{2,1} & a_{2,2} \\ a_{1,1} & a_{1,2}
\end{smallmatrix} \right)$ und damit
  	$$ \det{A'} = a_{2,1} a_{1,2} - a_{1,1} a_{2,2} = - \det{A} $$
}

\begin{beispiel} Ist $A = \left( \begin{smallmatrix} 1 & 2 \\ 3 & 0 \end{smallmatrix} \right)$, so ist
$A' = \left( \begin{smallmatrix} 3 & 0 \\ 1 & 2 \end{smallmatrix} \right)$ und 
  $$ \det{A'} = 3 \cdot 2 - 0 \cdot 1 = 6 = - \left( 1 \cdot 0 - 2 \cdot 3 \right) = \det{A} $$
\end{beispiel}

\medbreak

\begin{regel}\label{det_rule_2_4} Entsteht die $2 \times 2$--Matrix $A'$ aus $A$ durch Multiplikation 
einer Zeile von $A$ mit 
einer Zahl $r$, so gilt
  	$$ \det{A'} = r \cdot \det{A} $$
Das Gleiche gilt, wenn $A'$ aus $A$ durch Multiplikation einer Spalte von $A$ mit 
einer Zahl $r$ hervorgeht.
\end{regel}

\beweis{ Ist $A = \left( \begin{smallmatrix} a_{1,1} & a_{1,2} \\ a_{2,1} & a_{2,2}
\end{smallmatrix} \right)$ und entsteht $A'$ durch Multiplikation der ersten Zeile von $A$ mit $r$ aus $A$, so 
ist $A' = \left( \begin{smallmatrix} r \cdot a_{1,1} & r \cdot a_{1,2} \\ a_{2,1} & a_{2,2} \end{smallmatrix} 
\right)$ und damit
  	$$ \det{A'} = r \cdot a_{1,1} \cdot a_{2,2} - r \cdot a_{1,2} \cdot a_{2,1} 
	= r \cdot \left( a_{1,1} a_{2,2} - a_{1,2} a_{2,1} \right) 
	= r \cdot \det{A} $$
Die anderen Fälle behandelt man genauso.
} 

\begin{beispiel} Ist $A = \left( \begin{smallmatrix} 1 & 2 \\ 1 & 3 \end{smallmatrix} \right)$, und ist
$A' = \left( \begin{smallmatrix} 4 & 8 \\ 1 & 3 \end{smallmatrix} \right)$, so ist
  	$$ \det{A'} = 4 \cdot 3 - 8 \cdot 1 = 4 = 4 \cdot \left( 1 \cdot 3 - 2 \cdot 1 \right) = 
  	4 \cdot \det{A} $$
\end{beispiel}

\medbreak

Die erste nichttriviale Regel ist die folgende

\begin{regel}\label{det_rule_2_5} Entsteht die $2 \times 2$--Matrix $A'$ aus $A$ dadurch, dass wir ein 
Vielfaches einer Zeile von 
$A$ zur anderen Zeile von $A$ addieren, so gilt
  	$$ \det{A'} =  \det{A} $$
Das Gleiche gilt, wenn $A'$ aus $A$ dadurch entsteht, dass wir ein Vielfaches einer Spalte von 
$A$ zur anderen Spalte von $A$ addieren.
\end{regel}

\beweis{ Sei $A = \left( \begin{smallmatrix} a_{1,1} & a_{1,2} \\ a_{2,1} & a_{2,2}
\end{smallmatrix} \right)$.
Wir betrachten den Fall, dass das $r$--fache der ersten Zeile von $A$ zur zweiten addiert wird, so dass 
  	$$A' = \left( \begin{matrix} a_{1,1} & a_{1,2} \\ a_{2,1} + r \cdot a_{1,1} & a_{2,2} + r \cdot a_{1,2}
	\end{matrix} \right). $$ 
Dann gilt 
  	$$ \begin{array} {l c l}
  	\det{A'} & = & a_{1,1} \cdot \left( a_{2,2} + r \cdot a_{1,2} \right) - a_{1,2} \cdot 
 	\left( a_{2,1} + r \cdot a_{1,1} \right) \\[0.2em]
  	& = & a_{1,1} \cdot a_{2,2} - a_{1,2} a_{2,1} + \left( a_{1,1} \cdot r \cdot a_{1,2} - a_{1,2} \cdot r 
 	\cdot a_{1,1} \right) \\[0.2em]
  	& = & a_{1,1} \cdot a_{2,2} - a_{1,2} a_{2,1} \\[0.2em]
  	& = & \det{A} 
  	\end{array} $$
}

\begin{beispiel} Ist $A = \left( \begin{smallmatrix} -1 & 1 \\ 2 & 3 \end{smallmatrix} \right)$, und ist
$A' = \left( \begin{smallmatrix} -1 & 1 \\ 0 & 1 \end{smallmatrix} \right)$, so entsteht $A'$ aus $A$ durch 
Addition des doppelten der ersten Zeile zur zweiten, und es gilt
  	$$ \det{A'} = (-1) \cdot 1 - 0 \cdot 1 = 1 = (-1) \cdot 3 - 1 \cdot 2 = \det{A} $$
\end{beispiel}

\begin{notiz} Die Regels erlauben es immer, die Berechnung von Determinanten auf Dreiecksmatrizen zu 
reduzieren. In diesem Fall ist die Determinantenberechnung besonders einfach.
\end{notiz}

\medbreak

\begin{regel}\label{det_rule_2_6} Sind $A$, $B$ zwei $2 \times 2$--Matrizen, so gilt
   	$$ \det{A \cdot B} = \det{A} \cdot \det{B} $$
\end{regel}

\beweis{ Falls $A = \left( \begin{smallmatrix} a_{1,1} & a_{1,2} \\ a_{2,1} & a_{2,2}
\end{smallmatrix} \right)$ und $B = \left( \begin{smallmatrix} b_{1,1} & b_{1,2} \\ b_{2,1} & b_{2,2}
\end{smallmatrix} \right)$, so ist
  	$$ A \cdot B =  \left( \begin{matrix} a_{1,1} b_{1,1} + a_{1,2} b_{2,1}  
	& a_{1,1} b_{1,2} + a_{1,2} b_{2,2} \\ a_{2,1} b_{1,1} + a_{2,2} b_{2,1} 
	& a_{2,1} b_{1,2} + a_{2,2} b_{2,2} 
  	\end{matrix} \right) $$
und damit
  	$$ \begin{array} {l c l}
  	\det{A \cdot B} & = & \left( a_{1,1} b_{1,1} + a_{1,2}b_{2,1} \right) \cdot \left( a_{2,1} b_{1,2} + 
    	a_{2,2} b_{2,2} \right) \\
   	& = & - \left( a_{1,1} b_{1,2} + a_{1,2} b_{2,2} \right) \cdot 
    	\left( a_{2,1} b_{1,1} + a_{2,2} b_{2,1} \right) \\
    	& = & a_{1,1} b_{1,1} a_{2,1} b_{1,2} + a_{1,1} b_{1,1} a_{2,2} b_{2,2} \\
    	&  & + a_{1,2} b_{2,1} a_{2,1} b_{1,2} 
	+ a_{1,2} b_{2,1} a_{2,2} b_{2,2} \\
    	& & - a_{1,1} b_{1,2} a_{2,1}b_{1,1} - a_{1,1} b_{1,2} a_{2,2} b_{2,1} \\
    	& & - a_{1,2} b_{2,2} a_{2,1} b_{1,1}  - a_{1,2} b_{2,2} a_{2,2} b_{2,1} \\
   	& = & a_{1,1} b_{1,1} a_{2,2} b_{2,2} + a_{1,2} b_{2,1} a_{2,1} b_{1,2} \\
   	& & - a_{1,1} b_{1,2} a_{2,2} b_{2,1} - a_{1,2} b_{2,2} a_{2,1} b_{1,1} \\
   	& = & \left( a_{1,1} a_{2,2} - a_{1,2} a_{2,1} \right) \cdot \left(b_{1,1} b_{2,2} - b_{1,2} b_{2,1} \right) \\
   	& = & \det{A} \cdot \det{B}
   	\end{array} $$
}

\begin{beispiel} Für $A = \left( \begin{smallmatrix} 1 & 3 \\ 0 & 2 \end{smallmatrix} \right)$ und 
$B = \left( \begin{smallmatrix} 3 & 0 \\ 4 & 2 \end{smallmatrix} \right)$ gilt
  	$$ A \cdot B = \left( \begin{matrix} 15 & 6 \\ 8 & 4 \end{matrix} \right) $$
und 
  	$$ \det{A \cdot B} = 15 \cdot 4 - 6 \cdot 8 = 12 = 2 \cdot 6 = \det{A} \cdot \det{B} $$
\end{beispiel}

\bigbreak

Die Determinante kann auch geometrisch interpretiert werden:

%Beziehung zwischen $\vert \vektor{v} \vert \cdot \vert \vektor{w} \vert$, 
%$\langle \vektor{v}, \vektor{w} \rangle$ und $\det{A}$.

\begin{satz}\label{det_2_geom} Sind 
  	$$\vektor{v} = \left( \begin{matrix} v_1 \\ v_2 \end{matrix} \right), \qquad 
     	\vektor{w} = \left( \begin{matrix} w_1 \\ w_2 \end{matrix} \right) $$ 
zwei nicht kollineare Vektoren, ist $P$ das Parallelogramm, dessen Seiten durch die beiden Vektoren 
$\vektor{v}$ und $\vektor{w}$ gegeben ist, und ist 
  	$$ A = \left( \begin{matrix} v_1 & w_1 \\ v_2 & w_2 \end{matrix} \right) $$
die Matrix mit $\vektor{v}$ und $\vektor{w}$ als Splaten, so ist $\vert \det{A} \vert$ die Fläche 
des Parallelogramms $P$.
\end{satz}

%\begin{figure}[h!] 
%  \centering
%     \scalebox{0.70}{\input{determinant1.tex}}
%\end{figure}
\begin{figure}[h]
	\vspace{-0.5cm}
	\begin{center}
	\begin{scaletikzpicturetowidth}{\hsize}
     		\begin{tikzpicture}[gnuplot,scale= \tikzscale]
%% generated with GNUPLOT 5.0p4 (Lua 5.3; terminal rev. 99, script rev. 100)
%% 21.11.2016 18:01:10
\path (0.000,0.000) rectangle (12.500,8.750);
%
  \gpfill{rgb color={1.000,0.000,0.000},opacity=0.30} (1.211,0.895)--(8.726,4.058)--(10.873,8.276)--(3.359,5.113)--cycle;
%
  \gpfill{rgb color={0.000,0.000,0.000}} (11.587,0.895)--(11.587,0.999)--(11.840,0.895)--(11.587,0.791)--cycle;
\gpcolor{rgb color={0.000,0.000,0.000}}
\gpsetlinetype{gp lt border}
\gpsetdashtype{gp dt solid}
\gpsetlinewidth{1.00}
\draw[gp path] (11.587,0.895)--(11.587,0.999)--(11.840,0.895)--(11.587,0.791)--cycle;
\draw[gp path] (0.567,0.895)--(11.587,0.895);
%
  \gpfill{rgb color={0.000,0.000,0.000}} (1.211,8.023)--(1.107,8.023)--(1.211,8.276)--(1.315,8.023)--cycle;
\draw[gp path] (1.211,8.023)--(1.107,8.023)--(1.211,8.276)--(1.315,8.023)--cycle;
\draw[gp path] (1.211,0.473)--(1.211,8.023);
%
  \gpfill{rgb color={1.000,0.000,0.000}} (8.482,3.956)--(8.440,4.057)--(8.726,4.058)--(8.525,3.854)--cycle;
\gpcolor{rgb color={1.000,0.000,0.000}}
\gpsetlinewidth{2.00}
\draw[gp path] (8.482,3.956)--(8.440,4.057)--(8.726,4.058)--(8.525,3.854)--cycle;
\draw[gp path] (1.211,0.895)--(8.482,3.956);
%
  \gpfill{rgb color={1.000,0.000,0.000}} (3.239,4.877)--(3.141,4.927)--(3.359,5.113)--(3.337,4.827)--cycle;
\draw[gp path] (3.239,4.877)--(3.141,4.927)--(3.359,5.113)--(3.337,4.827)--cycle;
\draw[gp path] (1.211,0.895)--(3.239,4.877);
\gpsetlinewidth{1.00}
\draw[gp path] (8.726,4.058)--(10.873,8.276)--(3.359,5.113);
\gpcolor{color=gp lt color border}
\node[gp node left] at (0.460,7.748) {$\footnotesize{y}$};
\node[gp node left] at (10.981,0.368) {$\footnotesize{x}$};
\node[gp node left] at (6.579,2.688) {$\vektor{a}$};
\node[gp node left] at (1.963,4.058) {$\vektor{b}$};
\node[gp node left] at (6.042,4.585) {$P$};
%% coordinates of the plot area
\gpdefrectangularnode{gp plot 1}{\pgfpoint{0.460cm}{0.368cm}}{\pgfpoint{11.947cm}{8.381cm}}
\end{tikzpicture}
%% gnuplot variables
 
	\end{scaletikzpicturetowidth}
	\vspace{-0.5cm}
	\caption{Determinante und Parallelogramm}\label{lin:05_det_1000}
	\end{center}
	\vspace{-0.6cm}
\end{figure}

\beweis{ 
Wir schreiben die Vektoren in ihrer Polarkoordinatendarstellung
  	$$\vektor{v} = r \cdot \left( \begin{matrix} \cos(\varphi) \\ \sin(\varphi) 
      \end{matrix} \right), \qquad  \vektor{w} = 
     	s \cdot \left( \begin{matrix} \cos(\psi) \\ \sin(\psi) \end{matrix} \right) $$ 
und erhalten damit nach den Additionstheoremen für Sinus und Kosinus (vergleiche Vorkurs \textit{Trigonometrie})
  	$$ \begin{array} {l c l}
   	\det{A} & = & r \cdot \cos(\varphi) \cdot s \cdot \sin(\psi) -
	s \cdot \cos(\psi) \cdot r \cdot \sin(\varphi) \\
   	& = & r \cdot s \cdot \sin(\psi - \varphi)
   	\end{array} $$
Dabei ist entweder $\psi - \varphi$ oder $-\left(\psi - \varphi \right)$ der Winkel zwischen 
$\vektor{v}$ und $\vektor{w}$. Damit ist $s \cdot \vert \sin(\psi - \varphi) \vert$ 
die Höhe des Parallelogramms $P$ dessen Grundlinie durch den Vektor $\vektor{v}$ 
und dessen Seitenlinie durch den Vektor $\vektor{w}$ gegeben ist, 
wie sofort aus der Definition des Sinus folgt, 
und dementsprechend ist $r \cdot s \cdot \vert \sin(\psi - \varphi) \vert = \vert \det{A} \vert$ seine 
Fläche.
}

\bigbreak

\begin{korollar} In der Situation von Satz~\ref{det_2_geom} gilt:
  	$$ \det{A}^2 + \langle \vektor{v}, \vektor{w} \rangle^2 = 
     \vert \vektor{v} \vert^2 \cdot \vert \vektor{w} \vert^2 $$
\end{korollar}

\beweis{ Das kann man direkt nachrechnen. Wegen Satz~\ref{det_2_geom} gilt aber auch 
   	$$ \vert \det{A} \vert = \vert \vektor{v} \vert \cdot \vert \vektor{w} \vert
      \cdot \sin(\alpha) $$
wobei $\alpha \in [0, \pi]$ den Winkel zwischen $\vektor{v}$ und $\vektor{w}$ bezeichnet.
Damit folgt % erhalten wir aus Satz~\ref{2d_vec_angle}
  	$$ \begin{array} {l c l}
     	\det{A}^2 + \langle \vektor{v}, \vektor{w} \rangle^2 & = &
	\vert \vektor{v} \vert^2 \cdot \vert \vektor{w} \vert^2 \cdot \sin^2(\alpha)
	+ \vert \vektor{v} \vert^2 \cdot \vert \vektor{w} \vert^2 \cdot \cos^2(\alpha) \\
    	& = & \vert \vektor{v} \vert^2 \cdot \vert \vektor{w} \vert^2 \left(
	\cos^2(\alpha) + \sin^2(\alpha) \right) \\
    	& = & \vert \vektor{v} \vert^2 \cdot \vert \vektor{w} \vert^2
  	\end{array} $$
wie gewünscht.
}

\bigbreak

\begin{aufgabe} Berechnen Sie $\det{A}$ für $A = \left( \begin{smallmatrix} 2 & 2 \\ 
-2 & 2 \end{smallmatrix} \right)$. 
\end{aufgabe}

\begin{aufgabe} Berechnen Sie $\det{A}$ für $A = \left( \begin{smallmatrix} 2 & 2 \\ 
3 & 3 \end{smallmatrix} \right)$. 
\end{aufgabe}


\begin{aufgabe} Berechnen Sie $\det{A}$ für $A = \left( \begin{smallmatrix} 1 & 3 \\ 
2 & 2 \end{smallmatrix} \right)$. Benutzen Sie die Cramersche Regel, um die Lösung von 
  $$ A \cdot \vektor{x} =  \left( \begin{smallmatrix} 4 \\ 
     6 \end{smallmatrix} \right) $$
zu bestimmen. 
\end{aufgabe}

\begin{aufgabe} Berechnen Sie $\det{A}$ für $A = \left( \begin{smallmatrix} 2 & 3 \\ 
1 & -1 \end{smallmatrix} \right)$. Hat das Gleichungssystem 
  $$ A \cdot \vektor{x} =  \left( \begin{smallmatrix} 0 \\ 
     5 \end{smallmatrix} \right) $$
eine eindeutige Lösung? 
\end{aufgabe}

\begin{aufgabe} Zeigen Sie: Ist $A = \left( \begin{smallmatrix} a_{1,1} & a_{1,2} \\ a_{2,1} & a_{2,2}
\end{smallmatrix} \right)$ eine $2 \times 2$--Matrix mit $a_{1,1} = 0$ oder $a_{2,2} = 0$, so gilt
  $$ \det{A} = - a_{1,2} a_{2,1} $$
\end{aufgabe}

\begin{aufgabe} Zeigen Sie: Genau dann sind die beiden Zeilen von $A$ kollinear, wenn $\det{A} = 0$. 
\end{aufgabe}

\bigbreak

\bigbreak


\subsection{Dreidimensionale Determinanten}\label{section_det_3}

\setcounter{definition}{0}
\setcounter{beispiel}{0}
\setcounter{notiz}{0}

Der Fall von Gleichungssystemen mit zwei Unbekannten und zwei Gleichungen oder - äquivalent - der Fall 
von $2 \times 2$--Matrizen war also noch recht einfach, und wir wollen uns jetzt dem dreidimensionalen 
Fall zuwenden.

\medbreak

Ausgangspunkt unserer Überlegungen ist wieder ein Gleichungssystem, diesmal in drei Unbekannten und 
mit drei Gleichungen:
  	\begin{equation}\label{lin_gls_3_gen} 
  	\begin{array} {l c l c l c l}
  	a_{1,1} x_1 & + & a_{1,2} x_2 & + & a_{1,3} x_3 & = & b_1 \\
  	a_{2,1} x_1 & + & a_{2,2} x_2 & + & a_{2,3} x_3 & = & b_2 \\
  	a_{3,1} x_1 & + & a_{3,2} x_2 & + & a_{3,3} x_3 & = & b_3 \\
  	\end{array} 
  	\end{equation}
dessen Koeffizientenmatrix wir wie üblich mit $A$ bezeichnen. 

Unser Ziel ist es wieder, dem Gleichungssystem (~\ref{lin_gls_3_gen}) bzw. seiner Koeffizientenmatrix $A$ 
eine Zahl $\det{A}$ zuzuordnen, die genau dann von Null verschieden ist, wenn das Gleichungssystem für 
jede Wahl von $b_1, b_2$ und $b_3$ eine eindeutige Lösung hat. Darüberhinaus sollen auch die Regeln, 
die wir für zweidimensionale Determinanten abgeleitet haben, in analoger Weise im dreidimensionalen Fall 
gelten. 

Für die Berechnung der Determinante einer $2 \times 2$--Matrix haben die Diagonale und die Gegendiagonale 
eine besondere Rolle gespielt. Betrachten wir im dreidimensionalen die speziellen Gleichungssysteme
  	\begin{equation}\label{lin_gls_3_main_diag} 
  	\begin{array} {l c l c l c l}
  	a_{1,1} x_1 &   &   &   &   & = & b_1 \\
    	&   & a_{2,2} x_2 &  &   & = & b_2 \\
    	&   &   &   & a_{3,3} x_3 & = & b_3 \\
  	\end{array} 
  	\end{equation}
mit $a_{1,1} \cdot a_{2,2} \cdot a_{3,3} \neq 0$ und 
  	\begin{equation}\label{lin_gls_3_gegen_diag} 
  	\begin{array} {l c l c l c l}
    	&   &   &  & a_{1,3} x_3 & = & b_1 \\
    	&   & a_{2,2} x_2 &   &   & = & b_2 \\
	a_{3,1} x_1 &   &  & &  & = & b_3 \\
  	\end{array} 
  	\end{equation}
mit $a_{1,3} \cdot a_{2,2} \cdot a_{3,1} \neq 0$, so haben auch diese stets eine eindeutige Lösung. Damit 
ist zu erwarten, dass auch bei der Bildung der dreidimensionalen Determinante die Produkte der Diagonalelemente, 
$a_{1,1} \cdot a_{2,2} \cdot a_{3,3}$ und der Nebendiagonalelemente $a_{1,3} \cdot a_{2,2} \cdot a_{3,1}$ eine 
besondere Rolle spielen. Eine Bildung der Form
  	$$ \mathrm{''det''}(A) = a_{1,1} \cdot a_{2,2} \cdot a_{3,3} - a_{1,3} \cdot a_{2,2} \cdot a_{3,1} $$
die der zweidimensionalen Determinante recht nahe kommt, greift aber leider zu kurz. 
Sie liefert zwar für die Matrix 
  	$$ A = \left( \begin{matrix} 1 & 0 & 0 \\ 0 & 1 & 0 \\ 0 & 0 & 1 \end{matrix} \right) $$
ein Ergebnis wie gewünscht, nämlich den Wert $1$, aber für Matrizen
  	$$ A' = \left( \begin{matrix} 0 & 1 & 0 \\ 0 & 0 & 1 \\ 1 & 0 & 0 \end{matrix} \right), \quad 
   	A^{\prime \prime} = \left( \begin{matrix} 0 & 0 & 1 \\ 1 & 0 & 0 \\ 0 & 1 & 0 \end{matrix} \right) 
  	\, \textrm{ oder } \, 
  	A^{\prime \prime \prime} = \left( \begin{matrix} 0 & 1 & 0 \\ 1 & 0 & 0 \\ 0 & 0 & 1 \end{matrix} \right) $$
erhalten wir immer den Wert $0$, obwohl auch hier die zugehörigen Gleichungssysteme eindeutig lösbar sind, 
und obwohl diese Matrizen aus $A$ durch Vertauschung von Zeilen oder Spalten hervorgegangen sind und wir daher 
aufgrund der Regel~\ref{det_rule_2_3} im Dreidimensionalen den Wert $1$ oder $-1$ erwarten. Diese Beispiele 
zeigen bereits, dass die Definition der  dreidimensionalen Determinante nicht nur die Diagonale und die 
Gegendiagonale berücksichtigen muss, sondern auch die Nebendiagonalen und die Nebengegendiagonalen der Matrix. 
Es bleibt zu klären, wie diese anzuordnen sind, und ob noch weitere Einflussfaktoren zu berücksichtigen 
sind. Das ist nicht unmittelbar klar. Darum wollen wir uns noch einen anderen Zugang zuwenden. 

Sind $\vektor{v}$ und $\vektor{w}$ zwei nicht kollineare ebene Vektoren, und ist $A$ die 
$2 \times 2$--Matrix, die diese Vektoren als Spalten hat, so haben wir in 
Satz~\ref{det_2_geom} gesehen, dass $\vert \det{A} \vert$ der Flächeninhalt des von 
$\vektor{v}$ und $\vektor{w}$ aufgespannten Parallelograms ist. Daher ist es naheliegend, 
im Dreidimensionalen das folgende zu verlangen: Sind $\vektor{u}, \vektor{v}$ und 
$\vektor{w}$ drei nicht komplanare räumliche Vektoren, und ist $A$ die $3 \times 3$--Matrix, die 
diese Vektoren als Spalten hat, so ist $\vert \det{A} \vert$ der Inhalt des von $\vektor{u}, 
\vektor{v}$ und $\vektor{w}$ aufgespannten Parallelotops. Erfreulicherweise haben 
Sie im Vorkurs \textit{Lineare Algebra} % wir in Satz~\ref{spat_inhalt_quader} 
schon eine Größe kennengelernt, die diese Bedingung erfüllt, nämlich 
das Spatprodukt $[\vektor{u}, \vektor{v}, \vektor{w} ]$ der Vektoren 
$\vektor{u}, \vektor{v}$ und $\vektor{w}$. Wollen wir also diese geometrische 
Interpretation auch im Dreidimensionalen, so ergibt sich daraus: 

Sind die drei Spaltenvektoren $\vektor{u}, \vektor{v}$ und $\vektor{w}$ der Matrix 
$A$ linear unabhängig, so muss gelten
  	$$ \det{A} = \pm [\vektor{u}, \vektor{v}, \vektor{w}] $$
Wenn wir auch noch die Normierungseigenschaft der Regel~\ref{det_norm_2} im Dreidimensionalen, also 
$\det{E_3} = 1$, so erhalten wir sogar
  	$$ \det{A} = [\vektor{u}, \vektor{v}, \vektor{w}] $$
und das ist tatsächlich der richtige Ansatz, der zu folgender Formel führt:

\begin{definition} Die \index{Determinante!dreidimensional}\textbf{Determinante} $\det{A}$ eine 
$3 \times 3$--Matrix 
  	$$A = \left( \begin{matrix} a_{1,1} & a_{1,2} & a_{1,3} \\ 
	a_{2,1} & a_{2,2} & a_{2,3} \\ a_{3,1} & a_{3,2} & a_{3,3}
\end{matrix} \right) $$ 
ist der Ausdruck
  	$$ \begin{array}{l c l }
    	 \det{A} & = & a_{1,1} a_{2,2} a_{3,3} + a_{1,2} a_{2,3} a_{3,1} + a_{1,3} a_{2,1} a_{3,2} \\
     	& &  - a_{1,3} a_{2,2} a_{3,1} - a_{1,2} a_{2,1} a_{3,3} - a_{1,1} a_{2,3} a_{3,2} 
  	\end{array} $$
\end{definition}

%\begin{notiz} Das es sich bei dieser Definition tatsächlich um das Spatprodukt der Spaltenvektoren von $A$ 
%handelt, ergibt sich aus der Koordinatenbeschreibung des Spatprodukts in Regel~\ref{spat_koordinaten}.
%\end{notiz}

\begin{notiz} Wie wir schon in unseren Vorüberlegungen festgestellt haben, spielen alle Diagonalen und alle
Nebendiagonalen bei der Bildung der Determinante eine Rolle. 
\end{notiz}

\begin{beispiel}
	$$ \begin{array} {l c l}
  	\det{ \begin{matrix} 1 & 2 & 0 \\ 2 & 3 & 4 \\ 1 & 0 & 3 \end{matrix} } & = &
   	1 \cdot 3 \cdot 3 + 2 \cdot 4 \cdot 1 + 0 \cdot 2 \cdot 0  - 0 \cdot 3 \cdot 1 - 2 \cdot 2 \cdot 3 
  	- 1 \cdot 4 \cdot 0 \\
  	& = & 5
   	\end{array} $$
\end{beispiel}

\begin{beispiel}
	$$ \begin{array} {l c l}
  	\det{ \begin{matrix} 1 & 2 & 3 \\ 0 & 3 & 4 \\ 0 & 0 & 3 \end{matrix} } & = &
   	1 \cdot 3 \cdot 3 + 2 \cdot 4 \cdot 0 + 3 \cdot 0 \cdot 0  - 3 \cdot 3 \cdot 0 - 2 \cdot 0 \cdot 3 
  	- 1 \cdot 4 \cdot 0 \\
  	& = & 9
   	\end{array} $$
\end{beispiel}

Die Determinante lässt sich ganz allgemein nach einem geometrischen Schema, der 
\index{Determinante!Sarrus}\textbf{Regel von Sarrus} ermitteln

\begin{regel} Zu einer $3 \times 3$--matrix $A = \left( a_{i,j} \right)$ betrachte das Schema
$$ \setlength{\unitlength}{0.4cm}
%\left\vert 
\begin{picture}(22,16)(-4,-4)
%  \put(-1,0){\line(0,1){13}}
  \put(0,0){$a_{3,1}$}
  \put(1,1){\line(1,1){2}}
  \put(0,4){$a_{2,1}$}
  \put(9,9){\line(1,1){1}}
  \put(0,8){$a_{1,1}$}
  \put(4,0){$a_{3,2}$}
  \put(5,1){\line(1,1){2}}
  \put(4,4){$a_{2,2}$}
  \put(5,5){\line(1,1){2}}
  \put(4,8){$a_{1,2}$}
  \put(8,0){$a_{3,3}$}
  \put(9,1){\line(1,1){2}}
  \put(8,4){$a_{2,3}$}
  \put(9,5){\line(1,1){2}}
  \put(8,8){$a_{1,3}$}
  \put(12,0){$a_{3,1}$}
  \put(13,9){\line(1,1){1}}
  \put(12,4){$a_{2,1}$}
  \put(17,9){\line(1,1){1}}
  \put(12,8){$a_{1,1}$}
  \put(16,0){$a_{3,2}$}
%  \put(17,1){\line(1,1){2}}
  \put(16,4){$a_{2,2}$}
%  \put(17,5){\line(1,1){2}}
  \put(16,8){$a_{1,2}$}
  \put(11,11){$-$}
  \put(15,11){$-$}
  \put(19,11){$-$}
  \put(1,7){\line(1,-1){0.9}}
  \put(2,6){\line(1,-1){0.9}}
  \put(5,3){\line(1,-1){0.9}}
  \put(6,2){\line(1,-1){0.9}}
  \put(9,-1){\line(1,-1){0.9}}
%  \put(10,-2){\line(1,-1){0.9}}
  \put(5,7){\line(1,-1){0.9}}
  \put(6,6){\line(1,-1){0.9}}
  \put(9,3){\line(1,-1){0.9}}
  \put(10,2){\line(1,-1){0.9}}
  \put(13,-1){\line(1,-1){0.9}}
%  \put(14,-2){\line(1,-1){0.9}}
  \put(9,7){\line(1,-1){0.9}}
  \put(10,6){\line(1,-1){0.9}}
  \put(13,3){\line(1,-1){0.9}}
  \put(14,2){\line(1,-1){0.9}}
  \put(17,-1){\line(1,-1){0.9}}
%  \put(18,-2){\line(1,-1){0.9}}
  \put(11,-3){$+$}
  \put(15,-3){$+$}
  \put(19,-3){$+$}
%  \put(21,0){\line(0,1){13}} 
\end{picture} 
%\right\vert 
\setlength{\unitlength}{1pt}  $$
das man erhält indem man die Spalten von $A$ hinschreibt und die ersten beiden Spalten von $A$ nocheinmal 
anfügt. Die Determinante von $A$ berechnet sich dadurch, dass man jeweils die Elemente entlang der drei 
Diagonalen dieses $3 \times 5$--Schemas miteinander multipliziert und aufaddiert und jeweils die Elemente 
entlang der drei Gegendiagonalen miteinander multipliziert und davon abzieht.
\end{regel}

\begin{beispiel}
Wir betrachten die Matrix 
  $$ A = \left( \begin{matrix} 1 & 2 & 3 \\ 2 & 3 & 4 \\ 3 & 4 & 5 \end{matrix} \right) $$
Dazu gehört das Sarrus--Schema
$$ \setlength{\unitlength}{0.4cm}
%\left\vert 
\begin{picture}(22,16)(-4,-4)
%  \put(-1,0){\line(0,1){13}}
  \put(0,0){$3$}
  \put(1,1){\line(1,1){2}}
  \put(0,4){$2$}
  \put(9,9){\line(1,1){1}}
  \put(0,8){$1$}
  \put(4,0){$4$}
  \put(5,1){\line(1,1){2}}
  \put(4,4){$3$}
  \put(5,5){\line(1,1){2}}
  \put(4,8){$2$}
  \put(8,0){$5$}
  \put(9,1){\line(1,1){2}}
  \put(8,4){$4$}
  \put(9,5){\line(1,1){2}}
  \put(8,8){$3$}
  \put(12,0){$3$}
  \put(13,9){\line(1,1){1}}
  \put(12,4){$2$}
  \put(17,9){\line(1,1){1}}
  \put(12,8){$1$}
  \put(16,0){$4$}
%  \put(17,1){\line(1,1){2}}
  \put(16,4){$3$}
%  \put(17,5){\line(1,1){2}}
  \put(16,8){$2$}
  \put(11,11){$-$}
  \put(15,11){$-$}
  \put(19,11){$-$}
  \put(1,7){\line(1,-1){0.9}}
  \put(2,6){\line(1,-1){0.9}}
  \put(5,3){\line(1,-1){0.9}}
  \put(6,2){\line(1,-1){0.9}}
  \put(9,-1){\line(1,-1){0.9}}
%  \put(10,-2){\line(1,-1){0.9}}
  \put(5,7){\line(1,-1){0.9}}
  \put(6,6){\line(1,-1){0.9}}
  \put(9,3){\line(1,-1){0.9}}
  \put(10,2){\line(1,-1){0.9}}
  \put(13,-1){\line(1,-1){0.9}}
%  \put(14,-2){\line(1,-1){0.9}}
  \put(9,7){\line(1,-1){0.9}}
  \put(10,6){\line(1,-1){0.9}}
  \put(13,3){\line(1,-1){0.9}}
  \put(14,2){\line(1,-1){0.9}}
  \put(17,-1){\line(1,-1){0.9}}
%  \put(18,-2){\line(1,-1){0.9}}
  \put(11,-3){$+$}
  \put(15,-3){$+$}
  \put(19,-3){$+$}
%  \put(21,0){\line(0,1){13}} 
\end{picture} 
%\right\vert 
\setlength{\unitlength}{1pt}  $$
und damit die Determinante
  	$$ 1 \cdot 3 \cdot 5 + 2 \cdot 4 \cdot 3 + 3 \cdot 2 \cdot 4 - 3 \cdot 3 \cdot 3 - 1 \cdot 4 \cdot 4
  	- 2 \cdot 2 \cdot 5 = 0 $$
\end{beispiel}

\medbreak 

Die Berechnung der Determinante im Dreidimensionalen ist also wesentlich komplizierter als im 
Zweidimensionalen. Daher ist es in diesem Fall besonders wichtig, Formeln und Regeln zu finden, die 
die Berechnungen vereinfachen. Eine Beziehung, die folgende \textbf{Entwicklungsformel}, ergibt sich 
unmittelbar aus der Definition des Spatprodukts, die ja der Determinante zugrunde liegt:

\begin{satz}[Entwicklungssatz von Laplace (Spezialfall)]\label{det_laplace_3_spat} Ist $A = \left(a_{i,j}\right)$ eine 
$3 \times 3$--Matrix, so gilt
  	$$ \begin{array} {l c l c l }
  	\det{A} & = & a_{1,1} \cdot \left( a_{2,2} a_{3,3} - a_{2,3} a_{3,2} \right) & - & 
    	a_{2,1} \cdot \left( a_{1,2} a_{3,3} - a_{3,2} a_{1,3} \right) \\
  	& &   & + &
  	a_{3,1} \cdot \left( a_{1,2} a_{2,3} - a_{2,2} a_{1,3} \right) \\
  	& = & a_{1,1} \cdot \det{ \begin{matrix} a_{2,2} & a_{2,3} \\ a_{3,2} & a_{3,3} \end{matrix}}  & - &
  	a_{2,1} \cdot \det{ \begin{matrix} a_{1,2} & a_{1,3} \\ a_{3,2} & a_{3,3} \end{matrix}}  \\
 	& &  & + &
   	a_{3,1} \cdot \det{ \begin{matrix} a_{1,2} & a_{1,3} \\ a_{2,2} & a_{2,3} \end{matrix}}
  	\end{array} $$
\end{satz}

Die Berechnung der dreidimensionalen Determinante lässt sich also auf die Berechnung von zweidimensionalen 
Determinanten zurückführen. Diese Regel gilt sogar noch etwas allgemeiner. Dazu sei wieder
  	$$ A = \left( \begin{matrix} a_{1,1} & a_{1,2} & a_{1,3} \\ a_{2,1} & a_{2,2} & a_{2,3} \\
    	a_{3,1} & a_{3,3} & a_{3,3} \end{matrix} \right) $$
eine beliebige $3 \times 3$--Matrix.

\begin{definition} Für $i, j \in \{1, 2, 3\}$ bezeichnen wir mit $A_{i,j}$ die 
\index{Matrix!Streichungsmatrix}$i,j$--Strei\-chungs\-matrix, 
also die $2 \times 2$--Matrix, die wir aus $A$ durch Streichen der $i$--ten Zeile und der $j$--ten Spalte 
erhalten.
\end{definition}

\begin{beispiel}\label{det_3_bsp_str_1} Für
  	$$ A = \left( \begin{matrix} 1 & 2 & 3 \\ 4 & 5 & 6 \\ 7 & 8 & 9 \end{matrix} \right) $$
gilt etwa
  	$$ \begin{array} {c c c }
     	A_{1,1} = \left( \begin{matrix} 5 & 6 \\ 8 & 9 \end{matrix} \right) \quad &
     	A_{2,1} = \left( \begin{matrix} 2 & 3 \\ 8 & 9 \end{matrix} \right) \quad &
     	A_{2,2} = \left( \begin{matrix} 1 & 3 \\ 7 & 9 \end{matrix} \right) \\
   	A_{2,3} = \left( \begin{matrix} 1 & 2 \\ 7 & 8 \end{matrix} \right) \quad &
 	A_{3,1} = \left( \begin{matrix} 2 & 3 \\ 5 & 6 \end{matrix} \right) \quad &
	A_{3,3} = \left( \begin{matrix} 1 & 2 \\ 4 & 5 \end{matrix} \right) 
	\end{array} $$
\end{beispiel}

\begin{beispiel}\label{det_3_bsp_str_2} Für eine allgemeine Matrix
  	$$ A = \left( \begin{matrix} a_{1,1} & a_{1,2} & a_{1,3} \\ a_{2,1} & a_{2,2} & a_{2,3} \\
	a_{3,1} & a_{3,3} & a_{3,3} \end{matrix} \right) $$
gilt etwa
  	$$ A_{3,1} = \left( \begin{matrix} a_{1,2} & a_{1,2}\\ a_{2,2} & a_{2,3} \end{matrix} \right) \quad 
   	A_{2,2} = \left( \begin{matrix} a_{1,1} & a_{1,3} \\ a_{3,1} & a_{3,3} \end{matrix} \right) \quad
  	A_{2,3} = \left( \begin{matrix} a_{1,1} & a_{1,2} \\ a_{3,1} & a_{3,2} \end{matrix} \right) $$
\end{beispiel}

\medbreak

Diese Teilmatrizen können nun wie folgt in der Berechnung der 
\index{Determinante!Laplace Entwicklung}Determinanten verwendet werden:

\begin{satz}[Entwicklungssatz von Laplace]\label{det_laplace_3} \index{Satz!Laplace--Entwicklung}
Für eine allgemeine $3 \times 3$--Matrix
  	$$ A = \left( \begin{matrix} a_{1,1} & a_{1,2} & a_{1,3} \\ a_{2,1} & a_{2,2} & a_{2,3} \\
	a_{3,1} & a_{3,2} & a_{3,3} \end{matrix} \right) $$
kann $\det{A}$ nach den folgenden Regeln berechnet werden:

\begin{enumerate}
\item \textbf{Entwicklung nach der $i$--ten Zeile}:

Für jedes $i \in \{1, 2,3 \}$ gilt
  	$$ \det{A} = (-1)^{i+1} a_{i,1}  \det{A_{i,1}} + (-1)^{i+2} a_{i,2}  \det{A_{i,2}} + 
	(-1)^{i+3} a_{i,3}  \det{A_{i,3}}  $$

\item \textbf{Entwicklung nach der $j$--ten Spalte}

Für jedes $j \in \{1, 2, 3 \}$ gilt
  	$$ \det{A} = (-1)^{1+j} a_{1,j} \det{A_{1,j}} + (-1)^{2+j} a_{2,j} \det{A_{2,j}} + 
	(-1)^{3+j} a_{3,j} \det{A_{3,j}}  $$
\end{enumerate}
\end{satz}

\begin{notiz} Die Formel aus Satz~\ref{det_laplace_3_spat} ist der Spezialfall der Entwicklung nach 
der ersten Spalte.

Zu beachten ist, dass die Unterdeterminanten nicht einfach nur aufaddiert werden, vielmehr sind dabei 
Vorzeichen zu berücksichtigen.
\end{notiz}

\beweis Der direkte Nachweis aller Formeln ist mühsam und rechenintensiv aber wenig herausfordernd und 
bleibt dem Leser überlassen.

\medbreak

\begin{beispiel} Wir betrachten wieder die Matrix
  	$$ A = \left( \begin{matrix} 1 & 2 & 3 \\ 4 & 5 & 6 \\ 7 & 8 & 9 \end{matrix} \right) $$
Hierfür gilt
  	$$ A_{2,1} = \left( \begin{matrix} 2 & 3 \\ 8 & 9 \end{matrix} \right) \quad 
   	A_{2,2} = \left( \begin{matrix} 1 & 3 \\ 7 & 9 \end{matrix} \right) \quad
   	A_{2,3} = \left( \begin{matrix} 1 & 2 \\ 7 & 8 \end{matrix} \right) $$
wie wir schon in Beispiel~\ref{det_3_bsp_str_1} gesehen haben. Entwickeln wir die Determinante also 
nach der zweiten Spalte, so erhalten wir
  	$$ \begin{array} {l c l} 
  	\det{A} & = & 2 \cdot (-1)^{2+1} \det{\left( \begin{matrix} 2 & 3 \\ 8 & 9 \end{matrix} \right)}
	+ 5 \cdot (-1)^{2+2} \det{\left( \begin{matrix} 1 & 3 \\ 7 & 9 \end{matrix} \right)} \\
  	& & \qquad \qquad + 8 \cdot (-1)^{2+3} \det{\left( \begin{matrix} 1 & 2 \\ 7 & 8 \end{matrix} \right)} \\
  	& = & - 2 \cdot (-6) + 5 \cdot (-12) - 8 \cdot (-6) \\
  	& = & 0
 	\end{array} $$
\end{beispiel}

\bigbreak

Trotz der Entwicklungsregel bleibt das Berechnen von Determinanten mühsam und vor allem fehleranfällig. 
Daher werden weitere Regeln benötigt, die Vereinfachungen erlauben. Es gilt nämlich wie im zweidimensionalen 
Fall:

\begin{regel}\label{det_3_rule} Es sei $A$ eine $3 \times 3$--Matrix. 

\begin{enumerate}
\item Für die $3 \times 3$--Einheitsmatrix
$E_3 = \left( \begin{smallmatrix} 1 & 0 & 0 \\ 0 & 1 & 0 \\ 0 & 0 & 1 \end{smallmatrix} \right)$ gilt
  	$$ \det{E_3} = 1 $$
\item $\det{A^T} = \det{A}$.
\item Es gilt schon $\det{A} = 0$, wenn 

\begin{itemize}
\item Eine Zeile von $A$ die Nullzeile ist.
\item Eine Spalte von $A$ die Nullspalte ist.
\item Eine Zeile von $A$ ein Vielfaches einer anderen Zeile ist.
\item Eine Spalte von $A$ ein Vielfaches einer anderen Spalte ist. 
\end{itemize}
\item  Ist $A$ eine obere oder untere Dreiecksmatrix (d.h. gilt $a_{1,2} = a_{1,3} = a_{2,3} = 0$ 
oder $a_{2,1} = a_{3,1} = a_{3,2} = 0$), so gilt
  	$$ \det{A} = a_{1,1} \cdot a_{2,2} \cdot a_{3,3} $$
\item Entsteht $A'$ aus $A$ durch Vertauschen von zwei Zeilen, so gilt
  	$$ \det{A'} = - \det{A} $$
Das Gleiche gilt, wenn $A'$ durch Vertauschung von zwei Spalten aus $A$ entsteht.
\item Entsteht $A'$ aus $A$ durch Multiplikation einer Zeile von $A$ mit einer Zahl $r$, so gilt
  	$$ \det{A'} = r \cdot \det{A} $$
Das Gleiche gilt, wenn $A'$ aus $A$ durch Multiplikation einer Spalte von $A$ mit 
einer Zahl $r$ hervorgeht.
\item Entsteht $A'$ aus $A$ dadurch, dass wir ein 
Vielfaches einer Zeile von $A$ zu einer anderen Zeile von $A$ addieren, so gilt
  	$$ \det{A'} =  \det{A} $$
Das Gleiche gilt, wenn $A'$ aus $A$ dadurch entsteht, dass wir ein Vielfaches einer Spalte von 
$A$ zu einer anderen Spalte von $A$ addieren.
\item Ist $B$ eine weitere $3 \times 3$--Matrix, so gilt
  	$$ \det{ A \cdot B } = \det{A} \cdot \det{B} $$
\end{enumerate}
\end{regel}

\beweis{ 
Der Nachweis dieser Eigenschaften ist ähnlich, wenn auch aufwendiger als im zweidimensionalen Fall.
}

\bigbreak

\begin{notiz} Die Regeln erlauben es immer, die Berechnung der Determinante einer Matrix auf die Berechnung der 
Determinante einer Dreiecksmatrix zurückzuführen. Dieser Ansatz ist in der Praxis in der Tat der einfachste 
und praktibelste Weg zur Berechnung von Determinanten.
\end{notiz}

\begin{beispiel} Wir betrachten die Matrix 
  	$$ A = \left( \begin{matrix} 2 & 3 & 4 \\  4 & 6 & 9 \\ 2 & 4 & 5 \end{matrix} \right) $$

\begin{itemize}
\item[i)] Wir subtrahieren das Doppelte der erste Zeile von der zweiten Zeile und die erste Zeile von der 
dritten Zeile und erhalten
  	$$ A' = \left( \begin{matrix} 2 & 3 & 4 \\  0 & 0 & 1 \\ 0 & -2 & -3 \end{matrix} \right) $$
\item[ii)] Wir vertauschen die Zeilen 2 und 3 der Matrix und beachten dabei, dass sich dadurch das Vorzeichen 
der Determinante ändert. Wir erhalten
  	$$ A^{\prime \prime} = \left( \begin{matrix} 2 & 3 & 4 \\ 0 & -2 & -3 \\  0 & 0 & 1 \end{matrix} \right) $$
\item[iii)] $A^{\prime \prime}$ ist eine Dreiecksmatrix, hat also Determinante
  	$$ \det{A^{\prime \prime}} = 2 \cdot (-2) \cdot 1 = -4 $$
\item[iv)] Die Operation in Schritt ii) hat zu einem Vorzeichenwechsel der Determinante geführt. Wir machen 
diesen rückgängig und erhalten
  	$$ \det{A} = - \det{A^{\prime \prime}} =  4 $$
\end{itemize}
\end{beispiel}

\bigbreak

Unser ursprüngliches Ziel war es, eine Größe zu finden, die uns erlaubt zu entscheiden, ob ein 
lineares Gleichungssystem 
  	$$ A \cdot \vektor{x} = \vektor{b} $$
für alle Wahlen von $\vektor{b}$ eine eindeutige Lösung hat oder nicht. Wir wollen nun 
überprüfen, ob wir dieses Ziel mit dieser Definition der Determinante erreicht haben. Wir betrachten 
dazu die augmentierte Matrix $\left( A \,\, \vert \, \vektor{b} \right)$ und führen diese in Normalform 
$\left( A' \,\, \vert \, \vektor{b'} \right)$ über. Hierzu benötigen wir die Operationen

\begin{itemize}
\item[-] Addition des Vielfachen einer Zeile von $\left( A \,\, \vert \, \vektor{b} \right)$ zu einer anderen Zeile.
\item[-] Multiplikation einer Zeile von $\left( A \,\, \vert \, \vektor{b} \right)$ mit einer Zahl $r \neq 0$.
\item[-] Vertauschung von zwei Zeilen von $\left( A \,\, \vert \, \vektor{b} \right)$.
\item[-] Vertauschung von zwei Spalten von $\left( A \,\, \vert \, \vektor{b} \right)$, von denen aber keine die 
letzte sein darf.
\end{itemize}

Bei all diesen Operationen wissen wir jetzt, wie sich die Determinante ändert, und wir erhalten aus 
Regel~\ref{det_3_rule}:

\begin{notiz} Genau dann ist $\det{A} \neq 0$ wenn $\det{A'} \neq 0$.
\end{notiz}

Wir können uns daher bei unseren Überlegungen auf Gleichungssysteme $ A \cdot \vektor{x} = 
\vektor{b}$ in Normalform beschränken. In diesem Fall ist aber $A$ eine Dreieckmatrix und hat eine 
der folgenden Gestalten
  	$$ \left( \begin{matrix} 0 & 0 & 0 \\ 0 & 0 & 0 \\ 0 & 0 & 0 \end{matrix} \right), \qquad 
   	\left( \begin{matrix} 1 & $*$ & $*$ \\ 0 & 0 & 0 \\ 0 & 0 & 0 \end{matrix} \right), \qquad
  	\left( \begin{matrix} 1 & $*$ & $*$ \\ 0 & 1 & $*$ \\ 0 & 0 & 0 \end{matrix} \right), \qquad
  	\left( \begin{matrix} 1 & $*$ & $*$ \\ 0 & 1 & $*$ \\ 0 & 0 & 1 \end{matrix} \right) $$
In den ersten drei Fällen gilt $\det{A} = 0$, nur im vierten Fall  haben wir 
  	$$ \det{A} = 1 \neq 0 $$
Dieser Fall ist genau die Situation, in der $\mathrm{rg}(A) = 3$, und in der das Gleichungssystem
  $$ A \cdot \vektor{x} = \vektor{b} $$
eine eindeutige Lösung hat. Damit kann also die eindeutige Lösbarkeit tatsächlich mit Hilfe der 
Determinante entschieden werden und es gilt

\begin{satz}\label{det_3_gls_unique}
Ist $A$ eine $3 \times 3$--Matrix, so hat das Gleichungssystem 
  	$$ A \cdot \vektor{x} = \vektor{b} $$
genau dann eine eindeutige Lösung für jeden Vektor $\vektor{b}$, wenn 
  $$ \det{A} \neq 0 $$
\end{satz}

\bigbreak 

\begin{aufgabe} Berechnen Sie die Determinante der folgenden Matrizen
  	$$ A = \left( \begin{matrix} 1 & 2 & 5 \\ 3 & 1 & 4 \\ 1 & 4 & 1 \end{matrix} \right), \quad 
     	B = \left( \begin{matrix} 0 & 0 & 5 \\ 2 & 3 & 1 \\ 4 & 0 & 2 \end{matrix} \right), \quad
     	C = \left( \begin{matrix} 6 & 1 & 3 \\ 3 & 0 & 1 \\ 2 & 3 & -1 \end{matrix} \right) $$
\end{aufgabe}

\begin{aufgabe} Bestimmen Sie die Streichungsmatrizen $A_{1,1}, A_{1,2}, A_{1,3}$, $A_{2,1}$, $A_{2,2}$ und
$A_{2,3}$ von 
  	$$ A = \left( \begin{matrix} 3 & 0 & 1 \\ 1 & 1 & 1 \\ 2 & -1 & 1 \end{matrix} \right), $$
berechnen Sie die Determinanten dieser Streichungsmatrizen und die Determinante von $A$.
\end{aufgabe}

\begin{aufgabe} Es sei 
  	$$  A = \left( \begin{matrix} 0 & 1 & 5 \\ 4 & 1 & 2 \\ 2 & 3 & 1 \end{matrix} \right). $$
Entscheiden Sie, ob das Gleichungssystem 
  	$$ A \cdot \vektor{x} = \vektor{b} $$
für jede Wahl von $\vektor{b}$ eine eindeutige Lösung besitzt.
\end{aufgabe}

\begin{aufgabe} Es sei $A = \left(a_{i,j}\right)$ eine $3 \times 3$--Matrix mit 
  	$$ a_{1,1} = a_{1,2} = a_{2,1} = 0 $$
Zeigen Sie 
  	$$ \det{A} = - a_{1,3} \cdot a_{2,2} \cdot a_{3,1} $$
\end{aufgabe}

\begin{aufgabe} Zeigen Sie, dass genau dann $\det{A} = 0$ ist, wenn die Spaltenvektoren von $A$ linear 
abhängig sind.
\end{aufgabe}

\bigbreak

\bigbreak

\subsection{Die $n$--dimensionale Determinante}\label{section_det_n}

\setcounter{definition}{0}
\setcounter{beispiel}{0}
\setcounter{notiz}{0}

Analog zum Begriff der Vektoren, den wir uns auch zunächst im zwei--und dreidimensionalen Fall studierten 
ehe wir zur $n$--dimensionalen Sitution übergingen, wollen wir jetzt auch die Determinantentheorie in der 
allgemeinen Situation betrachten. Der Übergang vom Zwei-- zum Dreidimensionalen legt aber schon nahe, dass 
hier die Situation etwas komplizierter sein wird. 

Es gibt viele Methoden, Determinanten zu motivieren und einzuführen. Wir wollen hier einen rekursiven 
Zugang verfolgen, mit dem wir auch schon die Berechnung dreidimensionaler Determinanten auf die Berechnung 
zweidimensionaler Determinanten zurückgeführt haben. 

Sei dazu $A = \left( a_{i,j} \right)$ eine beliebige $n \times n$--Matrix. Für  $l,k \in \{1, \ldots, n \}$ 
bezeichnen wir wieder mit $A_{l,k}$ die $l,k$--Streichungsmatrix von $A$, also die $(n-1) \times (n-1)$--Matrix, 
die aus $A$ durch Streichung der $l$--ten Zeile und der $k$--ten Spalte hervorgeht. Wir definieren rekursiv

\begin{definition}  
Die \index{Determinante!$n$--dimensional}\textbf{Determinante} $\det{A}$ von $A$ ist definiert wie folgt:
\begin{itemize}
\item Ist $n =  1$, also $A = \left( a_{1,1} \right)$, so ist $\det{A} = a_{1,1}$.
\item Ist $n = 2$ oder $n=3$ so ist $\det{A}$ definiert wie in den 
Abschnitten~\ref{section_det_2} und~\ref{section_det_3}.
\item Ist $n > 3$ und die Determinante für $(n-1) \times (n-1)$--Matrizen schon erklärt, so setzen wir
  	$$ \begin{array} {l c l}
  	\det{A} & = &  a_{1,1} \cdot \det{A_{1,1}} - a_{1,2} \cdot \det{A_{1,2}} + \cdots \\
	& &  \quad + (-1)^{n+1} \cdot a_{1,n} \cdot \det{A_{1,n}} \\
	& = & \sum\limits_{l=1}^n (-1)^{1+l} a_{1,l} \cdot \det{A_{1,l}}
  	\end{array} $$
\end{itemize}
\end{definition}

\begin{notiz}\label{determinant_permut} In diese Rekursionsformel können wir weiter einsetzen. So ist etwa
  	$$ \begin{array} {l c l}
	\det{A_{1,1}} & = &  a_{2,2} \cdot \det{\left(A_{1,1}\right)_{1,1}} 
	- a_{2,3} \cdot \det{\left(A_{1,1}\right)_{1,2}} + \cdots  \\
	& & \quad + (-1)^{n} \cdot a_{2,n} \cdot \det{\left(A_{1,1}\right)_{1,n-1}} 
	\end{array} $$
Diesen Prozess können wir fortsetzen bis wir den Fall von Teilmatrizen der Größe $1 \times 1$ erreicht 
haben. Führen wir das tatsächlich durch, so erhalten wir
  	$$ \det{A} = \sum_{\sigma \in S_n} \textrm{sign}(\sigma) a_{1,\sigma(1)} \cdot  a_{2, \sigma(2)} \cdots 
	a_{n \sigma(n)} $$
mit einer Summe über alle Permutationen $\sigma \in S_n$ (vergleiche etwa Beispiel~\ref{gruppe_permutation} 
aus Abschnitt~\ref{section_gruppe}), also einer Summe mit $n!$ vielen Summanden. 

Da $n!\,$ sehr schnell wächst, wenn $n$ groß wird, kann man schon erkennen, dass der Rechenaufwand zur 
Ermittlung der Determinante sehr groß werden kann.
\end{notiz}

\begin{beispiel} Wir betrachten die Matrix
  	$$ A = \left( \begin{matrix} 1 & 2 & -2 & - 1 \\ 0 & 1 & 2 & 0 \\ 1 & 0 & 1 & 0 \\ 1 & -1 & 1 & 0 
	\end{matrix} \right) $$
Die Teilmatrizen, die wir benötigen, sind
  	$$ \begin{array} {l c c}
   	A_{1,1} = \left( \begin{matrix} 1 & 2 & 0 \\ 0 & 1 & 0 \\ -1 & 1 & 0 \end{matrix} \right), \quad &
   	A_{1,2} = \left( \begin{matrix} 0 & 2 & 0 \\ 1 & 1 & 0 \\  1 & 1 & 0 \end{matrix} \right), \quad &
   	A_{1,3} = \left( \begin{matrix} 0 & 1 & 0 \\ 1 & 0 & 0 \\  1 & -1 & 0 \end{matrix} \right), \quad \\
  	A_{1,4} = \left( \begin{matrix} 0 & 1 & 2 \\ 1 & 0 & 1 \\  1 & -1 & 1 \end{matrix} \right) 
  	\end{array} $$
mit 
  $$ \det{A_{1,1}} = 0, \quad \det{A_{1,2}} = 0, \quad \det{A_{1,3}} = 0, \quad \det{A_{1,4}} = -2. $$
Also gilt
  $$ \det{A} = (-1)^{1+4}\cdot (-1) \cdot (-2) = -2 $$
\end{beispiel}

\medbreak 

Aufgrund des mit $n$ sehr schnell wachsenden Rechenaufwands ist es notwendig, Formeln und Regeln zu finden, 
die das Berechnen von Determinanten vereinfachen. 

\begin{lemma}\label{det_lemma1} 
Geht $A'$ aus $A$ durch das Vertauschen von zwei Spalten hervor, so gilt 
  	$$ \det{A'} = -\det{A} $$
\end{lemma}

\beweis{ Wir betrachten dazu die Darstellung  

  	$$ \det{A} = \sum_{\sigma \in S_n} \textrm{sign}(\sigma) a_{1,\sigma(1)} a_{2, \sigma(2)} \cdots 
 	a_{n \sigma(n)} $$
Falls $A'$ aus $A$ durch Vertauschen der Spalten $l, k$ mit $l < k$ entsteht, so gilt also 
  	$$ \det{A'} = \sum_{\sigma \in S_n} \textrm{sign}(\sigma) \cdot a_{1,\sigma(1)} \cdots a_{l, \sigma(k)} \cdots 
	a_{k, \sigma(l)} \cdots a_{n \sigma(n)} $$
Bezeichnet nun $\tau = \langle l \, k \rangle$ die Transposition von $l$ und $k$, so schreibt sich dies als
  	$$ \begin{array} {l c l}
	\det{A'} & = & \sum\limits_{\sigma \in S_n} \textrm{sign}(\sigma) \cdot a_{1,\sigma(1)} \cdots 
	a_{l, \sigma(\tau(l))} 	\cdots  a_{k, \sigma(\tau(k))} \cdots a_{n, \sigma(n)} \\
      & = &  \sum\limits_{\sigma \in S_n} \textrm{sign}(\sigma) \cdot a_{1,\sigma(\tau(1))} \cdots 
	a_{l, \sigma(\tau(l))}  \cdots  a_{k, \sigma(\tau(k))} \cdots a_{n, \sigma(\tau(n))}
  	\end{array} $$
(da ja $\tau$ die Zahlen $\neq l, k$ unberührt lässt). Nun haben wir aber, dass gilt
  	$$ \textrm{sign}(\sigma) = - \textrm{sign}(\sigma \circ \tau) $$ 
und mit $\sigma$ durchläuft auch $\sigma \circ \tau$ alle Permutationen. Daher schreibt sich diese 
Beziehung auch als
  	$$ \begin{array} {l c l}
   	\det{A'} & = & \sum\limits_{\sigma \in S_n} - \textrm{sign}(\sigma \circ \tau) \cdot a_{1,\sigma(\tau(1))} 
  	\cdots a_{l, \sigma(\tau(l))} \cdots  a_{k, \sigma(\tau(k))} \cdots a_{n \sigma(\tau(n))} \\
  	& = & \sum\limits_{\sigma \in S_n} -\textrm{sign}(\sigma) \cdot a_{1,\sigma(1)} \cdots a_{l, \sigma(l)} 
  	\cdots  a_{k, \sigma(k)} \cdots a_{n \sigma(n)} \\
   	& = &  - \det{A}
  	\end{array} $$
wie gewünscht.
}

\medbreak

\begin{lemma}\label{det_lemma2} Ist $A^{T}$ die zu $A$ transponierte Matrix, so gilt 
	$$ \det{A} = \det{A^T}  $$
\end{lemma}

\beweis{ Dazu benutzen wir wieder die Darstellung 
  	$$ \det{A} = \sum_{\sigma \in S_n} \textrm{sign}(\sigma) a_{1,\sigma(1)} a_{2, \sigma(2)} \cdots 
	a_{n \sigma(n)} $$
und erhalten daraus
  	$$ \begin{array} {l c l}
    	\det{A} & = & \sum\limits_{\sigma \in S_n} \textrm{sign}(\sigma) \cdot 
	a_{1,\sigma(1)} \cdot a_{2, \sigma(2)} \cdots  a_{n \sigma(n)} \\
   	& = &  \sum\limits_{\sigma \in S_n} \textrm{sign}(\sigma) \cdot 
   	a_{(\sigma^{-1} \circ{\sigma})(1), \sigma((\sigma^{-1} \circ{\sigma})(1))} \cdot 
  	a_{(\sigma^{-1} \circ{\sigma})(2), \sigma((\sigma^{-1} \circ{\sigma})(2))} 
   	\cdots   \\ & & \phantom{serh viel Platz} \cdots
   	a_{(\sigma^{-1} \circ{\sigma})(n) \sigma((\sigma^{-1} \circ{\sigma})(n))} \\
 	& = & \sum\limits_{\sigma \in S_n} \textrm{sign}(\sigma) \cdot 
   	a_{\sigma^{-1}(1), \sigma(\sigma^{-1}(1))} \cdot 
   	a_{\sigma^{-1}(2), \sigma((\sigma^{-1} (2)} \cdots 
   	a_{\sigma^{-1}(n) \sigma(\sigma^{-1}(n))} \\
   	& = & \sum\limits_{\sigma \in S_n} \textrm{sign}(\sigma) \cdot 
    	a_{\sigma(1), 1} \cdot a_{\sigma(2), 2} \cdots a_{\sigma(n), n} \\
   	& = & \det{A^T} 
   	\end{array} $$
Dabei haben wir beim Übergang von Zeile 2 zu Zeile 3 die Terme im Produkt umsortiert, und beim Übergang 
von Zeile 3 zu Zeile 4 ausgenutzt, dass mit $\sigma$ auch $\sigma^{-1}$ alle Permutationen durchläuft, 
und dass $\textrm{sign}(\sigma) = \textrm{sign}(\sigma^{-1})$, wie wir schon in Beispiel~\ref{gruppe_permutation} 
aus Abschnitt~\ref{section_gruppe} gesehen haben.

Damit ist auch dieses Lemma gezeigt.
}

\bigbreak

\begin{korollar} Entsteht $A'$ aus $A$ durch Vertauschen von zwei Zeilen, so gilt
  	$$ \det{A'} = - \det{A} $$
\end{korollar}

\beweis{ Das folgt jetzt sofort aus den beiden Hilfssätzen~\ref{det_lemma1} und~\ref{det_lemma2}, da das 
Vertauschen von Zeilen von $A$ dem Vertauschen von Spalten von $A^T$ entspricht.
}

\medbreak

Damit sind wir jetzt in der Lage einige Regeln für Determinanten herzuleiten und zusammenzustellen. 

\begin{regel}\label{det_n_rule} Es sei $A$ eine $n \times n$--Matrix.

\begin{enumerate}
\item Für die $n \times n$--Einheitsmatrix $E_n$ gilt
  	$$ \det{E_n} = 1 $$
\item $\det{A^T} = \det{A}$.
\item Es gilt schon $\det{A} = 0$, wenn 

\begin{itemize}
\item Eine Zeile von $A$ die Nullzeile ist.
\item Eine Spalte von $A$ die Nullspalte ist.
\item Eine Zeile von $A$ ein Vielfaches der anderen Zeile ist.
\item Eine Spalte von $A$ ein Vielfaches der anderen Spalte ist. 
\end{itemize}
\item  Ist $A$ eine obere oder untere Dreiecksmatrix, so gilt
  	$$ \det{A} = a_{1,1} \cdot a_{2,2} \cdots a_{n,n} $$
\item Entsteht $A'$ aus $A$ durch Vertauschen von zwei Zeilen, so gilt
  	$$ \det{A'} = - \det{A} $$
Das Gleiche gilt, wenn $A'$ durch Vertauschung von zwei Spalten aus $A$ entsteht.
\item Entsteht $A'$ aus $A$ durch Multiplikation einer Zeile von $A$ mit einer Zahl $r$, so gilt
  	$$ \det{A'} = r \cdot \det{A} $$
Das Gleiche gilt, wenn $A'$ aus $A$ durch Multiplikation einer Spalte von $A$ mit 
einer Zahl $r$ hervorgeht.
\item Entsteht $A'$ aus $A$ dadurch, dass wir ein 
Vielfaches einer Zeile von $A$ zu einer anderen Zeile von $A$ addieren, so gilt
  	$$ \det{A'} =  \det{A} $$
Das Gleiche gilt, wenn $A'$ aus $A$ dadurch entsteht, dass wir ein Vielfaches 
einer Spalte von $A$ zu einer anderen Spalte von $A$ addieren.
\item Schreiben wir $A$ als $\left(\vektor{a_1} \, \ldots \, \vektor{a_n}
\right)$ mit Spaltenvektoren $\vektor{a_1}, \ldots , \vektor{a_n}$, so 
gilt
  	$$ \begin{array} {l c l}
   	\det{\vektor{a_1} \, \ldots \, \vektor{a_i} + \vektor{a'_i} 
  	\, \ldots \, \vektor{a_n}} &  = &  
  	\det{\vektor{a_1} \, \ldots \, \vektor{a_i} \, \ldots \, 
   	\vektor{a_n}} \\
   	& & \quad + \det{\vektor{a_1} \, \ldots \, \vektor{a'_i} 
  	\, \ldots \, \vektor{a_n}} 
  	\end{array} $$
Die analoge Aussage gilt für die Zeilen von $A$.
\end{enumerate}
\end{regel}

\beweis{ Einige der Regeln wurden ja schon in den vorangegangenen Lemmata bewiesen. 

Wir zeigen hier beispielshaft (6) und überlassen die Nachweise 
der anderen Regeln, die nach einem ähnlichen Muster erfolgen, dem Leser. 
$A'$ entsteht aus $A$ durch Addition eines Vielfachen der 
$l$--ten Zeile zur $k$--ten Zeile, wobei $l \neq k$. 

Wir führen den Beweis durch vollständige Induktion. 

Die Fälle $n = 2$ und $n= 3$ sind uns schon aus den Regeln~\ref{det_rule_2_5} 
bzw.~\ref{det_3_rule} bekannt
Wir können annehmen, dass $n \geq 4$, und dass wir die Aussage für Determinanten 
kleinerer Dimension schon kennen. 

Wir können auch annehmen, dass $l > 1$ und $k > 1$. Anderfalls vertauschen wir 
in $A$ und $A'$ geeignete Zeilen und ändern dadurch die Determinanten  
jeweils um ein Vorzeichen. Dann gilt aber 
 	\begin{equation}\label{det_ind_zeilen}
  	\det{A'}= \sum_{l=1}^n (-1)^{1+l} \cdot a_{1,l} \cdot \det{A^{\prime}_{1,l}} 
 	\end{equation}
wobei $A^{\prime}_{i,l}$ aus $A_{i,l}$ jeweils durch Addition eines Vielfachen 
der $(k-1)$--ten Zeile zur $(l-1)$--ten Zeile entsteht. Damit gilt also nach 
Induktionsvoraussetzung
 	$$ \det{A^{\prime}_{i,l}} = \det{A_{i,l}} \qquad \textrm{ für } \, i = 1, \ldots n $$
und daher folgt aus~\ref{det_ind_zeilen}
  	$$ \det{A'} = \det{A} $$
}

\bigbreak

\begin{beispiel} Wir betrachten die Matrix 
  	$$ A = \left( \begin{matrix} 1 & 2 & 3 \\ 3 & 2 & 1 \\ 2 & 4 & 6 \end{matrix} \right) $$
Ohne jegliche explizite Rechnung erhalten wir, dass $\det{A} = 0$, denn die dritte Zeile ist das doppelte der 
ersten Zeile, und wir können Punkt (3) der Regel~\ref{det_n_rule} anwenden.
\end{beispiel}

\medbreak

Eine wiederholte Anwendung der Punkte (3) und (8) dieser Regel liefert sofort folgendes hilfreiche 
Resultat

\begin{korollar} Ist eine Zeile der Matrix $A$ eine Linearkombination der anderen Zeilen, so gilt
  	$$ \det{A} = 0$$
\end{korollar}

\medbreak

\begin{beispiel} Wir betrachten die Matrix 
  	$$ A = \left( \begin{matrix} 1 & 2 & 3 \\ 3 & 2 & 1 \\ 4 & 4 & 4 
  	\end{matrix} \right) $$
Ohne jegliche Rechnung erhalten wir, dass $\det{A} = 0$, denn die dritte Zeile ist die Summer der ersten 
beiden Zeilen.
\end{beispiel}

\bigbreak

Hieraus erhalten wir das $n$--dimensionale 
\index{Determinante!Laplace Entwicklung}Analog von Satz~\ref{det_laplace_3}:

\begin{satz}[Entwicklungssatz von Laplace]\label{det_laplace_n} \index{Satz!Laplace--Entwicklung}
Für eine allgemeine $n \times n$--Matrix
	$$ A = \left( \begin{matrix} a_{1,1} & a_{1,2} & \ldots & a_{1,n} \\
  	a_{2,1} & a_{2,2} & \ldots & a_{2,n} \\
   	\vdots & & \ddots & \vdots \\
  	a_{n,1} & a_{n,2} & \ldots & a_{n,n} \end{matrix} \right) $$
kann $\det{A}$ nach den folgenden Regeln berechnet werden:

\begin{enumerate}
\item \textbf{Entwicklung nach der $i$--ten Zeile}:

Für jedes $i \in \{1, \ldots,n \}$ gilt
  	$$ \begin{array} { l c r}
   	\det{A} & = & (-1)^{i+1} \cdot a_{i,1} \cdot \det{A_{i,1}} + (-1)^{i+2} \cdot  a_{i,2} \cdot \det{A_{i,2}} \\
      	& & \,\, + \cdots + (-1)^{i+n} \cdot a_{i,n}  \cdot \det{A_{i,n}} 
  	\end{array} $$

\item \textbf{Entwicklung nach der $j$--ten Spalte}

Für jedes $j \in \{1, \ldots, n \}$ gilt
  	$$ \begin{array} { l c r}
 	\det{A} & = & (-1)^{1+j} \cdot a_{1,j} \cdot \det{A_{1,j}} + (-1)^{2+j} \cdot a_{2,j}\cdot \det{A_{2,j}} \\
  	& &  \,\,+  \cdots + (-1)^{n+j} \cdot a_{n,j} \cdot \det{A_{n,j}} 
  \end{array} $$
\end{enumerate}
\end{satz}

\beweis{ Durch $(i-1)$-faches Vertauschen können wir die $i$--te Zeile zur ersten 
machen, ohne dadurch die Reihenfolge der anderen zu verändern. Dadurch ändert 
sich das Vorzeichen der Determinante um den Faktor $(-1)^{i-1} = (-1)^{i+1}$. Wenden 
wir jetzt die Definition der Determinante an, so erhalten wir die Formel für 
die Entwicklung nach der $i$--ten Zeile.

Die Formel für die Entwicklung nach der $j$--ten Spalte ergibt sich wegen 
Lemma~\ref{det_lemma2} aus der Zeilenentwicklung, angewendet auf $A^T$.
}

\medbreak

Ferner können wir aus den Regeln folgendes nützliche Resultat ableiten:

\begin{lemma}\label{det_lemma3} 
Entsteht $A'$ aus $A$ durch eine der elementaren Matrizenoperationen
aus Satz~\ref{gls_mat_operation}, 
so gilt
  	$$ \det{A'} \neq 0 \, \iff \, \det{A} \neq 0 $$
Ist $A^{\prime \prime}$ eine Normalform von $A$, so gilt
  	$$ \det{A^{\prime \prime}} \neq 0 \, \iff \, \det{A} \neq 0 $$
\end{lemma}

\beweis{ Nach den Regeln~\ref{det_n_rule} ändert sich bei Durchführung einer 
elementaren Matrizenoperation aus Satz~\ref{gls_mat_operation} allenfalls um ein 
Vorzeichen oder einen von Null verschiedenen Faktor. Hieraus folgt die Aussage 
für $A'$. Da $A^{\prime \prime}$ aus $A$ durch wiederholte Anwendung dieser 
Operationen entsteht, folgt auch die Aussage für $A^{\prime \prime}$.
}

\medbreak

\begin{definition} Eine $n \times n$--Matrix heißt 
\index{Matrix!regulär}\textbf{regulär}, wenn $\det{A} \neq 0$
\end{definition}

\begin{satz}\label{det_n_regular} 
Für eine $n \times n$--Matrix $A$ sind äquivalent:

\begin{enumerate}
\item $A$ ist regulär.
\item $\mathrm{rg}(A) = n$.
\end{enumerate}
\end{satz}

\beweis{  Wegen Lemma~\ref{det_lemma3} ist $A$ genau dann regulär, wenn das für 
eine Normalform von $A$ gilt, und auch der Rang einer Matrix ändert sich nicht, 
wenn wir zur Normalform übergehen. Wir können deshalb annehmen, dass $A$ in Normlaform vorliegt. Dann ist aber $A$ eine obere Dreiecksmatrix der Form 
  	$$ \left(\begin{matrix} 1 & a_{1,2} & \ldots & a_{1,t} & a_{1,t+1} & \ldots 
 	& a_{1,n} \\
 	0 & 1 & \ldots & a_{2,t} & a_{2,t+1} & \ldots & a_{2,n} \\
 	\vdots & & \ddots & & & & \vdots \\
	0 & 0 & \ldots & 1 & a_{t, t+1} & \ldots & a_{t, n} \\
 	0 & 0 & \ldots & 0 & 0 & \ldots & 0 \\
 	\vdots & & \ddots & & & & \vdots \\
 	0 & 0 & \ldots & 0 & 0 & \ldots & 0  \end{matrix} \right) \qquad \textrm{ für ein }
 	t < n $$
oder 
  	$$ \left(\begin{matrix} 1 & a_{1,2} & \ldots & a_{1,n} \\
  	0 & 1 & \ldots & a_{2, n} \\
  	\vdots & & \ddots & \vdots \\
  	0 & 0 & \ldots & 1 \end{matrix} \right) $$
Im ersten Fall gilt $\mathrm{rg}(A) < n$ und $\det{A} = 0$ (etwa wegen (3) aus 
Regel~\ref{det_n_rule}) und im zweiten Fall gilt $\mathrm{rg}(A) = n$ und 
$\det{A} = 1$ nach (4) aus Regel~\ref{det_n_rule}.
}
\medbreak

\begin{notiz}
Die Definition der Determinante lässt sich sehr einfach als Algorithmus umsetzen:

\makebox[\textwidth]{\hrulefill}  
\lstinputlisting{LA_determinante.m}
\makebox[\textwidth]{\hrulefill} 

\end{notiz}

\bigbreak

Für große $n$ ist das Berechnen der Determinante sowohl über die 
Laplace--Entwicklung als auch über die Darstellung mit der Permutationsgruppe 
sehr arbeitsintensiv. In diesem Fall wird in der Praxis die Determinante dadurch 
berechnet, dass man die Matrix in Normalform überführt und sich die Operationen, 
die dabei durchgeführt werden und den Wert der Determinante ändern, merkt. 
Dieses Verfahren wächst in seiner Komplexität viel langsamer als etwa die 
Laplace--Entwicklung.

\begin{beispiel}\label{det_n_einfach} Wir betrachten die Matrix 
  	$$ A = \left( \begin{matrix} 1 & 2 & 3 \\ 2 & 4 & 4 \\ 3 & 4 & 4 
    	\end{matrix} \right) $$
Im ersten Schritt subtrahieren wir das Doppelte der ersten Zeile von der zweiten und 
das Dreifache der ersten Zeile von der Dritten, 
wodurch sich die Determinante nicht ändert, und erhalten 
  	$$ A' = \left( \begin{matrix} 1 & 2 & 3 \\ 0 & -2 & -2 \\ 0 & -2 & -5 
   	\end{matrix} \right) $$
Im zweiten Schritt multiplizieren wir die zweite Zeile mit $-\frac{1}{2}$ und 
merken uns, dass dadurch die Determinante mit $-\frac{1}{2}$ multipliziert wird. 
Wir erhalten 
  	$$ A^{\prime \prime} = \left( \begin{matrix} 1 & 2 & 3 \\ 0 & 1 & 1 \\ 0 & -2 & -5 
   	\end{matrix} \right) $$
Wir addieren jetzt das Doppelte der zweiten Zeile zur Dritten,
wodurch sich die Determinante nicht ändert, und erhalten 
  	$$ A^{\prime \prime \prime} = \left( \begin{matrix} 1 & 2 & 3 \\ 0 & 1 & 1 
   	\\ 0 & 0 & -1  \end{matrix} \right) $$
Diese Matrix $A^{\prime \prime \prime}$ ist eine obere Dreiecksmatrix, hat also 
Determinante $1 \cdot 1 \cdot (-1) = -1$. Um die Determinante von $A$ zu bekommen, 
müssen wir berücksichtigen, dass in Schritt 2 die Determinante mit dem Faktor 
$-\frac{1}{2}$ multipliziert wurde und dies rückgängig machen. Wir erhalten also
  	$$ \det{A} = 2 $$
\end{beispiel}
 
\begin{beispiel} In Beispiel~\ref{det_n_einfach} hätten wir die Determinante auch noch 
leicht nach dem Entwicklungssatz von Laplace ausrechnen können. Schon etwas komplizierter ist 
das im Fall
  	$$   B = \left( \begin{matrix} 5 & 1 & 5 & 4 \\ 4 & -4 & -3 & -2 \\ 1 & 0 & -2 & 5 \\
  	3 & 1 & 2 & 4  \end{matrix} \right) $$
Hier empfiehlt es sich bereits, die Regeln aus~\ref{det_n_rule}anzuwenden.

Zunächst vertauschen wir die erste und die dritte Spalte und erhalten 
  	$$ B_1 =  \left( \begin{matrix} 1 & 0 & -2 & 5 \\  4 & -4 & -3 & -2 \\ 5 & 1 & 5 & 4 \\
   	3 & 1 & 2 & 4  \end{matrix} \right) $$
Hierdurch ändert sich das Vorzeichen, $\det{B_1} = - \det{B}$. Wir subtrahieren viermal die 
erste Zeile von der zweiten, fünfmal die erste von der dritten und dreimal die erste von der vierten und 
erhalten
  	$$ B_2 =  \left( \begin{matrix} 1 & 0 & -2 & 5 \\  0 & -4 & 5 & -22 \\ 0 & 1 & 15 & -21 \\
   	0 & 1 & 8 & -11  \end{matrix} \right) $$
Hierdurch ändert sich die Determinante nicht,  $\det{B_2} = \det{B_1} = - \det{B}$. 
Nun können wir $\textrm{det}(B_2)$ schon recht einfach durch Entwicklung nach der ersten Spalte berechnen. Wir 
können aber auch weiter vereinfachen. Dazu vertauschen wir zunächst die zweite und die vierte Spalte:
  	$$ B_3 =  \left( \begin{matrix} 1 & 0 & -2 & 5 \\ 0 & 1 & 8 & -11  \\ 0 & 1 & 15 & -21 \\
   	0 & -4 & 5 & -22  \end{matrix} \right) $$
Hierdurch ändert sich das Vorzeichen der Determinante, $\det{B_3}= -\det{B_2} = \det{B}$. 
Jetzt subtrahieren wir die zweite Zeile von der dritten und addieren sie viermal zur vierten:
  	$$ B_4 = \left( \begin{matrix} 1 & 0 & -2 & 5 \\ 0 & 1 & 8 & -11  \\ 0 & 0 & 7 & -10 \\
     	0 & 0 & 37 & -66  \end{matrix} \right) $$
Hierdurch ändert sich die Determinante nicht, $\det{B_4} = \det{B_3} = \det{B}$. Um uns 
das Leben noch einfacher zu machen, subtrahieren wir fünfmal die dritte Spalte von der Vierten und erhalten
  	$$ B_5 = \left( \begin{matrix} 1 & 0 & -2 & 5 \\ 0 & 1 & 8 & -11  \\ 0 & 0 & 7 & -10 \\
    	0 & 0 & 2 & -16  \end{matrix} \right) $$
Hierdurch ändert sich die Determinante nicht, $\det{B_5}= \det{B_4} = \det{B}$. Nun 
berechnen wir $\det{B_5}$ indem wir zweimal nach der ersten Spalte entwickeln:
  	$$ \begin{array} {l c l}
   	\det{B_5} & = & 1 \cdot \textrm{det} \left( \begin{matrix}  1 & 8 & -11  \\  0 & 7 & -10 \\
   	0 & 2 & -16  \end{matrix} \right) - 0 + 0 - 0\\[2.0em]
     	& = & 1 \cdot \textrm{det} \left( \begin{matrix}  7 & -10 \\
  	2 & -16  \end{matrix} \right) - 0 + 0 \\[1.2em]
  	& = & 7 \cdot (-16) - (-10) \cdot 2 \\
 	& = & -92 
  	\end{array} $$
Also gilt $\det{B} = -92$.
\end{beispiel}

\medbreak

\begin{beispiel} Wir wollen die Determinante der Matrix
  	$$ A = \left( \begin{matrix} 1 & 2 & 3 & 2 & 1 \\ 2 & 1 & 4 & 2 & 3 \\ 
    	7 & 6 & 5 & 4 & 3 \\ 1 & -1 & 1 & -1 & 1 \\ 2 & 3 & 1 & 0 & -1 \end{matrix} \right) $$
berechnen. In diesem Fall ist die Entwicklung nach einer Zeile oder einer Spalte nicht zu empfehlen. 
Hierzu müssten zunächst fünf Determinanten von $4 \times 4$--Matrizen berechnet werden, und jede dieser 
Determinanten erfordert wiederum die Ermittlung von vier Determinanten von $3 \times 3$--Matrizen. 
Neben dem Rechenaufwand ist hier auch die Gefahr, sich zu verrechnen, zu berücksichtigen.
Die numerische einfachere Methode ist es, die Matrix auf obere Dreiecksform zu bringen.

Dazusubtrahieren wir geeignete Vielfache der ersten Zeile von den Zeilen 2 - 5 und erhalten 
  	$$ A_1 = \left( \begin{matrix} 1 & 2 & 3 & 2 & 1 \\ 0 & -3 & -2 & - 2 & 1 \\ 
     	0 & - 8 & -16 & -10 & -4 \\0 & -3 & -2 & -3 & 0  \\ 0 & -1 & -5 & -4 & -3 \end{matrix} \right) $$
Dadurch ändert sich die  Determinante nicht, also $\det{A_1}= \det{A}$. 

Nun vertauschen wir die zweite und die fünfte Spalte und erhalten
  	$$ A_2 = \left( \begin{matrix} 1 & 1 & 3 & 2 & 2 \\ 0 & 1 & -2 & - 2 & -3\\ 
  	0 & -4 & -16 & -10 & -8 \\0 & 0 & -2 & -3 & -3  \\ 0 & -3 & -5 & -4 & -1 \end{matrix} \right) $$
Dadurch ändert die Determinante das Vorzeichen, $\det{A_2} = -\det{A_1} = -\det{A}$.

Wir addieren viermal die zweite Zeile zur dritten und dreimal die zweite Zeile zur vierten, und erhalten
  	$$ A_3 = \left( \begin{matrix} 1 & 1 & 3 & 2 & 2 \\ 0 & 1 & -2 & - 2 & -3\\ 
    	0 & 0 & -24 & -18 & -20 \\0 & 0 & -2 & -3 & -3  \\ 0 & 0 & -11 & -10 & -10 \end{matrix} 
    	\right) $$
Dadurch ändert sich die  Determinante nicht, also $\det{A_3} = \det{A_2} = - \det{A}$. 

Nun bietet sich zunächst keine Zeile oder Spalte direkt zur Weiterarbeit an. Wir könnten die dritte Zeile durch $20$ 
dividieren, uns diesen Faktor für die Determinantenberechnung merken und dann mit der modifizierten dritten Zeile 
weiterarbeiten. Das wäre die Methode, die ein Computerprogramm benutzen würde, bei manuellen Berechnungen 
ist das jedoch aufgrund der auftretenden Brüche fehleranfällig. Daher schieben wir einen Zwischenschritt ein und 
subtrahieren sechsmal die vierte Zeile von der fünften:
 	$$ A_4 = \left( \begin{matrix} 1 & 1 & 3 & 2 & 2 \\ 0 & 1 & -2 & - 2 & -3\\ 
    	0 & 0 & -24 & -18 & -20 \\0 & 0 & -2 & -3 & -3  \\ 0 & 0 & 1 & 8 & 8  \end{matrix} \right) $$
Dadurch ändert sich die  Determinante nicht, also $\det{A_4} = \det{A_3} = - \det{A} $. 

Nun vertauschen wir die Zeilen 3 und 5:
 	$$ A_5 = \left( \begin{matrix} 1 & 1 & 3 & 2 & 2 \\ 0 & 1 & -2 & - 2 & -3\\  0 & 0 & 1 & 8 & 8 \\
  	0 & 0 & -2 & -3 & -3 \\ 0 & 0 & -24 & -18 & -20 \\   \end{matrix} \right) $$
Dadurch ändert die Determinante das Vorzeichen, $\det{A_5} = -\det{A_4} =  \det{A} $.

Wir addieren das doppelte der dritten Zeile zur vierten und das vierundzwanzigfache der dritten zur fünften und erhalten
  	$$ A_6 = \left( \begin{matrix} 1 & 1 & 3 & 2 & 2 \\ 0 & 1 & -2 & - 2 & -3\\  0 & 0 & 1 & 8 & 8 \\
 	0 & 0 & 0 & 13 & 13 \\ 0 & 0 & 0 & 174 & 172 \\   \end{matrix}  \right) $$
Dadurch ändert sich die  Determinante nicht, also $\det{A_6} = \det{A_5} = \det{A}$. 

Nun können wir die Determinante bereits explizit berechnen, indem wir sukkzessive nach der ersten Spalte entwickeln:
  	$$  \det{A_6} = 1 \cdot 1 \cdot 1 \cdot \left( 13 \cdot 172 - 13 \cdot 174 \right) = -26 $$

Alternativ können wir aber auch noch die fünfte Spalte von der vierten subtrahieren und erhalten
  	$$ A_7 = \left( \begin{matrix} 1 & 1 & 3 & 0 & 2 \\ 0 & 1 & -2 & 1 & -3\\  0 & 0 & 1 & 0 & 8 \\
  	0 & 0 & 0 & 0 & 13 \\ 0 & 0 & 0 & 2 & 172 \\   \end{matrix} 
   	\right) $$
Dadurch ändert sich die  Determinante nicht, also $\det{A_7} = \det{A_6} = \det{A}$.

Vertauschen wir jetzt die Zeilen 4 und 5, so ergibt sich  
  	$$ A_8 = \left( \begin{matrix} 1 & 1 & 3 & 0 & 2 \\ 0 & 1 & -2 & 1 & -3\\  0 & 0 & 1 & 0 & 8 \\
   	0 & 0 & 0 & 2 & 172 \\ 0 & 0 & 0 & 0 & 13 \\   \end{matrix} \right) $$
und ein Vorzeichenwechsel in der Determinante, $\det{A_8}= -\det{A_7} = -\det{A}$.
Jetzt haben wir obere Dreicksform erreicht und lesen also ab, dass
  	$$ \det{A_8} = 1 \cdot 1 \cdot 1 \cdot 2 \cdot 13 = 26 $$
also auch hier $\det{A}= -26$.
\end{beispiel}

\bigbreak

Bei der Definition der Determinante haben die $(i,j)$--Steichungsmatrizen $A_{i,j}$ 
und ihre Determinanten eine wichtige Rolle gespielt. Wir setzen 
  	$$ \widetilde{a_{i,j}} = (-1)^{i+j} \cdot \det{A_{j,i}} $$
(beachten Sie dabei die Vertauschung der Indizes).

\begin{definition} Die Matrix $\widetilde{A} := \left( \widetilde{a_{i,j}} \right)$ 
heißt die zu $A$ \index{Matrix!komplementär}\textbf{komplementäre} Matrix
\end{definition}

\begin{beispiel}\label{det_kompl_2} Sei 
  	$$ A = \left( \begin{matrix} 1 & 2 \\ 1 & 3 \end{matrix} \right) $$
Dann ist 
  	$$ \widetilde{A} = \left( \begin{matrix} 3 & -2 \\ -1 & 1 \end{matrix} \right) $$
\end{beispiel}

\begin{beispiel}\label{det_kompl_3} Sei 
  	$$ A = \left( \begin{matrix} 1 & 0 & -1 \\ 8 & -4 & -1 \\ -2 & 1 & 0 \end{matrix} \right) $$
Dann ist 
  	$$ \widetilde{A} = \left( \begin{matrix} 1 & -1 & -4 \\ 2 & -2 & -7 \\ 0 & -1 & -4 \end{matrix} \right) $$
\end{beispiel}

\medbreak

\begin{satz}\label{det_n_komplement} Es gilt
 	$$ \widetilde{A} \cdot A = A \cdot \widetilde{A} = \det{A} \cdot E_n $$
ist eine Diagonalmatrix mit $\det{A}$ als Eintrag an jeder Stelle der Diagonale.
\end{satz}

\beweis{ Wir berechnen dazu explizit den Matrixeintrag $\left( \widetilde{A} 
\cdot A \right)_{k,l}$ an jeder Stelle $k,l$.

Ist $l = k$, so gilt:
 	$$ \begin{array} {l c l}
  	\left( \widetilde{A} \cdot A \right)_{l,l} & = & \sum\limits_{i=1}^n \widetilde{a_{l,i}} \cdot a_{i,l} \\
  	& = & \sum\limits_{i=1}^n (-1)^{i+l} \det{A_{i,l}} \cdot a_{i,l} \\
  	& = & \sum\limits_{i=1}^n (-1)^{i+l} a_{i,l} \cdot \det{A_{i,l}} \\
  	& = & \det{A} 
  	\end{array} $$
nach dem Laplace--Entwicklungssatz (Entwicklung nach der $k$--ten Spalte).

Ist $l \neq k$ so betrachten wir die Matrix $B$, die aus $A$ dadurch entsteht, dass 
wir die $k$--te Spalte von $A$ durch die $l$--te Spalte von $A$ ersetzen. Damit hat 
$B$ zwei identische Spalten (an der Stelle $k$ und der Stelle $l$), und daher gilt 
nach Regel~\ref{det_n_rule}
  	$$ \det{B} = 0 $$
Andererseits gilt $B_{i,k} = A_{i,k}$, da $B$ und $A$ in allen Spalten ausser der 
$k$--ten übereinstimmen.
Entwickeln wir daher $B$ nach der $k$-ten Spalte, so erhalten wir
  	$$ \begin{array} {l c l}
  	\det{B} & = & \sum\limits_{i=1}^n (-1)^{i+k} b_{i,k} \cdot \det{B_{i,k}} \\
   	& = & \sum\limits_{i=1}^n (-1)^{i+k} a_{i,l} \cdot \det{A_{i,k}} \\
    	& = & \sum\limits_{i=1}^n (-1)^{i+k} \widetilde{a_{k,i}} \cdot a_{i,l} \\
   	& = & \left( \widetilde{A} \cdot A \right)_{k,l}
  	\end{array} $$
Daraus folgt $\widetilde{A} \cdot A = \det{A} \cdot E_n$, und die Gleichheit 
$A \cdot \widetilde{A} = \det{A} \cdot E_n$ sieht man genauso. 
}

\medbreak

\begin{beispiel} Für die Matrix $A$ aus Beispiel~\ref{det_kompl_2} gilt
  	$$ A \cdot \widetilde{A} = \left( \begin{matrix} 1 & 2 \\ 1 & 3 \end{matrix} \right)
  	\cdot \left( \begin{matrix} 3 & -2 \\ -1 & 1 \end{matrix} \right)
  	= \left( \begin{matrix} 1 & 0 \\ 0 & 1 \end{matrix} \right) $$
und $\det{A} = 1$.
\end{beispiel}

\begin{beispiel} Für die Matrix $A$ aus Beispiel~\ref{det_kompl_3} gilt
  	$$ A \cdot \widetilde{A} =  \left( \begin{matrix} 1 & 0 & -1 \\ 8 & -4 & 1 \\ 
   	-2 & 1 & 0 \end{matrix} \right) \cdot 
   	\left( \begin{matrix} 1 & 0 & -1 \\ 8 & -4 & -1 \\ -2 & 1 & 0 \end{matrix} \right) =
  	\left( \begin{matrix} 1 & -1 & -4 \\ 2 & -2 & -7 \\ 0 & -1 & -4 \end{matrix} \right) = 1 \cdot E_3  $$ 
und $\det{A} = 1$.           
\end{beispiel}
\bigbreak

Ein wichtiges aber schwieriges \index{Determinante!Produktsatz}Resultat ist 

\begin{satz}\label{det_n_produkt} \textbf{(Produktsatz)} 

Sind $A$ und  $B$ zwei $n \times n$--Matrizen, so gilt
  	$$ \det{ A \cdot B } = \det{A} \cdot \det{B} $$
\end{satz}

Wir werden den Beweis dieses Satzes im Abschnitt~\ref{section_det_inverse} 
skizzieren.

\bigbreak

\begin{aufgabe} Berechnen Sie die Determinanten der Matrizen
  	$$ A = \left( \begin{smallmatrix} 1 & 2 & 3 & 5 \\ 2 & 5 & 7 & 9\\ -1 & 2 & -3 & 4 \\
	0 & 1 & 0 & 2  \end{smallmatrix} \right),\quad 
  	B = \left( \begin{smallmatrix} 7 & 1 & 5 & 2 \\ 4 & -4 & 2 & -2 \\ 1 & 0 & -2 & 5 \\
	1 & 1 & 2 & 2  \end{smallmatrix} \right),\quad 
  	C =  \left( \begin{smallmatrix} 6 & -6 & 0 & 2 \\ 4 & 1 & 1 & 4 \\ -1 & -2 & 3 & 4 \\
  	2 & 1 & 2 & 1  \end{smallmatrix} \right)  $$
\end{aufgabe}

\begin{aufgabe} Berechnen Sie die Determinante der Matrix
  	$$ A = \left( \begin{matrix} 2 & 2 & 3 & 1 & 5 \\4  & -3 & 6 & 1 & 3 \\ 
  	4 & 6 & 3 & 5 & 2 \\ 1 & -1 & 1 & -1 & 1 \\ 2 & -3 & 1 & 4 & 2 \end{matrix} \right) $$
\end{aufgabe}

\begin{aufgabe} Ist $A$ eine $n \times n$ Matrix, die sich in der Form 
  	$$ A = \left( \begin{matrix} B & C \\ 0 & D \end{matrix} \right) $$
mit einer $n_1 \times n_1$--Matrix $B$, einer $n_1 \times (n-n_1)$--Matrix $C$ und 
einer $(n-n_1) \times (n-n_1)$--Matix $D$ (sowie einer 
$(n-n_1) \times n_1$--Nullmatrix 0) schreiben lässt, so gilt
  	$$ \det{A} = \det{B} \cdot \det{C} $$
\end{aufgabe}

\begin{aufgabe} Berechnen Sie die Komplementärmatrizen zu den Matrizen
  	$$ A = \left( \begin{matrix} 1 & 2 & 3 \\ 2 & 3 & 4 \\ 4 & 4 & 4 
	\end{matrix} \right), \quad B = \left( \begin{matrix} 1 & 2 & 3 & 4 \\
  	0 & 2 & 4 & 6 \\ 0 & 0 & 3 & 6 \\ 0 & 0 & 0 & 4 \end{matrix} \right) $$
\end{aufgabe}

\begin{aufgabe} Für reelle Zahlen $r_1, \ldots ,r_n$ ($n \geq 2$) definieren wir die $n \times n$--Matrix
  	$$ V(r_1, \ldots, r_n) = \left( \begin{matrix} 1 & 1 & \ldots & 1 \\r_1 & r_2 & \ldots & r_n \\
	r_1^2 & r_2^2 & \ldots & r_n^2 \\  \vdots & & \ddots & \vdots \\
	r_1^{n-1} & r_2^{n-1} & \ldots & r_n^{n-1} \end{matrix} \right) $$
Zeigen Sie: $\det{V(r_1, \ldots, r_n)} = \prod\limits_{1 \leq i < j \leq n} (r_j-r_i)$.
\end{aufgabe}


\bigbreak

\newpage

\section{Determinanten und invertierbare Matrizen}\label{section_det_inverse}

\setcounter{definition}{0}
\setcounter{beispiel}{0}
\setcounter{notiz}{0}

Wir haben in den vorangegangenen Abschnitten gesehen, dass $n \times n$--Matrizen $A$ mit $\det{A} \neq 0$ 
in der Menge der Matrizen eine ausgezeichnete Rolle spielen und diese Matrizen als reguläre Matrizen 
bezeichnet. In diesem Abschnitt wollen wir uns einer anderen ausgezeichneten Eigenschaft einiger Matrizen 
zuwenden.

\begin{definition} Eine $n \times n$--Matrizen $A$ heißt \index{Matrix!invertierbar}\textbf{invertierbar}, 
wenn es eine $n \times n$--Matrizen $B$ gibt mit
  	$$ A \cdot B = E_n, \qquad   B \cdot A = E_n $$
\end{definition}

\bemerkung{Bezeichnung} Ist $A$ invertierbar, so nennen wir die Matrix $B$ aus der Definition die \textit{zu $A$ 
inverse Matrix} und bezeichnen \index{Matrix!inverse Matrix}sie mit $A^{-1}$.

\begin{notiz} Zunächst stellt sich die Frage, ob vielleicht jede Matrix invertierbar ist. Betrachten wir 
dazu zunächst $A = \left( \begin{smallmatrix} 0 & 0 \\ 0 & 0 \end{smallmatrix} \right)$, also die Nullmatrix, 
so gilt für jede Matrix $B$: 
  	$$A \cdot B = \left( \begin{matrix} 0 & 0 \\ 0 & 0 \end{matrix} \right) $$ 
und damit kann es sicherlich keine Matrix mit $A \cdot B = E_2$ geben. 

Bei den reellen Zahlen ist jede von Null verschiedene reelle Zahl invertierbar. Ist also vielleicht jede von der
Nullmatrix verschiedene Matrix invertierbar? Auch das ist nicht der Fall, wie das Beispiel
$A = \left( \begin{smallmatrix} 1 & 1 \\ 0 & 0 \end{smallmatrix} \right)$ zeigt. Ist nämlich $B$ eine 
beliebige $2 \times 2$--Matrix, so gilt immer
  	$$ A \cdot B = \left( \begin{matrix} $*$ & $*$ \\ 0 & 0 \end{matrix} \right) $$
die zweite Zeile ist also immer die Nullzeile, und damit kann es auch hier keine Matrix $B$ mit $A \cdot B = E_2$
geben. 
\end{notiz}

\begin{notiz} Auf dem Vektorraum $\textrm{Matr}(n,n)$ haben wir neben der Matrizenaddition auch noch die 
Matrizenmulitplikation. Es ist leicht einzusehen, dass $(\textrm{Matr}(n,n), +, \cdot)$ ein (nicht--kommutativer) 
Ring mit Einselement $E_n$ ist. Wir suchen nun die Elemente in diesem Ring, die ein multiplikatives Inverses 
haben.
\end{notiz}

\begin{notiz}\label{det_rechts_links_inv} 
Es reicht, zu einer $n \times n$--Matrix $A$ ein Linksinverses und ein Rechtsinverses zu finden, 
also eine Matrix $B$ mit $B \cdot A = E_n$ und eine Matrix $C$ mit $A \cdot C = E_n$. Dann gilt schon 
  	$$ B =  B \cdot E_n = B \cdot A \cdot C = E_n \cdot C = C$$
Genauso zeigt man, dass eine inverse Matrix, falls sie existiert, eindeutig ist: Sind $B$ und $C$ 
zwei inverse Matrizen zu $A$ , so gilt
  	$$ B = B \cdot E_n = B \cdot A \cdot C = E_n \cdot C = C$$
\end{notiz}

Inverse Matrizen sind ein hilfreiches Instrument bei der Lösung linearer Gleichungssysteme
 
\begin{notiz} Ist $A$ eine invertierbare Matrix mit Inverser $A^{-1}$ so hat das Gleichungssystem 
  	$$ A \cdot \vektor{x} = \vektor{b} $$
die eindeutige Lösung 
  	$$ \vektor{x} = A^{-1} \cdot \vektor{b} $$
\end{notiz} 

\beweis{ Ist $\vektor{x}$ eine Lösung des Gleichungssystems, so muss gelten
  	$$ \vektor{x} = A^{-1} \cdot A \cdot \vektor{x} = A^{-1} \cdot \vektor{b} $$
Setzen wir umgekehrt $\vektor{x} = A^{-1} \cdot \vektor{b}$, so erhalten wir
  	$$ A \cdot \vektor{x} = A \cdot A^{-1} \cdot \vektor{b} = \vektor{b} $$
}

\bigbreak

Wir konzentrieren uns zunächst auf das Finden einer Rechtsinversen, also auf das Finden einer Matrix $B$ 
mit $A \cdot B = E_n$. Dazu sei $\vektor{e_1}, \ldots , \vektor{e_n}$ die Standardbasis 
des $\mathbb R^n$. 

\begin{satz}\label{det_rechtsinverse} Genau dann existiert eine rechtsinverse Matrix $B$ von $A$, wenn die 
Gleichungssysteme 
  	$$ A \cdot \vektor{x} = \vektor{e_i} $$
für jedes $i = 1, \ldots, n$ lösbar ist. Ist in diesem Fall $\vektor{v_i}$ eine Lösung von 
$A \cdot \vektor{x} = \vektor{e_i}$ und ist $B$ die Matrix mit den Spaltenvektoren 
$\vektor{v_1}, \vektor{v_2}, \ldots, \vektor{v_n}$, so gilt
  	$$ A \cdot B = E_n $$
\end{satz}

\beweis{ Ist $B$ eine Rechtsinverse zu $A$ und ist $B_{\bullet, i}$ der $i$--te Spaltenvektor von $B$, so 
gilt nach Definition der Matrizenmultiplkation gerade $A \cdot B_{\bullet, i} = \vektor{e_i}$, und 
damit sind alle fraglichen Gleichungssysteme lösbar.

Sind umgekehrt $\vektor{v_1}, \ldots, \vektor{v_n}$ Lösungen von $A \cdot \vektor{x} = \vektor{e_i}$ und 
ist $B$ die Matrix mit den $\vektor{v_1}, \ldots , \vektor{v_n}$ als Spaltenvektoren, so ist - 
wiederum nach Definition der Matrizenmultiplikation - $A \cdot B$ die Matrix mit den Spaltenvektoren $A \cdot 
\vektor{v_1}$, $\ldots$ , $A \cdot \vektor{v_n}$, also 
  	$$ A \cdot B = \left( \vektor{e_1} \, \vektor{e_2} \, \ldots \, \vektor{e_n} \right) = E_n $$ 
}

\begin{korollar} Genau dann existiert zu $A$ eine rechtsinverse Matrix, wenn $\mathrm{rg}(A) = n$.
\end{korollar} 

\beweis{ Nach Satz~\ref{gls_loesung_ker} aus Abschnitt~\ref{section_lin_abb} sind die Gleichungssysteme 
$A \vektor{x} = \vektor{e_i}$ genau dann für alle $i$ 
lösbar, wenn $\vektor{e_i} \in \mathrm{Im}(A)$ ist für alle $i$, und das ist äquivalent zu
  	$$ \mathrm{Im}(A) = \mathbb R^n $$
also zu $\mathrm{rg}(A) = n$. Aus dem Satz folgt jetzt die Behauptung.
}
\bigbreak

Wir haben also die Frage nach einer rechtsinversen Matriz zu $A$ gelöst und können uns jetzt der Suche 
nach einer linksinversen Matrix zuwenden. Hierfür benötigen wir einen kleinen Hilfssatz
 
\begin{lemma}\label{det_links_transp}
 Genau dann hat $A$ eine linksinverse Matrix, wenn $A^T$ eine rechtsinverse Matrix hat.

Ist in diesem Fall $C$ eine rechtsinverse Matrix von $A^T$, so ist $C^T$ eine linksinverse Matrix von $A$.
\end{lemma}

\beweis{ Ist $C$ eine rechtsinverse Matrix zu $A^T$, so gilt nach den Regeln zu Transponieren von Matrizen
  	$$ C^{T} \cdot A = \left( \left( C^T \cdot A \right)^T\right)^T = \left( A^T \cdot \left(C^T\right)^T 
	\right)^T = \left(A^T \cdot C\right)^T = E_n^T = E_n $$
Ist umgekehrt $D$ eine linksinverse Matrix zu $A$ so rechnet man genauso nach, dass $D^T$ eine rechtsinverse 
Matrix zu $A^T$ ist.
}

\medbreak

\begin{satz}\label{det_inverse_matrix} Für eine $n \times n$--Matrix $A$ sind 
äquivalent:

\begin{enumerate}
\item $A$ ist regulär.
\item $A$ ist invertierbar.
\item $\mathrm{rg}(A) = n$.
\end{enumerate}
\end{satz}

\beweis{ Wir haben in Satz~\ref{det_n_regular} schon gesehen, dass die 
Regularität von $A$ äquivalent zu
$\mathrm{rg}(A) = n$ ist und brauchen also nur noch zeigen, dass $A$ genau dann 
invertierbar ist, wenn $\mathrm{rg}(A) = n$.

Nach Satz ~\ref{det_rechtsinverse} existiert zu $A$ genau dann eine rechtsinverse 
Matrix, wenn $\mathrm{rg}(A) = n$. Aus Lemma~\ref{det_links_transp} folgt, 
dass $A$ genau dann eine linksinverse Matrix hat, wenn 
$\mathrm{rg}(A^T) = n$. Da nach Satz~\ref{gls_rang_transp} aus Abschnitt~\ref{section_lin_abb} gilt
  	$$ \mathrm{rg}(A) = \mathrm{rg}(A^T) $$
folgt hieraus mit Bemerkung~\ref{det_rechts_links_inv} die Behauptung. 
}
\bigbreak

Zur Bestimmung einer inversen Matrix reicht es also, eine rechtsinverse Matrix von $A$ zu ermitteln. Dazu liefert 
Satz~\ref{det_rechtsinverse} ein probates Mittel. Sowohl um zu entscheiden, ob es eine zu $A$ inverse Matrix gibt 
als auch um diese zu finden müssen wir nur die Gleichungssysteme $A \cdot 
\vektor{x} = \vektor{e_i}$ für alle $i = 1, \ldots, n$ lösen. Nach 
Bemerkung~\ref{gls_augmentiert_simultan} aus Abschnitt~\ref{gls_matrix}
können wir diese Gleichungssysteme alle in einem Durchgang durch 
Betrachtung der erweiterten augmentierten Matrix  $ \left( A \, \vektor{e_1} \, \vektor{e_2}
\, \ldots \, \vektor{e_n}\right)$ und Überführen dieser Matrix in Normalform lösen. Ferner 
können wir diese erweiterte Matrix gemäß Satz~\ref{gls_voller_rang_n} nur durch Zeilenoperationen auf 
die Normalform
  	$$ \left( \begin{matrix} 1 & 0 & \ldots & 0 \\ 0 & 1 & \ldots & 0 \\ \vdots & & \ddots & \vdots \\
	0 & 0 & \ldots & 1 \end{matrix} \quad \, \, \vektor{b_1} \, \vektor{b_2} \, \ldots \, \vektor{b_n}  \right) $$
bringen, und daraus folgt

\begin{korollar} $A^{-1} = \left( \vektor{b_1} \, \ldots \, \vektor{b_n} \right)$ ist die Matrix 
mit den $\vektor{b_i}$ als Spaltenvektoren.
\end{korollar}

\begin{beispiel} Wir betrachten die Matrix
  	$$ A = \left(\begin{matrix} 1 & 2 \\ 2 & 3 \end{matrix} \right) $$
und gehen hier vor wie folgt:
  	$$ \begin{array} {l c l}
     	\left( A \quad \, \vektor{e_1} \, \vektor{e_2}\right) & \quad = \quad & 
     	\left( \begin{array} {c c | c c} 1 & 2 \phantom{x} &  \quad 1 & 0 \\ 
	2 & 3 \phantom{x} &  \quad  0 & 1 \end{array} \right)  \\[2.0em]
     	II - 2 \cdot I  & & \left( \begin{array} {c c | c c} 1 &  2 \phantom{x} &  \quad  1 & 0 \\ 
	0 & -1 \phantom{x} &  \quad  -2 & 1 \end{array} \right)  \\[2.0em]
    	(-1) \cdot II & & \left( \begin{array} {c c | c c} 1 &  2 \phantom{x} &  \quad  1 & 0 \\ 
	0 &  1 \phantom{x} &  \quad   2 & -1 \end{array} \right)  \\[2.0em]
     	I - 2 \cdot II & & \left(\begin{array} {c c | c c} 1 &  0 \phantom{x} &  \quad  -3 & 2 \\ 
	0 &  1 \phantom{x} &  \quad   2 & -1 \end{array} \right) 
   	\end{array} $$
wobei wir in der linken Spalte die durchgeführten Zeilenoperationen notieren (''$II - 2 \cdot I$'' bedeutet 
also: Subtrahiere das Doppelte der ersten Zeile von der zweiten). Wir erhalten, dass $A$ invertierbar ist mit
  	$$ A^{-1} = \left(\begin{matrix} -3 & 2 \\ 2 & -1 \end{matrix} \right) $$
und wir überzeugen uns davon, dass in der Tat
  	$$ A \cdot A^{-1} = E_2 = A^{-1} \cdot A $$
\end{beispiel}

\begin{beispiel} Wir betrachten die Matrix  
  	$$ A = \left(\begin{matrix} 0 & 1 & 2 \\ 1 & 2 & 2 \\ 1 & 3 & 3 \end{matrix} \right) $$
und gehen vor wie folgt:
  	$$ \begin{array} {l c l}
     	\left( A \, \, \vert \, \vektor{e_1} \, \vektor{e_2} \, \vektor{e_3} \right) 
      & \quad = \quad & 
     	\left(\begin{array} {c c c | c c c} 0 & 1 & 2 \phantom{x} &  \quad 1 & 0 & 0 \\ 
	1 & 2 & 2 \phantom{x} &  \quad 0 & 1 & 0 \\ 
	1 & 3 & 3 \phantom{x} &  \quad 0 & 0 & 1  \end{array} \right) \\[2.2em]
      I \leftrightarrow II & & 
     	\left(\begin{array} {c c c | c c c} 1 & 2 & 2 \phantom{x} &  \quad 0 & 1 & 0 \\ 
	0 & 1 & 2 \phantom{x} &  \quad 1 & 0 & 0 \\ 
	1 & 3 & 3 \phantom{x} &  \quad 0 & 0 & 1 \end{array} \right) \\[2.2em]
      III - I & & 
     	\left(\begin{array} {c c c | c c c} 1 & 2 & 2 \phantom{x} &  \quad 0 & 1 & 0 \\ 
	0 & 1 & 2 \phantom{x} &  \quad 1 & 0 & 0 \\ 
	0 & 1 & 1 \phantom{x} &  \quad 0 & -1 & 1 \end{array} \right) \\[2.2em]
   	III - II & & 
     	\left(\begin{array} {c c c | c c c} 1 & 2 & 2 \phantom{x} &  \quad 0 & 1 & 0 \\ 
	0 & 1 & 2 \phantom{x} &  \quad 1 & 0 & 0 \\ 
	0 & 0 & -1 \phantom{x} &  \quad -1 & -1 & 1\end{array} \right) \\[2.2em]
  	(-1) \cdot III & & 
 	\left(\begin{array} {c c c | c c c} 1 & 2 & 2 \phantom{x} &  \quad 0 & 1 & 0 \\ 
	0 & 1 & 2 \phantom{x} &  \quad 1 & 0 & 0 \\ 
	0 & 0 &  1 \phantom{x} &  \quad  1 &  1 & -1  \end{array} \right) \\[2.2em]
   	\begin{matrix} II - 2 \cdot III  \\ I - 2 \cdot III  \end{matrix} & &
  	\left(\begin{array} {c c c | c c c} 1 & 2 & 0 \phantom{x} &  \quad -2 & -1 & 2 \\ 
	0 & 1 & 0 \phantom{x} &  \quad -1 & -2 & 2 \\
	0 & 0 &  1 \phantom{x} &  \quad  1 &  1 & -1  \end{array} \right) \\[2.2em] 
   	I - 2 \cdot II & & 
  	\left(\begin{array} {c c c | c c c} 1 & 0 & 0 \phantom{x} &  \quad 0 & 3 & -2 \\ 
	0 & 1 & 0 \phantom{x} &  \quad -1 & -2 & 2 \\ 
	0 & 0 &  1 \phantom{x} &  \quad  1 &  1 & -1 \end{array} \right) 
   	\end{array} $$
Wir erhalten auch hier, dass $A$ invertierbar ist mit 
  	$$ A^{-1} = \left(\begin{matrix}  0 & 3 & -2 \\  -1 & -2 & 2 \\  1 &  1 & -1 \end{matrix} \right) $$
\end{beispiel}

\begin{beispiel} Nun betrachten wir die Matrix
   	$$ A = \left(\begin{matrix} 0 & 1 & 2 \\ 1 & 2 & 2 \\ 1 & 3 & 4 \end{matrix} \right) $$
und gehen vor wie folgt:
 	 $$ \begin{array} {l c l}
     	\left( A \quad  \, \, \vert \, \vektor{e_1} \, \vektor{e_2} \, \vektor{e_3} \right) 
    	& \quad = \quad & 
     	\left(\begin{array} {c c c | c c c}  0 & 1 & 2 \phantom{x} &  \quad  1 & 0 & 0 \\ 
	1 & 2 & 2 \phantom{x} &  \quad  0 & 1 & 0 \\ 
	1 & 3 & 4 \phantom{x} &  \quad  0 & 0 & 1  \end{array} \right) \\[2.2em] 
  	I \leftrightarrow II & & 
  	\left(\begin{array} {c c c | c c c}  1 & 2 & 2 \phantom{x} &  \quad  0 & 1 & 0 \\ 
	0 & 1 & 2 \phantom{x} &  \quad  1 & 0 & 0 \\ 
	1 & 3 & 4 \phantom{x} &  \quad  0 & 0 & 1   \end{array} \right) \\[2.2em] 
  	III - II & & 
  	\left(\begin{array} {c c c | c c c}  1 & 2 & 2 \phantom{x} &  \quad  0 & 1 & 0 \\ 
	0 & 1 & 2 \phantom{x} &  \quad  1 & 0 & 0 \\ 
	0 & 0 & 0 \phantom{x} &  \quad  -1 & -1 & 1  \end{array} \right) 
    	\end{array} $$
An dieser Stelle bricht der Algorithmus ab. Die letzte Zeile ist zur Nullzeile geworden, und damit hat die 
Matrix $A$ nicht den vollen Rang 3, ist also auch nicht invertierbar. 
\end{beispiel}

\bigbreak

Die elementaren Zeilenoperationen, die wir durchführen, um $A$ zu invertieren, 
lassen sich auch mit Hilfe von Matrizen beschreiben.

Wir betrachten dazu drei Typen von Elementarmatrizen, $M_i(\lambda)$ für ein 
$\lambda \in \mathbb R \setminus \{ 0 \}$, $P_{i,j}$ für $i \neq j$ und 
$S_{i,j}(\nu)$ für $i \neq j$ und $\nu \in \mathbb R$, die wie folgt 
definiert sind:

$M_i(\lambda) = \left(m_{l,k} \right)$ mit
  	$$ m_{l,k} = \left\{ \begin{array} {l c l} 1 & \quad & \textrm{ falls } l = k \, 
  	\textrm{ und } l \neq i \\
  	\lambda & & \textrm{ falls } l = k = i \\
  	0 & & \textrm{ sonst} \end{array} \right. $$

$P_{i,j} = \left(p_{l,k} \right)$ mit
  	$$  p_{l,k} = \left\{ \begin{array} {l c l} 1 & \quad & \textrm{ falls } l = k \, 
  	\textrm{ und } l \neq i, l \neq j \\
  	1 & & \textrm{ falls } l = i, k = j \\
  	1 & & \textrm{ falls } l = j, k = i \\
  	0 & & \textrm{ sonst} \end{array} \right. $$

$S_{i,j}(\nu) = \left(s_{l,k} \right)$ mit 
  	$$ s_{l,k} = \left\{ \begin{array} {l c l} 1 & \quad & \textrm{ falls } l = k \\
  	\nu & & \textrm{ falls } l = i, k = j \\
  	0 & & \textrm{ sonst} \end{array} \right. $$

\begin{beispiel} Es sei $n = 4$. 
  	$$ M_2\left(3\right) = \left( \begin{smallmatrix} 1 & 0 & 0 & 0 \\ 
     	0 & 3 & 0 & 0 \\ 0 & 0 & 1 & 0 \\ 0 & 0 & 0 & 1 \end{smallmatrix} \right), 
  	\quad P_{2,3} = \left( \begin{smallmatrix} 1 & 0 & 0 & 0 \\ 
     	0 & 0 & 1 & 0 \\ 0 & 1 & 0 & 0 \\ 0 & 0 & 0 & 1 \end{smallmatrix} \right), \quad 
  	S_{1,4}(7) = \left( \begin{smallmatrix} 1 & 0 & 0 & 7 \\ 
     	0 & 1 & 0 & 0 \\ 0 & 0 & 1 & 0 \\ 0 & 0 & 0 & 1 \end{smallmatrix} \right) $$ 
\end{beispiel}

\begin{notiz}\label{det_inverse_elementar} 
Diese Elementarmatrizen haben die folgende Eigenschaft:

\begin{itemize}
\item $M_i(\lambda) \cdot A$ ist die Matrix, die aus $A$ durch Mulitplikation der 
$i$--ten Zeile mit $\lambda$ hervorgeht.
\item $P_{i,j} \cdot A$ ist die Matrix, die aus $A$ durch Vertauschen der Zeilen $i$ 
und $j$ hervorgeht.
\item $S_{i,j}(\nu) \cdot A$ ist die Matrix, die aus $A$ durch Addition des 
$\nu$--fachen der $j$--ten Zeile zur $i$--ten Zeile hervorgeht. 
\end{itemize}

Damit können also die Zeilenoperationen, die eine invertierbare Matrix $A$ in die 
Einheitsmatrix $E_n$ überführen, als Multiplikationen mit diesen 
Elementarmatrizen beschrieben werden. Es gibt also Elementarmatrizen 
$\mathcal{E}_1, \ldots, \mathcal{E}_l$ mit 
  	$$ \mathcal{E}_1 \cdots \mathcal{E}_l \cdot A = E_n $$
und damit gilt 
  	$$ A^{-1} = \mathcal{E}_1 \cdots \mathcal{E}_l \quad \textrm{ und } 
  A = \mathcal{E}_l^{-1} \cdots \mathcal{E}_1^{-1} $$
Da die Inverse einer Elementarmatrix wieder eine Elementarmatrix ist, 
	$$ M_i(\lambda)^{-1} = M_i\left(\frac{1}{\lambda}\right), \quad P_{i,j}^{-1}=P_{i,j} \,\, \text{ und } \, 
	S_{i,j}(\\nu)^{-1} = S{i,j}(-\nu) $$ 
haben wir gezeigt:  Jede invertierbare Matrix $A$ ist Produkt von Elementarmatrizen, 
  	$$ A = \mathcal{E}_1 \cdots \mathcal{E}_l $$
\end{notiz}

\bigbreak

Invertierbare Matrizen $A$  sind dadurch charakterisiert, dass $\det{A} \neq 0$. 
Determinanten können aber auch dazu benutzt werden, um inverse Matrizen zu berechnen:

\begin{satz}\label{det_inverse_komplement} Ist $A$ invertierbar, so gilt
  	$$ A^{-1} = \frac {1}{\det{A}} \cdot  \widetilde{A} $$
mit der zu $A$ komplementären Matrix $\widetilde{A}$ aus 
Abschnitt~\ref{section_det_n}.
\end{satz}

\beweis{ Das haben haben wir im Prinzip schon mit Satz~\ref{det_n_komplement} gezeigt.
}

\begin{beispiel} Für die Matrix 
  	$ A = \left( \begin{matrix} 1 & 2 \\ 1 & 3 \end{matrix} \right) $
haben wir schon in Beispiel~\ref{det_kompl_2} gesehen, dass $\det{A} = 1$ und  
  	$$ \widetilde{A} = \left( \begin{matrix} 3 & -2 \\ -1 & 1 \end{matrix} \right) $$
Damit ist $A^{-1} = \widetilde{A}$.
\end{beispiel}

\begin{beispiel} Ist 
  	$ A = \left( \begin{matrix} a & b \\ c & d \end{matrix} \right) $
eine beliebige $2 \times 2$--Matrix so ist
  	$$ \widetilde{A} = \left( \begin{matrix} d & -b \\ -c & a \end{matrix} \right) $$
Ist also $ad - bc \neq 0$, so ist $A$ invertierbar mit
  	$$ A^{-1} = \frac {1}{ad - bc} \left( \begin{matrix} d & -b \\ -c & a  \end{matrix} \right) $$
\end{beispiel} 
  
\begin{beispiel} Ist 
  	$ A = \left( \begin{matrix} 1 & 0 & -1 \\ 8 & -4 & -1 \\ -2 & 1 & 0 \end{matrix} \right) $
so haben wir in Beispiel~\ref{det_kompl_2} gesehen, dass $\det{A} = 1$ und
  	$$ \widetilde{A} = \left( \begin{matrix} 1 & -1 & -4 \\ 2 & -2 & -7 \\ 0 & -1 & -4 \end{matrix} \right) $$
Daher ist $A^{-1} = \widetilde{A}$.
\end{beispiel}

\medbreak

\begin{notiz} Für große $n$ ist es unpraktikabel, die inverse Matrix über 
die Komplementärmatrix zu berechnen, da der Rechenaufwand zu hoch ist.
\end{notiz}

\bigbreak

Im Allgemeinen ist es nicht einfach, zu entscheiden, ob eine $n \times n$--Matrix invertierbar ist 
(speziell für großes $n$) und gegegebenenfalls ihr Inverses 
zu berechnen. Einige Regeln helfen dabei manchmal.

\begin{regel} 

\begin{itemize} 
\item Sind $A$ und $B$ invertierbare $n \times n$--Matrizen, so ist auch $A \cdot B$ invertierbar und 
  $$ \left( A \cdot B \right)^{-1} = B^{-1} \cdot A^{-1} $$
\item Ist $A$ eine Diagonalmatrix, 
  $$ A = \left(\begin{matrix} d_1 & 0 & \ldots & 0 \\ 0 & d_2 & \ldots & 0 \\
     0 & 0 & \ddots & 0 \\ 0 & 0 & \ldots & d_n \end{matrix} \right),$$ 
so ist $A$ genau dann invertierbar, wenn $d_i \neq 0$ für 
alle $i$, und in diesem Fall gilt
  $$ A^{-1} = \left( \begin{matrix} \frac {1}{d_1} & 0 & \ldots & 0 \\ 0 & \frac {1}{d_2} & \ldots & 0 \\
     0 & 0 & \ddots & 0 \\ 0 &  & \ldots &  \frac {1}{d_n} \end{matrix} \right) $$
\item Ist $A$ eine $n \times n$--Matrix mit $A^l = 0$ für ein $l \in \mathbb N$, so ist $E_n - A$ invertierbar 
und $\left(E_n - A\right)^{-1} = E_n + A + A^2 + \cdots + A^{l-1} $.
\end{itemize}
\end{regel}

\beweis{ Die Aussagen (1) und (2) lassen sich unmittelbar nachrechnen. Zum Nachweis von (3) rechnen wir
  	$$ \begin{array} {l c l } 
    	\left( E_n - A \right) &\cdot& \left( E_n + A + A^2 + \cdots + A^{l-1} \right) \\
    	&=& E_n + A + A^2 + \cdots + A^{l-1} - A - A^2 - A^3 - \cdots - A^l \\
    	&=& E_n -A^l \\
    	&=& E_n
  	\end{array} $$
Also ist $E_n - A$ invertierbar.
}

\begin{notiz} Eine Matrix $A$ mit $A^l = 0$ für ein $l \in \mathbb N$ heißt \textbf{nilpotent}.
\end{notiz}

\medbreak

\begin{definition} Eine Matrix $n \times n$--Matrix $A$ heißt \index{Matrix!orthogonal}\textbf{orthogonal}, 
wenn die Spalten von $A$ eine Orthonormalbasis von $\mathbb R^n$ bilden.
\end{definition}

\begin{beispiel}\label{det_matrix_orth} Die Matrix 
  $$A = \left( \begin{matrix} \frac {\sqrt{2}}{2} & \frac {\sqrt{2}}{2} \\
                              -\frac {\sqrt{2}}{2} & \frac {\sqrt{2}}{2} \end{matrix} \right) $$
ist orthogonal.
\end{beispiel}

\begin{satz}\label{det_matrix_orth_inv} Ist $A$ eine orthogonale Matrix, so ist $A$ invertierbar und 
  	$$ A^{-1} = A^T $$
ist die Transponierte von $A$.
\end{satz}

\beweis{ Wir bezeichnen mit $\vektor{u_1}, \ldots , \vektor{u_n}$ die Spaltenvektoren von $A$. 
Dann gilt nach Definition der Matrizenmultiplikation
  	$$ \begin{array}{l c l}
     	A^T \cdot A & = & \left( \begin{matrix} \langle \vektor{u_1}, \vektor{u_1} \rangle & 
     	\langle \vektor{u_2}, \vektor{u_1} \rangle & \ldots & \langle \vektor{u_n}, 
    	\vektor{u_1} \rangle  \\
    	\langle \vektor{u_1}, \vektor{u_2} \rangle & \langle \vektor{u_2}, 
     	\vektor{u_2} \rangle & \ldots & \langle \vektor{u_n}, \vektor{u_1} \rangle \\
    	\vdots &  & \ddots & \vdots \\
    	\langle \vektor{u_1}, \vektor{u_n} \rangle & \langle \vektor{u_2}, 
     	\vektor{u_n} \rangle & \ldots & \langle \vektor{u_n}, \vektor{u_n} \rangle 
    	\end{matrix} \right) \\
   	& = & E_n 
  	\end{array} $$
nach Definition einer Orthonormalbasis.
}

\medbreak

\begin{beispiel} Die Matrix
  	$$A = \left( \begin{matrix} \frac {\sqrt{2}}{2} & \frac {\sqrt{2}}{2} \\
	-\frac {\sqrt{2}}{2} & \frac {\sqrt{2}}{2} \end{matrix} \right) $$
ist invertierbar mit inverser Matrix 
  	$$A^{-1} = \left( \begin{matrix} \frac {\sqrt{2}}{2} & -\frac {\sqrt{2}}{2} \\
	\frac {\sqrt{2}}{2} & \frac {\sqrt{2}}{2} \end{matrix} \right) $$
\end{beispiel}

\begin{korollar}\label{det_matrix_orth_skalar}
Ist $A$ eine orthogonale $n \times n$--Matrix, so gilt  
  	$$ \langle A \cdot \vektor{v},\, A \cdot \vektor{w} \rangle = 
     	\langle \vektor{v},\, \vektor{w} \rangle \qquad \textrm{ für alle }
     	\vektor{v},\, \vektor{w} \in \mathbb R^n $$
\end{korollar}

\beweis{ Wir benutzen Regel~\ref{matrix_skalarprod} und Satz~\ref{det_matrix_orth_inv} und erhalten
  	$$ \begin{array} {l c l}
  	\langle A \cdot \vektor{v},\, A \cdot \vektor{w} \rangle & = & 
  	\langle \vektor{v},\, A^T \cdot A \cdot \vektor{w} \rangle \\
  	& = & \langle \vektor{v},\, A^{-1} \cdot A \cdot \vektor{w} \rangle \\
  	& = & \langle \vektor{v},\, \vektor{w} \rangle
  	\end{array} $$
}
	
\bigbreak

\begin{notiz} Orthogonale Matrizen haben folgende interessante Eigenschaft: Ist $A$ eine orthogonale 
Matrix und $\vektor{v}$ ein beliebiger Vektor, so gilt
  	$$ \vert A \cdot \vektor{v} \vert = \vert \vektor{v} \vert $$
Das folgt jetzt sofort aus Korollar~\ref{det_matrix_orth_skalar}, denn damit gilt
  	$$ \begin{array} {l c l}
   	\vert A \cdot \vektor{v} \vert^2 
   	& = & \langle A \cdot \vektor{v}, \, A \cdot \vektor{v} \rangle \\
   	& = & \langle \vektor{v},\, \vektor{v} \rangle \\
   	& = & \vert \vektor{v} \vert^2
  	\end{array} $$
Durch Wurzelziehen erhalten wir die Behauptung.

Eine Matrix mit $ \vert A \cdot \vektor{v} \vert = \vert \vektor{v} \vert $ für jeden 
Vektor $\vektor{v} \in \R^n$ nennen wir auch eine \index{Isometria}\textbf{Isometrie}. Jede orthogonale Matrix 
ist also eine Isometrie. 
\end{notiz}

\begin{beispiel}\label{lin_alg_ortho_dreh} 
Ist $A$ eine orthogonale $2 \times 2$--Matrix, so gibt es ein $\alpha \in [0, \, 2 \pi[$ mit
  	$$ A = \left( \begin{matrix} \cos(\alpha) & -\sin(\alpha) \\ \sin(\alpha) & \cos(\alpha) \end{matrix} \right) 
  	\quad \textrm{ oder } \, 
  	A = \left( \begin{matrix} \cos(\alpha) & \sin(\alpha) \\ \sin(\alpha) & -\cos(\alpha) \end{matrix} \right) $$
Dazu schreiben wir zunächst ganz allgemein 
  	$$ A = \left( \begin{matrix} a & b  \\ c & d \end{matrix} \right) $$
mit reellen Zahlen $a, b, c$ und $d$. Da $A$ orthogonal ist, muss gelten $A^T \cdot A = E_2$, also 
  	$$ \left( \begin{matrix} a & c  \\ b & d \end{matrix} \right) \cdot 
  	\left( \begin{matrix} a & b  \\ c & d \end{matrix} \right) = 
  	\left( \begin{matrix} 1 & 0  \\ 0 & 1 \end{matrix} \right) $$
Multiplizieren wir das aus, so ergeben sich folgende drei Beziehungen
  	$$ \begin{array} {l c l c l}
  	a^2 & + & c^2 & = & 1 \\
  	ab & + & cd & = & 0 \\
  	b^2 & + & d^2 & = & 1 
  	\end{array} $$
Die erste Beziehung besagt, dass $\vektor{v} = \left( \begin{matrix} a  \\ c \end{matrix} \right)$ ein 
Vektor der Länge $a$ ist. Schreiben wir als $\vektor{v}$ in Polarkoordinaten, so bedeutet das, dass 
es genau ein $\alpha \in [0, \, 2\pi)$ gibt mit
  	$$ \vektor{v} = \left( \begin{matrix} \cos(\alpha) \\ \sin(\alpha) \end{matrix} \right) $$
also $a = \cos(\alpha)$ und $c = \sin(\alpha)$. Genauso finden wir aus der dritten Beziehung ein $\beta$ mit 
$b = \sin(\beta)$ und $d = \cos(\beta)$.

Setzen wir diese Resultate in die zweite Beziehung ein und nutzen auch noch die Additionstheoreme für Sinus 
und Kosinus aus, so erhalten wir
  	$$ \begin{array} {l c l}
  	0 & = & \cos(\alpha) \sin(\beta) + \sin(\alpha) \cos(\beta) \\
  	& = & \sin(\alpha + \beta) 
  	\end{array} $$
Da $\alpha, \beta \in [0, 2\pi[$ kann das nur dann der Fall sein, wenn 
  	$$ \alpha + \beta \in \{0, \pi, 2 \pi, 3 \pi \} $$
Falls nun $\alpha + \beta = 0$ oder $\alpha + \beta = 2\pi$, so gilt
  	$$ \sin(\beta) = - \sin(\alpha), \quad \cos(\beta) = \cos(\alpha) $$
und wir erhalten also 
  	$$  A = \left( \begin{matrix} \cos(\alpha) & -\sin(\alpha) \\ \sin(\alpha) & \cos(\alpha) \end{matrix} \right) $$
Ist dagegen $\alpha + \beta = \pi$ oder $\alpha + \beta = 3 \pi$, so gilt
  	$$ \sin(\beta) = \sin(\alpha), \quad \cos(\beta) = -\cos(\alpha) $$
und wir erhalten also 
  	$$  A = \left( \begin{matrix} \cos(\alpha) & \sin(\alpha) \\ \sin(\alpha) & -\cos(\alpha) \end{matrix} \right) $$
 
Die orthogonale Matrix 
  	$$  A = \left( \begin{matrix} \cos(\alpha) & -\sin(\alpha) \\ \sin(\alpha) & \cos(\alpha) \end{matrix} \right) $$
beschreibt eine \index{Drehung}\textbf{Drehung} 
um den Winkel $\alpha$, und wir schreiben hierfür auch $D_{\alpha}$. Ist etwa 
$\alpha = \frac {\pi}{3}$ (also $\alpha = 60^{\circ}$ 
im Winkelmaß), so ist
  	$$ D_{60^{\circ}} =  \left( \begin{matrix} \frac {1}{2} & -\frac {\sqrt{3}}{2} \\ 
	\frac {\sqrt{3}}{2} & \frac {1}{2} \end{matrix} \right) $$
Das ergibt folgendes Bild

%\begin{figure}[H]
%  \centering
%     \scalebox{0.60}{\input{orthogonal_dreh.tex}}
%\end{figure}

\begin{figure}[H]
	\vspace{-0.5cm}
	\begin{center}
	\begin{scaletikzpicturetowidth}{\hsize}
     		\begin{tikzpicture}[gnuplot,scale= \tikzscale]
%% generated with GNUPLOT 5.0p4 (Lua 5.3; terminal rev. 99, script rev. 100)
%% 21.11.2016 18:03:41
\path (0.000,0.000) rectangle (12.500,8.750);
\gpcolor{color=gp lt color border}
\node[gp node left] at (11.216,0.882) {$x$};
\node[gp node left] at (4.533,7.970) {$y$};
\node[gp node left] at (8.292,3.964) {$\vektor{v}$};
\node[gp node left] at (1.504,5.196) {$A \cdot \vektor{v}$};
\gpcolor{rgb color={0.000,0.620,0.451}}
\gpsetlinetype{gp lt border}
\gpsetdashtype{gp dt solid}
\gpsetlinewidth{2.00}
\draw[gp path] (6.258,2.391)--(6.222,2.425)--(6.185,2.457)--(6.147,2.489)--(6.108,2.519)%
  --(6.068,2.548)--(6.027,2.576)--(5.985,2.602)--(5.942,2.627)--(5.899,2.651)--(5.855,2.673)%
  --(5.810,2.694)--(5.764,2.714)--(5.718,2.732)--(5.672,2.749)--(5.624,2.764)--(5.577,2.778)%
  --(5.529,2.790)--(5.480,2.801)--(5.431,2.810)--(5.382,2.818)--(5.333,2.824)--(5.283,2.829)%
  --(5.234,2.832)--(5.184,2.833)--(5.134,2.833)--(5.085,2.832)--(5.035,2.829)--(4.986,2.824)%
  --(4.936,2.818)--(4.887,2.810)--(4.838,2.801)--(4.790,2.790)--(4.742,2.778)--(4.694,2.764)%
  --(4.647,2.749)--(4.600,2.732)--(4.554,2.714)--(4.509,2.694)--(4.464,2.673)--(4.419,2.651)%
  --(4.376,2.627)--(4.333,2.602);
%
  \gpfill{rgb color={0.000,0.000,0.000}} (11.590,1.293)--(11.590,1.397)--(11.843,1.293)--(11.590,1.189)--cycle;
\gpcolor{rgb color={0.000,0.000,0.000}}
\gpsetlinewidth{1.00}
\draw[gp path] (11.590,1.293)--(11.590,1.397)--(11.843,1.293)--(11.590,1.189)--cycle;
\draw[gp path] (0.564,1.293)--(11.590,1.293);
%
  \gpfill{rgb color={0.000,0.000,0.000}} (5.159,8.025)--(5.055,8.025)--(5.159,8.278)--(5.263,8.025)--cycle;
\draw[gp path] (5.159,8.025)--(5.055,8.025)--(5.159,8.278)--(5.263,8.025)--cycle;
\draw[gp path] (5.159,0.471)--(5.159,8.025);
%
  \gpfill{rgb color={1.000,0.000,0.000}} (9.147,5.217)--(9.070,5.295)--(9.336,5.402)--(9.224,5.138)--cycle;
\gpcolor{rgb color={1.000,0.000,0.000}}
\gpsetlinewidth{2.00}
\draw[gp path] (9.147,5.217)--(9.070,5.295)--(9.336,5.402)--(9.224,5.138)--cycle;
\draw[gp path] (5.159,1.293)--(9.147,5.217);
%
  \gpfill{rgb color={1.000,0.000,0.000}} (2.338,6.098)--(2.243,6.042)--(2.204,6.326)--(2.432,6.153)--cycle;
\draw[gp path] (2.338,6.098)--(2.243,6.042)--(2.204,6.326)--(2.432,6.153)--cycle;
\draw[gp path] (5.159,1.293)--(2.338,6.098);
\gpcolor{color=gp lt color border}
\node[gp node left] at (5.264,3.142) {$60^{\circ}$};
%% coordinates of the plot area
\gpdefrectangularnode{gp plot 1}{\pgfpoint{0.460cm}{0.368cm}}{\pgfpoint{11.947cm}{8.381cm}}
\end{tikzpicture}
%% gnuplot variables
 
	\end{scaletikzpicturetowidth}
	\vspace{-0.5cm}
	\caption{Drehung um $60^{\circ}$}\label{lin:05_det_3100}
	\end{center}
	\vspace{-0.5cm}
\end{figure}

 
Die orthogonale Matrix 
  	$$  A = \left( \begin{matrix} \cos(\alpha) & \sin(\alpha) \\ \sin(\alpha) & -\cos(\alpha) \end{matrix} \right) $$
dagegen beschreibt eine \index{Spiegelung}\textbf{Spiegelung} an der Geraden mit Winkel $\frac {\alpha}{2}$ 
zur $x$--Achse, und wir schreiben hierfür $S_{\frac {\alpha}{2}}$. Ist wieder 
$\alpha = \frac {\pi}{3}$ (also $\alpha = 60^{\circ}$ im Winkelmaß), so ist
  	$$ S_{30^{\circ}} =  \left( \begin{matrix} \frac {1}{2} & \frac {\sqrt{3}}{2} \\ 
	\frac {\sqrt{3}}{2} & -\frac {1}{2} \end{matrix} \right) $$
Das ergibt dann Bild inAbbildung~\ref{lin_05_det_3110} (mit Spiegelungsachse $S$)

%\begin{figure}[H]
%  \centering
%     \scalebox{0.60}{\input{orthogonal_spiegel.tex}}
%\end{figure}

\begin{figure}[h]
	\begin{center}
	\vspace{-0.5cm}
	\begin{scaletikzpicturetowidth}{\hsize}
     		\begin{tikzpicture}[gnuplot,scale= \tikzscale]
%% generated with GNUPLOT 5.0p4 (Lua 5.3; terminal rev. 99, script rev. 100)
%% 21.11.2016 18:03:50
\path (0.000,0.000) rectangle (12.500,8.750);
\gpcolor{color=gp lt color border}
\node[gp node left] at (10.966,0.790) {$x$};
\node[gp node left] at (2.001,7.819) {$y$};
\node[gp node left] at (11.106,5.429) {$S$};
\node[gp node left] at (6.344,5.710) {$\vektor{v}$};
\node[gp node left] at (8.585,2.196) {$A \cdot \vektor{v}$};
\gpcolor{rgb color={0.000,0.620,0.451}}
\gpsetlinetype{gp lt border}
\gpsetdashtype{gp dt solid}
\gpsetlinewidth{2.00}
\draw[gp path] (5.434,1.831)--(5.413,1.918)--(5.390,2.004)--(5.363,2.089)--(5.334,2.173)%
  --(5.302,2.256)--(5.268,2.339)--(5.231,2.420)--(5.192,2.500)--(5.150,2.578)--(5.105,2.656)%
  --(5.058,2.732)--(5.009,2.806)--(4.957,2.878)--(4.904,2.949)--(4.848,3.019)--(4.789,3.086)%
  --(4.729,3.152)--(4.667,3.215);
%
  \gpfill{rgb color={0.000,0.000,0.000}} (11.554,1.211)--(11.554,1.315)--(11.807,1.211)--(11.554,1.107)--cycle;
\gpcolor{rgb color={0.000,0.000,0.000}}
\gpsetlinewidth{1.00}
\draw[gp path] (11.554,1.211)--(11.554,1.315)--(11.807,1.211)--(11.554,1.107)--cycle;
\draw[gp path] (0.600,1.211)--(11.554,1.211);
%
  \gpfill{rgb color={0.000,0.000,0.000}} (2.701,7.987)--(2.597,7.987)--(2.701,8.240)--(2.805,7.987)--cycle;
\draw[gp path] (2.701,7.987)--(2.597,7.987)--(2.701,8.240)--(2.805,7.987)--cycle;
\draw[gp path] (2.701,0.509)--(2.701,7.987);
%
  \gpfill{rgb color={1.000,0.000,0.000}} (8.118,6.648)--(8.041,6.725)--(8.305,6.835)--(8.196,6.570)--cycle;
\gpcolor{rgb color={1.000,0.000,0.000}}
\gpsetlinewidth{2.00}
\draw[gp path] (8.118,6.648)--(8.041,6.725)--(8.305,6.835)--(8.196,6.570)--cycle;
\draw[gp path] (2.701,1.211)--(8.118,6.648);
%
  \gpfill{rgb color={1.000,0.000,0.000}} (10.148,2.911)--(10.124,3.017)--(10.406,2.969)--(10.172,2.803)--cycle;
\draw[gp path] (10.148,2.911)--(10.124,3.017)--(10.406,2.969)--(10.172,2.803)--cycle;
\draw[gp path] (2.701,1.211)--(10.148,2.911);
\gpcolor{rgb color={0.000,0.000,1.000}}
\draw[gp path] (1.150,0.368)--(11.807,6.160);
\gpcolor{color=gp lt color border}
%\node[gp node left] at (5.293,2.336) {$\frac{\alpha}{2}$};
%\node[gp node left] at (4.873,3.180) {$\frac{\alpha}{2}$};
\node[gp node left] at (5.293,2.336) {$\beta$};
\node[gp node left] at (4.873,3.180) {$\beta$};
%% coordinates of the plot area
\gpdefrectangularnode{gp plot 1}{\pgfpoint{0.460cm}{0.368cm}}{\pgfpoint{11.947cm}{8.381cm}}
\end{tikzpicture}
%% gnuplot variables
 
	\end{scaletikzpicturetowidth}
	\vspace{-0.5cm}
	\caption{Spiegelung an einer Achse}\label{lin:05_det_3110}\label{lin_05_det_3110}
	\end{center}
	\vspace{-0.5cm}
\end{figure}

\end{beispiel}

\begin{notiz} 
%Spezielle orthogonale Matrizen, nämlich die Drehmatrizen aus Beispiel~\ref{lin_alg_ortho_dreh} 
%sind uns implizit schon im Abschnitt~\ref{section_kegel} über Kegelschnitte begegnet. Dort haben wir, um 
%die gemischten Produkte $xy$ zu eliminieren, eine Koordinatentransformation der Form
%  	$$ \begin{array} {l c l c l}
%  	\widetilde{x} & = & ax & - & by \\ 
%  	\widetilde{y} & = & bx & + & ay 
%  	\end{array} $$
%mit $a^2 + b^2 = 1$ durchgeführt. In Matrizenschreibweise bedeutet das 
% 	$$ \left( \begin{matrix} \widetilde{x} \\ \widetilde{y} \end{matrix} \right) = 
%  	\left( \begin{matrix} a & -b  \\ b & a \end{matrix} \right) \cdot \left( \begin{matrix} x \\ 
%   	y \end{matrix} \right) $$
%wobei 
%  	$$ A = \left( \begin{matrix} a & -b  \\ b & a \end{matrix} \right) $$
%eine Drehmatrix ist, denn wir können wieder ein $\alpha \in [0, \, 2 \pi)$ wählen mit $a = \cos(\alpha)$ 
%und $b = \sin(\alpha)$. Das sehen wir also, das die Koordinatentransformation, die wir im 
%Abschnitt~\ref{section_kegel} durchgeführt haben, tatsächlich eine Drehung der Koordinatenachsen um 
%den Winkel $\alpha$ ist. 

Bei der geometrischen Betrachtung komplexer Zahlen spielen Drehungen (zusammen mit Streckungen) eine 
wichtige Rolle. Ist eine komplexe Zahl $z$ in Polarkoordinatendarstellung durch die Länge $r$ und den Winkel 
$\alpha$ gegeben, so ist die Multiplikation mit $z$ eine Drehstreckung mit Winkel $\alpha$ und Streckfaktor $r$, 
schreibt sich also in Matrizenschreibweise als Multiplikation mit der Matrix
  	$$ \mu_z = r \cdot D_{\alpha} = r \cdot \left( \begin{matrix} \cos(\alpha) & -\sin(\alpha) \\ 
      \sin(\alpha) & \cos(\alpha) \end{matrix} \right) $$
\end{notiz}

\begin{notiz} Für jeden Winkel $\alpha$ gilt:
  	$$ \det{D_{\alpha}} = 1, \quad \det{S_{\alpha}} = -1 $$
\end{notiz}

\bigbreak

Im letzten Abschnitt haben wir für diesen Paragraphen eine Beweisskizze des 
Produktsatzes für Determinanten versprochen:

\begin{satz}[Produktsatz]\label{det_n_produkt_revisit}  
Sind $A$ und  $B$ zwei $n \times n$--Matrizen, so gilt
  	$$ \det{ A \cdot B } = \det{A} \cdot \det{B} $$
\end{satz}

\bemerkung{Beweisskizze \,} 

1. Fall: Wir betrachten zunächst den Fall, dass $\mathrm{rg}(A) < n$ oder $\mathrm{rg}(B) < n$. Dann ist auch
$\mathrm{rg}(A \cdot B) < n$.

Ist etwa $\mathrm{rg}(B) < n$, so ist nach dem Rangsatz $\textrm{nul}(B) > 0$, also 
gibt es ein $\vektor{x} \in \mathbb R^n$ mit $\vektor{x} \neq
\vektor{0}$ und $B \cdot \vektor{x} = \vektor{0}$. Dann gilt 
aber auch 
  	$$A \cdot B \cdot \vektor{x} = A \cdot \vektor{0} = 
   	\vektor{0}, $$ 
also $\textrm{nul}(A \cdot B) > 0$ und damit 
$\mathrm{rg}(A \cdot B) < n$. Ist $\mathrm{rg}(B) = n$ und 
$\mathrm{rg}(A) < n$ so gibt es zunächst ein $\vektor{y}$ mit 
$\vektor{y} \neq \vektor{0}$ und $A \cdot \vektor{y} = 
\vektor{0}$. Da $\mathrm{rg}(B) = n$, gibt es ein $\vektor{x}$ mit
$B \cdot \vektor{x} = \vektor{y}$ und hierfür gilt
  	$$ A \cdot B \cdot \vektor{x} = A \cdot \vektor{x} = 
   	\vektor{0}, $$
also auch in dieser Situation $\textrm{nul}(A \cdot B) > 0$ und damit 
$\mathrm{rg}(A \cdot B) < n$.

Damit gilt in diesem Fall also wegen Satz~\ref{det_n_regular}:
  	$$ \det{A \cdot B} = 0 = \det{A} \cdot \det{B} $$

\medbreak

2. Fall: Es gilt $\mathrm{rg}(A) = n$ und $\mathrm{rg}(B) = n$. Damit brauchen wir 
also nur noch invertierbare Matrizen betrachten. Nach 
Bemerkung~\ref{det_inverse_elementar} schreibt sich dann $A$ als Produkt von 
Elementarmatirzen, $ A = \mathcal{E}_1 \cdots \mathcal{E}_l $, und es reicht zu 
zeigen 
  	$$\det{\mathcal{E} \cdot B}  = \det{\mathcal{E}} \cdot \det{B} $$
für jede (invertierbare) Matrix $B$ und jede Elementarmatrix $\mathcal{E}$. 
Das rechnet man leicht mit den Regeln~\ref{det_n_rule} nach.

\bigbreak

\begin{korollar} Sind $A$ und  $B$ zwei $n \times n$--Matrizen, und ist $A \cdot B$ 
invertierbar, so sind auch $A$ und $B$ invertierbar. Entsprechend haben in diesem 
Fall auch $A$ und $B$ den vollen Rang $n$.
\end{korollar}

\beweis{ 
Die Invertierbarkeit von $A \cdot B$ ist äquivalent zu $\det{A \cdot B} 
\neq 0$. Nach dem Produktsatz folgt aber daraus $\det{A} \neq 0$ und $\det{B} 
\neq 0$.
}

\bigbreak

Zum Abschluß dieses Abschnitts und dieses Kapitels wollen wir nochmal zum 
Ausgangspunkt dieses Kapitels zurückkehren, nämlich zu linearen 
Gleichungssystemen 
  	$$ A \cdot \vektor{x} = \vektor{b} $$
mit einer $n \times n$--Koeffizientenmatrix $A$. Wir schreiben $A = \left(
\vektor{a_1} \,\, \vektor{a_2} \, \ldots \, \vektor{a_n} 
\right)$ mit Spaltenvektoren $\vektor{a_i}$ und setzen
  	$$ A_i(\vektor{b}) = \left( \begin{matrix}
  	\vektor{a_1} & \ldots & \vektor{a_{i-1}} &
   	\vektor{b} & \vektor{a_{i+1}} & \ldots & \vektor{a_n} 
  	\end{matrix} \right), $$
wir ersetzen also die $i$--te Spalte von $A$ durch $\vektor{b}$.

\begin{satz}[Cramersche Regel]\label{det_n_cramer} \index{Cramersche Regel!allgemein}  
Ist $A$ invertierbar, so ist die eindeutige Lösung $\vektor{x}$ von 
  	$$ A \cdot \vektor{x} = \vektor{b} $$
gegeben durch 
  	$$ x_i = \frac {\det{A_i(\vektor{b})}}{\det{A}}. $$
\end{satz} 

\beweis{ Da $A$ invertierbar ist, ist die Lösung des Gleichungssystems gegeben durch
  	$$ \vektor{x} = A^{-1} \vektor{b} $$
Drücken wir $A^{-1} $ gemäß Satz~\ref{det_inverse_komplement} mit der 
Komplementärmatrix aus, so bedeutet das 
  	$$ \begin{array} {l c l}
     	x_i & = & \left(A^{-1} \cdot \vektor{b}\right)_i \\[0.3em]
     	& = & \sum\limits_{l = 1}^n \frac {\widetilde{a_{i,l}}}{\det{A}} \cdot b_l \\[0.3em]
     	& = & \frac {1}{\det{A}} \cdot \sum\limits_{i=1}^n (-1)^{i+l} \cdot \det{A_{l,i}} 
     	\cdot b_l
   	\end{array} $$
Berechnen wir andererseits $\det{A_i(\vektor{b})}$ durch Entwicklung nach 
der $i$--ten Spalte, so erhalten wir
  	$$ \det{A_i(\vektor{b})} = \sum_{l=1}^n (-1)^{i+l} \cdot b_l \cdot \det{A_{l,i}} $$
da $A_i(\vektor{b})_{l,i} = A_{l,i}$, und es folgt die Behauptung.
}
   
\begin{beispiel} Wir betrachten das lineare Gleichungssystem
 	 $$ \begin{array} {r c r c r c l}
  	x_1  & + & x_2 & & & = & 2 \\
  	3x_1 & + & 2x_2 & + & x_3 & = & 0 \\
  	& & x_2 & + & x_3 & = & 2 
  	\end{array} $$
haben also 
  	$$ A = \left( \begin{matrix} 1 & 1 & 0 \\ 3 & 2 & 1 \\ 0 & 1 & 1
   	\end{matrix} \right), \qquad \vektor{b} 
    	= \left( \begin{matrix} 2 \\ 0 \\ 2 \end{matrix} \right) $$ 
Da $\det{A} = -2$ kann die Cramersche Regel angewendet werden und wir erhalten
  	$$ \begin{array}{l c l c l c c l c l c l }
  	x_1 & = & \dfrac {-1}{2} \cdot \det{ \begin{matrix} 2 & 1 & 0 \\ 0 & 2 & 1 \\ 2 & 1 & 1 
    	\end{matrix} } & = & - 2 & \quad &
  	x_2 & = & \dfrac {-1}{2} \cdot \det{ \begin{matrix} 1 & 2 & 0 \\ 3 & 0 & 1 \\ 2 & 2 & 1 
    	\end{matrix} } & = & 4 \\
  	x_3 & = &  \dfrac {-1}{2} \cdot \det{ \begin{matrix} 1 & 1 & 2 \\ 3 & 2 & 0 \\ 0 & 1 & 2 
    	\end{matrix} } & = & -2
 	\end{array} $$
\end{beispiel}

\begin{notiz} Die Cramersche Regel ist keine praktikable Methode, um die Lösung eines 
Gleichungssystems zu bestimmen (ausser vielleicht im Fall $n = 2$). Der Rechenaufwand, der 
damit verbunden ist, ist viel höher als der, der sich aus dem Gaußschen 
Eliminationsverfahren ergibt.
\end{notiz}

\bigbreak

\begin{notiz} Determinanten können mit den gleichen Berechnungsregeln auch für komplexe Matrizen 
definiert werden. Betrachten wir etwa die Matrix 
  	$$ A = \left( \begin{matrix} 1 & \ii \\ \ii & \ii-1 \end{matrix} \right) $$
so gilt hierfür
  	$$ \det{A} = 1 \cdot (\ii-1) - \ii \cdot \ii = \ii -1 - (-1) = \ii $$
Die Aussagen über Determinanten übertragen sich entsprechend. Insbesondere ist eine komplexe Matrix 
invertierbar, wenn ihre Determinante von Null verschieden ist. Das gilt also für unsere Matrix $A$, und 
ihre Inverse berechnet sich wie im reellen Fall mit der erweiterten Matrix:
  	$$ \left( A \, \vert \, E_2 \right) = \left( \begin{array} {c c | c c } 1 & \ii \phantom{x}& \quad  1 & 0 \\ 
	i & \ii-1 \phantom{x}& \quad 0 & 1 \end{array} \right) $$
Subtrahieren wir das $i$--fache der ersten Zeile von der zweiten, so erhalten wir
  	$$ \left(\begin{array} {c c | c c }  1 & \ii \phantom{x}& \quad  1 & 0 \\ 
	0 & \ii \phantom{x}& \quad  -\ii & 1 \end{array} \right) $$
Division der zweiten Zeile durch $i$ ergibt
  	$$ \left(\begin{array} {c c | c c }  1 & \ii \phantom{x}& \quad  1 & 0 \\ 
	0 & 1 \phantom{x}& \quad  -1 & -\ii \end{array} \right) $$
Subtrahieren wir das $\ii$--fache der zweiten Zeile von der ersten, so ergibt sich 
  	$$  \left( E_2 \, \, \vert \, A^{-1} \right) = \left( \begin{array} {c c | c c }  1 & 0 \phantom{x}& \quad  1+\ii & -1 \\ 
	0 & 1 \phantom{x}& \quad  -1 & -\ii \end{array} \right) $$
und damit
  	$$ A^{-1} = \left( \begin{matrix} 1+\ii & -1 \\  -1 & -\ii \end{matrix} \right) $$
\end{notiz}

\medbreak

\begin{notiz}
Die Theorie der Determinanten und invertierbaren Matrizen kann genauso auf jeden anderen Körper, etwa 
$\Q$ oder $\F_p$ für eine Primzahl $p> 0$ übertragen werden. Alle Regeln und Berechnungsformeln 
gelten entsprechend.
\end{notiz}

\bigbreak


\begin{aufgabe} Bestimmen Sie die Inverse der Matrix
  	$$ A = \left(\begin{matrix} 1 & -1 \\ -2 & 3 \end{matrix} \right) $$ 
\end{aufgabe}

\begin{aufgabe} Bestimmen Sie die Inverse der Matrix
  	$$ A = \left(\begin{matrix} 1 & 3 & 4 \\ 2 & 4 & 6 \\ 1 & 3 & 5 \end{matrix} \right) $$
\end{aufgabe}

\begin{aufgabe} Überprüfen Sie, ob die Matrix 
  	$$ A = \left(\begin{matrix} 2 & 0 & 2 & 0 \\ 0 & 1 & -1 & 0 \\ 0 & 0 & 1 & 1 \\
	1 & 1 & 1 & 1 \end{matrix} \right) $$
invertierbar ist und bestimmen Sie gegebenenfalls die inverse Matrix.
\end{aufgabe}

\begin{aufgabe} Zeigen Sie, dass eine $n \times n$--Matrix $A$ genau dann invertierbar ist, wenn ihre 
Spaltenvektoren eine Basis des $\mathbb R^n$ bilden. 
\end{aufgabe}

\begin{aufgabe} Ziegen Sie: Ist $A$ eine invertierbare Matrix, so gilt
  	$$ \det{A^{-1}} = \frac {1}{\det{A}} $$
\end{aufgabe}

\begin{aufgabe} Zeigen Sie: Ist $A$ eine orthogonale Matrix, so gilt
  	$$ \det{A} =\pm 1 $$
\end{aufgabe}