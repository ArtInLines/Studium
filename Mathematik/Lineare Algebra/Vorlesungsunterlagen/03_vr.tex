

\chapter{Allgemeine Vektorräume}

\section{Vektorräume und Untervektorräume}\label{secton_vr_uvr}

\setcounter{definition}{0}
\setcounter{beispiel}{0}
\setcounter{notiz}{0}

Im letzten Kapitel haben wir gesehen, dass Mengen unerschiedliche Strukturen haben können 
und mithilfe dieser Strukturen besser untersucht werden können. Je mehr Struktur ein Objekt 
dabei hat, desto einfacher ist es zu untersuchen. Am meisten kann man deshalb auch über 
Körper sagen. Es gibt aber auch viele Menge, die selbst keine Körper sind, die aber sehr gut mit 
Hilfe von Körpern untersucht werden können. Mit diesen Objekten, den 
\textbf{Vektorräumen}\index{Vektorraum} wollen wir uns in diesem Kapitel beschäftigen. 

Dazu betrachten wir zunächst einen beliebigen Körper $K$. 

%
%Im vorangegangenen Abschnitt haben wir - weder in den Aussagen noch in den Beweisen - 
%die spezielle Struktur von $\mathbb R$, also der Anordnung oder dem Dedekindschen 
%Schnittgesetz, gemacht. Alle Aussagen und Beweise benutzen lediglich die Regeln  
%für Addition, subtraktion, Multiplikation oder Division, also die allgemeinen 
%Axiome, die für jeden Körper gelten. Genauso gut hätten wir anstelle von 
%$\mathbb R$ also auch $\mathbb C$, $\mathbb Q$ oder gar einen der endlichen Körper  $\mathbb F_p$, die wir in Abschnitt~\ref{section_gruppe} kennengelernt haben, nehmen 
%können. Das wollen wir in diesem Abschnitt auch tun. Dazu fixieren wir einen 
%beliebigen Körper $K$.


\begin{definition} Ein \index{Vektorraum}\textbf{$K$--Vektorraum} ist eine 
nichtleere Menge $V$ zusammen mit einer Addition 
  	$$ '+': V \times V \longrightarrow V, \quad (\vektor{v}, \vektor{w}) 
     	\longmapsto \vektor{v} + \vektor{w} $$
und einer Skalarmultiplikation 
  	$$ '\cdot': K \times V \longrightarrow V, \quad (r, \vektor{v}) 
     	\longmapsto r \cdot \vektor{v} $$
so dass für $\vektor{u}, \vektor{v}$ und 
$\vektor{w}$ in $V$ und Skalare $r$ und $s$ in $K$ gilt:

\begin{tabular} {l l c l}
$V1:$ & $\left(\vektor{u} + \vektor{v} \right) + \vektor{w} = 
\vektor{u} + \left(\vektor{v} + \vektor{w} \right)$ & $\,$ &
(1. Assoziativgesetz) \\
$V2:$ & $\vektor{v} + \vektor{w} = \vektor{w} + 
\vektor{v}$ & & (Kommutativgesetz) \\
$V3:$ & Es existiert ein Element $\vektor{0} \in V$ & & \\
& mit  $\vektor{v} + \vektor{0} = \vektor{v}$ & & (neutrales 
Element \\
& & & der Addition) \\
$V4:$ & Zu jedem $\vektor{v} \in V$ existiert ein  
& & \\
& $-\vektor{v} \in V$ mit $\vektor{v} + (- \vektor{v}) = \vektor{0}$ & &
(inverses Element \\
& & & der Addition) \\
$V5:$ & $(r \cdot s) \cdot \vektor{v} = r \cdot ( s \cdot \vektor{v})$ 
& & (2. Assoziativgesetz) \\
$V6:$ & $r \cdot (\vektor{v} + \vektor{w}) = r \cdot 
\vektor{v} + r \cdot \vektor{w}$ & & (1. Distributivgesetz) \\
$V7:$ & $(r + s) \cdot \vektor{v} = r \cdot \vektor{v} + 
s \cdot \vektor{v}$ & & (2.Distributivgesetz) \\
$V8:$ & $ 1 \cdot \vektor{v} = \vektor{v}$ & & (neutrales Element \\
& & & der Multiplikation)
\end{tabular}

Die Elemente $\vektor{v} \in V$ heißen \textbf{Vektoren}\index{Vektor}.
\end{definition}


\begin{beispiel}\label{LA_vr_rn} 
Das $n$--fache kartesische Produkt von $\R$, also die Menge
  	$$ \R^n := \{ \left( \begin{smallmatrix} v_1 \\ \vdots \\ v_n 
     	\end{smallmatrix} \right) \vert \, v_i \in \R \} $$
zusammen mit der komponentenweisen Addition 
  	$$ \left( \begin{smallmatrix} v_1 \\ \vdots \\ v_n 
     	\end{smallmatrix} \right) + \left( \begin{smallmatrix} w_1 \\ \vdots \\ w_n 
     	\end{smallmatrix} \right) = \left( \begin{smallmatrix} v_1 + w_1 \\ \vdots \\ 
     	v_n + w_n  \end{smallmatrix} \right) $$ 
($v_i, w_i \in \R$) und der komponentenweisen Skalarmultiplikation
  	$$ r \cdot \left( \begin{smallmatrix} v_1 \\ \vdots \\ v_n 
     	\end{smallmatrix} \right) = \left( \begin{smallmatrix} r \cdot v_1 \\ \vdots \\ 
     	r \cdot v_n \end{smallmatrix} \right) $$
($r, v_i \in \R$) ist ein $\R$--Vektorraum. 

Ein Vektor 
	$$ \vektor{v} = \begin{pmatrix} v_1 \\ \vdots \\ v_n \end{pmatrix} $$
heißt \textbf{$n$--dimensionaler reeller Vektor}. 

Der Nullvektor 
	$$ \vektor{0} = \begin{pmatrix} 0 \\ \vdots \\ 0 \end{pmatrix} $$
ist dabei das neutrale Element der Addition, und der negative Vektor $- \vektor{v}$ zu 
einem $n$--dimensionalen reellen Vektor ist gegeben durch 
	$$ - \vektor{v} = \begin{pmatrix} -v_1 \\ \vdots \\ - v_n \end{pmatrix} $$
Die Axiome $V1$ bis $V8$ lassen sich in diesem Fall sehr einfach nachrechnen. 

Der Vektrorraum $\R^n$ wird auch als 
\textbf{$n$--dimensionaler reeller Vektorraum}\index{Vektorraum $\R^n$} bezeichnet.
\end{beispiel} 

\begin{beispiel}
Die Ebene kann mit $\R^2$ identifiziert werden, der Raum mit $\R^3$. Daher sind Ebene und 
Raum reelle Vektorräume. Die geometrischen Vektoren (Pfeile im Raum und in der Ebene), wie Sie 
sie aus der Schule (oder auch aus dem Vorkurs \textit{Lineare Algebra}) kennen, sind Vektoren im 
Sinne unserer neuen Definition.
\end{beispiel}

\begin{beispiel}\label{LA_vr_cn} Das $n$--fache kartesische Produkt von $\mathbb C$, also die Menge
  	$$ \mathbb C^n := \{ \left( \begin{smallmatrix} z_1 \\ \vdots \\ z_n 
     	\end{smallmatrix} \right) \vert \, z_i \in \mathbb C \} $$
zusammen mit der komponentenweisen Addition 
  	$$ \left( \begin{smallmatrix} y_1 \\ \vdots \\ y_n 
     	\end{smallmatrix} \right) + \left( \begin{smallmatrix} z_1 \\ \vdots \\ z_n 
     	\end{smallmatrix} \right) = \left( \begin{smallmatrix} y_1 + z_1 \\ \vdots \\ 
     	y_n + z_n  \end{smallmatrix} \right) $$ 
($y_i, z_i \in \mathbb C$) und der komponentenweisen Skalarmultiplikation
  	$$ a \cdot \left( \begin{smallmatrix} z_1 \\ \vdots \\ z_n 
     	\end{smallmatrix} \right) = \left( \begin{smallmatrix} a \cdot z_1 \\ \vdots \\ 
     	a \cdot z_n \end{smallmatrix} \right) $$
($a, z_i \in \mathbb C$) ist ein $\mathbb C$--Vektorraum. 
\end{beispiel} 

\begin{beispiel} Das $n$--fache kartesische Produkt von $\mathbb F_2$, also die Menge
  	$$ \mathbb F_2^n := \{ \left( \begin{smallmatrix} a_1 \\ \vdots \\ a_n 
     	\end{smallmatrix} \right) \vert \, a_i \in \mathbb C \} $$
zusammen mit komponentenweiser Addition und Skalarmultiplikation ist ein 
$\mathbb F_2$--Vektorraum. 

Bei $\mathbb F_2^n$ handelt es sich um die $n$--Tupel binärer Zahlen, die sich 
komplett auflisten lassen. So gilt etwa 
  	$$ \mathbb F_2^2 := \{ \left( \begin{smallmatrix} 0 \\ 0 \end{smallmatrix} \right),
   	\left( \begin{smallmatrix} 0 \\ 1 \end{smallmatrix} \right), 
   	\left( \begin{smallmatrix} 1 \\ 0 \end{smallmatrix} \right),
   	\left( \begin{smallmatrix} 1 \\ 1 \end{smallmatrix} \right) \} $$
Insbesondere ist also $\mathbb F_2^n$ eine endliche Menge mit 
  	$$ \vert \mathbb F_2^n \vert \, = 2^n $$
Der Körper $\mathbb F_2$ und die Vektorräume über ihm spielen eine fundamentale 
Rolle in der Informatik und der Computerprogrammierung.
\end{beispiel}

\begin{beispiel}\label{LA_vr_folgen1}
Wir betrachten die Menge 
	$$ V = \mathrm{Abb}(\N, \R) = \{ f: \N \longrightarrow \R \, \text{ Abbildung }\} $$
aller Abbildungen von $\N$ nach $\R$. Für $f, g \in V$ definieren wir $f+g$ durch 
	$$ (f+g)(n) = f(n) + g(n) \qquad \text{ für alle } n \in N $$
und für $f \in V$ und $r \in \R$ definieren wir $r \cdot f$ durch
	$$ (r \cdot f)(n) = r \cdot f(n) \qquad \text{ für alle } n \in N $$
Dann ist $V$ ein $\R$--Vektorraum. 

In dieser Situation schreibt man häufig $f_n$ für $f(n)$ und $\left(f_n\right)_{n \in \N}$ für $f$ und nennt 
ein Element $\left(f_n\right)_{n \in \N}$ eine \textbf{reelle Zahlenfolge}. Der Raum $V$ heißt auch 
Vektorraum der Folgen.
\end{beispiel}

\begin{beispiel}
Die Menge $V = \{ \vektor{0}\}$ mit den Operationen 
	$$ \vektor{0} + \vektor{0} = \vektor{0}, \quad r \cdot \vektor{0} = \vektor{0} $$
ist ein Vektorraum (über jedem Körper), der \textbf{Nullvektorraum} bzw. der \textbf{triviale Vektorraum}. 
\end{beispiel}

Auch Teilmengen von Vektorräumen können interessante Strukturen tragen:

\begin{definition} Ist $V$ ein $K$--Vektorraum und  ist $U \subseteq V$ eine 
lTeilmenge, so heißt $U$ \index{Untervektorraum}
\textbf{Untervektorraum} von $V$ wenn gilt:

\begin{enumerate}
\item $U \neq \emptyset$. 
\item Sind $\vektor{v}, \vektor{w} \in U$ so ist auch 
$\vektor{v} + \vektor{w} \in U$.
\item Ist $\vektor{v} \in U$ und ist $\kappa \in K$ ein Skalar, so ist 
auch $\kappa \cdot \vektor{v} \in U$.
\end{enumerate}
\end{definition}

\begin{beispiel}\label{uvr_ebene}
Jede Gerade durch den Koordinatenursprung ist ein Untervektorraum der Ebene (als reeller Vektorraum betrachtet). 

Alle Geraden und alle Ebenen durch den Koordinatenursprung sind Untervektorräume des Raums (als reeller 
Vektorraum betrachtet). 
\end{beispiel}

\begin{beispiel}
Ist $V$ ein beliebiger $K$--Vektorraum, so sind $U_1 = \{\vektor{0} \}$ und $U_2 = V$ (offensichtlich) 
Untervektorräume von $V$. Diese beiden Untervektoräume werden auch die trivialen Untervektorräume genannt. 
\end{beispiel}

\begin{beispiel} 
Bezeichnen wir mit 
	$$ V  = \{ f: \N \longrightarrow \R \, \text{ Abbildung }\} $$
den Vektorraum der reellen Zahlenfolgen (aus Beispiel~\ref{LA_vr_folgen1}), und mit 
	$$ U = \{ \left(f_n\right)_{n \in \N} \,\vert \,\, \lim\limits_{n \to \infty} f_n = 0 \} $$
die Teilmenge der Nullfolgen, so ist $U \subseteq V$ ein Untervektorraum. 

Dabei ist $ U \neq \emptyset$, denn die Nullfolge $\left( f_n \right)_{n \in \N}$ mit $f_n = 0$ ist in $U$. 
Die weiteren Bedingungen für einen Untervektorraum werden wir in der Vorlesung \textit{Analysis in einer 
Veränderlichen} im zweiten Semester nachrechnen.
\end{beispiel} 

\begin{notiz}
Ein Untervektorraum $U \subseteq V$ ist selbst wieder ein Vektorraum. Offensichtlich gelten nämlich die 
Axiome $V1$ bis $V8$ auch in $U$.
\end{notiz}

\bigbreak

\begin{aufgabe}
Zeigen Sie, dass jede Gerade durch den Koordinatenursprung tatsächlich ein Untervektorraum der 
Ebene (aufgefasst als reeller Vektorraum) ist. 
\end{aufgabe}

\begin{aufgabe}
Zeien Sie, dass jede Gerade bzw. jede Ebene durch den Koordinatenursprung tatsächlich ein Untervektorraum des 
Raums (aufgefasst als reeller Vektorraum) ist. 
\end{aufgabe} 

\begin{aufgabe}
Wir betrachten einen $K$--Vektorraum $V$ und zwei Untervektorräume $U_1, U_2 \subseteq V$ von $V$. Zeigen 
Sie dass dann auch $U_1 \cap U_2$ wieder ein Untervektorraum von $V$ ist. Geben Sie ein Beispiel an, das zeigt, 
dass $U_1 \cup U_2$ nicht notwendigerweise ein Untervektorraum von $V$ ist.
\end{aufgabe}

\section{Der $n$--dimensionale reelle Raum}\label{section_rn}

\setcounter{definition}{0}
\setcounter{beispiel}{0}
\setcounter{notiz}{0}

Im ersten Abschnitt haben wir als Beispiel (Beispiel~\ref{LA_vr_rn}) den reellen Vektorraum $\R^n$ 
betrachtet. Dieser Vektorraum ist für viele Anwendungen auch der wichtigste und wird daher nun in 
diesem Abschnitt ausführlicher behandelt werden. 

Die Vektorraumoperationen können in diesem Fall sehr einfach und explizit beschrieben werden: 

\begin{beispiel} Es ist
 	$$ \left( \begin{matrix} 1 \\ 2 \\ 3 \\ 4 \end{matrix} \right) 
	+ \left( \begin{matrix} 2 \\ -3 \\ 4 \\ -5 \end{matrix} \right)
	= \left( \begin{matrix} 1 + 2 \\ 2 + (-3) \\ 3 + 4 
   	\\ 4 + (-5) \end{matrix} \right) = \left( \begin{matrix} 3 \\ -1 \\ 7 
   	\\ -1 \end{matrix} \right) $$
und 
 	$$\left( \begin{matrix} 1 \\ 2 \\ 3 \\ 4 \end{matrix} \right) 
	+ \left( \begin{matrix} 2 \\ -3 \\ 4 \\ -5 \end{matrix} \right)
	= \left( \begin{matrix} 1 - 2 \\ 2 - (-3) \\ 3 - 4 
   	\\ 4 - (-5) \end{matrix} \right) = \left( \begin{matrix} -1 \\ 5 \\ -1 
   	\\ 9 \end{matrix} \right) $$

Ferner gilt 
  	$$  - \left( \begin{matrix} 2 \\ 3 \\ -1 \\ 2 \end{matrix} \right) 
	= \left( \begin{matrix} -2 \\ -3 \\ 1 \\ -2 \end{matrix} \right) $$
und 
	$$ 3 \cdot \left( \begin{matrix} 2 \\ 3 \\ -1 \\ 2 \end{matrix} \right) =
	\left( \begin{matrix} 3 \cdot 2 \\ 3 \cdot 3 \\ 3 \cdot (-1) \\ 3 \cdot 2 \end{matrix} \right) = 
	\left( \begin{matrix} 6 \\ 9 \\ -3 \\ 6 \end{matrix} \right) $$
\end{beispiel}

\begin{definition}
Ein Vektor 
	$$ \vektor{v} = \begin{pmatrix} v_1 \\ \vdots \\ v_n \end{pmatrix} \quad \in \R^n $$
heißt \textbf{$n$--dimensionaler reeller Vektor}, die Größe
	$$ \vert \vektor{v} \Vert = \sqrt{v_1^2 + \cdots v_n^2} $$
heißt \textbf{Länge}\index{Vektor!Länge} des Vektors $\vektor{v}$.
\end{definition}


\bigbreak

Wir betrachten nun Vektoren $\vektor{v_1}, \vektor{v_2}, \ldots, \vektor{v_m}$ in $\mathbb R^n$. 

\begin{definition}\label{uvr_lin_unab} 
Die $n$--Vektoren  $\vektor{v_1}, \vektor{v_2}, \ldots, \vektor{v_m}$ 
heißen \index{linear unabhängig}\index{Vektoren!Lineare Unabhängigkeit}\textbf{linear abhängig}, wenn 
es reelle Zahlen $r_1, r_2, \ldots, r_m$ gibt, von denen mindestens eine von Null verschieden ist, mit
  	$$ r_1 \cdot \vektor{v_1} + r_2 \cdot  \vektor{v_2} + \cdots + r_m 
     	\cdot \vektor{v_m} = \vektor{0} $$
Andernfalls heißen sie \textbf{linear unabhängig}.
\end{definition}

\begin{notiz} Die Bedingung für lineare Unabhängigkeit kann auch so formuliert werden: Sind $r_1,  r_2, 
\ldots, r_m$ reelle Zahlen mit 
  	$$ r_1 \cdot \vektor{v_1} + r_2 \cdot  \vektor{v_2} + \cdots + r_m \cdot \vektor{v_m} = \vektor{0} $$
so muss schon gelten: $r_1 = r_2 = \cdots = r_m = 0$.
\end{notiz}

\begin{beispiel} Die Vektoren $\vektor{v_1} = \left( \begin{smallmatrix} 1 \\ 2 \\ 3 \\ 1
\end{smallmatrix} \right)$ und $\vektor{v_2} = \left( \begin{smallmatrix} -2 \\ -4 \\ -6 \\ -2 
\end{smallmatrix} \right)$ sind linear abhängig: Für $r_1 = 2$ und $r_2 = 1$ gilt
  	$$ r_1 \cdot \vektor{v_1} + r_2 \cdot  \vektor{v_2} =
     	2 \cdot \left( \begin{smallmatrix} 1 \\ 2 \\ 3 \\ 1 \end{smallmatrix} \right) + 
     	1 \cdot \left( \begin{smallmatrix} -2 \\ -4 \\ -6 \\ -2 \end{smallmatrix} \right) =
     	\left( \begin{smallmatrix} 0 \\ 0 \\ 0 \\ 0 \end{smallmatrix} \right) $$
\end{beispiel}

\begin{beispiel}\label{lin_ab_drei}
Die Vektoren  $\vektor{v_1} = \left( \begin{smallmatrix} 1 \\ 2 \\ 3 \\ 1
\end{smallmatrix} \right)$, $\vektor{v_2} = \left( \begin{smallmatrix} 2 \\ 3 \\ 4 \\ 2 
\end{smallmatrix} \right)$ und $\vektor{v_3} = \left( \begin{smallmatrix} 1 \\ 1 \\ 1 \\ 1 
\end{smallmatrix} \right)$ sind linear abhängig: Für $r_1 = 1$ und $r_2 = -1$ und $r_3 = 1$ gilt:
  	$$  r_1 \cdot \vektor{v_1} + r_2 \cdot  \vektor{v_2} + r_3 \cdot \vektor{v_3} =  \vektor{0} $$
\end{beispiel} 

\begin{beispiel} Die Vektoren $\vektor{v_1} = \left( \begin{smallmatrix} 1 \\ 2 \\ 3 \\ 4
\end{smallmatrix} \right)$ und $\vektor{v_2} = \left( \begin{smallmatrix} 3 \\ 2 \\ 1 \\ 0 
\end{smallmatrix} \right)$ sind linear unabhängig: Sind nämlich $r_1, r_2 \in \mathbb R$ Skalare mit 
  	$$  r_1 \cdot \vektor{v_1} + r_2 \cdot  \vektor{v_2} =  \vektor{0} $$
so bedeutet dies 
  	$$ \left( \begin{matrix} r_1 \cdot 1 \\ r_1 \cdot 2 \\ r_1 \cdot 3 \\ r_1 \cdot 4 \end{matrix} 
    	\right) + \left( \begin{matrix} r_2 \cdot  3 \\ r_2 \cdot 2 
     	\\ r_2 \cdot 1 \\ r_2 \cdot 0 \end{matrix} \right) =
     	\left( \begin{matrix} 0 \\ 0 \\ 0 \\ 0 \end{matrix} \right) $$
also 
 	 $$ \begin{array} { l c l c l}
    	1 \cdot r_1 & + & 3 \cdot r_2 & = & 0 \\
    	2 \cdot r_1 & + & 2 \cdot r_2 & = & 0 \\
    	3 \cdot r_1 & + & 1 \cdot r_2 & = & 0 \\
    	4 \cdot r_1 & + & 0 \cdot r_2 & = & 0 \\
    	\end{array} $$
Aus der letzten Gleichung folgt sofort, dass $r_1 = 0$ gelten muss. Setzt man das in die vorletzte Gleichung ein, so 
ergibt sich unmittelbar $r_2 = 0$, und damit sind $\vektor{v_1}$ und $\vektor{v_2}$ linear unabhängig.
\end{beispiel}

\begin{beispiel}\label{lin_un_standard} Die $n$ Vektoren  $\vektor{e_1} = 
\left( \begin{smallmatrix} 1 \\ 0 \\ \vdots \\ 0
\end{smallmatrix} \right)$, $\vektor{e_2} = \left( \begin{smallmatrix} 0 \\ 1 \\ \vdots \\ 0 
\end{smallmatrix} \right), \ldots , \vektor{e_n} = \left( \begin{smallmatrix} 0 \\ \vdots \\ 0 \\ 1 
\end{smallmatrix} \right)$ im $\R^n$ sind linear unabhängig. Sind nämlich $r_1, r_2, \ldots, r_n$ 
Skalare mit 
  	$$  r_1 \cdot \vektor{e_1} + r_2 \cdot  \vektor{e_2} + \ldots +
      	r_n \cdot \vektor{e_n} =  \vektor{0} $$
so bedeutet dies
  	$$ \left( \begin{matrix} r_1 \cdot 1 + r_2 \cdot 0 + \cdots + r_n \cdot 0 \\ 
     	r_1 \cdot 0 + r_2 \cdot 1 + \cdots + r_n \cdot 0 \\ \vdots \\ 
     	r_1 \cdot 0 + r_2 \cdot 0 + \cdots + r_n \cdot 1 \end{matrix} \right)
    	= \left( \begin{matrix} 0 \\ 0 \\ \vdots \\ 0 \end{matrix} \right) $$
also 
  	$$ \left( \begin{matrix} r_1  \\ r_2 \\ \vdots \\  r_n  \end{matrix} \right)
    	= \left( \begin{matrix} 0 \\ 0 \\ \vdots \\ 0 \end{matrix} \right) $$
und damit $r_1 = r_2 = \ldots = r_n = 0$.
\end{beispiel} 

\begin{notiz} Ist $\vektor{v_k} = \vektor{0}$ für ein $k$, so sind die 
Vektoren $\vektor{v_1}, \vektor{v_2}, \ldots, \vektor{v_m}$ 
schon linear abhängig. Hierzu können wir etwa $r_k = 1$ und $r_i = 0$ für $i \neq k$ setzen 
und erhalten eine nicht--triviale Linearkombination
  	$$  r_1 \cdot \vektor{v_1} + \cdots + r_n \cdot  \vektor{v_n} =  \vektor{0} $$
\end{notiz}

\begin{notiz}
Ein einzelner Vektor $\vektor{v} \in \R^n$ ist genau dann linear abhängig, wenn $\vektor{v} = \vektor{0}$. 
Entsprechend ist er genau dann linear unabhängig, wenn $\vektor{v} \neq \vektor{0}$.
\end{notiz}

\begin{notiz} Sind die Vektoren $\vektor{v_1}, \vektor{v_2}, \ldots, \vektor{v_m}$ 
linear unabhängig, so auch jede Teilmenge davon, d.h. für $1 \leq i_1 < i_2 < \ldots < i_t \leq m$ 
sind auch die Vektoren 
$\vektor{v_{i_1}}, \vektor{v_{i_2}}, \ldots, \vektor{v_{i_t}}$ linear unabhänig.

Die entsprechende Aussage für linear Abhängigkeit ist nicht richtig.
\end{notiz}

\begin{beispiel} Die Vektoren  $\vektor{e_1} = \left( \begin{smallmatrix} 1 \\ 0 \\ 0 \\ 0
\end{smallmatrix} \right)$, $\vektor{e_4} = \left( \begin{smallmatrix} 0 \\ 0 \\ 0 \\ 1 
\end{smallmatrix} \right)$ im $\mathbb R^4$ sind linear unabhängig. Sie sind Teil des Systems linear 
unabhängiger Vektoren aus Beispiel ~\ref{lin_un_standard}.

Natürlich kann die lineare Unabhängigkeit in diesem Fall auch direkt nachgerechnet werden.
\end{beispiel}

\begin{beispiel} Die Vektoren  $\vektor{v_1} = \left( \begin{smallmatrix} 1 \\ 2 \\ 3 \\ 1
\end{smallmatrix} \right)$ und  $\vektor{v_2} = \left( \begin{smallmatrix} 2 \\ 3 \\ 4 \\ 2 
\end{smallmatrix} \right)$ sind linear unabhängig, wie man leicht nachrechnet, obwohl sie eine Teilmenge der 
linear abhängigen Vektoren $\vektor{v_1}, \vektor{v_2}, \vektor{v_3}$ aus Beispiel
~\ref{lin_ab_drei} bilden.   
\end{beispiel}

\begin{notiz} Zwei Vektoren $\vektor{v}$ und $\vektor{w}$ sind genau dann linear abhängig, wenn 
einer von beiden ein Vielfaches des anderen ist. Gilt nämlich 
  	$$ r_1 \cdot \vektor{v} + r_2 \cdot \vektor{w} = \vektor{0} $$
und ist etwa $r_1 \neq 0$, so gilt
  	$$ \vektor{v} = - \frac {r_2}{r_1} \cdot \vektor{w} $$
Analog ist natürlich $\vektor{w}$ ein Vielfaches von $\vektor{v}$ falls $r_2 \neq 0$.
\end{notiz}

\begin{notiz} Die lineare Unabhängigkeit von drei oder mehr Vektoren ist in der Regel schwer elementar und 
direkt festzustellen. Mit den linearen Gleichungssystemen werden wir das richtige Werkzeug zur Behandlung 
dieser Frage noch kennenlernen. 
\end{notiz}

Eine wichtige Rolle beim Studium linearer Gleichungssysteme werden die Untervektorräume der $\R^n$ bilden.

\begin{beispiel}\label{uvr_null_uvr} 
Ist $V = \{ \left( \begin{smallmatrix} 0 \\ 0 \\ 0 \\ 0 \end{smallmatrix} \right) \} \subseteq \mathbb R^4$, so  
ist $V$ ein Untervektorraum von $\R^4$.
\end{beispiel}


\begin{beispiel}\label{uvr_y_axis} 
Ist $V = \{ \left( \begin{smallmatrix} 0 \\ r \\ 0 \\ 0 \end{smallmatrix} \right) \vert \, 
r \in \mathbb R\} \subseteq \mathbb R^4$, so ist $V$ ein Untervektorraum von $\R^4$.
\end{beispiel}

\begin{beispiel} Ist $V = \{ \left( \begin{smallmatrix} 0 \\ r \\ 0 \\ 0 \end{smallmatrix} \right) \vert \,  
r \in \mathbb R\ \, \textrm{ und } r \geq 0 \} \subseteq \mathbb R^4$, so ist $V$ kein Untervektorraum. Es ist 
nämlich $\left( \begin{smallmatrix} 0 \\ 1 \\ 0 \\ 0 \end{smallmatrix} \right) \in V$ aber nicht 
$(-1) \cdot \left( \begin{smallmatrix} 0 \\ 1 \\ 0 \\ 0 \end{smallmatrix} \right)$
\end{beispiel}

\begin{beispiel}
Ist $V = \{ \left( \begin{smallmatrix} 0 \\ r \\ 1 \\ 0 \end{smallmatrix} \right) \vert \, 
r \in \mathbb R\} \subseteq \mathbb R^4$, so ist $V$ kein Untervektorraum. Es ist 
nämlich $\left( \begin{smallmatrix} 0 \\ 2 \\ 1 \\ 0 \end{smallmatrix} \right) \in V$ aber nicht 
$2 \cdot \left( \begin{smallmatrix} 0 \\ 2 \\ 1 \\ 0 \end{smallmatrix} \right)$ 
\end{beispiel}


%\begin{beispiel} Jede Gerade in der Ebene, die durch den Nullpunkt geht, ist ein Untervektorraum von 
%$\mathbb R^2$.
%\end{beispiel}

\begin{beispiel}\label{uvr_gen_gerade} Ist $\vektor{v} \in \mathbb R^n$ ein beliebiger Vektor, so ist 
  	$$ V := \{ r \cdot \vektor{v} \vert \, r \in \mathbb R \} $$
ein Untervektorraum (auch, falls $\vektor{v}$ der Nullvektor ist).
\end{beispiel}

\begin{beispiel}
Ist $V = \{ \left( \begin{smallmatrix} r \\ s \\ r+s \end{smallmatrix} \right) \vert \, 
r,s \in \mathbb R\} \subseteq \mathbb R^3$, so ist $V$ ein Untervektorraum.
\end{beispiel}

%\begin{beispiel}\label{uvr_ebene} 
%Jede Gerade in der Ebene, die durch den Nullpunkt geht, ist ein Untervektorraum von 
%$\mathbb R^2$.
%
%Jede Ebene im Raum, die den Nullpunkt enthält, ist ein Untervektorraum von $\R^3$.
%\end{beispiel}

\medbreak

Gesucht ist häufig eine effiziente und knappe Beschreibung von Untervektorräumen mit möglichst wenig Daten. 
Dazu geben wir uns einen Untervektorraum 
$V \subseteq \mathbb R^n $ vor und betrachten Vektoren $\vektor{v_1}, \vektor{v_2}, \ldots, 
\vektor{v_m}$ in $V$.

\begin{definition} Die Vektoren $\vektor{v_1}, \vektor{v_2}, \ldots, 
\vektor{v_m}$ heißen \index{Untervektorraum!Erzeugendensystem} \textbf{Erzeugendensystem} von $V$, 
wenn gilt:

\begin{enumerate}
\item $\vektor{v_1}, \vektor{v_2}, \ldots, \vektor{v_m} \in V$.
\item Zu jedem $\vektor{w} \in V$ gibt es Skalare $r_1, r_2, \ldots,  r_m$ mit
  	$$  \vektor{w} = r_1 \cdot \vektor{v_1} + r_2 \cdot \vektor{v_2} + \ldots + 
      r_m \cdot \vektor{v_m} $$
\end{enumerate}
Wir sagen in diesem Fall auch,  $\vektor{v_1}, \vektor{v_2}, \ldots, 
\vektor{v_m}$ erzeugen $V$.
\end{definition}

\begin{beispiel}\label{erz_standard} Die Vektoren  $\vektor{e_1} = 
\left( \begin{smallmatrix} 1 \\ 0 \\ \vdots \\ 0
\end{smallmatrix} \right)$, $\vektor{e_2} = \left( \begin{smallmatrix} 0 \\ 1 \\ \vdots \\ 0 
\end{smallmatrix} \right), \ldots , \vektor{e_n} = \left( \begin{smallmatrix} 0 \\ \vdots \\ 0 \\ 1 
\end{smallmatrix} \right)$ erzeugen $V = \mathbb R^n$ (als Untervektroraum von $\mathbb R^n$). Es gilt 
nämlich für einen beliebigen Vektor 
$\vektor{v} = \left( \begin{smallmatrix} v_1 \\ v_2 \\ \vdots \\ v_n \end{smallmatrix} \right)$ 
in $\R^n$:
  	$$ \vektor{v} = v_1 \cdot \vektor{e_1} + v_2 \cdot \vektor{e_2} + \cdots + v_n \cdot \vektor{e_n} $$
\end{beispiel}

\begin{beispiel} Ist $V = \{ \left( \begin{smallmatrix} 0 \\ r \\ 0 \\ 0 \end{smallmatrix} \right) \vert \, 
r \in \mathbb R\} \subseteq \mathbb R^4$ der Untervektorraum aus Beispiel ~\ref{uvr_y_axis} und ist
$\vektor{v} = \left( \begin{smallmatrix} 0 \\ 1 \\ 0 \\ 0 \end{smallmatrix} \right)$, 
so erzeugt $\vektor{v}$ den Untervektorraum $V$. Ebenso wird $V$ erzeugt von
$\vektor{w} = \left( \begin{smallmatrix} 0 \\ -3 \\ 0 \\ 0 \end{smallmatrix} \right)$, und auch 
$\{ \vektor{v}, \vektor{w} \}$ bildet ein Erzeugendensystem von $V$.
Erzeugendensysteme sind also nicht eindeutig und können unterschiedlich viele Elemente enthalten.
\end{beispiel}

\begin{beispiel}\label{uvr_null_erz} 
Ist $V = \{ \vektor{0} \}$ der Nullvektorraum, so bildet der Nullvektor 
$\vektor{0}$ ein Erzeugendensystem von $V$. Es ist aber auch üblich, die leere Menge als 
Erzeugendensystem von $V$ zu betrachten. 
\end{beispiel}

%\begin{beispiel} Ist $V = \mathbb R^n$ (als Untervektorraum von $\mathbb R^n$), so erzeugen die 
%Vektoren  $\vektor{e_1} = 
%\left( \begin{smallmatrix} 1 \\ 0 \\ \vdots \\ 0
%\end{smallmatrix} \right)$, $\vektor{e_2} = \left( \begin{smallmatrix} 0 \\ 1 \\ \vdots \\ 0 
%\end{smallmatrix} \right), \ldots , \vektor{e_n} = \left( \begin{smallmatrix} 0 \\ \vdots \\ 0 \\ 1 
%\end{smallmatrix} \right)$ aus Beispiel ~\ref{lin_un_standard} $V$.
%\end{beispiel}

\begin{beispiel}\label{uvr_ebene_bsp} 
Ist $E$ die Ebene durch die Punkte $O = (0, 0, 0)$, $P = (1,2,3)$ und $Q = (-3,2,-1)$, so ist 
$E$ ein Untervektorraum von $\mathbb R^3$ (siehe Beispiel~\ref{uvr_ebene}), der erzeugt wird von den Vektoren
  	$$ \vektor{v_1} = \left( \begin{matrix} 1 \\ 2 \\ 3  \end{matrix} \right), \quad  
  	\vektor{v_2} = \left( \begin{matrix} -3 \\ 2 \\ -1  \end{matrix} \right) $$
also von den beiden Vektoren, die (zusammen mit $\vektor{0}$ als Stützvektor) $E$ auch als Ebene 
erzeugen. Ebenso bildet auch jedes andere System von Vektoren, das mit $\vektor{0}$ als Stützvektor die 
Ebene $E$ erzeugt, ein Untervektorraumerzeugendensystem von $E$, etwa
  	$$ \vektor{w_1} = \left( \begin{matrix} 1 \\ 2 \\ 3 \end{matrix} \right), \quad  
    	\vektor{w_2} = \left( \begin{matrix} 0 \\ 4 \\ 4  \end{matrix} \right), \quad 
     	\vektor{w_3} = \left( \begin{matrix} 	2 \\ 0 \\ 2  \end{matrix} \right) $$
\end{beispiel}

\medbreak

Wir betrachten nun Vektoren $\vektor{v_1}, \vektor{v_2}, \ldots, 
\vektor{v_m} \in \mathbb R^n$ gegeben, und setzen
  	$$ U = \{ \vektor{v} \in \mathbb R^n \, \vert \, \, \exists r_1, \ldots, r_m \in 
	\mathbb R \, \textrm{ mit } \,
   	\vektor{v} = r_1 \cdot \vektor{v_1} + \cdots + r_m \cdot \vektor{v_m} \} $$

\begin{satz} $U$ ist ein Untervektorraum von $\mathbb R^n$.
\end{satz}

\beweis{ Die Unterraumeigenschaften sind sehr leicht nachzurechnen. Wir zeigen, dass $U$ 
abgeschlossen unter Addition ist. Seien dazu $\vektor{v}$ und 
$\vektor{w}$ in $U$. Schreibe
  	$$ \begin{array}{l c l}
  	\vektor{v} & = & r_1 \cdot \vektor{v_1} + \cdots + r_m \cdot \vektor{v_m} \\
  	\vektor{w} & = & s_1 \cdot \vektor{v_1} + \cdots + s_m \cdot \vektor{v_m} 
  	\end{array} $$
Dann gilt
 	$$ \vektor{v} + \vektor{w} = 
    	(r_1+s_1) \cdot \vektor{v_1} + \cdots + (r_m+s_m) \cdot \vektor{v_m} $$
und da $r_i + s_i \in \mathbb R$ folgt
 	$$ \vektor{v} + \vektor{w} \in U $$
Die Abgeschlossenheit unter Skalarmultiplikation zeigt man ähnlich.
}

\medbreak  

\bemerkung{Bezeichnung:} $U$ heißt das \textbf{Erzeugnis} von $\vektor{v_1}, \vektor{v_2}, \ldots, \vektor{v_m}$ 
oder der von $\vektor{v_1}, \vektor{v_2}, \ldots, \vektor{v_m}$
\index{Vektorraum!aufgespannter Unterraum}\textbf{aufgespannte Unterraum} von $\mathbb R^n$. Wir 
schreiben hierfür $\langle \{ \vektor{v_i} \}_{i = 1, \ldots, m} \rangle$ 
oder auch $\textrm{Span}( \{ \vektor{v_i} \}_{i = 1, \ldots, m} )$.

\medbreak

\begin{beispiel} Wir betrachten die Vektoren 
  	$$ \vektor{v_1} = \left( \begin{matrix} -1 \\ 2 \\ 0  \end{matrix} \right), \quad  
  	\vektor{v_2} = \left( \begin{matrix} 0 \\ 4 \\ 0  \end{matrix} \right) $$
in $\mathbb R^3$. Dann gilt hierfür
  	$$ \langle \{ \vektor{v_1}, \vektor{v_2} \} \rangle = 
 	\{ \left( \begin{matrix} x_1 \\ x_2 \\ 0  \end{matrix} \right) \, \vert \, x_1, x_2 \in \mathbb R \} $$
Klar ist dabei, dass 
  	$$ \langle \{ \vektor{v_1}, \vektor{v_2} \} \rangle \subseteq 
    	\{ \left( \begin{matrix} x_1 \\ x_2 \\ 0  \end{matrix} \right) \, \vert \, x_1, x_2 \in \mathbb R \} $$
denn mit $\vektor{v_1}$ und $\vektor{v_2}$ wird auch jede 
Linearkombination von $\vektor{v_1}$ und $\vektor{v_2}$ in der 
letzen Komponente eine ''0'' stehen haben. Ist umgekehrt ein Vektor 
$\vektor{v} = \left( \begin{smallmatrix} x_1 \\ x_2 \\ 0  \end{smallmatrix} 
\right)$ gegeben, so können wir schreiben 
  	$$ \vektor{v} = (-x_1) \cdot \vektor{v_1} + \frac {x_2-2x_1}{4} \cdot \vektor{v_2} $$
und erhalten $\vektor{v} \in \langle \{ \vektor{v_1}, 
\vektor{v_2} \} \rangle$, also 
  	$$ \langle \{ \vektor{v_1}, \vektor{v_2} \} \rangle \subseteq 
    	\{ \left( \begin{matrix} x_1 \\ x_2 \\ 0  \end{matrix} \right) \, \vert \, 
   	x_1, x_2 \in \mathbb R \} $$
und damit Gleichheit der beiden Mengen.
\end{beispiel}

\bigbreak

\begin{definition} Die Vektoren $\vektor{v_1}, \vektor{v_2}, \ldots, 
\vektor{v_m} \in V$ heißen \index{Untervektorraum!Basis} \textbf{Basis} von $V$, wenn 
sie ein Erzeugendensystem von $V$ bilden, 
und wenn sie linear unabhängig sind.
\end{definition}

\begin{beispiel}\label{uvr_standard_basis} Die Vektoren  $\vektor{e_1} = 
\left( \begin{smallmatrix} 1 \\ 0 \\ \vdots \\ 0
\end{smallmatrix} \right)$, $\vektor{e_2} = \left( \begin{smallmatrix} 0 \\ 1 \\ \vdots \\ 0 
\end{smallmatrix} \right), \ldots , \vektor{e_n} = \left( \begin{smallmatrix} 0 \\ \vdots \\ 0 \\ 1 
\end{smallmatrix} \right)$ bilden eine Basis von $\mathbb R^n$. In Beispiel ~\ref{lin_un_standard} haben wir 
schon gesehen, dass sie linear unabhängig sind, und in Beispiel ~\ref{erz_standard} haben wir festgestellt, 
dass sie den $\mathbb R^n$ erzeugen. 

Die Vektoren $\vektor{e_1}$, $\vektor{e_2}, \ldots , \vektor{e_n}$ heißen 
\index{Standardbasis} \textbf{Standardbasis des $\mathbb R^n$}.
\end{beispiel}

\begin{beispiel}\label{uvr_ebene_erz} 
Ist $E$ die Ebene aus Beispiel ~\ref{uvr_ebene_bsp}, aufgefasst als Untervektorraum von $\mathbb R^n$. 
Dann bilden die Vektoren
  	$$ \vektor{v_1} = \left( \begin{matrix} 1 \\ 2 \\ 3  \end{matrix} \right), \quad  
     	\vektor{v_2} = \left( \begin{matrix} -3 \\ 2 \\ -1  \end{matrix} \right) $$
eine Basis von $E$. Wie wir in Beispiel ~\ref{uvr_ebene} gesehen haben, erzeugen sie $E$, und wie man leicht sieht, 
sind sie linear unabhängig, da keiner der Vektoren ein Vielfaches des anderen ist. Dagegen bilden die 
Vektoren
  	$$ \vektor{w_1} = \left( \begin{matrix} 1 \\ 2 \\ 3 \end{matrix} \right), \quad  
    	\vektor{w_2} = \left( \begin{matrix} 0 \\ 4 \\ 4  \end{matrix} \right), \quad 
     	\vektor{w_3} = \left( \begin{matrix} 	2 \\ 0 \\ 2  \end{matrix} \right) $$
keine Basis von $E$. Sie erzeugen zwar, aber es gilt
  	$$ 2 \cdot \vektor{w_1} + (-1) \cdot \vektor{w_2} + (-1) \cdot \vektor{w_3} = \vektor{0}, $$
sie sind also nicht linear unabhängig.
\end{beispiel}

\medbreak

Im nächsten Abschnitt werden wir den folgenden hilfreichen Satz beweisen:

\begin{satz}\label{uvr_existenz_basis} 
Für einen Untervektorraume $V$ von $\mathbb R^n$ gilt:

\begin{enumerate}
\item Ist $\{\vektor{v_1}, \vektor{v_2}, \ldots, \vektor{v_m} \}$ ein Erzeugendensystem 
von $V$, so enthält $\{ \vektor{v_1}$, $\vektor{v_2}$, $\ldots$, $\vektor{v_m} \}$ eine 
Basis von $V$, d.h. es gibt $1 \leq i_1 < i_2 < \cdots < i_t \leq m$ so dass
$ \vektor{v_{i_1}}, \vektor{v_{i_2}}, \ldots, \vektor{v_{i_t}}$ eine Basis von 
$V$ ist.
\item $V$ hat eine Basis.
\item Je zwei Basen von $V$ sind gleich lang, d.h. sind 
$\{\vektor{v_1}, \vektor{v_2}, \ldots, \vektor{v_m} \}$ 
und $\{\vektor{w_1}, \vektor{w_2}, \ldots, \vektor{w_t} \}$ Basen von $V$, so gilt
$m = t$.
\end{enumerate}
\end{satz}

Dieser Satz rechtfertigt die folgende Festsetzung

\begin{definition} Die Länge einer Basis eines Untervektorraums $V$ heißt \index{Untervektorraum!Dimension}
die \textbf{Dimension} von $V$ und wird mit $\textrm{dim}(V)$ bezeichnet.
\end{definition}

\begin{beispiel} Der $\mathbb R^n$ hat die Dimension $n$.
\end{beispiel}

\begin{beispiel} Die Ebene $E$ aus Beispiel~\ref{uvr_ebene} hat die Dimension 2, da die Vektoren 
$ \vektor{v_1}, \vektor{v_2}$ aus Beispiel~\ref{uvr_ebene_erz} eine Basis von $E$ bilden. 
Es ist also $\textrm{dim}(V) = 2$. 

Die Vektoren 
$\vektor{w_1}, \vektor{w_2}$ und $\vektor{w_3}$ aus Beispiel~\ref{uvr_ebene_erz} bilden 
eine  Erzeugendensystem von $V$ aber keine Basis. $\vektor{w_2}$ und $\vektor{w_3}$ bilden eine 
in diesem Erzeugendensystem enthaltene Basis von $E$.
\end{beispiel}

\begin{beispiel} Sei $\vektor{v} \in \mathbb R^n$ ein beliebiger Vektor und sei
  	$$ V := \{ r \cdot \vektor{v} \vert \, r \in \mathbb R \} $$
der Untervektorraum aus ~\ref{uvr_gen_gerade}. Ist $\vektor{v} \neq \vektor{0}$ so ist 
$\vektor{v}$ eine Basis von $V$ und $\textrm{dim}(V) = 1$.
\end{beispiel}

\begin{beispiel} Wie wir in Beispiel ~\ref{uvr_null_erz} festgesetzt haben, bildet die leere Menge ein 
Erzeugendensystem des Nullvektorraums $V = \{\vektor{0} \}$. Die leere Menge ist auch ein System 
linear unabhängiger Vektoren (da keine Bedingungen zu verifizieren sind), und daher ist die leere Menge 
eine Basis des Nullvektorraums. Damit gilt $\textrm{dim}(V) = 0$.
\end{beispiel}

\bigbreak

Eine Besonderheit der Vektorräume $\R^n$ ist die Möglichkeit, dort auch geometrische Überlegungen durchzuführen. 
Speziell in der Ebene und im Raum sind Begriffe wie Abstand und Winkel bekannt. Teilweise übertragen sich diese 
Konzepte auch auf höhere Dimensionen. Ein wichtiges Hilfsmittel dabei ist das Skalarprodukt:

\begin{definition}
Für zwei $n$--Vektoren $\vektor{v} = \left( \begin{smallmatrix} v_1 \\ \vdots \\  v_n \end{smallmatrix} 
\right)$ und  $\vektor{w} = \left( \begin{smallmatrix} w_1 \\  \vdots \\ w_n \end{smallmatrix} \right)$ 
heißt
  	$$ \langle \vektor{v}, \vektor{w} \rangle := v_1 w_1 + \cdots + v_n w_n $$
das \index{Skalarprodukt}\textbf{Skalarprodukt} von $\vektor{v}$ und $\vektor{w}$.
\end{definition}

\begin{beispiel}
$ \langle  \left( \begin{smallmatrix} 1 \\ 2 \\ 3 \\ 1
\end{smallmatrix} \right),  \left( \begin{smallmatrix} 2 \\ 3 \\ 4 \\ 2 
\end{smallmatrix} \right) \rangle = 1 \cdot 2 + 2 \cdot 3 + 3 \cdot 4 + 1 \cdot 2 = 22$.
\end{beispiel}

\begin{notiz}
Für einen Vektor $\vektor{v} \in \R^n$ gilt 
	$$ \vert \vektor{v} \vert = \sqrt{ \langle \vektor{v}, \vektor{v} \rangle} $$
Zunächst ist nämlich 
	$$  \langle \vektor{v}, \vektor{v} \rangle = v_1 \cdot v_1 + \cdots + v_n \cdot v_n = 
	v_1^2 + \cdots + v_n^2  \geq 0 $$
(sodass wir die Wurzel daraus ziehen können) und damit 
	$$ \sqrt{ \langle \vektor{v}, \vektor{v} \rangle}  = \sqrt{ v_1^2 + \cdots + v_n^2 } 
	= \vert \vektor{v} \vert $$
\end{notiz}

\medbreak

Einige elementaren Eigenschaften des Skalarprodukts folgen sofort aus 
der Definition

\begin{regel}\label{uvr_regel_scalar} Für Vektoren $\vektor{v}$, $\vektor{w}$, 
$\vektor{w_1}$ und $\vektor{w_2}$ und einen Skalar $r \in \mathbb R$ gilt:
\begin{enumerate}
\item $\langle \vektor{v}, \vektor{w} \rangle = \langle \vektor{w}, 
\vektor{v} \rangle \qquad$ (\textit{Kommutativgesetz}).
\item $\langle \vektor{v}, \vektor{w_1} + \vektor{w_2} \rangle = 
\langle \vektor{v}, \vektor{w_1} \rangle + \langle \vektor{v}, 
\vektor{w_2} \rangle \qquad$ (\textit{Distributivgesetz}).
\item $\langle r \cdot \vektor{v}, \vektor{w} \rangle = r \cdot \langle \vektor{v}, 
\vektor{w} \rangle \qquad$ (\textit{Skalarmultiplikation}).
\item $\langle \vektor{v}, \vektor{v} \rangle = \vert \vektor{v} \vert^2$.
\end{enumerate}
\end{regel}

Wie im zwei-- oder dreidimensionalen Fall gilt auch hier der  
\index{Satz!von Cauchy--Schwarz}Satz von Cauchy--Schwarz 

\begin{satz}[Satz von Cauchy--Schwarz]\label{uvr_cauchy_schwarz} 
Für zwei Vektoren $\vektor{v}$ und $\vektor{w}$ gilt:
\begin{enumerate}
\item $ \vert \langle \vektor{v},  \vektor{w} \rangle \vert \leq \vert \vektor{v} 
        \vert \cdot \vert \vektor{w} \vert $.
\item Genau dann sind $\vektor{v}$ und $\vektor{w}$ parallel, wenn $\langle \vektor{v}, 
\vektor{w} \rangle = \vert \vektor{v} \vert \cdot  \vert \vektor{w} \vert$.
\item Genau dann sind $\vektor{v}$ und $\vektor{w}$ anti--parallel, wenn $\langle 
\vektor{v}, \vektor{w} \rangle = 
- \vert \vektor{v} \vert \cdot  \vert \vektor{w} \vert$.
\item Genau dann sind $\vektor{v}$ und $\vektor{w}$ kollinear, wenn 
$\vert \langle \vektor{v}, \vektor{w} \rangle \vert \, 
= \vert \vektor{v} \vert \cdot  \vert \vektor{w} \vert$.
\end{enumerate}
\end{satz}

%Der Beweis hierfür kann wortwörtlich aus dem zweidimensionalen Fall übertragen werden.
\beweis{ 
Der direkte Beweis dieser Aussage ist - sogar schon im einfachen zweidimensionalen Fall - recht komplex 
und un\"ubersichtlich. Daher benutzen wir einen kleine Trick und setzen $r = \langle \vektor{w}, 
\vektor{w} \rangle$ und $s = - \langle \vektor{v}, \vektor{w} \rangle$. Wir k\"onnen 
annehmen, dass $\vektor{w} \neq 0$, also $r > 0$. Dann gilt
  	$$ \begin{array} {l c l}
  	0 & \leq & \vert r \vektor{v} + s \vektor{w} \vert^2 \\
    	& = & \langle r \vektor{v} + s \vektor{w}, r \vektor{v} + s \vektor{w} 
	\rangle \\
    	& = & r^2 \langle \vektor{v}, \vektor{v} \rangle + 2rs \langle \vektor{v}, 
	\vektor{w} \rangle + s^2 \langle \vektor{w}, \vektor{w} \rangle \\
    	& = & r^2 \langle \vektor{v}, \vektor{v} \rangle - 2r \langle \vektor{v}, 
	\vektor{w} \rangle^2 + s^2 \langle \vektor{w}, \vektor{w} \rangle \\
    	& = & r^2 \langle \vektor{v}, \vektor{v} \rangle - 2r \langle \vektor{v}, 
	\vektor{w} \rangle^2 + \langle \vektor{v}, \vektor{w} \rangle^2 \cdot 
   	\langle \vektor{w}, \vektor{w} \rangle \\
    	& = & r \langle \vektor{w}, \vektor{w} \rangle \cdot \langle \vektor{v}, 
   	\vektor{v} \rangle - 2r \langle \vektor{v}, 
  	\vektor{w} \rangle^2 + r \langle \vektor{v},  \vektor{w} \rangle^2 \\
    	& = & r \cdot \left( \langle \vektor{w}, \vektor{w} \rangle \cdot \langle \vektor{v}, 
   	\vektor{v} \rangle -  \langle \vektor{v},  \vektor{w} \rangle^2 \right)
  \end{array} $$
Da $r > 0$, folgt
  	$$ \langle \vektor{v},  \vektor{w} \rangle^2 \leq \langle 
   	\vektor{w}, \vektor{w} 
  	\rangle \cdot \langle \vektor{v}, \vektor{v} \rangle 
   	= \vert \vektor{v} \vert^2 
   	\cdot  \vert \vektor{w} \vert^2 $$
und daher (durch Ziehen der Quadratwurzel)
  	$$ \vert \langle \vektor{v},  \vektor{w} \rangle \vert \leq \vert \vektor{v} \vert 
    	\cdot \vert \vektor{w} \vert $$
also Aussage 1., und genau dann ist
  	$$ \vert \langle \vektor{v},  \vektor{w} \rangle \vert = \vert \vektor{v} \vert \cdot 
     	\vert \vektor{w} \vert $$
wenn 
  	$$ \vert r \vektor{v} + s \vektor{w} \vert^2 = 0 $$
also 
  	$$ r \vektor{v} + s \vektor{w} = 0 $$ 
oder
  	$$ \vektor{v} = - \frac {s}{r} \cdot  \vektor{w} $$
d.h. $\vektor{v}$ und $\vektor{w}$ sind kollinear. 

Falls $- \frac {s}{r} > 0$, so sind $\vektor{v}$ und $\vektor{w}$ parallel, und man rechnet 
unmittelbar nach, dass in diesem Fall  
  	$$ \langle \vektor{v},  \vektor{w} \rangle = \vert \vektor{v} \vert \cdot 
	\vert \vektor{w} \vert $$
w\"ahrend im Fall $- \frac {s}{r} < 0$ die Vektoren $\vektor{v}$ und $\vektor{w}$ anti--parallel 
sind und 
  	$$ \langle \vektor{v},  \vektor{w} \rangle = - \vert \vektor{v} \vert \cdot 
   	\vert \vektor{w} \vert $$
}
\medbreak

Aus den Regeln ergeben sich sofort einige interessante Konsequenzen für die Länge von Vektoren:

\begin{satz}\label{uvr_dreiecks_ungleichung} 
 Für Vektoren $\vektor{v}$, $\vektor{w}$ und einen Skalar $r \in \mathbb R$ gilt:

\begin{enumerate}
\item $\vert r \cdot \vektor{v} \vert \, = \, \vert r \vert \cdot \vert \vektor{v} \vert$.
\item $\vert \vektor{v} + \vektor{w} \vert \, \leq \, \vert \vektor{v} \vert \, + 
     \, \vert \vektor{w} \vert$.
\end{enumerate}
\end{satz}


\beweis{ In der Tat gilt
  	$$ \begin{array} {l c l}
  	\vert r \cdot \vektor{v} \vert^2  
  	& = &  \langle r \cdot \vektor{v}, r \cdot \vektor{v} \rangle \\
  	& = &  r^2 \cdot \langle  \vektor{v},  \vektor{v} \rangle \\
  	& = &  r^2 \cdot \vert \vektor{v} \vert^2
  \end{array} $$	
und durch Ziehen der Quadratwurzel folgt die erste Behauptung.

Zum Nachweis der zweiten Aussage gehen wir ähnlich vor:
  	$$ \begin{array} {l c l}
  	\vert  \vektor{v} + \vektor{w} \vert^2 
  	& = & \langle \vektor{v} + \vektor{w}, \vektor{v} + \vektor{w} \rangle \\
  	& = & \langle  \vektor{v},  \vektor{v} \rangle + 
	\langle  \vektor{v},  \vektor{w} \rangle + 
	\langle  \vektor{w},  \vektor{v} \rangle + 
	\langle  \vektor{w},  \vektor{w} \rangle  \\
  	& = & \vert  \vektor{v} \vert^2 + 2 \cdot \langle  \vektor{v}, \vektor{w} \rangle 
	+ \vert  \vektor{w} \vert^2 \\
  	& \leq & \vert  \vektor{v} \vert^2 + 2 \cdot \vert \vektor{v} 
 	\vert \cdot \vert \vektor{w} \vert + \vert  \vektor{w} \vert^2 \\
  	& = & \left( \vert  \vektor{v} \vert + \vert  \vektor{w} \vert \right)^2
 	\end{array} $$
wobei wir für die Ungleichung den Satz von Cauchy--Schwarz benutzt haben. Wiederum erhalten wir 
die Behauptung durch Ziehen der Quadratwurzel.
}
         
\bigbreak

\begin{notiz} Aus der ersten Bedingung des Satzes von Cauchy--Schwarz folgt wieder, dass es ein $\varphi \in 
[0, \pi]$ gibt mit 
  	$$ \vert \langle \vektor{v},  \vektor{w} \rangle \vert = \cos(\varphi) \vert \vektor{v} \vert 
  	\cdot  \vert \vektor{w} \vert $$
Dieses $\varphi$ heißt der \textbf{Winkel} zwischen $\vektor{v}$ und 
$\vektor{w}$. In der Ebene handelt es sich dabei in der Tat um den Winkel zwischen den beiden 
Vektoren $\vektor{v}$ und $\vektor{w}$. 
\end{notiz}

\begin{definition} Zwei $n$--dimensionale Vektoren $\vektor{v}$ und $\vektor{w}$ heißen 
\index{Vektoren!orthogonal} \textbf{orthogonal}, wenn gilt
  	$$ \langle \vektor{v},  \vektor{w} \rangle = 0 $$
Wir sagen in diesem Fall auch, dass $\vektor{v}$ senkrecht auf $\vektor{w}$ steht und schreiben 
$\vektor{v} \perp \vektor{w}$.
\end{definition}

\begin{beispiel} Die Vektoren  $\vektor{v} = \left( \begin{smallmatrix} 1 \\ 2 \\ -3 \\ 1
\end{smallmatrix} \right)$ und  $\vektor{w} = \left( \begin{smallmatrix} 2 \\ 4 \\ 4 \\ 2 
\end{smallmatrix} \right)$ sind orthogonal. 
\end{beispiel}

\begin{beispiel} Die Vektoren  $\vektor{e_1} = 
\left( \begin{smallmatrix} 1 \\ 0 \\ \vdots \\ 0
\end{smallmatrix} \right)$, $\vektor{e_2} = \left( \begin{smallmatrix} 0 \\ 1 \\ \vdots \\ 0 
\end{smallmatrix} \right), \ldots , \vektor{e_n} = \left( \begin{smallmatrix} 0 \\ \vdots \\ 0 \\ 1 
\end{smallmatrix} \right)$ sind paarweise orthogonal, d.h. für $i \neq j$ gilt
  $$ \langle \vektor{e_i}, \vektor{e_j}  \rangle =  0 $$
\end{beispiel}

\medbreak

\begin{definition}\label{linalg_vr_def_orth_komplement}
Ist $V \subseteq \mathbb R^n$ ein Untervektorraum, so heißt die Menge
  	$$ V^{\perp} = \{ \vektor{u} \in \mathbb R^n \vert \, 
      \langle \vektor{u}, \vektor{v}  \rangle =  0 \, \textrm{ für alle } 
      \vektor{v} \in V \} $$
das \index{orthogonales Komplement}\textbf{orthogonale Komplement} von $V$.
\end{definition}

\begin{notiz}\label{linalg_vr_orth_komplement} Ist $V \subseteq \mathbb R^n$ ein Untervektorraum der 
Dimension $l$, so ist $V^{\perp} \subseteq \mathbb R^n$ ein Untervektorraum der Dimension $n-l$.
\end{notiz}

\beweis Wir weisen die Untervektorraumaxiome für $V^{\perp}$ nach:

\begin{enumerate}
\item Sicherlich ist $\vektor{0} \in V^{\perp}$, und damit ist $V^{\perp}$ nicht leer.
\item Sind $\vektor{u_1}, \vektor{u_2} \in V^{\perp}$, und ist $\vektor{v} \in V$ 
beliebig, so gilt
  	$$  \langle \vektor{u_1} +  \vektor{u_2}, \vektor{v}  \rangle =  
    	\langle \vektor{u_1}, \vektor{v}  \rangle 
    	+  \langle \vektor{u_2}, \vektor{v}  \rangle = 0 $$
und damit ist $\vektor{u_1} +  \vektor{u_2} \in V^{\perp}$.
\item Ist $\vektor{u} \in V^{\perp}$ und $r \in \mathbb R$, und ist $\vektor{v} \in V$ 
beliebig, so gilt
  	$$  \langle r \cdot \vektor{u}, \vektor{v}  \rangle =  
     	r \cdot  \langle \vektor{u}, \vektor{v}  \rangle  = 0 $$
und damit ist $ r \cdot \vektor{u} \in V^{\perp}$.
\end{enumerate}
Also ist $V^{\perp}$ ein Untervektorraum von $\mathbb R^n$. Die Dimensionsformel ergibt sich aus dem 
Gram--Schmidt Orthonormalisierungsverfahren, dass wir in Satz~\ref{uvr_gram_schmidt} behandeln werden 
(vergleiche auch Aufgabe~\ref{linalg_vr_orth_komplement_aufgabe}).
\bigbreak

\bigbreak

Mit dem Begriff der Basis haben wir ein gutes Mittel kennengelernt, um Untervektorräume $V \subseteq 
\mathbb R^n$ zu beschreiben. Doch auch Basen sind nicht alle gleich gut geeignet. 

\begin{beispiel}\label{uvr_ortho_non_ortho} Eine Basis der $\mathbb R^3$ ist die Standardbasis 
	$$ \vektor{e_1} = \left( \begin{matrix} 1 \\ 0 \\  0 \end{matrix} \right), \quad  
	\vektor{e_2} = \left( \begin{matrix} 0 \\ 1 \\ 0 \end{matrix} \right), \quad 
	\vektor{e_3} = \left( \begin{matrix} 0  \\ 0 \\ 1 \end{matrix} \right) $$
eine weitere Basis des $\mathbb R^3$ ist 
	$$ \vektor{v_1} =  \left( \begin{matrix} 5 \\ 4 \\  4 \end{matrix} \right)\, , \quad  
	\vektor{v_2} = \left( \begin{matrix} 1 \\ 1 \\ 1 \end{matrix} \right)\, , \quad 
	\vektor{v_3} = \frac {1}{10} \cdot \left( \begin{matrix} 4  \\ 4 \\ 5 \end{matrix} \right) \ $$
wie man sich leicht überzeugt. Für viele Anwendungen ist die erste Basis besser geeignet als die zweite. 
So sind etwa in der ersten Basis alle Vektoren normiert, haben also die Länge 1. In der zweiten Basis sind 
die Vektoren nicht nur nicht normiert, sie sind auch noch alle unterschiedlich lang. Ein weiterer entscheidender 
Unterschied fällt sofort ins Auge, wenn wir uns die Basen graphisch vorstellen. Für die Standardbasis 
des $\mathbb R^3$ erhalten wir dann folgendes Bild

%\begin{figure}[h!] 
%  \centering
%     \scalebox{0.6}{\input{geom_vector_orth.tex}}
%\end{figure}

\begin{figure}[H]
	\vspace{-0.9cm}
	\begin{center}
	\begin{scaletikzpicturetowidth}{0.7\textwidth}
     		\begin{tikzpicture}[gnuplot,scale=\tikzscale]
%% generated with GNUPLOT 5.0p4 (Lua 5.3; terminal rev. 99, script rev. 100)
%% 23.08.2020 17:20:41
\path (0.000,0.000) rectangle (12.500,8.750);
\gpcolor{color=gp lt color border}
\node[gp node left] at (7.158,1.668) {$\footnotesize{x}$};
\node[gp node left] at (6.070,4.417) {$\footnotesize{y}$};
\node[gp node left] at (3.120,5.596) {$\footnotesize{z}$};
%
  \gpfill{rgb color={0.000,0.000,0.000}} (7.608,2.032)--(7.642,2.171)--(7.943,1.950)--(7.574,1.894)--cycle;
\gpcolor{rgb color={0.000,0.000,0.000}}
\gpsetlinetype{gp lt border}
\gpsetdashtype{gp dt solid}
\gpsetlinewidth{1.00}
\draw[gp path] (7.608,2.032)--(7.642,2.171)--(7.943,1.950)--(7.574,1.894)--cycle;
\draw[gp path] (2.483,3.285)--(7.608,2.032);
%
  \gpfill{rgb color={0.000,0.000,0.000}} (6.161,4.648)--(6.076,4.764)--(6.439,4.852)--(6.246,4.533)--cycle;
\draw[gp path] (6.161,4.648)--(6.076,4.764)--(6.439,4.852)--(6.246,4.533)--cycle;
\draw[gp path] (3.330,2.571)--(6.161,4.648);
%
  \gpfill{rgb color={0.000,0.000,0.000}} (3.849,5.484)--(3.706,5.484)--(3.849,5.829)--(3.992,5.484)--cycle;
\draw[gp path] (3.849,5.484)--(3.706,5.484)--(3.849,5.829)--(3.992,5.484)--cycle;
\draw[gp path] (3.849,2.376)--(3.849,5.484);
%
  \gpfill{rgb color={1.000,0.000,0.000}} (5.277,2.602)--(5.317,2.764)--(5.669,2.506)--(5.238,2.440)--cycle;
\gpcolor{rgb color={1.000,0.000,0.000}}
\gpsetlinewidth{2.00}
\draw[gp path] (5.277,2.602)--(5.317,2.764)--(5.669,2.506)--(5.238,2.440)--cycle;
\draw[gp path] (3.849,2.952)--(5.277,2.602);
%
  \gpfill{rgb color={1.000,0.000,0.000}} (5.005,3.800)--(4.906,3.935)--(5.330,4.038)--(5.104,3.665)--cycle;
\draw[gp path] (5.005,3.800)--(4.906,3.935)--(5.330,4.038)--(5.104,3.665)--cycle;
\draw[gp path] (3.849,2.952)--(5.005,3.800);
%
  \gpfill{rgb color={1.000,0.000,0.000}} (3.849,4.192)--(3.682,4.192)--(3.849,4.595)--(4.016,4.192)--cycle;
\draw[gp path] (3.849,4.192)--(3.682,4.192)--(3.849,4.595)--(4.016,4.192)--cycle;
\draw[gp path] (3.849,2.952)--(3.849,4.192);
\gpcolor{color=gp lt color border}
\node[gp node left] at (4.645,1.974) {$\vektor{e_1}$};
\node[gp node left] at (4.737,3.357) {$\vektor{e_2}$};
\node[gp node left] at (2.756,4.288) {$\vektor{e_3}$};
%% coordinates of the plot area
\gpdefrectangularnode{gp plot 1}{\pgfpoint{1.800cm}{0.771cm}}{\pgfpoint{10.700cm}{8.287cm}}
\end{tikzpicture}
%% gnuplot variables
 
	\end{scaletikzpicturetowidth}
	\end{center}
	\vspace{-1.2cm}
	\caption[]{Standardbasis}
\end{figure}

und für die zweite Basis 

%\begin{figure}[h!] 
%  \centering
%     \scalebox{0.6}{\input{geom_vector_non_orth.tex}}
%\end{figure}

\begin{figure}[H]
	\vspace{-0.9cm}
	\begin{center}
	\begin{scaletikzpicturetowidth}{0.7\textwidth}
     		\begin{tikzpicture}[gnuplot,scale=\tikzscale]
%% generated with GNUPLOT 5.0p4 (Lua 5.3; terminal rev. 99, script rev. 100)
%% 23.08.2020 17:20:47
\path (0.000,0.000) rectangle (12.500,8.750);
\gpcolor{color=gp lt color border}
\node[gp node left] at (6.900,1.393) {$\footnotesize{x}$};
\node[gp node left] at (5.232,4.943) {$\footnotesize{y}$};
\node[gp node left] at (2.511,5.459) {$\footnotesize{z}$};
%
  \gpfill{rgb color={0.000,0.000,0.000}} (7.432,1.706)--(7.466,1.845)--(7.767,1.624)--(7.398,1.568)--cycle;
\gpcolor{rgb color={0.000,0.000,0.000}}
\gpsetlinetype{gp lt border}
\gpsetdashtype{gp dt solid}
\gpsetlinewidth{1.00}
\draw[gp path] (7.432,1.706)--(7.466,1.845)--(7.767,1.624)--(7.398,1.568)--cycle;
\draw[gp path] (2.307,2.960)--(7.432,1.706);
%
  \gpfill{rgb color={0.000,0.000,0.000}} (5.712,4.567)--(5.627,4.683)--(5.990,4.771)--(5.797,4.452)--cycle;
\draw[gp path] (5.712,4.567)--(5.627,4.683)--(5.990,4.771)--(5.797,4.452)--cycle;
\draw[gp path] (2.870,2.483)--(5.712,4.567);
%
  \gpfill{rgb color={0.000,0.000,0.000}} (3.217,5.471)--(3.074,5.471)--(3.217,5.816)--(3.360,5.471)--cycle;
\draw[gp path] (3.217,5.471)--(3.074,5.471)--(3.217,5.816)--(3.360,5.471)--cycle;
\draw[gp path] (3.217,2.352)--(3.217,5.471);
%
  \gpfill{rgb color={1.000,0.000,0.000}} (10.186,6.544)--(10.106,6.691)--(10.540,6.737)--(10.266,6.397)--cycle;
\gpcolor{rgb color={1.000,0.000,0.000}}
\gpsetlinewidth{2.00}
\draw[gp path] (10.186,6.544)--(10.106,6.691)--(10.540,6.737)--(10.266,6.397)--cycle;
\draw[gp path] (3.217,2.737)--(10.186,6.544);
%
  \gpfill{rgb color={1.000,0.000,0.000}} (4.483,3.571)--(4.391,3.711)--(4.820,3.793)--(4.575,3.432)--cycle;
\draw[gp path] (4.483,3.571)--(4.391,3.711)--(4.820,3.793)--(4.575,3.432)--cycle;
\draw[gp path] (3.217,2.737)--(4.483,3.571);
%
  \gpfill{rgb color={1.000,0.000,0.000}} (3.566,3.009)--(3.464,3.140)--(3.858,3.237)--(3.667,2.879)--cycle;
\draw[gp path] (3.566,3.009)--(3.464,3.140)--(3.858,3.237)--(3.667,2.879)--cycle;
\draw[gp path] (3.217,2.737)--(3.566,3.009);
\gpcolor{color=gp lt color border}
\node[gp node left] at (8.135,5.230) {$\vektor{v_1}$};
\node[gp node left] at (3.906,4.131) {$\vektor{v_2}$};
\node[gp node left] at (3.217,3.661) {$\vektor{v_3}$};
%% coordinates of the plot area
\gpdefrectangularnode{gp plot 1}{\pgfpoint{1.800cm}{0.771cm}}{\pgfpoint{10.700cm}{8.287cm}}
\end{tikzpicture}
%% gnuplot variables
 
	\end{scaletikzpicturetowidth}
	\end{center}
	\vspace{-1.2cm}
	\caption[]{andere Basis}
\end{figure}


Wir sehen also, dass die Vektoren der ersten Basis jeweils senkrecht aufeinanderstehen, während die Vektoren 
der zweiten Basis relativ eng beieinander liegen und alle in einen Quadranten des Raums zeigen. Solche Basen 
sind vor allem für numerische Berechnungen und Näherungen sehr schlecht geeignet, da sie dazu tendieren, 
Messfehler zu verstärken. 
\end{beispiel}

Dieses Beispiel motiviert die folgende Definition:

\begin{definition} Eine Basis $\vektor{v_1}, \ldots, \vektor{v_m}$ eines Untervektorraums 
$U \subseteq \mathbb R^n$ heißt \index{Untervektorraum!Orthonormalbasis}\textbf{Orthonormalbasis} von 
$U$ oder kurz \textbf{ONB} von $U$, wenn gilt:

\begin{itemize}
\item $\vert \vektor{v_i} \vert = 1$ für $i = 1, \ldots, m$, d.h. alle Vektoren der Basis sind 
normiert.
\item $\langle \vektor{v_i}, \vektor{v_j} \rangle = 0$ für $i \neq j$, d.h. die Vektoren 
stehen paarweise senkrecht aufeinander.
\end{itemize}
\end{definition}

\begin{beispiel} Die Standardbasis $\vektor{e_1}, \vektor{e_2}, \vektor{e_3}$ des 
$\mathbb R^3$ ist eine ONB von $\mathbb R^3$, die Basis $\vektor{v_1}, \vektor{v_2}, 
\vektor{v_3}$ aus Beispiel~\ref{uvr_ortho_non_ortho} ist es nicht.
\end{beispiel}

\begin{beispiel} Die Standardbasis $\vektor{e_1}, \vektor{e_2}, \ldots,  \vektor{e_n}$ 
des $\mathbb R^n$ ist eine ONB.
\end{beispiel}

\medbreak

Ganz offensichtlich sind Orthonormlabasen anderen Basen vorzuziehen. Damit stellt sich die Frage, wann solche 
existieren, und wie wir sie finden können. Antwort darauf liefert uns das Orthonormalisierungsverfahren von 
Gram--Schmidt und der folgende 

\begin{satz}\label{uvr_gram_schmidt} Jeder Untervektorraum $U$ des $\mathbb R^n$ besitzt eine 
Orthonormalbasis.
\end{satz}

\beweis{ Der Beweis liefert und gleichzeitig einen Algorithmus, mit dem wir aus einer beliebigen Basis von $U$ 
eine Orthonormalbasis von $U$ machen können.

Wir betrachten dazu eine beliebige Basis $\vektor{u_1}, \vektor{u_2}, \ldots, \vektor{u_m}$ von 
$U$ und wandeln diese Basis in zwei Schritten in eine Orthonormalbasis um. Zunächst bilden wir die Vektoren:
  	$$ \begin{array}{l c l}
   	\vektor{v_1} & = & \vektor{u_1} \\
   	\vektor{v_2} & = & \vektor{u_2} - \frac {\langle \vektor{u_2}, 
   	\vektor{v_1}\rangle}{ \langle \vektor{v_1}, \vektor{v_1}\rangle} \cdot 
   	\vektor{v_1} \\
   	\vektor{v_3} & = & \vektor{u_3} - \frac {\langle \vektor{u_3}, 
   	\vektor{v_1}\rangle}{ \langle \vektor{v_1}, \vektor{v_1}\rangle} \cdot 
   	\vektor{v_1} - \frac {\langle \vektor{u_3}, 
   	\vektor{v_2}\rangle}{ \langle \vektor{v_2}, \vektor{v_2}\rangle} \cdot 
   	\vektor{v_2}  \\
   	& \vdots & \\
  	 \vektor{v_m} & = & \vektor{u_m} - \frac {\langle \vektor{u_m}, 
   	\vektor{v_1}\rangle}{ \langle \vektor{v_1}, \vektor{v_1}\rangle} \cdot 
   	\vektor{v_1} - \frac {\langle \vektor{u_m}, 
   	\vektor{v_2}\rangle}{ \langle \vektor{v_2}, \vektor{v_2}\rangle} \cdot 
   	\vektor{v_2}  - \ldots - \frac {\langle \vektor{u_m}, 
   	\vektor{v_{m-1}}\rangle}{ \langle \vektor{v_{m-1}}, \vektor{v_{m-1}}\rangle} \cdot 
   	\vektor{v_{m-1}}
   	\end{array} $$
dh. wir übernehmen $\vektor{u_1}$ unverändert als $\vektor{v_1}$, wir ziehen von 
$\vektor{u_2}$ seinen Anteil ''in Richtung $\vektor{v_1}$'' ab usw. Dann gilt:

Die Vektoren $\vektor{v_1}, \ldots , \vektor{v_m}$ bilden eine Basis von $U$, bestehend aus Vektoren, 
die paarweise senkrecht aufeinander stehen. 

Der Nachweis, dass $\langle \vektor{v_i}, \vektor{v_j} \rangle = 0$ für $i \neq j$ ist eine 
einfache aber sehr langwierige Rechnung, die wir dem Leser überlassen. Um zu zeigen, dass $\vektor{v_1}, 
\ldots , \vektor{v_m}$ den Unterraum $U$ erzeugen, reicht es zu zeigen, dass sich die ursprünglichen 
Basisvektoren $\vektor{u_1}, \vektor{u_2}, \ldots, \vektor{u_m}$ damit darstellen 
lassen. Das ist aber klar, da sich ja jede der definerenden Gleichungen für ein $\vektor{v_i}$ direkt
nach $\vektor{u_i}$ auflösen lässt. Die lineare Unabhängigkeit der Vektoren folgt etwa aus 
ihrer paarweisen Orthogonalität (siehe auch Aufgabe~\ref{uvr_aufg_orthogonal_lu}). Es ist aber auch 
nicht schwer, sie direkt nachzurechnen. 

Wir haben also nun eine Basis von $U$ gefunden, die aus Vektoren besteht, die paarweise senkrecht 
aufeinander stehen. Um eine Orthonormalbasis zu finden, reicht es also, diese Basis zu normieren:
  	$$ \begin{array}{l c l}
  	\vektor{w_1} & = & \frac {\vektor{v_1}}{\vert \vektor{v_1} \vert} \\
  	\vektor{w_2} & = & \frac {\vektor{v_2}}{\vert \vektor{v_2} \vert} \\
   	& \vdots & \\
  	\vektor{w_m} & = & \frac {\vektor{v_m}}{\vert \vektor{v_m} \vert}
  	\end{array} $$
Es ist klar, dass die Vektoren $\vektor{w_1}, \ldots , \vektor{w_m}$ normiert sind. Ebenfalls 
klar ist, dass sich durch die Normierung nichts daran ändert, dass sie paarweise orthogonal sind und dass 
sich dadurch auch nichts an ihrer Eigenschaft, Basis von $U$ zu sein, ändert.
}

\medbreak

\begin{beispiel} Wir betrachten den Untervektorraum $U \subseteq \mathbb R^3$, der von den Vektoren 
  	$$ \vektor{u_1} = \left( \begin{matrix} 1 \\ 2 \\ 3 \end{matrix} \right), \qquad 
   	\vektor{u_1} = \left( \begin{matrix} 2 \\ 3 \\ 4 \end{matrix} \right) $$
erzeugt wird. Zunächst überzeugen wir uns, dass diese beiden Vektoren auch eine Basis von $U$ bilden. 
Das ist klar, denn keiner ist ein Vielfaches des anderen. 

\textbf{Orthogonalisierung:}
  	$$ \begin{array}{l c l c l }
   	\vektor{v_1} & = & \vektor{u_1} & = & \, \, \left( \begin{matrix} 1 \\ 2 \\ 3 \end{matrix} 
	\right) \\
   	\vektor{v_2} & = & \vektor{u_2} - \frac {\langle \vektor{u_2}, 
   	\vektor{v_1}\rangle}{ \langle \vektor{v_1}, \vektor{v_1}\rangle} \cdot 
   	\vektor{v_1} & = & \left( \begin{matrix} \frac {8}{14} \\ \frac {2}{14} \\ -\frac {4}{14} 
	\end{matrix}  \right)  
  	\end{array} $$

\textbf{Normalisierung:}
  	$$ \begin{array}{l c l c l}
  	\vektor{w_1} & = & \frac {\vektor{v_1}}{\vert \vektor{v_1} \vert}
    	& = & \, \left( \begin{matrix} \frac {1}{\sqrt{14}} \\ \frac {2}{\sqrt{14}} \\ \frac {3}{\sqrt{14}} 
	\end{matrix}  \right) \\
  	\vektor{w_2} & = & \frac {\vektor{v_2}}{\vert \vektor{v_2} \vert}
    	& = & \left( \begin{matrix} \frac {8}{\sqrt{84}} \\ \frac {2}{\sqrt{84}} \\ -\frac {4}{\sqrt{84}} 
	\end{matrix}  \right)  
  	\end{array} $$
und wir haben die gewünschte Orthonormalbasis gefunden.
\end{beispiel}


\bigbreak

\begin{aufgabe} Zeigen Sie, dass die Menge  $V = \{ \vektor{v} = \left( \begin{smallmatrix} 
v_1 \\ v_2 \\ v_3 \\ v_4  \end{smallmatrix} \right) \, \vert \, 2 v_1 + 3 v_2 - 4 v_3 + v_4 = 0 \}$ 
ein Untervektorraum von $\mathbb R^4$ ist.
\end{aufgabe}

\begin{aufgabe} Zeigen Sie, dass die Menge  $V = \{ \vektor{v} = \left( \begin{smallmatrix} t 
\\ t^2 \\ t^3 \\ t^4\end{smallmatrix} \right) \, \vert \, t \in \mathbb R \}$ 
kein Untervektorraum von $\mathbb R^4$ ist.
\end{aufgabe}

\begin{aufgabe} Überprüfen Sie, ob die Menge $ V = \{ \vektor{v} = 
\left( \begin{smallmatrix} r \\ s \\ r+s \\ r-s  \end{smallmatrix} \right) \, \vert \, 
r, s \in \mathbb R \}$ Ein Untervektorraum von $\mathbb R^4$ ist.
\end{aufgabe}

\begin{aufgabe} Zeigen Sie, dass die Vektoren  $\vektor{v_1} = 
\left( \begin{smallmatrix} 1 \\ 0 \\ 2 \\ 0
\end{smallmatrix} \right)$, $\vektor{v_2} = \left( \begin{smallmatrix} 0 \\ 1 \\ 0 \\ 2 
\end{smallmatrix} \right)$ und $\vektor{v_3} = \left( \begin{smallmatrix} 1 \\ 1 \\ 1 \\ 1 
\end{smallmatrix} \right)$ linear unabhängig sind.
\end{aufgabe}

\begin{aufgabe} Zeigen Sie, dass die Vektoren $\vektor{v_1} = 
\left( \begin{smallmatrix} 1 \\ 2 \\ 3 \\ 4
\end{smallmatrix} \right)$, $\vektor{v_2} = \left( \begin{smallmatrix} 4 \\ 2 \\ 0 \\ 0 
\end{smallmatrix} \right)$ und $\vektor{v_3} = \left( \begin{smallmatrix} 3 \\ 3 \\ 3 \\ 4 
\end{smallmatrix} \right)$ linear abhängig sind.
\end{aufgabe}

\begin{aufgabe} Zeigen Sie, dass die Vektoren  $\vektor{v_1} = 
\left( \begin{smallmatrix} 1 \\ 0 \\ 1 \end{smallmatrix} \right)$, 
$\vektor{v_2} = \left( \begin{smallmatrix} 0 \\ 1 \\ 1 
\end{smallmatrix} \right)$ und $\vektor{v_3} = \left( \begin{smallmatrix} 1 \\ 1 \\  1 
\end{smallmatrix} \right)$ eine Basis von $\mathbb R^3$ bilden.
\end{aufgabe}

\begin{aufgabe} Gegeben seien die beiden Vektoren
  $$ \vektor{v_1} = \left( \begin{matrix} 2 \\ -1 \\ 0  \end{matrix} \right), \quad  
     \vektor{v_2} = \left( \begin{matrix} -3 \\ 0 \\ 1  \end{matrix} \right) $$
Zeigen Sie 
  $$ \langle \{ \vektor{v_1}, \vektor{v_2} \} \rangle = 
    \{  \left( \begin{matrix} x_1 \\ x_2 \\ x_3  \end{matrix} \right) \in \mathbb R^n 
    \, \vert \, x_1 + 2 x_2 + 3 x_3 = 0 \} $$
Bilden $\vektor{v_1}$ und $\vektor{v_2}$ auch eine Basis dieses 
Untervektorraums?
\end{aufgabe}

\begin{aufgabe}  Eine Basis von $\mathbb R^3$ ist 
$$ \vektor{v_1} = 
\left( \begin{matrix} 3 \\ 4 \\  5
\end{matrix} \right), \quad  \vektor{v_2} = \left( \begin{matrix} 3 \\ 5 \\ 6 
\end{matrix} \right), \quad \vektor{v_3} = \left( \begin{matrix} 1  \\ 2 \\ 2 
\end{matrix} \right) $$
(vergleiche Beispiel~\ref{uvr_ortho_non_ortho}). Orthonormalisieren Sie diese Basis.
\end{aufgabe}

\begin{aufgabe} Zeigen Sie, dass 
$$ \vektor{v_1} = 
\left( \begin{matrix} 1 \\ 2 \\  3
\end{matrix} \right), \quad  \vektor{v_2} = \left( \begin{matrix} 3 \\ 2 \\ 1 
\end{matrix} \right), \quad \vektor{v_3} = \left( \begin{matrix} 1  \\ 0 \\ 1 
\end{matrix} \right) $$ eine Basis von $\mathbb R^3$ ist und orthonormalisieren Sie 
diese Basis.
\end{aufgabe}

\begin{aufgabe}\label{uvr_aufg_orthogonal_lu} 
Zeigen Sie: Stehen zwei vom Nullvektor verschiedene Vektoren $\vektor{v}$ und 
$\vektor{w}$ senkrecht aufeinander, so sind sie linear unabhängig. 
Verallgemeinern Sie diese Aussage auf $n$ Vektoren, die paarweise orthogonal sind.
\end{aufgabe}  

\begin{aufgabe}\label{linalg_vr_orth_komplement_aufgabe} Weisenn Sie die Dimensionsformel 
aus Bemerkung~\ref{linalg_vr_orth_komplement} nach.
\end{aufgabe}

\newpage

\section{Allgemeine reelle Vektorräume}\label{section_vr_reell}

\setcounter{definition}{0}
\setcounter{beispiel}{0}
\setcounter{notiz}{0}

Im vergangenen Abschnitt haben wir speziell den reellen Vektorraum $\R^n$ untersucht, da diese 
für die Anwendungen die wichtigsten reellen Vektorräume bilden. Es gibt jedoch auch noch 
viele andere Mengen, die sich mithilfe der Vektorraumtheorie studieren und strukturieren lassen. 
Deshalb wollen wir uns in diesem Abschnitt mit allgemeinen reellen Vektorräumen beschäftigen. 

%
%Im vorangegangenen Abschnitt haben wir das Konzept der ebenen und räumlichen Vektoren 
%schon stark verallgemeinert und den $\mathbb R^n$ mit einer Vektoraddition und einer 
%Skalarmultiplikation versehen. Darüberhinaus haben wir aber auch Teilmengen $V$ des 
%$\mathbb R^n$ betrachtet, die Untervektorräume sind. Diese Untervektorräume $V$ sind 
%in der Regel vom $\mathbb R^n$ verschieden, trotzdem haben sie aber ähnliche 
%Eigenschaften: Zwei Vektoren in $V$ lassen sich addieren und ergeben wieder einen Vektor 
%in $V$, ein Vektor in $V$ kann mit einem Skalar aus $\mathbb R$ multipliziert werden und 
%das Ergebnis ist wieder in $V$. Damit haben diese Untervektorräume ähnliche 
%strukturielle Eigenschaften wie $\mathbb R^n$ selbst, und es liegt daher nahe, diese 
%strukturiellen Eigenschaften zu abstrahieren. 
%
%\begin{definition} Ein (reeller) \index{Vektorraum}\textbf{Vektorraum} ist eine 
%nichtleere Menge $V$ zusammen mit einer Addition 
%  $$ '+': V \times V \longrightarrow V, \quad (\vektor{v}, \vektor{w}) 
%     \longmapsto \vektor{v} + \vektor{w} $$
%und einer Skalarmultiplikation 
%  $$ '\cdot': \mathbb R \times V \longrightarrow V, \quad (r, \vektor{v}) 
%     \longmapsto r \cdot \vektor{v} $$
%so dass für $\vektor{u}, \vektor{v}$ und 
%$\vektor{w}$ in $V$ und Skalare $r$ und $s$ gilt:
%
%\begin{tabular} {l l c l}
%$V1:$ & $\left(\vektor{u} + \vektor{v} \right) + \vektor{w} = 
%\vektor{u} + \left(\vektor{v} + \vektor{w} \right)$ & $\,$ &
%(1. Assoziativgesetz) \\
%$V2:$ & $\vektor{v} + \vektor{w} = \vektor{w} + 
%\vektor{v}$ & & (Kommutativgesetz) \\
%$V3:$ & Es existiert ein Element $\vektor{0} \in V$ & & \\
%& mit  $\vektor{v} + \vektor{0} = \vektor{v}$ & & (neutrales 
%Element \\
%& & & der Addition) \\
%$V4:$ & Zu jedem $\vektor{v} \in V$ existiert ein  
%& & \\
%& $-\vektor{v} \in V$ mit $\vektor{v} + (- \vektor{v}) = \vektor{0}$ & &
%(inverses Element \\
%& & & der Addition) \\
%$V5:$ & $(r \cdot s) \cdot \vektor{v} = r \cdot ( s \cdot \vektor{v})$ 
%& & (2. Assoziativgesetz) \\
%$V6:$ & $r \cdot (\vektor{v} + \vektor{w}) = r \cdot 
%\vektor{v} + r \cdot \vektor{w}$ & & (1. Distributivgesetz) \\
%$V7:$ & $(r + s) \cdot \vektor{v} = r \cdot \vektor{v} + 
%s \cdot \vektor{v}$ & & (2.Distributivgesetz) \\
%$V8:$ & $ 1 \cdot \vektor{v} = \vektor{v}$ & & (neutrales Element \\
%& & & der Multiplikation)
%\end{tabular}
%
%\end{definition}
%
%\begin{beispiel} $\mathbb R^n$ zusammen mit den Operationen aus dem letzten Abschnitt 
%ist ein Vektoraum.
%\end{beispiel}
%
%\begin{beispiel} Jeder Untervektorraum $V \subseteq \mathbb R^n$ ist ein Vektorraum.
%\end{beispiel}

\begin{beispiel}\label{vectorraum_abbild} Wir betrachten die Menge $V$ aller Abbildungen
  	$$ f : [0,1] \longrightarrow \mathbb R $$
Für zwei Abbildungen $f, g \in V$ definieren wir die Abbildung $f + g$ durch
  	$$ (f+g)(x) = f(x) + g(x) \quad \textrm{ für alle } x \in [0, 1] $$
und für eine Abbildung $f \in V$ und ein $r \in \mathbb R$ erklären wir $r \cdot f$ 
durch 
  	$$ (r \cdot f)(x) = r \cdot f(x) \quad \textrm{ für alle } x \in [0, 1] $$
Dann ist $(V, +, \cdot )$ ein reeller Vektorraum, wie man leicht nachrechnet.

Für diesen Vektorraum schreibt man auch $\mathrm{Abb}([0,1], \mathbb R)$ oder 
$[0,1]^{\mathbb R}$ und nennt ihn den Vektorraum der (reellwertigen) Abbildungen auf 
dem Intervall $[0,1]$. 
\end{beispiel}

\begin{beispiel}\label{vectorraum_poly} Eine Abbildung 
  	$$ f : [a,b] \longrightarrow \mathbb R $$
auf einem Intervall $[a,b]$ heißt Polynomabbildung, wenn sich $f$ in der Form 
  	$$ f(x) = a_n x^n + a_{n-1}x^{n-1} + \cdots + a_1 x + a_0 $$
für ein geeignetes $n$ und mit geeigneten $a_0, a_1, \ldots, a_n \in \mathbb R$ schreiben lässt, 
vergleiche Beispiel~\ref{funktion_polynomial}. Beispiele hierfür sind etwa $f(x) = x^2+2$ 
oder $f(x) = 6x^3 - 3x + 4$.
 
Wir betrachten die Menge $U$ der Polynomabbildungen 
  	$$ f : [0,1] \longrightarrow \mathbb R $$
Wie in Beispiel~\ref{vectorraum_abbild} definieren wir für
zwei polynome Abbildungen $f, g \in U$ die Abbildung $f + g$ durch
  	$$ (f+g)(x) = f(x) + g(x) \quad \textrm{ für alle } x \in [0, 1] $$
und für eine polynome Abbildung $f \in U$ und ein $r \in \mathbb R$ erklären wir 
$r \cdot f$ durch 
  	$$ (r \cdot f)(x) = r \cdot f(x) \quad \textrm{ für alle } x \in [0, 1] $$ 
Dann ist zunächst zu beachten, dass sowohl $f+g$ als auch $r \cdot f$ wieder 
polynomiale Abbildungen sind. So ist etwa $(x^2+2) + (6x^3 - 3x + 4) = 6x^3 + x^2 
- 2x + 6$ oder $3(6x^3 - 3x + 4) = 18 x^3 - 9 x + 12$. 

Hierfür rechnen wir sofort nach, dass $U \subseteq \mathrm{Abb}([0,1], \mathbb R)$ ein 
Untervektorraum ist, und damit insbesondere auch selbst ein reeller Vektorraum. 
%Auch hier rechnet man sofort 
%nach, dass $(U, +, \cdot)$ ein Vektorraum ist. 

Für diesen Vektorraum schreiben wir $\textrm{Pol}([0,1], \mathbb R)$ und nennen ihn 
den Vektorraum der (reellwertigen) poylnomialen Abbildungen auf 
dem Intervall $[0,1]$. (Genauso gut hätten wir natürlich jedes andere Intervall 
$[a,b]$ nehmen können.)
\end{beispiel}

%Der Vektorraum $U$ aus Beispiel ~\ref{vectorraum_poly} ist eine Teilmenge des 
%Vektorraums $V$ aus Beispiel ~\ref{vectorraum_abbild}, und die Vektorraumoperationen 
%auf $U$ sind genauso definiert wie die auf $V$. Solche Paare sind besonders interessant
%für uns.
%
%\begin{definition} Sei $(V, + , \cdot)$ ein Vektorraum und sei $U \subseteq V$ eine 
%nichtleere Teilmenge. Dann heißt $U$ \index{Untervektorraum}
%\textbf{Untervektorraum} von $V$ wenn gilt:
%
%\begin{enumerate}
%\item Sind $\vektor{v}, \vektor{w} \in U$ so ist auch 
%$\vektor{v} + \vektor{w} \in U$.
%\item Ist $\vektor{v} \in U$ und ist $r$ ein Skalar, so ist auch $r \cdot 
%\vektor{v} \in U$.
%\end{enumerate}
%\end{definition}
%
%\begin{beispiel} Die Beispiele von Untervektorräumen des $\mathbb R^n$ die wir im 
%letzten Abschnitt kennengelernt haben, sind auch Untervektorräume des $\mathbb R^n$ 
%im Sinne dieser Definition.
%\end{beispiel}
%
%\begin{beispiel} Der Vektorraum $\textrm{Pol}([0,1], \mathbb R)$ aus Beispiel 
%~\ref{vectorraum_poly} ist ein Untervektorraum von $\textrm{Abb}([0,1], \mathbb R)$ 
%aus Beispiel ~\ref{vectorraum_abbild}
%\end{beispiel}

Wir betrachten nun  einen Vektorraum $V$  und Vektoren $\vektor{w}, \vektor{v_1}, \ldots, 
\vektor{v_n} \in V$.

\begin{definition}\label{vectorraum_lin_comp_fin} $\vektor{w}$ heißt 
Linearkombination von $\vektor{v_1}, \ldots , \vektor{v_n}$, wenn es 
$r_1, \ldots , r_n \in \mathbb R$ gibt mit 
  	$$ \vektor{w} = r_1 \cdot \vektor{v_1} + \cdots + r_n \cdot \vektor{v_n} $$
\end{definition}

\begin{beispiel} Sei $V = \mathbb R^n$ und seien $\vektor{w} = 
\left( \begin{smallmatrix} 1 \\ 1 \\ 1 \\ 1 \end{smallmatrix}  \right)$, 
$\vektor{v_1} = \left( \begin{smallmatrix} 1 \\ 2 \\ 3 \\ 4 \end{smallmatrix} 
\right)$ und  $\vektor{v_2} =
\left( \begin{smallmatrix} 8 \\ 6 \\ 4 \\ 2 \end{smallmatrix} \right)$. Dann ist 
$\vektor{w}$ eine Linearkombination von $\vektor{v_1}$ und 
$\vektor{v_2}$, denn es gilt:
  	$$ \vektor{w} = \frac {1}{5} \cdot \vektor{v_1} + \frac {1}{10} \cdot \vektor{v_2} $$
\end{beispiel}

\begin{beispiel} Sei $V = \mathbb R^n$ und seien $\vektor{w} = 
\left( \begin{smallmatrix} 1 \\ 1 \\ 2 \\ 2 \end{smallmatrix}  \right)$, 
$\vektor{v_1} = \left( \begin{smallmatrix} 1 \\ 2 \\ 3 \\ 4 \end{smallmatrix} 
\right)$ und  $\vektor{v_2} =
\left( \begin{smallmatrix} 8 \\ 6 \\ 4 \\ 2 \end{smallmatrix} \right)$. Dann ist 
$\vektor{w}$ keine Linearkombination von $\vektor{v_1}$ und 
$\vektor{v_2}$. Das ist so unmittelbar nicht sofort einzusehen. Wir werden 
diese Beispiel später im Zusammenhang mit linearen Gleichungssystemen noch einmal 
untersuchen und überlassen es hier dem Leser, zu zeigen, dass es keine reellen 
Zahlen $r, s$ gibt mit 
  	$$ \vektor{w} = r \cdot \vektor{v_1} + s \cdot \vektor{v_2} $$
\end{beispiel}

\begin{beispiel} Sei $V = \textrm{Pol}([0,1], \mathbb R)$ der Vektorraum aus Beispiel 
~\ref{vectorraum_poly} und seien $f(x) = 2x^2 - 4$, $f_0(x) = 1$, $f_1(x) = x$, 
$f_2(x) = x^2$ und $f_3(x) = x^3$. Dann ist $f$ Linearkombination von $f_0, f_1, f_2, 
f_3$, denn es gilt
  	$$ f(x) = -4 \cdot f_0(x) + 2 \cdot f_2(x) \,  = -4 \cdot f_0(x) + 0 \cdot f_1(x) 
   	+ 2 \cdot f_2(x) + 0 \cdot f_3(x) $$
\end{beispiel}

Das Konzept der Linearkombination lässt sich auch auf unendliche Familien von 
Vektoren ausdehnen. Dazu sei $\Lambda$ eine Indexmenge, etwa $\Lambda = \{ 1, 
\ldots , n \}$ oder $\Lambda = \mathbb N$ und es sei $\left\{ 
\vektor{v_{\lambda}} \right\}_{\lambda \in \Lambda}$ eine Familie von 
Elementen von $V$ (d.h. für jedes $\lambda \in \Lambda$ ist ein Element 
$\vektor{v_{\lambda}} \in V$ gegeben). Für $\Lambda = \{ 1, 
\ldots , n \}$ ist das einfach eine Menge $\{ \vektor{v_1}, \ldots , 
\vektor{v_n} \}$ von $n$ Elementen aus $V$. Für $\Lambda = \mathbb N = 
\{ 0, 1, 2, \ldots  \}$ ist eine Menge $\{ \vektor{v_0}, 
\vektor{v_1}, \vektor{v_2} \ldots  \}$ von 
unendliche vielen Elementen aus $V$. 

\begin{definition}\label{vectorraum_lin_comp_infin} Ein Vektor $\vektor{w}$ 
heißt \index{Linearkombination}\textbf{Linearkombination} von $\left\{ 
\vektor{v_{\lambda}} \right\}_{\lambda \in \Lambda}$, wenn es \textit{endlich 
viele} $\lambda_1, \ldots , \lambda_n \in \Lambda$  und $r_1, \ldots, r_n \in 
\mathbb R$ gibt mit 
  	$$ \vektor{w} = r_1 \cdot \vektor{v_{\lambda_1}} + \cdots +  r_n \cdot \vektor{v_{\lambda_n}} $$
\end{definition}

\begin{notiz} Ist $\Lambda$ endlich, $\lambda = \{ 1, \ldots, n \}$, so stimmt diese 
Definition mit der aus Definition ~\ref{vectorraum_lin_comp_fin} überein.
\end{notiz}

\begin{beispiel}\label{vectorraum_poly_infin} Sei $V = \textrm{Pol}([0,1], \mathbb R)$ und sei 
$\Lambda = \mathbb N = \{0, 1, 2, \ldots \}$. Definiere $f_n$ durch 
  	$$ f_n(x) = x^n \quad \textrm{ für alle } n \in \Lambda = \mathbb N $$
(also $f_0(x) = 1$, $f_1(x) = 1$, $f_2(x) = x^2, \ldots$). Ferner sei $f \in V$ die 
Polynomfunktion mit $f(x) = 3x^5 -2 x^2 + 7$. Dann ist $f$ eine Linearkombination der 
$\left( f_{\lambda} \right)_{\lambda \in \Lambda}$. 

Dazu betrachte die endliche Teilmenge von $\Lambda$ mit den Elementen $\lambda_1 = 0$, $\lambda_2 = 2$ und 
$\lambda_3 = 5$. Dann gilt hierfür
 	$$ f(x) = 7 \cdot f_{\lambda_1}(x) + (-2) \cdot f_{\lambda_2}(x) + 3 \cdot 
    	f_{\lambda_3}(x) $$

Allgemein ist jede beliebige Polynomfunktion $f: [0, 1] \longrightarrow \mathbb R$ eine 
Linearkombination der $\left\{ f_{\lambda} \right\}_{\lambda \in \Lambda}$. Schreibt sich dazu $f(x)$ etwa als
  	$$ f(x) = a_n x^n + a_{n-1} x^{n-1} + \cdots + a_1 x + a_0, $$
so betrachte die endliche Teilmenge $\{0, 1, \ldots, n \} \subseteq \Lambda$ und erhalte mit
  	$$ f(x) = a_n \cdot f_n(x) + a_{n-1} \cdot f_{n-1}(x)  + \cdots + a_1 \cdot f_1(x) + a_0 \cdot f_0(x) $$ 
eine Darstellung wie gewünscht.

Die Exponentialfunktion $e^x$, eingeschränkt auf das Intervall $[0,1]$ hat eine 
Beschreibung durch die sogenannte Exponentialreihe
  	$$ e^x = \sum_{n = 0}^{\infty} \frac {x^n}{n!} = 1 + x + \frac {x^2}{2} + 
	\frac {x^3}{6} + \frac {x^4}{24} + \ldots $$
hat also auch eine Summendarstellung mit den $f_{\lambda}$. Trotzdem ist das keine 
Darstellung der Exponentialfunktion als Linearkombination der $f_{\lambda}$, da 
unendlich viele der $f_{\lambda}$ benötigt werden. 

Man benötigt etwas Analysis um zu zeigen, dass es auch keine andere Darstellung der 
Exponentialfunktion als Linearkombination der $f_{\lambda}$ gibt, die 
Exponentialfunktion ist also keine Polynomfunktion.
\end{beispiel}

\begin{definition} Eine Familie  $\left\{ \vektor{v_{\lambda}} \right\}_{\lambda \in \Lambda}$ von 
Vektoren in $V$ heißt ein \index{Vektorraum!Erzeugendensystem}\textbf{Erzeugendensystem} 
von $V$, wenn sich jedes Element $\vektor{v} \in V$ als 
Linearkombination von $\left\{ \vektor{v_{\lambda}} \right\}_{\lambda \in \Lambda}$ schreiben lässt.
\end{definition}

\begin{beispiel} sei $V = \{ \left( \begin{smallmatrix} x \\ y \\ 0 \end{smallmatrix} \right) \vert \, x, y 
\in \mathbb R \}$. Dann ist $V$ ein Vektorraum (als untervektorraum von $\mathbb R^3$) und 
  	$$ \vektor{v_1} = \left( \begin{matrix} 1 \\ 0 \\ 0 \end{matrix} \right), \quad
   	\vektor{v_2} = \left( \begin{matrix} 0 \\ 1 \\ 0 \end{matrix} \right) $$
ist ein Erzeugendensystem von $V$. Ebenso ist 
  	$$ \vektor{w_1} = \left( \begin{matrix} 1 \\ 1 \\ 0 \end{matrix} \right), \quad
	\vektor{w_2} = \left( \begin{matrix} 1 \\ -1 \\ 0 \end{matrix} \right), \quad
  	\vektor{w_3} = \left( \begin{matrix} 3 \\ 2 \\ 0 \end{matrix} \right) $$
ein Erzeugendensystem von $V$. 
\end{beispiel}

\begin{beispiel}\label{vectorraum_poly_erz} Ist $U = \textrm{Pol}([0,1], \mathbb R)$ der Vektorraum der 
polynomialen Funktionen aus Beispiel ~\ref{vectorraum_poly} und isti $\left( f_{n} \right)_{n \in \mathbb N}$
die Familie polynomialer Funktionen aus Beispiel~\ref{vectorraum_poly_infin}, so ist 
$\left( f_{n} \right)_{n \in \mathbb N}$ ein Erzeugendensystem von $U$, wie wir in Beispiel 
~\ref{vectorraum_poly_infin} nachgerechnet haben.
\end{beispiel}

\begin{beispiel} Ist $V = \textrm{Abb}([0,1], \mathbb R)$ der Vektorraum aus Beispiel~\ref{vectorraum_abbild} 
und ist $\left( f_{n} \right)_{n \in \mathbb N}$
die Familie (polynomialer) Abbildungen aus Beispiel~\ref{vectorraum_poly_infin}, so ist 
$\left( f_{n} \right)_{n \in \mathbb N}$ kein Erzeugendensystem von $V$, da die Exponentialfunktion in $V$ liegt, 
aber keine Linearkombination der $\left( f_{n} \right)_{n \in \mathbb N}$ ist.
\end{beispiel}

\begin{definition}  Eine Familie  $\left\{ \vektor{v_{\lambda}} \right\}_{\lambda \in \Lambda}$ von 
Vektoren in $V$ heißt \index{Vektoren!linear unabhängig}\index{lineare Unabhängigkeit}\textbf{linear 
unabhänigig}, wenn gilt:

Ist $\Lambda_0 = \{\lambda_1, \ldots , \lambda_n \} \subseteq \Lambda$ eine beliebige endliche Teilmenge von 
$\Lambda$ und sind $r_1, \ldots, r_n$ reelle Zahlen mit 
  	$$ r_1 \cdot \vektor{v_{\lambda_1}} + r_2 \cdot \vektor{v_{\lambda_2}} + \cdots 
  	+ r_n \cdot \vektor{v_{\lambda_n}} = \vektor{0} $$
so muss schon gelten:
  	$$ r_1 = r_2 = \ldots = r_n = 0 $$
Andernfalls heißt die Familie $\left\{ \vektor{v_{\lambda}} \right\}_{\lambda \in \Lambda}$ linear 
abhängig.
\end{definition}

\begin{notiz} Ist $\Lambda$ insgesamt schon eine endlich Menge, so reicht es,d ie Bedingung aus der Definition 
nur für $\Lambda_0 = \Lambda$ zu prüfen. Wir erhalten dann wieder die Definition ~\ref{uvr_lin_unab}, die 
wir im letzten Abschnitt für Vektoren im $\mathbb R^n$ gemacht haben.
\end{notiz}

\begin{notiz}\label{lin_un_teil} Ist $\left\{ \vektor{v_{\lambda}} \right\}_{\lambda \in \Lambda}$ eine 
Familie linear unabhängiger Vektoren und ist $\Gamma \subseteq \Lambda$ eine Teilmenge von $\Lambda$, so ist 
auch $\left\{ \vektor{v_{\lambda}} \right\}_{\lambda \in \Gamma}$ linear unabhängig. 
\end{notiz}

\begin{beispiel} Die Vektoren $\vektor{v_1} = \left( \begin{smallmatrix} 1 \\ 0 \\ 0 \\ 0 
\end{smallmatrix} \right)$ und $\vektor{v_2} = \left( \begin{smallmatrix} 0 \\ 1 \\ 0 \\ 0 
\end{smallmatrix} \right)$ sind linear unabhängig, wie man sofort sieht.
\end{beispiel}

\begin{beispiel} Die Vektoren $\vektor{v_1} = \left( \begin{smallmatrix} 1 \\ 1 \\ 1 \\ 1 
\end{smallmatrix} \right)$ und $\vektor{v_2} = \left( \begin{smallmatrix} 1 \\ -1 \\ -1 \\ 1 
\end{smallmatrix} \right)$ und $\vektor{v_3} = \left( \begin{smallmatrix} 4 \\ 3 \\ 3 \\ 4 
\end{smallmatrix} \right)$ sind linear abhängig, denn 
  	$$ 7 \cdot \vektor{v_1} + \vektor{v_1} + (-2) \cdot \vektor{v_1} = \vektor{0} $$
\end{beispiel}

\begin{beispiel} Alle Beispiele linear unabhängiger Vektoren aus dem letzten Abschnitt sind auch linear 
unabhängig in der neuen Definition; alle Beispiele linear abhängiger Vektoren aus dem letzten Abschnitt sind 
auch linear abhängig im Sinne der neuen Definition.
\end{beispiel}

\begin{beispiel}\label{vectorraum_poly_lin_un} ist $U = \textrm{Pol}([0,1], \mathbb R)$ der Vektorraum der 
polynomialen Funktionen aus Beispiel~\ref{vectorraum_poly} und ist $\left( f_{n} \right)_{n \in \mathbb N}$
die Familie polynomialer Funktionen aus Beispiel~\ref{vectorraum_poly_infin}, so ist 
$\left( f_{n} \right)_{n \in \mathbb N}$ eine Familie linear unabhängiger Vektoren. Dies nachzurechnen 
erfordert ein wenig Analysis: 
Sei $\{ i_1, \ldots , i_n \} \subseteq \mathbb N$ eine endliche Teilmenge, wobei wir annehmen können, dass $i_1 
< i_2 < \cdots < i_n$, und seien $r_1, \ldots , r_n \in \mathbb R$ mit 
  	$$ 0 = r_1 \cdot f_{i_1}(x) + \cdots r_n \cdot f_{i_n}(x) $$
Wir differenzieren nun beide Seiten wiederholt und beachten dabei, dass $f_{i_l}^{(i_n)}(x) = 0$ für $l < n$, 
da $i_l < i_n$, und dass $f_{i_n}^{(i_n)}(x) = i_n! \, (= i_n(i_n-1) \cdots 1)$. Damit erhalten wir also durch 
$i_n$--faches Differenzieren
  	$$ 0 = r_n i_n! $$
woraus folgt, dass $r_n = 0$ gelten muss. Damit haben wir schon eine Linearkombination
  	$$ 0 = r_1 \cdot f_{i_1}(x) + \cdots r_{n-1} \cdot f_{i_{n-1}}(x) $$ 
mit der wir genauso vorgehen und durch $i_{n-1}$--faches Ableiten zeigen, dass auch $r_{n-1} = 0$. Nach $n$ 
Schritten haben wir gezeigt, dass $r_1 = r_2 = \ldots = r_n = 0$.
\end{beispiel}

\begin{beispiel} Isti wieder $U = \textrm{Pol}([0,1], \mathbb R)$ der Vektorraum der 
polynomialen Funktionen aus Beispiel~\ref{vectorraum_poly} und ist $\left( g_{n} \right)_{n \in \mathbb N}$
die Familie polynomialer Funktionen mit 
  	$$ \begin{array} {l c l l }
  	g_0(x) & = & 2 x^2 + 3 x^4 &  \\
  	g_n(x) & = & x^{2n} & \, \textrm{ für } n \geq 1 
  	\end{array} $$
so sind die $\left( g_{n} \right)_{n \in \mathbb N}$ linear abhängig, denn $g_0(x) - 2g_1(x) - 3 g_2(x) = 0$.

Definieren wir stattdessen 
  	$$ g_0(x) = 1 + 2 x^2 + 3 x^4 $$
so ist die Familie $\left( g_{n} \right)_{n \in \mathbb N}$ linear unabhängig. Das nachzurechnen überlassen 
bleibt Ihnen als Übung überlassen.
\end{beispiel}

\begin{definition} Eine Familie  $\left\{ \vektor{v_{\lambda}} \right\}_{\lambda \in \Lambda}$ von 
Vektoren in $V$ heißt \index{Vektorraum!Basis}\index{Basis} \textbf{Basis} von $V$, wenn die Vektoren
$\left\{ \vektor{v_{\lambda}} \right\}_{\lambda \in \Lambda}$ sowohl linear unabhängig als auch ein 
Erzeugendensystem von $V$ ist.
\end{definition}

\begin{beispiel}\label{vectorraum_standard_basis} Die Vektoren  $\vektor{e_1} = 
\left( \begin{smallmatrix} 1 \\ 0 \\ \vdots \\ 0
\end{smallmatrix} \right)$, $\vektor{e_2} = \left( \begin{smallmatrix} 0 \\ 1 \\ \vdots \\ 0 
\end{smallmatrix} \right), \ldots , \vektor{e_n} = \left( \begin{smallmatrix} 0 \\ \vdots \\ 0 \\ 1 
\end{smallmatrix} \right)$ bilden eine Basis von $\mathbb R^n$ wie wir schon im Beispiel ~\ref{uvr_standard_basis} 
aus dem letzten Abschnitt gesehen haben. 
\end{beispiel}

\begin{beispiel} Die Vektoren  $\vektor{v_1} = 
\left( \begin{smallmatrix} 1 \\ 2 \\ 3 \\ 4
\end{smallmatrix} \right)$, $\vektor{v_2} = \left( \begin{smallmatrix} 0 \\ 1 \\ 2 \\ 3 
\end{smallmatrix} \right)$, $\vektor{v_3} = \left( \begin{smallmatrix} 0 \\ 0 \\ 1 \\ 2 
\end{smallmatrix} \right)$ und  $\vektor{v_4} = \left( \begin{smallmatrix} 0 \\ 0 \\ 0 \\ 1 
\end{smallmatrix} \right)$ bilden eine Basis von $\mathbb R^4$, wie man leicht nachrechnet.
\end{beispiel}

\begin{beispiel} Die Vektoren  $\vektor{w_1} = 
\left( \begin{smallmatrix} 1 \\ 2 \\ 3 \\ 4
\end{smallmatrix} \right)$, $\vektor{w_2} = \left( \begin{smallmatrix} 1 \\ 0 \\ 1 \\ 0 
\end{smallmatrix} \right)$, $\vektor{w_3} = \left( \begin{smallmatrix} 0 \\ 1 \\ 0 \\ 1 
\end{smallmatrix} \right)$ $\vektor{w_4} = \left( \begin{smallmatrix} 2 \\ 0 \\ 1 \\ 0 
\end{smallmatrix} \right)$ und  $\vektor{w_5} = \left( \begin{smallmatrix} 0 \\ 1 \\ 0 \\ 2 
\end{smallmatrix} \right)$ bilden keine eine Basis von $\mathbb R^4$. Diese Vektoren erzeugen zwar 
den $\mathbb R^4$, aber sie sind nicht linear unabhängig, da etwa
  	$$ \vektor{w_1} - 5 \cdot \vektor{w_2} + 2  \cdot \vektor{w_3} - 2  \cdot\vektor{w_4} = \vektor{0} $$
\end{beispiel}

\begin{beispiel} Die Vektoren  $\vektor{u_1} = 
\left( \begin{smallmatrix} 1 \\ 2 \\ 3 \\ 4
\end{smallmatrix} \right)$, $\vektor{u_2} = \left( \begin{smallmatrix} 1 \\ 0 \\ 1 \\ 0 
\end{smallmatrix} \right)$ und $\vektor{u_3} = \left( \begin{smallmatrix} 0 \\ 1 \\ 0 \\ 1 
\end{smallmatrix} \right)$ sind keine Basis von $\mathbb R^4$. Sie sind zwar linear unabhängig, erzuegen aber 
$\mathbb R^4$ nicht. So ist etwa $\vektor{v} = 
\left( \begin{smallmatrix} 1 \\ 2 \\ 3 \\ 5 \end{smallmatrix} \right)$ keine Linearkombination von 
$\vektor{u_1}, \vektor{u_2}, \vektor{u_3}$. 
\end{beispiel}

\begin{beispiel}\label{vectorraum_poly_basis} Sei $U = \textrm{Pol}([0,1], \mathbb R)$ der Vektorraum der 
polynomialen Funktionen aus Beispiel ~\ref{vectorraum_poly} und sei $\left( f_{n} \right)_{n \in \mathbb N}$
die Familie polynomialer Funktionen aus Beispiel ~\ref{vectorraum_poly_infin}. Dann ist 
$\left( f_{n} \right)_{n \in \mathbb N}$ eine Basis von $U$, wie aus den Beispielen ~\ref{vectorraum_poly_erz} 
und ~\ref{vectorraum_poly_lin_un} folgt.
\end{beispiel}

\bigbreak

Ein wichtiges allgemeines Ergebnis über Basen liefert der folgende Satz (das im Fall unendlicher Indexfamilien 
auf dem sogenannten Auswahlaxiom beruht):

\begin{satz}\label{vectorraum_basis_allgemein} 
Für eine Familie $\left\{ \vektor{v_{\lambda}} \right\}_{\lambda \in \Lambda}$ von 
Vektoren in $V$ sind äquivalent:
\begin{enumerate}
\item $\left\{ \vektor{v_{\lambda}} \right\}_{\lambda \in \Lambda}$ ist eine Basis von $V$.
\item $\left\{ \vektor{v_{\lambda}} \right\}_{\lambda \in \Lambda}$ ist ein unverkürzbares 
Erzeugendensystem von $V$, d.h. $\left\{ \vektor{v_{\lambda}} \right\}_{\lambda \in \Lambda}$ erzeugt $V$ 
und für jede echte Teilmenge $\Lambda_0 \subset \Lambda$ gilt:
$\left\{ \vektor{v_{\lambda}} \right\}_{\lambda \in \Lambda_0}$ erzeugt $V$ nicht mehr.
\item $\left\{ \vektor{v_{\lambda}} \right\}_{\lambda \in \Lambda}$ ist eine maximale linear unabhängige 
Familie von Vektoren on $V$, d.h. nimmt man noch einen Vektor $\vektor{w}$ hinzu, so ist die 
resultierende Familie niemals mehr linear unabhängig.
\item $\left\{ \vektor{v_{\lambda}} \right\}_{\lambda \in \Lambda}$ ist ein Erzeugendensystem von $V$ mit 
dem sich jedes Element von $V$ auf genau eine Weise darstellen lässt.
\end{enumerate}
\end{satz}

\beweis{ Wir zeigen $1. \Longrightarrow 2. \Longrightarrow 4. \Longrightarrow 3. \Longrightarrow 1.$ Damit sind 
alle Äquivalenzen gezeigt.

$1. \Longrightarrow 2.$: Angenommen wir könnten ein $\vektor{v_{\lambda_0}}$ weglassen und erhielten 
immer noch ein Erzeugendensystem $\left\{ \vektor{v_{\lambda}} \right\}_{\lambda \in \Lambda \setminus 
\{ \lambda_0 \}}$ von $V$. Dann könnten wir insbesondere $\vektor{v_{\lambda_0}}$ als Linearkombination 
dieses Erzeugendensystems schreiben: 
  	$$ \vektor{v_{\lambda_0}} = r_1 \vektor{v_{\lambda_1}} + r_2 \vektor{v_{\lambda_2}} 
    	+ \cdots + r_n \vektor{v_{\lambda_n}} $$
mit $\lambda_1, \ldots , \lambda_n \in \Lambda \setminus \{ \lambda_0 \}$. Dann ist aber 
  	$$ \vektor{v_{\lambda_0}} - r_1 \vektor{v_{\lambda_1}} - r_2 \vektor{v_{\lambda_2}} 
    	- \cdots - r_n \vektor{v_{\lambda_n}} = \vektor{0} $$
im Widerspruch zur linearen Unabhäniggkeit von $\left\{ \vektor{v_{\lambda}} \right\}_{\lambda \in 
\Lambda}$.

$2. \Longrightarrow 4.$:  Angenommen, ein $\vektor{v}$ hat zwei Darstellungen, 
  	$$ \begin{array} {l c l c l }
  	\vektor{v} & = & r_1 \vektor{v_{\lambda_1}} & + \ldots + & r_n \vektor{v_{\lambda_n}} \\
  	\vektor{v} & = & s_1 \vektor{v_{\mu_1}} & + \ldots + & s_n \vektor{v_{\mu_n}}
  	\end{array} $$ 
Dann erhalten wir durch Differenzbilden
  	$$ \vektor{0} = \left( r_1 \vektor{v_{\lambda_1}}  + \ldots +  r_n 
     	\vektor{v_{\lambda_n}} \right) - 
     	\left(s_1 \vektor{v_{\mu_1}}  + \ldots +  s_n \vektor{v_{\mu_n}}\right) $$
also (nach eventueller Umbenennung und Umsortierung) eine Linearkombination
  	$$  \vektor{0} =  t_1 \vektor{v_{\nu_1}}  + \ldots +  t_l
     	\vektor{v_{\nu_l}} $$
in der nach Voraussetzung mindestens einer der Koeffizienten $t_i$ von Null verschieden ist. Wir können 
annehmen, dass das $t_1$ ist. Dann gilt aber
  	$$ \vektor{v_{\nu_1}} = - \frac {1}{t_1} \left( t_2  \vektor{v_{\nu_2}} + \ldots + 
	t_l  \vektor{v_{\nu_l}} \right) $$
und damit könnte $ \vektor{v_{\nu_1}}$ aus dem Erzeugendensystem entfernt werden und die verbleibenden 
vektoren wären immer noch erzeugend. Ein Widerspruch zur Annahme.

$4. \Longrightarrow 3.$: Wenn sich jedes Element in $V$ eindeutig mit $\left\{ \vektor{v_{\lambda}} 
\right\}_{\lambda \in \Lambda}$ darstellen lässt, dann gilt das insbesondere für den Nullvektor. 
  	$$  \vektor{0} =  r_1 \vektor{v_{\lambda_1}}  + \ldots +  r_n \vektor{v_{\lambda_n}} $$
Da sich der Nullvektor aber in trivialer Weise 
  	$$  \vektor{0} =  0 \cdot  \vektor{v_{\lambda_1}}  + \ldots +  0 \cdot \vektor{v_{\lambda_n}} $$
darstellen lässt, folgt aus der Eindeutigkeit schon, dass $r_1 = \ldots = r_n = 0$. Damit ist die Familie 
$\left\{ \vektor{v_{\lambda}} \right\}_{\lambda \in \Lambda}$ linear unabhängig. Da sie aber schon 
erzeugt, kann man kein weiteres Element aus $V$ mehr dazu nehmen, ohne die lineare Unabhängigkeit zu 
zerstören. Ist $\vektor{w}$ ein beliebiger weiterer Vektor, so schreibt sich $\vektor{w}$ als
  	$$ \vektor{w} = r_1 \vektor{v_{\lambda_1}} + \ldots + r_n \vektor{v_{\lambda_n}} $$
was zu einer nicht-trivialen Linearkombination
  	$$ \vektor{w} = \vektor{w} - r_1 \vektor{v_{\lambda_1}} - \ldots - r_n \vektor{v_{\lambda_n}} $$
so dass die Familie $\left\{ \vektor{v_{\lambda}} \right\}_{\lambda \in \Lambda} \cup 
\{\vektor{w} \}$ nicht mehr linear unabhängig ist.

$3. \Longrightarrow 1.$: Es bleibt noch zu zeigen, dass $\left\{ \vektor{v_{\lambda}} \right\}_{\lambda 
\in \Lambda}$ den Vektorraum $V$ erzeugt. Nehmen wir dazu an, dass es ein $\vektor{v}$ gibt, dass keine  
Linearkombination von $\left\{ \vektor{v_{\lambda}} \right\}_{\lambda \in \Lambda}$ ist, so wäre auch 
die Familie $\left\{ \vektor{v_{\lambda}} \right\}_{\lambda \in \Lambda} \cup \{\vektor{v} \}$ 
linear unabhängig:

Klar ist, das es keine Beziehung der Form
  	$$ \vektor{0} =  r_1 \cdot  \vektor{v_{\lambda_1}}  + \ldots +  r_n \cdot
	\vektor{v_{\lambda_n}} $$
mit $\lambda_1, \ldots , \lambda_n \in \Lambda$ geben kann, da ja $\left\{ \vektor{v_{\lambda}} 
\right\}_{\lambda \in \Lambda}$ eine Familie linear unabhängiger Vektoren ist. Es kann also höchstens 
eine Beziehung der Form
  	$$ \vektor{0} =  r_1 \cdot  \vektor{v_{\lambda_1}}  + \ldots +  r_n \cdot
	\vektor{v_{\lambda_n}} + s \cdot \vektor{v} $$
mit $s \neq 0$ geben. Aber dann lässt sich diese Beziehung auflösen zu 
  	$$ \vektor{v} = - \frac {r_1}{s} \cdot  \vektor{v_{\lambda_1}}  - \ldots -  
	\frac {r_n}{s} \cdot \vektor{v_{\lambda_n}} $$
und $\vektor{v}$ wäre eine Linearkombination der $\left\{ 
\vektor{v_{\lambda}} \right\}_{\lambda \in \Lambda}$, im Widerspruch zur Wahl von $\vektor{v}$.
 
Das wiederum ist ein Widerspruch dazu, dass $\left\{ \vektor{v_{\lambda}} \right\}_{\lambda \in 
\Lambda}$ ein nicht--verlängerbares System linear unabhängiger Vektoren ist.
}

\begin{korollar} Ist $\{ \vektor{v_1}, \vektor{v_2}, \ldots , \vektor{v_m} \}$ ein 
Erzeugendensystem eines Vektorraums $V$, so enthält es eine Basis, d.h. es gibt $i_1, i_2, \ldots, i_n \in 
\{1, \ldots , m \}$, so dass $\{ \vektor{v_{i_1}}, \vektor{v_{i_2}}, \ldots , 
\vektor{v_{i_n}} \}$ eine Basis von $V$ ist.
\end{korollar}

\beweis{ Das folgt sofort aus Satz ~\ref{vectorraum_basis_allgemein}, indem man aus dem Erzeugendensystem so lange 
Vektoren weglässt, bis es nicht mehr verkürzt werden kann.
}

\begin{korollar}\label{lin_un_basiserg} Ist $\left\{ \vektor{v_{\lambda}} \right\}_{\lambda \in \Lambda}$ 
eine Familie linear unabhängiger Vektoren in $V$, so kann $\left\{ \vektor{v_{\lambda}} 
\right\}_{\lambda \in \Lambda}$ zu einer Basis von $V$ ergänzt werden.
\end{korollar}

\beweis{ Das folgt sofort aus Satz ~\ref{vectorraum_basis_allgemein}, indem man das System linear unabhängiger 
Vektoren so lange verlängert, bis es nicht maximal ist.
}

Hieraus erhält man nun unmittelbar

\begin{korollar} Jeder Vektorraum hat eine Basis.
\end{korollar}

Zur Erinnerung: Für eine Menge $M$ bezeichnet $\vert \, M \ \vert$ die Mächtigkeit dieser Menge. Das kann eine 
endliche Zahl sein oder auch $\infty$.


\begin{definition} Ist $\left\{ \vektor{v_{\lambda}} \right\}_{\lambda \in \Lambda}$ eine Basis von $V$, 
so heißt $\vert \, \Lambda \, \vert$ die Länge der Basis.
\end{definition}

\begin{beispiel} Die Standardbasis $\{ \vektor{e_1}, \vektor{e_2}, \ldots , \vektor{e_n}\}$ 
des $\mathbb R^n$ hat die Länge $n$.
\end{beispiel}

\begin{beispiel} Die Basis $\{ f_n \}_{n \in \mathbb N}$ des Vektorraums $U = \textrm{Pol}([0,1], \mathbb R)$ der 
polynomialen Funktionen aus Beispiel ~\ref{vectorraum_poly_basis} hat die Länge $\infty$.
\end{beispiel}

\begin{satz}\label{vektorraum_basis_lange} Je zwei Basen $\left\{ \vektor{v_{\lambda}} \right\}_{\lambda 
\in \Lambda}$ und 
$\left\{ \vektor{w_{\gamma}} \right\}_{\gamma \in \Gamma}$ sind gleich lang, d.h. sie sind entweder 
beide unendlich oder beide endlich mit der selben Anzahl von Elementen.
\end{satz}

\begin{definition} Die Länge einer Basis von $V$ heißt die \index{Vektorraum!Dimension} 
\textbf{Dimension} des Vektorraums $V$ und wird mit $\textrm{dim}(V)$ bezeichnet.
\end{definition} 

\begin{notiz} $\textrm{dim}(V)$ ist eine endliche Zahl oder $\infty$.
\end{notiz}

Der entscheidende Punkt im Beweis von Satz ~\ref{vektorraum_basis_lange} ist der \index{Satz!Austauschsatz von 
Steinitz}\textit{Basisaustauschsatz von Steinitz}, den wir nur im endlichdimensionalen Fall formulieren:

\begin{satz}[Basisaustauschsatz von Steinitz] \label{steinitz} Ist $\{ \vektor{v_1}, \ldots , \vektor{v_n} \}$ eine 
Basis von $V$ und sind $\{ \vektor{w_1}, \ldots , \vektor{w_m} \}$ linear unabhängige Vektoren, so gibt es 
$i_1, \ldots, i_m \in \{1, \ldots, n \}$ derart, dass die $\vektor{v_{i_l}}$ durch die 
$\vektor{w_l}$ ersetzt werden können und wieder eine Basis entsteht. In anderen Worten: Die Vektoren 
$\vektor{u_1}, \ldots, \vektor{u_n}$ mit 
  	$$ \vektor{u_j} = \left\{ \begin{array} {l l} 
	\vektor{w_t} & \textrm{ falls } j = i_t \textrm{ für ein } t \\ 
	\vektor{v_j} & \textrm{ sonst}
	\end{array} \right. $$
bilden wieder eine Basis von $V$.

Speziell gilt also $m \leq n$.
\end{satz}

\beweisvon{~\ref{steinitz}} Zunächst zeigen wir: Ist $\vektor{w} \in V$ ein beliebiger, vom Nullvektor 
verschiedener Vektor, und gilt
  	\begin{equation}\label{eq: for w}
  	\vektor{w} = r_1 \vektor{v_1} + \cdots + r_n \vektor{v_n}  
  	\end{equation} 
uznd gilt $r_i \neq 0$ für ein $i$, so ist $\{ \vektor{v_1}, \ldots , \vektor{v_{i-1}}, 
\vektor{w}, \vektor{v_{i+1}}, \ldots, \vektor{v_n} \}$ wieder eine Basis von V.

Dazu nehmen wir an, dass $i = 1$. Das können wir erreichen, indem wir die $\vektor{v_i}$ anders 
durchnummerieren. Dann gilt also 
$ \vektor{w} = r_1 \vektor{v_1} + \cdots + r_n \vektor{v_n} $
mit $r_1 \neq 0$ und daher 
   	$$ \vektor{v_1} = \frac {1}{r_1} \cdot \left( \vektor{w} - r_2 \vektor{v_1} - \cdots - 
  	r_n \vektor{v_n} \right) $$   
und es folgt leicht, dass $\{ \vektor{w}, \vektor{v_2}, \ldots, \vektor{v_n} \}$ den 
Vektorraum $V$ erzeugen. Es bleibt zu zeigen, dass sie auch linear unabhängig sind. Dazu sei eine Gleichung
  	$$ \vektor{0} = s_1\vektor{w} + s_2 \vektor{v_2} + \cdots + s_n \vektor{v_n} $$
gegeben. Für $\vektor{w}$ setzen wir nun (~\ref{eq: for w}) ein und erhalten
  	$$ \vektor{0} = s_1 r_1 \vektor{v_1} + 
   	\left(s_2 + s_1 r_2 \right) \vektor{v_2} + \cdots + 
   	\left(s_n + s_1 r_n \right) \vektor{v_n} $$
Aus der linearen Unabhängigkeit von $\vektor{v_1}, \ldots , \vektor{v_n}$ folgt
  	$$ r_1 s_1 = 0, \quad s_i + s_1 r_i = 0 \, \textrm{ für } \, i \geq 2 $$
Da $r_1 \neq 0$ erhalten wir hieraus zunächst $s_1 = 0$, und damit durch Einsetzen in die anderen Gleichungen 
auch sofort $s_i = 0$ für $i = 2, \ldots, n$. 

Wir benutzen vollständige Induktion über die Anzahl $m$ der linear 
unabhängigen Vektoren.

Für $m = 0$ ist nichts zu zeigen.

Sei also $m \geq 1$ und die Behauptung für $m-1$ linear unabhängige Vektoren schon bewiesen. Durch geschicktes 
Umnummerieren und Anwendung der Induktionsvoraussetzung können wir erreichen, dass $\{ \vektor{w_1}, 
\ldots , \vektor{w_{m-1}}, \vektor{v_m}, \ldots, \vektor{v_n} \}$ eine Basis von $V$ ist. 
Mit dieser Basis können wir auch $\vektor{w_m}$ darstellen und schreiben
  	$$\vektor{w_m} = r_1 \vektor{w_1} + \cdots + r_{m-1} \vektor{w_{m-1}} + 
  	r_m \vektor{v_m} + \cdots + r_n \vektor{v_n} $$
Dabei muss mindestens einer der Koeffizienten $r_m, \ldots, r_n$ von Null verschieden sein, den andernfalls 
bekämen wir eine nichttriviale Relation zwischen $\vektor{w_1}, \ldots,  \vektor{w_m}$, 
  	$$ \vektor{w_m} - r_1 \vektor{w_1} - \cdots - r_{m-1} \vektor{w_{m-1}} = 
   	\vektor{0} $$
im Widerspruch zu deren linearer Unabhängigkeit. Nach dem ersten Teil des Beweises können wir das 
entsprechende $\vektor{v_i}$ durch $\vektor{w_m}$ ersetzen und erhalten wieder eine Basis. 
Damit ist der Basisaustauschsatz von Steinitz beweisen.

\beweisvon{~\ref{vektorraum_basis_lange}} Sei zunächst 
$\{ \vektor{v_1}, \ldots , \vektor{v_n} \}$ eine endliche Basis von $V$ und sei 
$\{ \vektor{w_{\lambda}} \}_{\lambda \in \Lambda}$ eine beliebige weitere Basis von  
$V$. Dann muss nach dem Satz von Steinitz schon gelten, dass $\vert \, \lambda \, \vert \leq n$ ist. (Falls 
$\Lambda$ endlich ist, folgt das sofort aus ~\ref{steinitz}. Wäre $\Lambda$ unendlich, so könnten wir darin 
eine $n+1$--elementige Teilmenge $\Lambda_0$ auswählen, und nach ~\ref{lin_un_teil} wäre dann auch 
$\{ \vektor{w_{\lambda}} \}_{\lambda \in \Lambda_0}$ linear unabhängig, im Widerspruch zu 
~\ref{steinitz}). Damit können wir also $\{ \vektor{w_{\lambda}} \}_{\lambda \in \Lambda}$ schreiben 
als $\{ \vektor{w_1}, \ldots , \vektor{w_m} \}$ mit $m \leq n$. Wenn wir nun die Rollen von
$\{ \vektor{w_1}, \ldots , \vektor{w_m} \}$ und 
$\{ \vektor{v_1}, \ldots , \vektor{v_n} \}$ vertauschen und wieder den Satz von Steinitz 
anwenden, so erhalten wir auch $n \leq m$, also insgesamt $n = m$.

Damit haben wir also gezeigt: Hat $V$ eine endliche Basis, so sind alle anderen Basen von $V$ ebenfalls endlich 
und haben die gleiche Länge wie die vorgegebene Basis. Hieraus folgt aber auch: Hat $V$ eine Basis unendlicher 
Länge, so sind auch alle anderen Basen unendlich lang.

\bigbreak

\begin{korollar} Ist $U \subseteq V$ eine Untervektorraum, so gilt:
  	$$ \textrm{dim}(U) \leq \textrm{dim}(V) $$
\end{korollar}

\beweis Eine Basis von $U$ ist eine Familie linear unabhängiger Vektoren in $V$, lässt sich also nach 
Korollar~\ref{lin_un_basiserg} zu einer Basis von $V$ ergänzen. Daraus folgt die Behauptung.

\bigbreak

\begin{aufgabe}\label{aufgabe_vr_1} Sei $V = \textrm{Abb}(I, \mathbb R)$ die Menge der Abbbildungen
  	$$ f: I \longrightarrow \mathbb R $$
wobei $I$ ein beliebiges Intervall ist (geschlossen, offen oder halboffen). Zeigen Sie, dass $V$ ein 
Vektorraum wird, wenn man die Vektorraumoperationen wie im Beispiel ~\ref{vectorraum_abbild} für 
$\textrm{Abb}([0,1], \mathbb R)$ definiert.
\end{aufgabe}

\begin{aufgabe}\label{aufgabe_vr_2}  Sei $V = \textrm{Abb}(\mathbb R, \mathbb R)$ der Vektorraum der Abbildungen
  	$$ f : \mathbb R \longrightarrow \mathbb R $$
(vergleiche Aufgabe ~\ref{aufgabe_vr_1}). Zeigen Sie, dass die Elemente $f, g \in V$ mit 
  	$$ f(x) = \cos(x), \quad g(x) = \sin(x) \, \textrm{ für alle } x \in \mathbb R $$
linear unabhängig sind.
\end{aufgabe} 

\begin{aufgabe}\label{aufgabe_vr_3}  Sei $V = \textrm{Abb}((0, \infty), \mathbb R)$ der Vektorraum der Abbildungen
  	$$ f : (0, \infty) \longrightarrow \mathbb R $$
(vergleiche Aufgabe ~\ref{aufgabe_vr_1}). Zeigen Sie, dass die Elemente $f, g \in V$ mit 
  	$$ f(x) = \exp(x), \quad g(x) = \ln(x) \, \textrm{ für alle } x \in (0, \infty) $$
linear unabhängig sind.
\end{aufgabe} 

\begin{aufgabe}\label{aufgabe_vr_4} Sei $V = \textrm{Abb}((1, \infty), \mathbb R)$ der Vektorraum der Abbildungen
  	$$ f : (1, \infty) \longrightarrow \mathbb R $$
(vergleiche Aufgabe ~\ref{aufgabe_vr_1}). Zeigen Sie, dass die Elemente $f, g, h \in V$ mit 
  	$$ f(x) = \frac {1}{x+1}, \quad g(x) = \frac {1}{x-1}, \quad h(x) = \frac {x}{x^2-1} 
	\, \textrm{ für alle } x \in (1, \infty) $$
linear abhängig sind.
\end{aufgabe}

\begin{aufgabe} Benutzen Sie den Satz von Steinitz ~\ref{steinitz} um folgende Aussage zu beweisen:
Ist $E_1$ parallel zu $E_2$, so ist auch $E_2$ parallel zu $E_1$.
\end{aufgabe}

\bigbreak

\newpage


\section{Vektorräume über beliebigen Körpern}\label{sect_vr_beliebig}

\setcounter{definition}{0}
\setcounter{beispiel}{0}
\setcounter{notiz}{0}

Im vorangegangenen Abschnitt haben wir - weder in den Aussagen noch in den Beweisen - 
die spezielle Struktur von $\mathbb R$, also der Anordnung oder dem Dedekindschen 
Schnittgesetz, gemacht. Alle Aussagen und Beweise benutzen lediglich die Regeln  
für Addition, subtraktion, Multiplikation oder Division, also die allgemeinen 
Axiome, die für jeden Körper gelten. Genauso gut hätten wir anstelle von 
$\mathbb R$ also auch $\mathbb C$, $\mathbb Q$ oder gar einen der endlichen Körper  
$\mathbb F_p$, die wir in Abschnitt~\ref{section_gruppe} kennengelernt haben, nehmen 
können. Das wollen wir in diesem Abschnitt auch tun. Dazu fixieren wir einen 
beliebigen Körper $K$.



\begin{beispiel} Wir haben im Beispiel~\ref{vectorraum_poly} in Abschnitt\ref{section_vr_reell} 
schon allgemeine polynomiale Abbildungen 
  	$$ f : [0, 1] \longrightarrow \mathbb R $$
betrachtet. In der Praxis treten dabei meistens Polynome mit rationalen Koeffizienten 
auf, also Abbildungen $f(x)$, die sich als 
  	$$ f(x) = a_n x^n + a_{n-1} x^{n-1} + \cdots + a_1 x + a_0 $$
mit $a_i \in \mathbb Q$ schreiben lassen. Die Menge dieser Polynome mit rationalen 
Koeffizienten wollen wir mit $\textrm{Pol}_{\mathbb Q}([0,1], \mathbb R)$ bezeichnen. 
Dann ist $\textrm{Pol}_{\mathbb Q}([0,1], \mathbb R)$ mit der Addition wie wir sie in 
Beispiel~\ref{vectorraum_poly} definiert haben und der Skalarmultiplikation
  	$$ (q \cdot f)(x) = q \cdot f(x) \qquad \textrm{ für alle } x \in [0,1] $$
für $q \in \mathbb Q$ ein $\mathbb Q$--Vektorraum. 

Beachten Sie dabei, dass $\textrm{Pol}_{\mathbb Q}([0,1], \mathbb R)$ kein reeller 
Vektorraum ist, denn die Abbildung $f$ mit  $f(x) = x$ ist sicherlich ein Element von 
$\textrm{Pol}_{\mathbb Q}([0,1], \mathbb R)$, aber $\sqrt{2} \cdot f$ ist es nicht.
\end{beispiel}

\begin{beispiel} Die Menge $\mathbb R$ der reellen Zahlen, zusammen mit der üblichen 
Addition reeller Zahlen und der (üblichen) Multiplikation mit rationalen Zahlen ist 
ein $\mathbb Q$--Vektorraum.
\end{beispiel}


Nun betrachten wir in einem beleibigen $K$--Vektorraum noch ein Familie $\left\{ 
\vektor{v_{\lambda}} \right\}_{\lambda \in \Lambda}$ von 
Elementen von $V$.

\begin{definition}\leavevmode\newline
\vspace{-0.8cm} 

\begin{itemize}
\item Ein Vektor $\vektor{w}$ 
heißt \index{Linearkombination}\textbf{Linearkombination} von $\left\{ 
\vektor{v_{\lambda}} \right\}_{\lambda \in \Lambda}$, wenn es \textit{endlich 
viele} $\lambda_1, \ldots , \lambda_n \in \Lambda$  und $\kappa_1, \ldots, 
\kappa_n \in K$ gibt mit 
  	$$ \vektor{w} = \kappa_1 \cdot \vektor{v_{\lambda_1}} + \cdots 
	+ \kappa_n \cdot \vektor{v_{\lambda_n}} $$
\item Eine Familie  $\left\{ \vektor{v_{\lambda}} \right\}_{\lambda \in 
\Lambda}$ von Vektoren in $V$ heißt ein \index{Vektorraum!Erzeugendensystem}\textbf{Erzeugendensystem} 
von $V$, wenn sich jedes Element $\vektor{v} \in V$ als 
Linearkombination von $\left\{ \vektor{v_{\lambda}} \right\}_{\lambda 
\in \Lambda}$ schreiben lässt.
\item Eine Familie  $\left\{ \vektor{v_{\lambda}} \right\}_{\lambda \in \Lambda}$ von 
Vektoren in $V$ heißt \index{Vektoren!linear unabhängig}\index{lineare Unabhängigkeit}\textbf{linear 
unabhänigig}, wenn gilt:

Ist $\Lambda_0 = \{\lambda_1, \ldots , \lambda_n \} \subseteq \Lambda$ eine beliebige endliche Teilmenge von 
$\Lambda$ und sind $\kappa_1, \kappa_2 \ldots, \kappa_n \in K$ Skalare mit 
  	$$ \kappa_1 \cdot \vektor{v_{\lambda_1}} + \kappa_2 \cdot 
    	\vektor{v_{\lambda_2}} + \cdots 
 	+ \kappa_n \cdot \vektor{v_{\lambda_n}} = \vektor{0} $$
so muss schon gelten:
  	$$ \kappa_1 = \kappa_2 = \ldots = \kappa_n = 0 $$
Andernfalls heißt die Familie $\left\{ \vektor{v_{\lambda}} \right\}_{\lambda \in \Lambda}$ linear 
abhängig.
\item Eine Familie  $\left\{ \vektor{v_{\lambda}} \right\}_{\lambda \in 
\Lambda}$ von Vektoren in $V$ heißt \index{Vektorraum!Basis}\index{Basis} 
\textbf{Basis} von $V$, wenn die Vektoren $\left\{ \vektor{v_{\lambda}} 
\right\}_{\lambda \in \Lambda}$ sowohl linear unabhängig als auch ein 
Erzeugendensystem von $V$ ist.
\end{itemize}
\end{definition}

\begin{beispiel} Über jeden Körper $K$ hat das $n$--dimensionale kartesische Produkt 
$K^n$ die Basis
  	$$ \vektor{e_1} = 
 	\left( \begin{smallmatrix} 1 \\ 0 \\ \vdots \\ 0 \end{smallmatrix} \right), \quad 
  	\vektor{e_2} = \left( \begin{smallmatrix} 0 \\ 1 \\ \vdots \\ 0 
  	\end{smallmatrix} \right), \ldots , \vektor{e_n} = \left( \begin{smallmatrix} 
   	0 \\ \vdots \\ 0 \\ 1 \end{smallmatrix} \right) $$
\end{beispiel}

\begin{beispiel} Der $\mathbb Q$--Vektorraum $\textrm{Pol}_{\mathbb Q}([0,1], 
\mathbb R)$ hat die Basis $1, x, x^2, x^3, \ldots $.
\end{beispiel}

\begin{aufgabe} Der Begriff der linearen Unabhängigkeit hängt ab von den Skalaren, 
die wir betrachten. Zeigen Sie dazu:

Betrachten wir $\mathbb R$ als $\mathbb Q$--Vektorraum, so sind die Elemente $1, 
\sqrt{2} \in \mathbb R$ linear unabhängig. Betrachten wir $\mathbb R = \mathbb R^1$ 
als $\mathbb R$--Vektorraum, so sind sie lineare abhängig.
\end{aufgabe}

\bigbreak

Alle Aussagen, die wir über allgemeine reelle Vektorräume getroffen haben, gelten 
vollkommen analog auch für Vektorräume über beliebigen Körpern. Speziell gilt 
das für Satz~\ref{vectorraum_basis_allgemein}, der sich  - inklusive Beweis - 
wortwörtlich auf unsere Situation überträgt. Wir wollen hier seine für uns 
wichtigsten Folgerungen auflisten:

\begin{satz}\label{basis_allg_vr_allg_koerper} Für einen beliebigen $K$--Vektorraum 
$V$ gilt:

\begin{enumerate}
\item Ist $\{ \vektor{v_1}, \vektor{v_2}, \ldots , \vektor{v_m} \}$ ein 
Erzeugendensystem eines Vektorraums $V$, so enthält es eine Basis, d.h. es gibt $i_1, i_2, \ldots, i_n \in 
\{1, \ldots , m \}$, so dass $\{ \vektor{v_{i_1}}, \vektor{v_{i_2}}, \ldots , 
\vektor{v_{i_n}} \}$ eine Basis von $V$ ist.
\item Ist $\left\{ \vektor{v_{\lambda}} \right\}_{\lambda \in \Lambda}$ 
eine Familie linear unabhängiger Vektoren in $V$, so kann $\left\{ \vektor{v_{\lambda}} 
\right\}_{\lambda \in \Lambda}$ zu einer Basis von $V$ ergänzt werden.
\item  Jeder Vektorraum hat eine Basis.
\end{enumerate}
\end{satz}

Wie auch für reelle Vektorräume zeigt man im allgemeinen Fall

\begin{satz} Je zwei $K$--Basen $\left\{ \vektor{v_{\lambda}} \right\}_{\lambda \in \Lambda}$ und $\left\{ 
\vektor{w_{\gamma}} \right\}_{\gamma \in \Gamma}$ eines $K$--Vektorraums 
$V$ sind gleich lang, d.h. sie sind entweder 
beide unendlich oder beide endlich mit der selben Anzahl von Elementen.
\end{satz}

\begin{definition} Die Länge einer $K$--Basis von $V$ heißt die \index{Vektorraum!Dimension}\textbf{Dimension} 
des $K$--Vektorraums $V$ und wird mit $\textrm{dim}_K(V)$ bezeichnet.
\end{definition} 

\begin{notiz} $\textrm{dim}_K(V)$ ist eine endliche Zahl oder $\infty$.
\end{notiz}

\begin{beispiel} $\textrm{dim}_K(K^n) = n$.
\end{beispiel}

\bigbreak

Allgemeine reelle Vektorräume unterscheiden sich von den speziellen reellen 
Vektorräumen $\mathbb R^n$ in ihrer geometrischen Interpretation. Auf $\mathbb R^n$ 
konnten wir ein Skalarprodukt definieren und damit Begriffe wie \textit{Abstand}, 
\textit{Winkel} oder \textit{Orthogonalität} einführen. Wir können uns jetzt 
natürlich ebenso fragen, ob eine geometrische Interpretation auch für die 
Körper $\mathbb K^n$ möglich ist. 

Zunächst scheint zumindest die formale Definition eines Skalarproduktes auf $K^n$ 
möglich zu sein, indem wir einfach die Definition aus dem $\mathbb R^n$ übertragen 
und setzen
  	$$ \langle \left( \begin{matrix} a_1 \\ \vdots \\ a_n \end{matrix} \right), 
     	\left( \begin{matrix} b_1 \\ \vdots \\ b_n \end{matrix} \right) \rangle^{\prime}
    	= a_1 b_1 + \cdots + a_n b_n$$
wir sehen aber sofort einen fundamentalen Unterschied zur reellen Situation:

\begin{beispiel} Im $\mathbb F_2$--Vektorraum $\mathbb F_2^2$ gilt:
  	$$ \langle \left( \begin{matrix} 1 \\ 1 \end{matrix} \right), 
    	\left( \begin{matrix} 1 \\ 1 \end{matrix} \right) \rangle^{\prime} = 1 + 1 = 0 $$

Im $\mathbb C$--Vektorraum $\mathbb C^2$ gilt:
  	$$ \langle \left( \begin{matrix} 1 \\ \ii \end{matrix} \right), 
     \left( \begin{matrix} 1 \\ \ii \end{matrix} \right) \rangle^{\prime} = 1 + \ii^2 
     = 1 - 1 = 0 $$
\end{beispiel}

Die von uns betrachtete Paarung $\langle \phantom{a}, \phantom{b} \rangle^{\prime}$ hat 
also nicht mehr die Eigenschaft $\langle \vektor{v}, \vektor{v} 
\rangle^{\prime} \neq 0$ (sogar $\langle \vektor{v}, \vektor{v} 
\rangle^{\prime} > 0$) für $\vektor{v} \neq \vektor{0}$. Und sogar 
in $\mathbb Q^n$ (wo diese Nichtverschwindungsaussage noch gilt) hat diese Paarung 
einen entscheidenden Nachteil:

\begin{beispiel} Wir betrachten den $\mathbb Q$--Vektorraum $\mathbb Q^2$ und den Vektor 
$\vektor{v} = \left( \begin{smallmatrix} 1 \\ 1 \end{smallmatrix} \right) $
Dann gibt es keine rationale Zahl $q$ mit 
  	$$ \langle q \cdot \vektor{v}, q \cdot \vektor{v} \rangle^{\prime} 
     	= 1 $$
\end{beispiel}

Im $\mathbb R^n$ konnten wir diese Normierung von $\vektor{v}$ immer erreichen, 
indem wir $\vektor{v}$ mit dem (reellen) Skalar
$r = \frac {1}{\vert \vektor{v} \vert }$ 
multiplizierten. Diese Eigenschaft war sehr wichtig für uns, etwa beim 
Gram--Schmidt--Orthonormalisierungsver\-fah\-ren.

Die Definition des Skalarprodukts überträgt sich also nicht unmittelbar auf 
beliebige $K$--Vektorräume $K^n$. Aber lässt sich die Definition dann vielleicht 
so abändern, dass wir eine  dem Skalarprodukt ähnliche Paarung erhalten? Zumindest 
im Fall der komplexen Vektorräume $C^n$ ist das möglich:

\begin{definition} Für zwei Vektoren $\vektor{v} = \left( \begin{smallmatrix} 
v_1 \\ \vdots \\ v_n \end{smallmatrix} \right), \vektor{w} = \left( 
\begin{smallmatrix} w_1 \\ \vdots \\ w_n \end{smallmatrix} \right) \in \mathbb C^n$ 
setzen  wir
  	$$ \langle \vektor{v}, \vektor{w} \rangle = v_1 \cdot \overline{w_1} + 
  	\cdots + v_n \cdot \overline{w_n} $$
(wobei $\overline{z}$ die zu einer komplexen Zahle $z = x + \ii \cdot  y$ konjugierte Zahl 
$x - \ii \cdot y$ bezeichnet)
und nennen es das \index{Skalarprodukt}\textbf{Skalarprodukt} von $\vektor{v}$ 
und $\vektor{w}$.
\end{definition}
 
Eine ganz wesentliche Eigenschaft dieses Skalarprodukts ist
 
\begin{satz}\label{vect_complex_norm} Für $\vektor{v} \neq \vektor{0}$ ist 
$\langle \vektor{v}, \vektor{v} \rangle$ eine reelle Zahl mit
  	$$ \langle \vektor{v}, \vektor{v} \rangle > 0 $$
\end{satz}

\beweis{ 
Wir haben 
  	$$ \langle \vektor{v}, \vektor{v} \rangle = v_1 \cdot \overline{v_1} 
   	+ \cdots + v_n \cdot \overline{v_n} = \vert v_1 \vert^2 + \cdots + \vert v_n 
   	\vert^2 $$
wobei $\vert z \vert \, = \sqrt{x^2 + y^2}$ den Betrag einer komplexen Zahl 
$z = x + \ii \cdot y$ bezeichnet. Hieraus folgt die Aussage.
}
\bigbreak

\begin{definition} Für ein $\vektor{v} \in \mathbb C^n$ heißt 
  	$$ \vert \vektor{v} \vert \, := \sqrt{\langle \vektor{v}, 
	\vektor{v} \rangle} $$ 
die \textbf{Länge} oder die \textbf{Norm} von $\vektor{v}$.
\end{definition}

Für das komplexe Skalarprodukt gelten die folgenden Regeln:

\begin{regel}\label{regel_skalar_produkt_komplex} 
Für Vektoren $\vektor{v}$, $\vektor{w}$, 
$\vektor{w_1}$ und $\vektor{w_2}$ und einen Skalar $z \in \mathbb C$ gilt:
\begin{enumerate}
\item $\langle \vektor{v}, \vektor{w} \rangle = \overline{\langle \vektor{w}, 
\vektor{v} \rangle}$.
\item $\langle \vektor{v}, \vektor{w_1} + \vektor{w_2} \rangle = 
\langle \vektor{v}, \vektor{w_1} \rangle + \langle \vektor{v}, 
\vektor{w_2} \rangle \quad$ (\textit{Distributivgesetz}).
\item $\langle z \cdot \vektor{v}, \vektor{w} \rangle = z \cdot \langle \vektor{v}, 
\vektor{w} \rangle \quad$ (\textit{Skalarmultiplikation im ersten Faktor}).
\item $\langle \vektor{v}, z \cdot \vektor{w} \rangle = 
\overline{z} \cdot \langle \vektor{v}, 
\vektor{w} \rangle \quad$ (\textit{Skalarmultiplikation im zweiten Faktor}).
\end{enumerate}
\end{regel}

Auch die Dreieicksungleichung und der \index{Satz!von Cauchy--Schwarz}Satz 
von Cauchy--Schwarz überträgen sich auf diese Situation:

\begin{satz}\label{cauchy_schwarz_komplex}
Für Vektoren $\vektor{v}$, $\vektor{w}$ und einen Skalar $z \in \mathbb C$ gilt:

\begin{enumerate}
\item $\vert z \cdot \vektor{v} \vert \, = \, \vert z \vert \cdot \vert \vektor{v} \vert$.
\item $\vert \vektor{v} + \vektor{w} \vert \, \leq \, \vert \vektor{v} \vert \, + 
     \, \vert \vektor{w} \vert$.
\item $ \vert \langle \vektor{v},  \vektor{w} \rangle \vert \leq \vert \vektor{v} 
        \vert \cdot \vert \vektor{w} \vert $.
\item Genau dann sind $\vektor{v}$ und $\vektor{w}$ kollinear, wenn 
$\vert \langle \vektor{v}, \vektor{w} \rangle \vert \, 
= \vert \vektor{v} \vert \cdot  \vert \vektor{w} \vert$.
\end{enumerate}
\end{satz}

\begin{definition} Zwei Vektoren $\vektor{v}$ und $\vektor{w}$ in  $\mathbb C^n$ heißen 
\index{Vektoren!orthogonal} \textbf{orthogonal}, wenn gilt
  	$$ \langle \vektor{v},  \vektor{w} \rangle = 0 $$
Wir sagen in diesem Fall auch, dass $\vektor{v}$ senkrecht auf $\vektor{w}$ steht und schreiben 
$\vektor{v} \perp \vektor{w}$.
\end{definition}

\begin{definition} Eine Basis $\vektor{v_1}, \ldots, \vektor{v_m}$ eines Untervektorraums $U \subseteq \mathbb C^n$ heißt \index{Untervektorraum!Orthonormalbasis}\textbf{Orthonormalbasis} von 
$U$ oder kurz \textbf{ONB} von $U$, wenn gilt:

\begin{itemize}
\item $\vert \vektor{v_i} \vert = 1$ für $i = 1, \ldots, m$, d.h. alle 
Vektoren der Basis sind normiert.
\item $\langle \vektor{v_i}, \vektor{v_j} \rangle = 0$ für 
$i \neq j$, d.h. die Vektoren stehen paarweise senkrecht aufeinander.
\end{itemize}
\end{definition}


\bigbreak


\begin{aufgabe} Wieviele Basen hat der $\mathbb F_3$--Vektorraum 
$\mathbb F_3^2$?
\end{aufgabe} 

\begin{aufgabe} Zeigen Sie $\textrm{dim}_{\mathbb Q}(\mathbb R) = \infty$.
\end{aufgabe}

\begin{aufgabe} Beweisen Sie die Regeln~\ref{regel_skalar_produkt_komplex}.
\end{aufgabe}

\begin{aufgabe} Übertagen Sie das Gram--Schmidt--Orthonormalsierungs\-ver\-fah\-ren auf 
den komplexen Vektorraum $\mathbb C^n$ und benutzen Sie es, um aus den Vektoren 
  $$ \vektor{v_1} = \left( \begin{matrix} 1 \\ i \\ i \end{matrix} \right), 
  \quad \vektor{v_1} = \left( \begin{matrix} 0 \\ i \\ -i \end{matrix} \right),
  \quad \vektor{v_1} = \left( \begin{matrix} 1 \\ 1 \\ 1 \end{matrix} \right) $$
eine Orthonormalbasis des $\mathbb C^n$ zu machen.
\end{aufgabe}
