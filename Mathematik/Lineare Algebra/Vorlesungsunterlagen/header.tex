% Headerdatei

% Allgemeines
\usepackage[utf8]{inputenc}	% zur richtigen Darstellung der Schriften/Sonderzeichen
\usepackage[T1]{fontenc}	% T1-Font auswählen
\usepackage[ngerman]{babel} % deutsche Rechtschreibung (u.a. Trennregeln)
\usepackage[dvips, dvipdfm]{graphicx}
% \usepackage{graphicx}		% Einbinden von Graphiken
\usepackage[dvipsnames]{xcolor}
%\usepackage{color}			% Farben
\usepackage{eurosym}

% Mathematisches
\usepackage[namelimits,sumlimits,intlimits]{amsmath}	% Matheumgebungen
\usepackage{amsfonts}		% mathematische Schriften
\usepackage{amssymb}		% mathematische Symbole
\usepackage{amsthm}			% Satzumgebungen
\usepackage{stmaryrd}		% Pfeil-Symbole (z.B.\lightning)
\usepackage{cancel}			% Formeln durchstreichen
\usepackage{calc}			% damit man rechnen kann
\usepackage{siunitx}		% \angle, \num, physikal. Einheiten
\usepackage{esvect}  		% Erzeugt Vektorpfeile
\usepackage{extarrows}		% erzeugt Text über/unter Folgepfeilen
\usepackage{polynom}		% Polynomdivision 
\usepackage{wasysym}		% Auflistungszeichen
\usepackage{fancybox}
\usepackage{fancyhdr}
\usepackage{float}
\usepackage{mcode}	
\usepackage{nicefrac}

\usepackage{mathrsfs}
\usepackage{makeidx}


% Umgebungen für Sätze, Bemerkungen, etc.
\theoremstyle{plain}% default
\newtheorem{satz}{Satz}[section]
\newtheorem{folgerung}[satz]{Folgerung}
\newtheorem{lemma}[satz]{Hilfssatz} 
\newtheorem{korollar}[satz]{Folgerung}
\newtheorem{regel}[satz]{Regel} 
\newtheorem{bezeichnung}[satz]{Bezeichnung}

\theoremstyle{definition}
\newtheorem{beispiel}{Beispiel}[section]
%\newtheorem{bemerkung}{Bemerkung}[section]
\newtheorem{definition}{Definition}[section]
\newtheorem{notiz}{Bemerkung}[section]
\newtheorem{aufgabe}{Aufgabe}[section]
\newtheorem{sol}{Lösung}

\def \ii{{\relax\ifmmode \mathrm{i} \else i \fi}}
\def \ee{{\relax\ifmmode \mathrm{e} \else e \fi}}
\def \dd{{\relax\ifmmode \mathrm{d} \else d \fi}}
\def \g{{\relax\ifmmode \mathrm{g} \else g \fi}}
\def \R{\mathbb R}
\def \C{\mathbb C}
\def \Q{\mathbb Q}
\def \N{\mathbb N}
\def \Z{\mathbb Z}
\def \F{\mathbb F}
\def \A{\mathbb A}

\newcommand{\vektor}[1]{\overrightarrow{#1}}             		% Vektoren mit Pfeil

\def \bemerkung#1{\vskip 6pt\relax{\bf #1: \\}}
\def \beweisvon#1{\medskip\noindent {\sl Beweis von #1:\ }}
\def \det#1{\medskip\noindent {\mathrm{det}\left(#1\right) }}
\newcommand{\beweis}[1]
{\begingroup \textbf{Beweis:} #1 \endgroup}

% Referenzen
\usepackage{hyperref}		% Referenzen

% Zeilenabstand 1,2 Zeilen
\linespread{1.2}

% KeinZeileneinzug
\parindent = 0pt

% Zahlenformatierung
%\sisetup{output-decimal-marker={,}}

% Stil der Aufgabenstellung

\newcommand{\aufgabenname}[1]{
		\begingroup
		\renewcommand{\textsc}[1]{{\rmfamily\scshape##1}}
		\renewcommand{\emph}[1]{{\normalfont##1}}
		\sffamily\slshape\color{green}
		\textbf{\hfill \\Aufgabenname: }#1
		\endgroup
}

% Stil der Aufgabenstellung
\newcommand{\aufgabenstellung}[1]{\begin{aufgabe}#1\end{aufgabe}}

% Stil des Lösungswegs
	\newcommand{\loesungsweg}[1]{
		\begingroup
		\renewcommand{\textsc}[1]{{\rmfamily\scshape##1}}
		\renewcommand{\emph}[1]{{\normalfont##1}}
		\addtokomafont{subsection}{\color{red}}
		\addtokomafont{subsubsection}{\color{red}}
		\addtokomafont{caption}{\color{red}}
		\addtokomafont{captionlabel}{\color{red}}
		\sffamily\slshape\color{red}
		\textbf{\hfill \\Lösungsweg/Erklärung: }#1
		\endgroup
}

% Stil der Lösung
	\newcommand{\loesung}[1]{
		\begingroup
		\renewcommand{\textsc}[1]{{\rmfamily\scshape##1}}
		\renewcommand{\emph}[1]{{\normalfont##1}}
		\addtokomafont{subsection}{\color{red}}
		\addtokomafont{subsubsection}{\color{red}}
		\addtokomafont{caption}{\color{red}}
		\addtokomafont{captionlabel}{\color{red}}
		\sffamily\slshape\color{red}
		\textbf{\hfill \\Lösung: }#1
		\endgroup
}

\DeclareTextCommandDefault{\textperiodcentered}
	{\rmfamily\UseTextSymbol{OMS}{\textperiodcentered}}

% tikz Anpassungen 

\usepackage{lscape}
\usepackage{gnuplot-lua-tikz}
\usepackage{pgf,tikz}
\usetikzlibrary{arrows,decorations.pathmorphing,backgrounds,positioning,fit,petri}

\usepackage{pgfplots}
\usepackage{pgfplotstable}
\pgfplotsset{compat=newest}

\usepackage{environ}
\makeatletter

\newsavebox{\measure@tikzpicture}
\NewEnviron{scaletikzpicturetowidth}[1]{%
  \def\tikz@width{#1}%
  \def\tikzscale{1}\begin{lrbox}{\measure@tikzpicture}%
  \BODY
  \end{lrbox}%
  \pgfmathparse{#1/\wd\measure@tikzpicture}%
  \edef\tikzscale{\pgfmathresult}%
  \BODY
}
\makeatother

