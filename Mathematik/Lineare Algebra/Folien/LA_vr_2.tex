\documentclass[hyperref={pdfpagelabels=false}]{beamer}
\providecommand\thispdfpagelabel[1]{}
% Die Hyperref Option hyperref={pdfpagelabels=false} verhindert die Warnung:
% Package hyperref Warning: Option `pdfpagelabels' is turned off
% (hyperref)                because \thepage is undefined. 
% Hyperref stopped early 
%

\usepackage{lmodern}
% Das Paket lmodern erspart die folgenden Warnungen:
% LaTeX Font Warning: Font shape `OT1/cmss/m/n' in size <4> not available
% (Font)              size <5> substituted on input line 22.
% LaTeX Font Warning: Size substitutions with differences
% (Font)              up to 1.0pt have occurred.
%
\usepackage{epstopdf}
\usepackage{german}
\usepackage{amssymb}
\usepackage{amsfonts}
\usepackage{amsmath}
\usepackage{amsthm}
\usepackage{latexsym}
\usepackage{newalg}
\usepackage{ifthen}
\usepackage{fancybox}
\usepackage{fancyhdr}
%\usepackage[dvips, dvipdfm]{graphicx}
%\usepackage[pdftex]{color,graphicx}
\usepackage{eurofont}
\usepackage{aeguill}
\usepackage[T1]{fontenc}
\usepackage{textcomp}
\usepackage{wrapfig}
\usepackage{mathrsfs}
\usepackage{float}
\usepackage{color}
\usepackage{siunitx}

\usepackage[utf8]{inputenc}
%\usepackage{hyperref}

\theoremstyle{plain}% default
\newtheorem*{satz}{Satz} 
\newtheorem*{hilf}{Lemma} 
\newtheorem*{korollar}{Folgerung}
\newtheorem*{regel}{Regel} 

\theoremstyle{definition}
%\newtheorem*{definition}{Definition} 
\newtheorem*{aufgabe}{Aufgabe} 
\newtheorem*{uebung}{Übung}
\newtheorem*{sol}{Lösung:}

\theoremstyle{remark}
\newtheorem*{beispiel}{Beispiel} 
\newtheorem*{notiz}{Bemerkung} 
\newtheorem*{notation}{Notation}
\newtheorem*{bez}{Bezeichnung} 

\def \ii{{\relax\ifmmode \mathrm{i} \else i \fi}}
\def \ee{{\relax\ifmmode \mathrm{e} \else e \fi}}
\def \dd{{\relax\ifmmode \mathrm{d} \else d \fi}}
\def \g{{\relax\ifmmode \mathrm{g} \else g \fi}}
\def \R{\mathbb R}
\def \C{\mathbb C}
\def \Q{\mathbb Q}
\def \N{\mathbb N}
\def \Z{\mathbb Z}
\def \F{\mathbb F}
\def \A{\mathbb A}

\newcommand{\vektor}[1]{\overrightarrow{#1}}   

% tikz Anpassungen 

\usepackage{lscape}
\usepackage{gnuplot-lua-tikz}
\usepackage{pgf,tikz}
\usetikzlibrary{arrows,decorations.pathmorphing,backgrounds,positioning,fit,petri}

\usepackage{pgfplots}
\usepackage{pgfplotstable}
\pgfplotsset{compat=newest}

\usepackage{environ}
\makeatletter

\newsavebox{\measure@tikzpicture}
\NewEnviron{scaletikzpicturetowidth}[1]{%
  \def\tikz@width{#1}%
  \def\tikzscale{1}\begin{lrbox}{\measure@tikzpicture}%
  \BODY
  \end{lrbox}%
  \pgfmathparse{#1/\wd\measure@tikzpicture}%
  \edef\tikzscale{\pgfmathresult}%
  \BODY
}
\makeatother

\setcounter{secnumdepth}{-1}


% Wenn \titel{\ldots} \author{\ldots} erst nach \begin{document} kommen,
% kommt folgende Warnung:
% Package hyperref Warning: Option `pdfauthor' has already been used,
% (hyperref) ... 
% Daher steht es hier vor \begin{document}

\title{Lineare Algebra \\ euklidische Vektorräume}   
\author{ Reinhold Hübl} 
\date{Wintersemester 2020/21} 

% Dadurch wird verhindert, dass die Navigationsleiste angezeigt wird.
%\setbeamertemplate{navigation symbols}{}

% zusaetzlich ist das usepackage{beamerthemeshadow} eingebunden 
\usepackage{beamerthemeshadow}
\usetheme{CambridgeUS}

%  \beamersetuncovermixins{\opaqueness<1>{25}}{\opaqueness<2->{15}}
%  sorgt dafuer das die Elemente die erst noch (zukuenftig) kommen 
%  nur schwach angedeutet erscheinen 
\beamersetuncovermixins{\opaqueness<1>{25}}{\opaqueness<2->{12}}
% klappt auch bei Tabellen, wenn teTeX verwendet wird\ldots
\begin{document}


\begin{frame}
\titlepage
\centering 
    \scalebox{0.40}{\includegraphics{dhbw_ma_logo.eps}}
 
\end{frame} 



\section{euklidische Vektorräume}


\begin{frame}
\frametitle{Skalarprodukt}

\begin{definition}
Für zwei $n$--Vektoren $\vektor{v} = \left( \begin{matrix} v_1 \\ \vdots \\  v_n \end{matrix} 
\right)$ und  $\vektor{w} = \left( \begin{matrix} w_1 \\  \vdots \\ w_n \end{matrix} \right)$ 
heißt
  	$$ \langle \vektor{v}, \vektor{w} \rangle := v_1 w_1 + \cdots + v_n w_n $$
das \textbf{Skalarprodukt} von $\vektor{v}$ und $\vektor{w}$.
\end{definition}

\pause 
\begin{beispiel}
$ \langle  \left( \begin{matrix} 1 \\ 2 \\ 3  \end{matrix} \right), \,
\left( \begin{matrix} 2 \\ 3 \\ 4   \end{matrix} \right) \rangle 
= 1 \cdot 2 + 2 \cdot 3 + 3 \cdot 4  = 20$.
\end{beispiel}

\end{frame}

\begin{frame}
\frametitle{Skalarprodukt}

\begin{regel}\label{uvr_regel_scalar} Für Vektoren $\vektor{v}$, $\vektor{w}$, 
$\vektor{w_1}$ und $\vektor{w_2}$ und einen Skalar $r \in \mathbb R$ gilt:
\begin{enumerate}
\item<2-> $\langle \vektor{v}, \vektor{w} \rangle = \langle \vektor{w}, 
\vektor{v} \rangle \qquad$ (\textit{Kommutativgesetz}).
\item<3-> $\langle \vektor{v}, \vektor{w_1} + \vektor{w_2} \rangle = 
\langle \vektor{v}, \vektor{w_1} \rangle + \langle \vektor{v}, 
\vektor{w_2} \rangle \qquad$ (\textit{Distributivgesetz}).
\item<4-> $\langle r \cdot \vektor{v}, \vektor{w} \rangle = r \cdot \langle \vektor{v}, 
\vektor{w} \rangle \qquad$ (\textit{Skalarmultiplikation}).
\item<5-> $\langle \vektor{v}, \vektor{v} \rangle = \vert \vektor{v} \vert^2$.
\end{enumerate}
\end{regel}

\end{frame}

\begin{frame}
\frametitle{Satz von Cauchy--Schwarz}

Zwei Vektoren $\vektor{v}$ und $\vektor{w}$ heißen \textbf{parallel} wenn einer ein positives 
Vielfaches des anderen ist, und \textbf{anti--parallel}, wenn einer ein negatives Vielfaches des anderen ist. 

\pause
\begin{satz}[Satz von Cauchy--Schwarz]\label{uvr_cauchy_schwarz} 
Für zwei Vektoren $\vektor{v}$ und $\vektor{w}$ gilt:
\begin{enumerate}
\item<3-> $ \vert \langle \vektor{v},  \vektor{w} \rangle \vert \leq \vert \vektor{v} 
        \vert \cdot \vert \vektor{w} \vert $.
\item<4-> Genau dann sind $\vektor{v}$ und $\vektor{w}$ parallel, wenn $\langle \vektor{v}, 
\vektor{w} \rangle = \vert \vektor{v} \vert \cdot  \vert \vektor{w} \vert$.
\item<5-> Genau dann sind $\vektor{v}$ und $\vektor{w}$ anti--parallel, wenn $\langle 
\vektor{v}, \vektor{w} \rangle = 
- \vert \vektor{v} \vert \cdot  \vert \vektor{w} \vert$.
\item<6-> Genau dann sind $\vektor{v}$ und $\vektor{w}$ kollinear, wenn 
$\vert \langle \vektor{v}, \vektor{w} \rangle \vert \, 
= \vert \vektor{v} \vert \cdot  \vert \vektor{w} \vert$.
\end{enumerate}
\end{satz}

\end{frame}

\begin{frame}
\frametitle{Skalarprodukt}

\begin{satz}[Dreiecksungleichung]\label{uvr_dreiecks_ungleichung} 
 Für Vektoren $\vektor{v}$, $\vektor{w}$ und einen Skalar $r \in \mathbb R$ gilt:

\begin{enumerate}
\item<2-> $\vert r \cdot \vektor{v} \vert \, = \, \vert r \vert \cdot \vert \vektor{v} \vert$.
\item<3-> $\vert \vektor{v} + \vektor{w} \vert \, \leq \, \vert \vektor{v} \vert \, + \, \vert \vektor{w} \vert$.
\end{enumerate}
\end{satz}

\pause \pause \pause 

\begin{beispiel}
Betrachte das Dreieck mit dem Ecken $(0,0)$, $(1,2)$, $(2,1)$. Dieses Dreieck wird beschrieben durch 
	$$ \vektor{v} = \begin{pmatrix} 1 \\ 2 \end{pmatrix}, \quad \vektor{w} = \begin{pmatrix} 1 \\ -1 \end{pmatrix}, 
	\quad \vektor{u} = \vektor{v} + \vektor{w} = \begin{pmatrix} 2 \\ 1 \end{pmatrix} $$
und 
	$$ \vert \vektor{u} \vert  = \sqrt{5} \leq \sqrt{5} + \sqrt{2} =  \vert \vektor{v} \vert +  \vert \vektor{w} \vert $$
\end{beispiel}

\end{frame}

\begin{frame}
\frametitle{Skalarprodukt}

Aus dem Satz vo Cauchy--Schwarz folgt: \pause 

Es gibt ein ein $\varphi \in [0, \pi]$ gibt mit 
  	$$  \cos(\varphi) = \frac {\vert \langle \vektor{v},  \vektor{w} \rangle \vert}{\vert \vektor{v} \vert 
  	\cdot  \vert \vektor{w} \vert} $$ \pause 
Dieses $\varphi$ heißt der \textbf{Winkel} zwischen $\vektor{v}$ und 
$\vektor{w}$. In der Ebene handelt es sich dabei in der Tat um den Winkel zwischen den beiden 
Vektoren $\vektor{v}$ und $\vektor{w}$. 

\pause 

\begin{beispiel}
Der Winkel zwischen $\vektor{v} = \begin{pmatrix} 1 \\ 1 \end{pmatrix}$ und $ \vektor{w} = \begin{pmatrix} 
1 \\ - \sqrt{3} \end{pmatrix}$ ist $\frac {7}{12} \cdot \pi$ bzw. $105^{\circ}$, denn 
	$$ \cos\left(\frac {7}{12} \cdot \pi\right) = \frac {1-\sqrt{3}}{2 \cdot \sqrt{2}} = 
	\frac {\vert \langle \vektor{v},  \vektor{w} \rangle \vert}{\vert \vektor{v} \vert 
  	\cdot  \vert \vektor{w} \vert} $$
\end{beispiel}

\end{frame}

\begin{frame}
\frametitle{Skalarprodukt}

\begin{definition} Zwei $n$--dimensionale Vektoren $\vektor{v}$ und $\vektor{w}$ heißen 
\index{Vektoren!orthogonal} \textbf{orthogonal}, wenn gilt
  	$$ \langle \vektor{v},  \vektor{w} \rangle = 0 $$
Wir sagen in diesem Fall auch, dass $\vektor{v}$ senkrecht auf $\vektor{w}$ steht und schreiben 
$\vektor{v} \perp \vektor{w}$.
\end{definition}

\pause 

\begin{beispiel} Die Vektoren  $\vektor{v} = \left( \begin{matrix} 1 \\ 2 \\ -3 \\ 1
\end{matrix} \right)$ und  $\vektor{w} = \left( \begin{matrix} 2 \\ 4 \\ 4 \\ 2 
\end{matrix} \right)$ sind orthogonal. 
\end{beispiel}


\end{frame}

\begin{frame}
\frametitle{orthogonales Komplement}

\begin{definition} 
Ist $V \subseteq \mathbb R^n$ ein Untervektorraum, so heißt die Menge
  	$$ V^{\perp} = \{ \vektor{u} \in \mathbb R^n \vert \, 
      \langle \vektor{u}, \vektor{v}  \rangle =  0 \, \textrm{ für alle } 
      \vektor{v} \in V \} $$
das \textbf{orthogonale Komplement} von $V$.
\end{definition}

\pause 
\begin{notiz} Ist $V \subseteq \mathbb R^n$ ein Untervektorraum der 
Dimension $l$, so ist $V^{\perp} \subseteq \mathbb R^n$ ein Untervektorraum der Dimension $n-l$.
\end{notiz}

\end{frame}

\begin{frame}
\frametitle{orthogonales Komplement}

\begin{beispiel}
Das orthogonale Komplement von 
 	$$ V = \left\{ \left( \begin{matrix} t \\ t \end{matrix} \right) \,\, \vert \,\, t \in \R \right\} $$
ist
	$$ V^{\perp} = \left\{ \left( \begin{matrix} -t \\ t \end{matrix} \right) \,\, \vert \,\, t \in \R \right\} $$

\end{beispiel}

\end{frame}

\begin{frame}
\frametitle{orthogonales Komplement}

\begin{uebung}
Bestimmen Sie das  orthogonale Komplement von 
 	$$ V = \left\{ \left( \begin{matrix} r+s \\ r \\ s  \end{matrix} \right) \,\, \vert \,\, t \in \R \right\} $$
\end{uebung}

\bigbreak
\pause \pause 
\begin{sol}
	$$ V^{\perp} = \left\{ \left( \begin{matrix} -t \\ t \\ t \end{matrix} \right) \,\, \vert \,\, t \in \R \right\} $$

\end{sol}
\end{frame}

\begin{frame}
\frametitle{Skalarprodukt}
numerisch stabile Basis

\begin{figure}[H]
	\vspace{-0.9cm}
	\begin{center}
	\begin{scaletikzpicturetowidth}{0.8\textwidth}
     		\begin{tikzpicture}[gnuplot,scale=\tikzscale]
%% generated with GNUPLOT 5.0p4 (Lua 5.3; terminal rev. 99, script rev. 100)
%% 23.08.2020 17:20:41
\path (0.000,0.000) rectangle (12.500,8.750);
\gpcolor{color=gp lt color border}
\node[gp node left] at (7.158,1.668) {$\footnotesize{x}$};
\node[gp node left] at (6.070,4.417) {$\footnotesize{y}$};
\node[gp node left] at (3.120,5.596) {$\footnotesize{z}$};
%
  \gpfill{rgb color={0.000,0.000,0.000}} (7.608,2.032)--(7.642,2.171)--(7.943,1.950)--(7.574,1.894)--cycle;
\gpcolor{rgb color={0.000,0.000,0.000}}
\gpsetlinetype{gp lt border}
\gpsetdashtype{gp dt solid}
\gpsetlinewidth{1.00}
\draw[gp path] (7.608,2.032)--(7.642,2.171)--(7.943,1.950)--(7.574,1.894)--cycle;
\draw[gp path] (2.483,3.285)--(7.608,2.032);
%
  \gpfill{rgb color={0.000,0.000,0.000}} (6.161,4.648)--(6.076,4.764)--(6.439,4.852)--(6.246,4.533)--cycle;
\draw[gp path] (6.161,4.648)--(6.076,4.764)--(6.439,4.852)--(6.246,4.533)--cycle;
\draw[gp path] (3.330,2.571)--(6.161,4.648);
%
  \gpfill{rgb color={0.000,0.000,0.000}} (3.849,5.484)--(3.706,5.484)--(3.849,5.829)--(3.992,5.484)--cycle;
\draw[gp path] (3.849,5.484)--(3.706,5.484)--(3.849,5.829)--(3.992,5.484)--cycle;
\draw[gp path] (3.849,2.376)--(3.849,5.484);
%
  \gpfill{rgb color={1.000,0.000,0.000}} (5.277,2.602)--(5.317,2.764)--(5.669,2.506)--(5.238,2.440)--cycle;
\gpcolor{rgb color={1.000,0.000,0.000}}
\gpsetlinewidth{2.00}
\draw[gp path] (5.277,2.602)--(5.317,2.764)--(5.669,2.506)--(5.238,2.440)--cycle;
\draw[gp path] (3.849,2.952)--(5.277,2.602);
%
  \gpfill{rgb color={1.000,0.000,0.000}} (5.005,3.800)--(4.906,3.935)--(5.330,4.038)--(5.104,3.665)--cycle;
\draw[gp path] (5.005,3.800)--(4.906,3.935)--(5.330,4.038)--(5.104,3.665)--cycle;
\draw[gp path] (3.849,2.952)--(5.005,3.800);
%
  \gpfill{rgb color={1.000,0.000,0.000}} (3.849,4.192)--(3.682,4.192)--(3.849,4.595)--(4.016,4.192)--cycle;
\draw[gp path] (3.849,4.192)--(3.682,4.192)--(3.849,4.595)--(4.016,4.192)--cycle;
\draw[gp path] (3.849,2.952)--(3.849,4.192);
\gpcolor{color=gp lt color border}
\node[gp node left] at (4.645,1.974) {$\vektor{e_1}$};
\node[gp node left] at (4.737,3.357) {$\vektor{e_2}$};
\node[gp node left] at (2.756,4.288) {$\vektor{e_3}$};
%% coordinates of the plot area
\gpdefrectangularnode{gp plot 1}{\pgfpoint{1.800cm}{0.771cm}}{\pgfpoint{10.700cm}{8.287cm}}
\end{tikzpicture}
%% gnuplot variables
 
	\end{scaletikzpicturetowidth}
	\end{center}
	\vspace{-1.2cm}
\end{figure}

\end{frame}

\begin{frame}
\frametitle{Skalarprodukt}
numerisch instabile Basis

\begin{figure}[H]
	\vspace{-0.9cm}
	\begin{center}
	\begin{scaletikzpicturetowidth}{0.8\textwidth}
     		\begin{tikzpicture}[gnuplot,scale=\tikzscale]
%% generated with GNUPLOT 5.0p4 (Lua 5.3; terminal rev. 99, script rev. 100)
%% 23.08.2020 17:20:47
\path (0.000,0.000) rectangle (12.500,8.750);
\gpcolor{color=gp lt color border}
\node[gp node left] at (6.900,1.393) {$\footnotesize{x}$};
\node[gp node left] at (5.232,4.943) {$\footnotesize{y}$};
\node[gp node left] at (2.511,5.459) {$\footnotesize{z}$};
%
  \gpfill{rgb color={0.000,0.000,0.000}} (7.432,1.706)--(7.466,1.845)--(7.767,1.624)--(7.398,1.568)--cycle;
\gpcolor{rgb color={0.000,0.000,0.000}}
\gpsetlinetype{gp lt border}
\gpsetdashtype{gp dt solid}
\gpsetlinewidth{1.00}
\draw[gp path] (7.432,1.706)--(7.466,1.845)--(7.767,1.624)--(7.398,1.568)--cycle;
\draw[gp path] (2.307,2.960)--(7.432,1.706);
%
  \gpfill{rgb color={0.000,0.000,0.000}} (5.712,4.567)--(5.627,4.683)--(5.990,4.771)--(5.797,4.452)--cycle;
\draw[gp path] (5.712,4.567)--(5.627,4.683)--(5.990,4.771)--(5.797,4.452)--cycle;
\draw[gp path] (2.870,2.483)--(5.712,4.567);
%
  \gpfill{rgb color={0.000,0.000,0.000}} (3.217,5.471)--(3.074,5.471)--(3.217,5.816)--(3.360,5.471)--cycle;
\draw[gp path] (3.217,5.471)--(3.074,5.471)--(3.217,5.816)--(3.360,5.471)--cycle;
\draw[gp path] (3.217,2.352)--(3.217,5.471);
%
  \gpfill{rgb color={1.000,0.000,0.000}} (10.186,6.544)--(10.106,6.691)--(10.540,6.737)--(10.266,6.397)--cycle;
\gpcolor{rgb color={1.000,0.000,0.000}}
\gpsetlinewidth{2.00}
\draw[gp path] (10.186,6.544)--(10.106,6.691)--(10.540,6.737)--(10.266,6.397)--cycle;
\draw[gp path] (3.217,2.737)--(10.186,6.544);
%
  \gpfill{rgb color={1.000,0.000,0.000}} (4.483,3.571)--(4.391,3.711)--(4.820,3.793)--(4.575,3.432)--cycle;
\draw[gp path] (4.483,3.571)--(4.391,3.711)--(4.820,3.793)--(4.575,3.432)--cycle;
\draw[gp path] (3.217,2.737)--(4.483,3.571);
%
  \gpfill{rgb color={1.000,0.000,0.000}} (3.566,3.009)--(3.464,3.140)--(3.858,3.237)--(3.667,2.879)--cycle;
\draw[gp path] (3.566,3.009)--(3.464,3.140)--(3.858,3.237)--(3.667,2.879)--cycle;
\draw[gp path] (3.217,2.737)--(3.566,3.009);
\gpcolor{color=gp lt color border}
\node[gp node left] at (8.135,5.230) {$\vektor{v_1}$};
\node[gp node left] at (3.906,4.131) {$\vektor{v_2}$};
\node[gp node left] at (3.217,3.661) {$\vektor{v_3}$};
%% coordinates of the plot area
\gpdefrectangularnode{gp plot 1}{\pgfpoint{1.800cm}{0.771cm}}{\pgfpoint{10.700cm}{8.287cm}}
\end{tikzpicture}
%% gnuplot variables
 
	\end{scaletikzpicturetowidth}
	\end{center}
	\vspace{-1.2cm}
\end{figure}

\end{frame}


\begin{frame}
\frametitle{Skalarprodukt}

Für numerische Berechnungen sind Basen $\vektor{v_1}, \ldots, \vektor{v_m}$  von Vektorräumen und 
Untervektorräumen $V$ besonders dann sinnvoll, wenn Sie die folgenden Bedingungen erfüllen: 

\begin{itemize} 
\item<2-> Alle Vektoren sind normiert, dh. $\vert \vektor{v_i} \vert = 1$ für alle $i$. 
\item<3-> Alle Vektoren stehen paarweise senkrecht aufeinander, dh. $\langle \vektor{v_i}, \, \vektor{v_j} \rangle = 0$ für 
$i \neq j$. 
\end{itemize}

\pause \pause \pause 
\bigbreak
\begin{definition} Eine Basis, die diese Bedingungen erfüllt, heißt eine \textbf{Orthonormalbasis} (ONB) von $V$. 
\end{definition}
\end{frame}

\begin{frame}
\frametitle{Orthonormalbasen}

\begin{satz}\label{uvr_gram_schmidt} Jeder Untervektorraum $U$ des $\mathbb R^n$ besitzt eine 
Orthonormalbasis.
\end{satz}

\pause

Das Vorgehen ist dabei wie folgt: 

\begin{itemize}
\item<3-> Der Vektor $ \vektor{v_1}$ wird zunächst unverändert als Vektor $\vektor{u_1}$ übernommen.
\item<4-> Die Vektoren $\vektor{v_2}, \ldots, \vektor{v_m}$ werden sukkzessive so zu Vektoren $\vektor{u_2}, 
\ldots, \vektor{u_m}$ abgeändert, dass sie paarweise orthogonal sind. 
\item<5-> Die Vektoren $\vektor{u_1}, \ldots, \vektor{u_m}$ werden normiert: $\vektor{w_i} = 
\frac {1}{\vert \vektor{u_i}} \cdot \vektor{u_i}$. 
\end{itemize}
\end{frame}

\begin{frame}
\frametitle{Orthonormalbasen}

\begin{beispiel}
Die Orthonormalbasis zur Basis 
  	$$ \vektor{u_1} = \left( \begin{matrix} 1 \\ 1 \\ 0 \end{matrix} \right), \quad 
   	\vektor{u_2} = \left( \begin{matrix} 1 \\ 0 \\ 1 \end{matrix} \right), \quad 
   	\vektor{u_3} = \left( \begin{matrix} 0 \\ 1\\ 1 \end{matrix} \right)  $$
des $\R^3$ ist \pause \pause 
  	$$ \vektor{w_1} = \frac {\sqrt{2}}{2} \cdot \left( \begin{matrix} 1 \\ 1 \\ 0 \end{matrix} \right), \quad 
   	\vektor{w_2} = \frac {\sqrt{6}}{6} \cdot \left( \begin{matrix} 1 \\ -1 \\ 2 \end{matrix} \right), \quad 
   	\vektor{w_3} = \frac {\sqrt{3}}{6} \cdot \left( \begin{matrix} -2 \\ 2 \\ 2 \end{matrix} \right)  $$
 
\end{beispiel}

\end{frame}

\begin{frame}
\frametitle{Orthonormalbasen}

\begin{uebung}
Bestimmen Sie die Orthormalbasis zur Basis 
  	$$ \vektor{u_1} = \left( \begin{matrix} 1 \\ 2 \\ 2 \end{matrix} \right), \quad 
   	\vektor{u_2} = \left( \begin{matrix} 2 \\ 0 \\ 1 \end{matrix} \right), \quad 
   	\vektor{u_3} = \left( \begin{matrix} 2 \\ 1\\ 1 \end{matrix} \right)  $$
des $\R^3$
\end{uebung}

\pause \pause 

\begin{sol}
Die zugehörige ONB ist 
  	$$ \vektor{w_1} = \frac {1}{3} \cdot \left( \begin{matrix} 1 \\ 2 \\ 2 \end{matrix} \right), \quad 
   	\vektor{w_2} = \frac {\sqrt{29}}{87} \cdot \left( \begin{matrix} 14 \\ -8 \\ 1 \end{matrix} \right), \quad 
   	\vektor{w_3} = \frac {\sqrt{29}}{783} \cdot \left( \begin{matrix} 54 \\ 81 \\ -108 \end{matrix} \right)  $$

\end{sol}

\end{frame}

\begin{frame}
\frametitle{Orthonormalbasen}

\begin{uebung}
Bestimmen Sie die Orthormalbasis zur Basis 
  	$$ \vektor{u_1} = \left( \begin{matrix} 1 \\ 2 \\ 3 \end{matrix} \right), \quad 
   	\vektor{u_2} = \left( \begin{matrix} 3 \\ 2 \\ 1 \end{matrix} \right), \quad 
   	\vektor{u_3} = \left( \begin{matrix} 1 \\ 1\\ 1 \end{matrix} \right)  $$
des $\R^3$
\end{uebung}

\pause \pause 

\begin{sol}
Das Verfahren bricht ab; Sie erhalten
  	$$ \vektor{v_3} = \vektor{0}  $$
Das bedeutet, dass die Vektoren $\vektor{u_1}$, $\vektor{u_2}$ und $\vektor{u_3}$ keine 
Basis von $\R^3$ bilden. 
\end{sol}

\end{frame}

\end{document}
