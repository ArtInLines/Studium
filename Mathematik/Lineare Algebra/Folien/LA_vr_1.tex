\documentclass[hyperref={pdfpagelabels=false}]{beamer}
\providecommand\thispdfpagelabel[1]{}
% Die Hyperref Option hyperref={pdfpagelabels=false} verhindert die Warnung:
% Package hyperref Warning: Option `pdfpagelabels' is turned off
% (hyperref)                because \thepage is undefined. 
% Hyperref stopped early 
%

\usepackage{lmodern}
% Das Paket lmodern erspart die folgenden Warnungen:
% LaTeX Font Warning: Font shape `OT1/cmss/m/n' in size <4> not available
% (Font)              size <5> substituted on input line 22.
% LaTeX Font Warning: Size substitutions with differences
% (Font)              up to 1.0pt have occurred.
%
\usepackage{epstopdf}
\usepackage{german}
\usepackage{amssymb}
\usepackage{amsfonts}
\usepackage{amsmath}
\usepackage{amsthm}
\usepackage{latexsym}
\usepackage{newalg}
\usepackage{ifthen}
\usepackage{fancybox}
\usepackage{fancyhdr}
%\usepackage[dvips, dvipdfm]{graphicx}
%\usepackage[pdftex]{color,graphicx}
\usepackage{eurofont}
\usepackage{aeguill}
\usepackage[T1]{fontenc}
\usepackage{textcomp}
\usepackage{wrapfig}
\usepackage{mathrsfs}
\usepackage{float}
\usepackage{color}
\usepackage{siunitx}

\usepackage[utf8]{inputenc}
%\usepackage{hyperref}

\theoremstyle{plain}% default
\newtheorem*{satz}{Satz} 
\newtheorem*{hilf}{Lemma} 
\newtheorem*{korollar}{Folgerung}
\newtheorem*{regel}{Regel} 

\theoremstyle{definition}
%\newtheorem*{definition}{Definition} 
\newtheorem*{aufgabe}{Aufgabe} 
\newtheorem*{uebung}{Übung}
\newtheorem*{sol}{Lösung:}

\theoremstyle{remark}
\newtheorem*{beispiel}{Beispiel} 
\newtheorem*{notiz}{Bemerkung} 
\newtheorem*{notation}{Notation}
\newtheorem*{bez}{Bezeichnung} 

\def \ii{{\relax\ifmmode \mathrm{i} \else i \fi}}
\def \ee{{\relax\ifmmode \mathrm{e} \else e \fi}}
\def \dd{{\relax\ifmmode \mathrm{d} \else d \fi}}
\def \g{{\relax\ifmmode \mathrm{g} \else g \fi}}
\def \R{\mathbb R}
\def \C{\mathbb C}
\def \Q{\mathbb Q}
\def \N{\mathbb N}
\def \Z{\mathbb Z}
\def \F{\mathbb F}
\def \A{\mathbb A}

\newcommand{\vektor}[1]{\overrightarrow{#1}}   

% tikz Anpassungen 

\usepackage{lscape}
\usepackage{gnuplot-lua-tikz}
\usepackage{pgf,tikz}
\usetikzlibrary{arrows,decorations.pathmorphing,backgrounds,positioning,fit,petri}

\usepackage{pgfplots}
\usepackage{pgfplotstable}
\pgfplotsset{compat=newest}

\usepackage{environ}
\makeatletter

\newsavebox{\measure@tikzpicture}
\NewEnviron{scaletikzpicturetowidth}[1]{%
  \def\tikz@width{#1}%
  \def\tikzscale{1}\begin{lrbox}{\measure@tikzpicture}%
  \BODY
  \end{lrbox}%
  \pgfmathparse{#1/\wd\measure@tikzpicture}%
  \edef\tikzscale{\pgfmathresult}%
  \BODY
}
\makeatother

\setcounter{secnumdepth}{-1}


% Wenn \titel{\ldots} \author{\ldots} erst nach \begin{document} kommen,
% kommt folgende Warnung:
% Package hyperref Warning: Option `pdfauthor' has already been used,
% (hyperref) ... 
% Daher steht es hier vor \begin{document}

\title{Lineare Algebra \\ Vektorräume}   
\author{ Reinhold Hübl} 
\date{Wintersemester 2020/21} 

% Dadurch wird verhindert, dass die Navigationsleiste angezeigt wird.
%\setbeamertemplate{navigation symbols}{}

% zusaetzlich ist das usepackage{beamerthemeshadow} eingebunden 
\usepackage{beamerthemeshadow}
\usetheme{CambridgeUS}

%  \beamersetuncovermixins{\opaqueness<1>{25}}{\opaqueness<2->{15}}
%  sorgt dafuer das die Elemente die erst noch (zukuenftig) kommen 
%  nur schwach angedeutet erscheinen 
\beamersetuncovermixins{\opaqueness<1>{25}}{\opaqueness<2->{15}}
% klappt auch bei Tabellen, wenn teTeX verwendet wird\ldots
\begin{document}


\begin{frame}
\titlepage
\centering 
    \scalebox{0.40}{\includegraphics{dhbw_ma_logo.eps}}
 
\end{frame} 



\section{Vektorräume}


\begin{frame}
\frametitle{Vektorraum}

Wir betrachten einen beliebigen Körper $K$.

\pause 

\begin{definition} Ein \textbf{$K$--Vektorraum} ist eine 
nichtleere Menge $V$ zusammen mit einer Addition 
  	$$ '+': V \times V \longrightarrow V, \quad (\vektor{v}, \vektor{w}) 
     	\longmapsto \vektor{v} + \vektor{w} $$
und einer Skalarmultiplikation 
  	$$ '\cdot': K \times V \longrightarrow V, \quad (r, \vektor{v}) 
     	\longmapsto r \cdot \vektor{v} $$
so dass für $\vektor{u}, \vektor{v}$ und 
$\vektor{w}$ in $V$ und Skalare $r$ und $s$ in $K$ gilt:

\end{definition}
\end{frame}

\begin{frame}
\frametitle{Vektorraum}

\begin{definition}
\begin{tabular} {l l c l}
$V1:$ & $\left(\vektor{u} + \vektor{v} \right) + \vektor{w} = 
\vektor{u} + \left(\vektor{v} + \vektor{w} \right)$ & $\,$ &
(1. Assoziativgesetz) \\
$V2:$ & $\vektor{v} + \vektor{w} = \vektor{w} + 
\vektor{v}$ & & (Kommutativgesetz) \\
$V3:$ & Es existiert ein Element $\vektor{0} \in V$ & & \\
& mit  $\vektor{v} + \vektor{0} = \vektor{v}$ & & (neutrales 
Element \\
& & & der Addition) \\
$V4:$ & Zu jedem $\vektor{v} \in V$ existiert ein  
& & \\
& $-\vektor{v} \in V$ mit $\vektor{v} + (- \vektor{v}) = \vektor{0}$ & &
(inverses Element \\
& & & der Addition) \\
$V5:$ & $(r \cdot s) \cdot \vektor{v} = r \cdot ( s \cdot \vektor{v})$ 
& & (2. Assoziativgesetz) \\
$V6:$ & $r \cdot (\vektor{v} + \vektor{w}) = r \cdot 
\vektor{v} + r \cdot \vektor{w}$ & & (1. Distributivgesetz) \\
$V7:$ & $(r + s) \cdot \vektor{v} = r \cdot \vektor{v} + 
s \cdot \vektor{v}$ & & (2. Distributivgesetz) \\
$V8:$ & $ 1 \cdot \vektor{v} = \vektor{v}$ & & (neutrales Element \\
& & & der Multiplikation)
\end{tabular}
\end{definition}

\end{frame}

\begin{frame}
\frametitle{Vektorraum}

Die ersten vier Bedingungen besagen, dass $(V,+)$ eine kommutative Gruppe ist. 

\bigbreak

\pause 
Die letzten vier Bedingungen sind Anforderungen an die Skalarmultiplikation. 

\end{frame}

\begin{frame}
\frametitle{Vektorraum}

\begin{beispiel}
Wir betrachten $K = \R$ und $V = \R^n$ zusammen mit der komponentenweisen Addition \pause 
  	$$ \left( \begin{matrix} v_1 \\ \vdots \\ v_n 
     	\end{matrix} \right) + \left( \begin{matrix} w_1 \\ \vdots \\ w_n 
     	\end{matrix} \right) = \left( \begin{matrix} v_1 + w_1 \\ \vdots \\ 
     	v_n + w_n  \end{matrix} \right) $$ 
%($v_i, w_i \in \R$) 
und der komponentenweisen Skalarmultiplikation \pause 
  	$$ r \cdot \left( \begin{matrix} v_1 \\ \vdots \\ v_n 
     	\end{matrix} \right) = \left( \begin{matrix} r \cdot v_1 \\ \vdots \\ 
     	r \cdot v_n \end{matrix} \right) $$
%($r, v_i \in \R$).

\pause 

\medbreak
Dann ist $\R^n$ ein $\R$--Vektorraum.  

\end{beispiel}

\end{frame}

\begin{frame}
\frametitle{Vektorraum}

\begin{beispiel}
Der Nullvektor 
	$$ \vektor{0} = \begin{pmatrix} 0 \\ \vdots \\ 0 \end{pmatrix} $$
ist dabei das neutrale Element der Addition. 

\pause
Der negative Vektor $- \vektor{v}$ zu 
einem $n$--dimensionalen reellen Vektor ist gegeben durch 
	$$ - \vektor{v} = \begin{pmatrix} -v_1 \\ \vdots \\ - v_n \end{pmatrix} $$
\end{beispiel}
\end{frame}


\begin{frame}
\frametitle{Vektorraum}

\begin{beispiel}\label{LA_vr_cn} Das $n$--fache kartesische Produkt von $\mathbb C$, also die Menge
  	$$ \mathbb C^n := \{ \left( \begin{matrix} z_1 \\ \vdots \\ z_n 
     	\end{matrix} \right) \vert \, z_i \in \mathbb C \} $$
zusammen mit der komponentenweisen Addition und der komponentenweisen Skalarmultiplikation
  	$$ \left( \begin{matrix} y_1 \\ \vdots \\ y_n 
     	\end{matrix} \right) + \left( \begin{matrix} z_1 \\ \vdots \\ z_n 
     	\end{matrix} \right) = \left( \begin{matrix} y_1 + z_1 \\ \vdots \\ 
     	y_n + z_n  \end{matrix} \right), \qquad 
	 a \cdot \left( \begin{matrix} z_1 \\ \vdots \\ z_n 
     	\end{matrix} \right) = \left( \begin{matrix} a \cdot z_1 \\ \vdots \\ 
     	a \cdot z_n \end{matrix} \right) $$
($a, y_i, z_i \in \mathbb C$) ist ein $\mathbb C$--Vektorraum. 
\end{beispiel} 

\end{frame}

\begin{frame}
\frametitle{Vektorraum}

\begin{uebung}
Bestimmen Sie 
  	$$ \left( \begin{matrix} 1 \\ 2 \\ 3 \\5  \end{matrix} \right) 
	+ \left( \begin{matrix} 3 \\  - 3 \\ 4 \\  2 \end{matrix} \right) $$
\end{uebung}

\pause \pause 

\begin{sol}
  	$$ \left( \begin{matrix} 1 \\ 2 \\ 3 \\5  \end{matrix} \right) 
	+ \left( \begin{matrix} 3 \\  - 3 \\ 4 \\  2 \end{matrix} \right) 
	= \left( \begin{matrix} 4 \\ -1 \\ 7 \\ 7  \end{matrix} \right) $$

\end{sol}

\end{frame}


\begin{frame}
\frametitle{Vektorraum}

\begin{beispiel} Die Zahlenebene $\R^2$ ist ein $\R$--Vektorraum, der Vektorraum der ebenen Vektoren, die 
durch Pfeile in der Ebenen dargestellt werden. 
\end{beispiel}

\pause 

\bigbreak

\begin{beispiel}
Der Raum $\R^3$ ist ein $\R$--Vektorraum, der Vektorraum der räumlichen Vektoren, die 
durch Pfeile im Raum dargestellt werden.
\end{beispiel}

\end{frame}

\begin{frame}
\frametitle{Vektorraum}

\begin{beispiel} Das $n$--fache kartesische Produkt von $\mathbb F_2$, also die Menge
  	$$ \mathbb F_2^n := \{ \left( \begin{matrix} a_1 \\ \vdots \\ a_n 
     	\end{matrix} \right) \vert \, a_i \in \mathbb C \} $$
zusammen mit komponentenweiser Addition und Skalarmultiplikation ist ein 
$\mathbb F_2$--Vektorraum.  \pause

Bei $\mathbb F_2^n$ handelt es sich um die $n$--Tupel binärer Zahlen, die sich 
komplett auflisten lassen. So gilt etwa 
  	$$ \mathbb F_2^2 := \{ \left( \begin{matrix} 0 \\ 0 \end{matrix} \right),
   	\left( \begin{matrix} 0 \\ 1 \end{matrix} \right), 
   	\left( \begin{matrix} 1 \\ 0 \end{matrix} \right),
   	\left( \begin{matrix} 1 \\ 1 \end{matrix} \right) \} $$ \pause 
Insbesondere ist also $\mathbb F_2^n$ eine endliche Menge mit $ \vert \mathbb F_2^n \vert \, = 2^n $.

\end{beispiel}

\end{frame}

\begin{frame}
\frametitle{Vektorraum}

\begin{beispiel} 
Wir betrachten die Menge 
	$$ V = \mathrm{Abb}(\N, \R) = \{ f: \N \longrightarrow \R \, \text{ Abbildung }\} $$
aller Abbildungen von $\N$ nach $\R$. \pause  Für $f, g \in V$ definieren wir $f+g$ durch 
	$$ (f+g)(n) = f(n) + g(n) \qquad \text{ für alle } n \in N $$ \pause
Für $f \in V$ und $r \in \R$ definieren wir $r \cdot f$ durch
	$$ (r \cdot f)(n) = r \cdot f(n) \qquad \text{ für alle } n \in N $$
Dann ist $V$ ein $\R$--Vektorraum. 

\pause 
In dieser Situation schreibt man häufig $f_n$ für $f(n)$ und $\left(f_n\right)_{n \in \N}$ für $f$ und nennt 
ein Element $\left(f_n\right)_{n \in \N}$ eine \textbf{reelle Zahlenfolge}. Der Raum $V$ heißt auch 
Vektorraum der Folgen.
\end{beispiel}

\end{frame}

\begin{frame}
\frametitle{Vektorraum}

\begin{beispiel}
Wir betrachten ein Intervall $ I \subseteq \mathbb R$ und die Menge 
	$$ V = \mathrm{Abb}(I, \R) = \{ f: I \longrightarrow \R \, \text{ Abbildung }\} $$
aller Abbildungen von $I$ nach $\R$. \pause  Für $f, g \in V$ definieren wir $f+g$ durch 
	$$ (f+g)(x) = f(x) + g(x) \qquad \text{ für alle } x \in I $$ \pause
Für $f \in V$ und $r \in \R$ definieren wir $r \cdot f$ durch
	$$ (r \cdot f)(x) = r \cdot f(x) \qquad \text{ für alle } x \in I $$
Dann ist $V$ ein $\R$--Vektorraum, der Vektorraum der reellwertigen Abbildungen auf $I$. 

\pause 
Fordern wir zusätzlich, dass die Abbildungen stetig, differenzierbar, beliebig oft differenzierbar sind, 
so erhalten wir ebenfalls Vektorräume. 
\end{beispiel}

\end{frame}

\section{Untervektorräume}

\begin{frame}
\frametitle{Untervektorraum}

\begin{definition} Ist $V$ ein $K$--Vektorraum und  ist $U \subseteq V$ eine 
Teilmenge, so heißt $U$ \textbf{Untervektorraum} von $V$ wenn gilt:

\begin{enumerate}
\item<2-> $U \neq \emptyset$. 
\item<3-> Sind $\vektor{v}, \vektor{w} \in U$ so ist auch 
$\vektor{v} + \vektor{w} \in U$.
\item<4-> Ist $\vektor{v} \in U$ und ist $\kappa \in K$ ein Skalar, so ist 
auch $\kappa \cdot \vektor{v} \in U$.
\end{enumerate}
\end{definition}

\pause 

\begin{notiz}
Ist $U \subseteq V$ ein Untervektorraum, so ist der Nullvektor $\vektor{0} \in U$. 
\end{notiz}


\end{frame}


\begin{frame}
\frametitle{Untervektorraum}

\begin{beispiel}
Die Menge 
	$$ U = \left\{ \begin{pmatrix} 2 t \\ 0 \end{pmatrix} \, \, \vert \, t \in \R \right\} 
	\subseteq \R^2 $$
ist ein Untervektorraum von $\R^2$. 
\end{beispiel}

\end{frame}


\begin{frame}
\frametitle{Untervektorraum}

\begin{uebung}
Überprüfen Sie, ob die Menge 
	$$ U = \left\{ \begin{pmatrix} -2 t \\ 2 \end{pmatrix} \, \, \vert \, t \in \R \right\} 
	\subseteq \R^2 $$
ein Untervektorraum von $\R^2$ ist. 
\end{uebung}

\pause \pause 
\bigbreak

\begin{sol}
 $U$ ist kein Untervektorraum von $\R^2$. 
\end{sol}
\end{frame}

\section{Der $n$--dimensionale reelle Raum}

\begin{frame}
\frametitle{Der $\R^n$}

Der vielleicht wichtigste Vektorraum in der Anwendung ist der $\R^n$, der \textbf{$n$--dimensionale 
reelle Raum}. \pause Die Vektorraumoperationen sind in diesem Fall sehr einfach zu beschreiben. 

\pause 

\begin{definition}
Ein Vektor 
	$$ \vektor{v} = \begin{pmatrix} v_1 \\ \vdots \\ v_n \end{pmatrix} \quad \in \R^n $$
heißt \textbf{$n$--dimensionaler reeller Vektor}, die Größe
	$$ \vert \vektor{v} \Vert = \sqrt{v_1^2 + \cdots v_n^2} $$
heißt \textbf{Länge} des Vektors $\vektor{v}$.
\end{definition}
\end{frame}

\begin{frame}
\frametitle{linear unabhängig}

\begin{definition}\label{uvr_lin_unab} 
Vektoren  $\vektor{v_1}, \vektor{v_2}, \ldots, \vektor{v_m} \in \R^n$ 
heißen \textbf{linear abhängig}, wenn es reelle Zahlen $r_1, r_2, \ldots, r_m$ gibt, 
von denen mindestens eine von Null verschieden ist, mit
  	$$ r_1 \cdot \vektor{v_1} + r_2 \cdot  \vektor{v_2} + \cdots + r_m 
     	\cdot \vektor{v_m} = \vektor{0} $$ \pause 
Andernfalls heißen sie \textbf{linear unabhängig}.
\end{definition}

\pause 
\begin{notiz} Die Bedingung für lineare Unabhängigkeit kann auch so formuliert werden: \pause Sind $r_1,  r_2, 
\ldots, r_m$ reelle Zahlen mit 
  	$$ r_1 \cdot \vektor{v_1} + r_2 \cdot  \vektor{v_2} + \cdots + r_m \cdot \vektor{v_m} = \vektor{0} $$
so muss schon gelten: $r_1 = r_2 = \cdots = r_m = 0$.
\end{notiz}

\end{frame}

\begin{frame}
\frametitle{linear unabhängig}

\begin{beispiel} 
Die Vektoren 
	$$ \vektor{v_1} = \left( \begin{matrix} 1 \\ 2 \\ -3 \\ 2 \end{matrix} \right), \qquad 
	\vektor{v_2} = \left( \begin{matrix} -2 \\ -4 \\ 6 \\ -4 \end{matrix} \right) $$ 
sind linear abhängig.
\end{beispiel}

\end{frame}

\begin{frame}
\frametitle{linear unabhängig}

\begin{beispiel} 
Die Vektoren 
	$$ \vektor{v_1} = \left( \begin{matrix} 1 \\ 2 \\ 3 \\ 4 \end{matrix} \right), \qquad 
	\vektor{v_2} = \left( \begin{matrix} 1 \\ 1 \\ 1 \\ 1 \end{matrix} \right) $$ 
sind linear unabhängig
\end{beispiel}

\end{frame}

\begin{frame}
\frametitle{linear unabhängig}

\begin{uebung}
Untersuchen Sie, ob die Vektoren 
	$$ \vektor{v_1} = \left( \begin{matrix} 1 \\ 2 \\ 3 \\ 4 \end{matrix} \right), \quad 
	\vektor{v_2} = \left( \begin{matrix} 4 \\ 3 \\ 2 \\ 1 \end{matrix} \right), \quad 
	\vektor{v_3} = \left( \begin{matrix} 1 \\ 1 \\ 1 \\ 1 \end{matrix} \right) $$ 
linear unabhängig sind.
\end{uebung}

\bigbreak

\pause \pause 

\begin{sol} Die Vektoren sind linear abhängig, denn 
	$$ \vektor{v_1} + \vektor{v_2} - 5 \cdot \vektor{v_3} = \vektor{0} $$
\end{sol}
\end{frame}

\begin{frame}
\frametitle{linear unabhängig}

\begin{regel}
Vektoren $\vektor{v_1}, \vektor{v_2}, \ldots, \vektor{v_m}$ sind 
schon dann linear abhängig, wenn eine der foglenden Bedingungen erfüllt ist. 
\begin{itemize}
\item<2-> $\vektor{v_k} = \vektor{0}$ für ein $k$.
\item<3-> $\vektor{v_k} = \vektor{v_l}$ für ein $k \neq l$.
\end{itemize}
\end{regel}

\pause \pause \pause 

\bigbreak

Das sind aber nur hinreichende Bedingung. In der Regel kann lineare Abhängigkeit nicht so einfach erkannt werden. 

\end{frame}

\begin{frame}
\frametitle{linear unabhängig}

\begin{notiz}
Ein einzelner Vektor $\vektor{v} \in \R^n$ ist genau dann linear abhängig, wenn $\vektor{v} = \vektor{0}$. \pause 
Entsprechend ist er genau dann linear unabhängig, wenn $\vektor{v} \neq \vektor{0}$.
\end{notiz}

\pause 

\begin{notiz} Zwei Vektoren $\vektor{v}$ und $\vektor{w}$ sind genau dann linear abhängig, wenn 
einer von beiden ein Vielfaches des anderen ist.
\end{notiz}


\end{frame}

\begin{frame}
\frametitle{linear unabhängig}

\begin{beispiel} Die $n$--dimensionalen Vektoren  
	$$\vektor{e_1} =  \left( \begin{matrix} 1 \\ 0 \\ \vdots \\ 0 \end{matrix} \right), \quad 
	\vektor{e_2} = \left( \begin{matrix} 0 \\ 1 \\ \vdots \\ 0  \end{matrix} \right), \ldots ,
	 \vektor{e_n} = \left( \begin{matrix} 0 \\ \vdots \\ 0 \\ 1 \end{matrix} \right) $$ 
im $\R^n$ sind linear unabhängig.
\end{beispiel}
\end{frame}

\section{Untervektorräume von $\R^n$}
\begin{frame}
\frametitle{Untervektorraum}

\begin{beispiel}
Die Menge $V = \{\vektor{0} \} \subseteq \R^4$ ist ein Untervektorraum.
\end{beispiel}

\pause 

\begin{beispiel}
Die Menge 
	$$ V = \left\{ \begin{pmatrix} t \\ 2t  \end{pmatrix} \,\vert \,\, t \in \R \right\} \subseteq \R^2 $$
ist ein Untervektorraum. 
\end{beispiel}

\pause 

\begin{beispiel}
Die Menge 
	$$ V = \left\{ \begin{pmatrix} x \\ y \\ z \end{pmatrix} \,\vert \,\, x + y + z = 0 \right\} \subseteq \R^3 $$
ist ein Untervektorraum. 
\end{beispiel}
\end{frame}

\begin{frame}
\frametitle{Untervektorraum}

\begin{uebung}
Überprüfen Sie, ob die Menge 
	$$ V = \left\{ \begin{pmatrix} x \\ y \\ z \end{pmatrix} \,\vert \,\, x^2 + y^2 + z^2 = 0 \right\} \subseteq \R^3 $$
ein Untervektorraum ist. 
\end{uebung}

\pause \pause 

\begin{sol}
$V$ ist kein Untervektroraum. 
\end{sol}
\end{frame}

\begin{frame}
\frametitle{Untervektorraum}
Betrachte eine Untervektorraum $V \subseteq \R^n$. 

\begin{definition} Die Vektoren $\vektor{v_1}, \vektor{v_2}, \ldots, 
\vektor{v_m}$ heißen \textbf{Erzeugendensystem} von $V$, wenn gilt:

\begin{enumerate}
\item<3-> $\vektor{v_1}, \vektor{v_2}, \ldots, \vektor{v_m} \in V$.
\item<4-> Zu jedem $\vektor{w} \in V$ gibt es Skalare $r_1, r_2, \ldots,  r_m$ mit
  	$$  \vektor{w} = r_1 \cdot \vektor{v_1} + r_2 \cdot \vektor{v_2} + \ldots + 
      r_m \cdot \vektor{v_m} $$
\end{enumerate}
\pause \pause \pause 
Wir sagen in diesem Fall auch,  $\vektor{v_1}, \vektor{v_2}, \ldots, 
\vektor{v_m}$ erzeugen $V$.
\end{definition}

\end{frame}

\begin{frame}
\frametitle{Untervektorraum}

\begin{beispiel} Die Vektoren  
	$$\vektor{e_1} = \left( \begin{matrix} 1 \\ 0 \\ 0 \end{matrix} \right), \quad 
	\vektor{e_2} = \left( \begin{matrix} 0 \\ 1 \\  0 \end{matrix} \right), \quad  
	\vektor{e_3} = \left( \begin{matrix} 0 \\ 0 \\ 1 \end{matrix} \right) $$ 
erzeugen $V = \mathbb R^3$
\end{beispiel}

\end{frame}

\begin{frame}
\frametitle{Untervektorraum}

\begin{beispiel} Die Vektoren  
	$$\vektor{v_1} = \left( \begin{matrix} 1 \\ 1 \\ 0 \end{matrix} \right), \quad 
	\vektor{v_2} = \left( \begin{matrix} 0 \\ 1 \\  1 \end{matrix} \right), \quad  
	\vektor{v_3} = \left( \begin{matrix} 1 \\ 0 \\ 1 \end{matrix} \right), , \quad  
	\vektor{v_4} = \left( \begin{matrix} 1 \\ 1 \\ 1 \end{matrix} \right)  $$ 
erzeugen $V = \mathbb R^3$. \pause Erzeugendensysteme sind also nicht eindeutig und können aus 
unterschiedlich vielen Vektoren bestehen. 
\end{beispiel}

\end{frame}

\begin{frame}
\frametitle{Untervektorraum}

\begin{beispiel} Die Untervektorraum
	$$ V  = \left\{ \begin{pmatrix} x \\ y \\ z \end{pmatrix} \,\vert \,\, x + y + z = 0 \right\} \subseteq \R^3 $$
von $\R^3$ wird erzeugt von den Vektoren 
	$$\vektor{v_1} = \left( \begin{matrix} -1 \\ 1 \\  0 \end{matrix} \right), \quad 
	\vektor{v_2} = \left( \begin{matrix} -1 \\ 0 \\  1 \end{matrix} \right) $$ 
\end{beispiel}

\end{frame}

\begin{frame}
\frametitle{Untervektorraum}

\begin{uebung} Untersuchen Sie, ob
	$$ V = \left\{ \begin{pmatrix} 2r + 3s \\ r-s \\ r+s \end{pmatrix} \,\vert \,\,r, s \in \R \right\} \subseteq \R^3 $$
ein Untervektorraum ist und bestimmen Sie gegebenenfalls ein Erzeugendensystem von $V$. 
\end{uebung}

\pause \pause 
\begin{sol}
$V$ ist ein Untervektorraum von $\R^3$ und wird erzeugt von 
	$$\vektor{v_1} = \left( \begin{matrix} 2 \\ 1 \\ 1 \end{matrix} \right), \quad 
	\vektor{v_2} = \left( \begin{matrix} 3 \\ -1 \\  1 \end{matrix} \right) $$ 
\end{sol}

\end{frame}

\begin{frame}
\frametitle{Untervektorraum}
 Für Vektoren $\vektor{v_1}, \vektor{v_2}, \ldots, 
\vektor{v_m} \in \mathbb R^n$ setze
  	$$ U = \{ \vektor{v} \in \mathbb R^n \, \vert \, \, \exists r_1, \ldots, r_m \in 
	\mathbb R \, \textrm{ mit } \,
   	\vektor{v} = r_1 \cdot \vektor{v_1} + \cdots + r_m \cdot \vektor{v_m} \} $$

\pause 
\begin{satz} $U$ ist ein Untervektorraum von $\mathbb R^n$.
\end{satz}

\pause 

$U$ heißt das \textbf{Erzeugnis} von $\vektor{v_1}, \vektor{v_2}, \ldots, \vektor{v_m}$ 
oder der von $\vektor{v_1}, \vektor{v_2}, \ldots, \vektor{v_m}$
\textbf{aufgespannte Unterraum} von $\mathbb R^n$
\end{frame}

\section{Basis}

\begin{frame}
\frametitle{Basis eines Vektorraums}
Wir betrachten wieder einen Untervektorraum $V \subseteq \R^n$

\pause
\begin{definition} Die Vektoren $\vektor{v_1}, \vektor{v_2}, \ldots, 
\vektor{v_m} \in V$ heißen \textbf{Basis} von $V$, wenn 
sie ein Erzeugendensystem von $V$ bilden, 
und wenn sie linear unabhängig sind.
\end{definition}

\pause 

\begin{beispiel} Die Vektoren  
	$$\vektor{e_1} = \left( \begin{matrix} 1 \\ 0 \\ 0 \end{matrix} \right), \quad 
	\vektor{e_2} = \left( \begin{matrix} 0 \\ 1 \\  0 \end{matrix} \right), \quad  
	\vektor{e_3} = \left( \begin{matrix} 0 \\ 0 \\ 1 \end{matrix} \right) $$ 
bilden eine Basis von $V = \mathbb R^3$
\end{beispiel}

\end{frame}

\begin{frame}
\frametitle{Basis eines Vektorraums}

\begin{beispiel}
Wir betrachten den Untervektorraum 
	$$ V = \left\{ \begin{pmatrix} 2r + 3s \\ r-s \\ r+s \end{pmatrix} \,\vert \,\,r, s \in \R \right\} \subseteq \R^3 $$ \pause
Dann bilden 
	$$\vektor{v_1} = \left( \begin{smallmatrix} 2 \\ 1 \\ 1 \end{smallmatrix} \right), \quad 
	\vektor{v_2} = \left( \begin{smallmatrix} 3 \\ -1 \\  1 \end{smallmatrix} \right) $$ 
eine Basis von $V$. \pause 
	$$\vektor{w_1} = \left( \begin{smallmatrix} 5 \\ 0 \\ 2 \end{smallmatrix} \right), \quad 
	\vektor{w_2} = \left( \begin{smallmatrix} 1 \\ -2 \\  0 \end{smallmatrix} \right), \quad 
	\vektor{w_3} = \left( \begin{smallmatrix} 7 \\ 1 \\  3 \end{smallmatrix} \right)  $$ 
erzeugen $V$, bilden aber keine Basis.  
\end{beispiel}
\end{frame}

\begin{frame}
\frametitle{Basis eines Vektorraums}

\begin{uebung}
Finden Sie eine Basis des Untervektorraums 
	$$ V =  \left\{ \begin{pmatrix} x \\ y \\ z \end{pmatrix} \,\vert \,\, x + y + z = 0 \right\} \subseteq \R^3 $$
\end{uebung}

\pause \pause 
\begin{sol}
Die Vektoren 
	$$\vektor{v_1} = \left( \begin{matrix} -1 \\ 1 \\  0 \end{matrix} \right), \quad 
	\vektor{v_2} = \left( \begin{matrix} -1 \\ 0 \\  1 \end{matrix} \right) $$ 
bilden eine Basis von $V$.
\end{sol}
\end{frame}

\begin{frame}
\frametitle{Basis eines Vektorraums}

\begin{satz}
Für einen Untervektorraume $V$ von $\mathbb R^n$ gilt:

\begin{enumerate}
\item<2-> Ist $\{\vektor{v_1}, \vektor{v_2}, \ldots, \vektor{v_m} \}$ ein Erzeugendensystem 
von $V$, so enthält $\{ \vektor{v_1}$, $\vektor{v_2}$, $\ldots$, $\vektor{v_m} \}$ eine 
Basis von $V$, d.h. es gibt $1 \leq i_1 < i_2 < \cdots < i_t \leq m$ so dass
$ \vektor{v_{i_1}}, \vektor{v_{i_2}}, \ldots, \vektor{v_{i_t}}$ eine Basis von $V$ ist.
\item<3-> $V$ hat eine Basis.
\item<4-> Je zwei Basen von $V$ sind gleich lang, dh. sind $\{\vektor{v_1}, \vektor{v_2}, \ldots, \vektor{v_m} \}$ 
und $\{\vektor{w_1}, \vektor{w_2}, \ldots, \vektor{w_t} \}$ Basen von $V$, so ist $m = t$.
\end{enumerate}
\end{satz}

\end{frame}

\begin{frame}
\frametitle{Basis eines Vektorraums}

\begin{definition} Die Länge einer Basis eines Untervektorraums $V$ heißt \index{Untervektorraum!Dimension}
die \textbf{Dimension} von $V$ und wird mit $\textrm{dim}(V)$ bezeichnet.
\end{definition}

\pause 

\begin{beispiel} Der $\mathbb R^3$ hat die Dimension $3$.
\end{beispiel}

\pause 

\begin{beispiel} Der $\mathbb R^n$ hat die Dimension $n$.
\end{beispiel}


\end{frame}

\begin{frame}
\frametitle{Basis eines Vektorraums}

\begin{beispiel}
Der Untervektorraum 
	$$ V = \left\{ \begin{pmatrix} 2r + 3s \\ r-s \\ r+s \end{pmatrix} \,\vert \,\,r, s \in \R \right\} \subseteq \R^3 $$
hat die Dimension $2$.
\end{beispiel}

\pause 

\begin{beispiel}
Der Untervektorraums 
	$$ V =  \left\{ \begin{pmatrix} x \\ y \\ z \end{pmatrix} \,\vert \,\, x + y + z = 0 \right\} \subseteq \R^3 $$
hat die Dimension $2$. 
\end{beispiel}

\end{frame}

\end{document}
