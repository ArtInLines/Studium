\documentclass[hyperref={pdfpagelabels=false}]{beamer}
\providecommand\thispdfpagelabel[1]{}
% Die Hyperref Option hyperref={pdfpagelabels=false} verhindert die Warnung:
% Package hyperref Warning: Option `pdfpagelabels' is turned off
% (hyperref)                because \thepage is undefined. 
% Hyperref stopped early 
%

\usepackage{lmodern}
% Das Paket lmodern erspart die folgenden Warnungen:
% LaTeX Font Warning: Font shape `OT1/cmss/m/n' in size <4> not available
% (Font)              size <5> substituted on input line 22.
% LaTeX Font Warning: Size substitutions with differences
% (Font)              up to 1.0pt have occurred.
%
\usepackage{epstopdf}
\usepackage{german}
\usepackage{amssymb}
\usepackage{amsfonts}
\usepackage{amsmath}
\usepackage{amsthm}
\usepackage{latexsym}
\usepackage{newalg}
\usepackage{ifthen}
\usepackage{fancybox}
\usepackage{fancyhdr}
%\usepackage[dvips, dvipdfm]{graphicx}
%\usepackage[pdftex]{color,graphicx}
\usepackage{eurofont}
\usepackage{aeguill}
\usepackage[T1]{fontenc}
\usepackage{textcomp}
\usepackage{wrapfig}
\usepackage{mathrsfs}
\usepackage{float}
\usepackage{color}
\usepackage{siunitx}

\usepackage[utf8]{inputenc}
%\usepackage{hyperref}

\theoremstyle{plain}% default
\newtheorem*{satz}{Satz} 
\newtheorem*{hilf}{Lemma} 
\newtheorem*{korollar}{Folgerung}
\newtheorem*{regel}{Regel} 

\theoremstyle{definition}
%\newtheorem*{definition}{Definition} 
\newtheorem*{aufgabe}{Aufgabe} 
\newtheorem*{uebung}{Übung}
\newtheorem*{sol}{Lösung:}

\theoremstyle{remark}
\newtheorem*{beispiel}{Beispiel} 
\newtheorem*{notiz}{Bemerkung} 
\newtheorem*{notation}{Notation}
\newtheorem*{bez}{Bezeichnung} 

\def \ii{{\relax\ifmmode \mathrm{i} \else i \fi}}
\def \ee{{\relax\ifmmode \mathrm{e} \else e \fi}}
\def \dd{{\relax\ifmmode \mathrm{d} \else d \fi}}
\def \g{{\relax\ifmmode \mathrm{g} \else g \fi}}
\def \R{\mathbb R}
\def \C{\mathbb C}
\def \Q{\mathbb Q}
\def \N{\mathbb N}
\def \Z{\mathbb Z}
\def \F{\mathbb F}
\def \A{\mathbb A}

% tikz Anpassungen 

\usepackage{lscape}
\usepackage{gnuplot-lua-tikz}
\usepackage{pgf,tikz}
\usetikzlibrary{arrows,decorations.pathmorphing,backgrounds,positioning,fit,petri}

\usepackage{pgfplots}
\usepackage{pgfplotstable}
\pgfplotsset{compat=newest}

\usepackage{environ}
\makeatletter

\newsavebox{\measure@tikzpicture}
\NewEnviron{scaletikzpicturetowidth}[1]{%
  \def\tikz@width{#1}%
  \def\tikzscale{1}\begin{lrbox}{\measure@tikzpicture}%
  \BODY
  \end{lrbox}%
  \pgfmathparse{#1/\wd\measure@tikzpicture}%
  \edef\tikzscale{\pgfmathresult}%
  \BODY
}
\makeatother

\setcounter{secnumdepth}{-1}


% Wenn \titel{\ldots} \author{\ldots} erst nach \begin{document} kommen,
% kommt folgende Warnung:
% Package hyperref Warning: Option `pdfauthor' has already been used,
% (hyperref) ... 
% Daher steht es hier vor \begin{document}

\title{Lineare Algebra \\ algebraische Strukturen 3}   
\author{ Reinhold Hübl} 
\date{Wintersemester 2020/21} 

% Dadurch wird verhindert, dass die Navigationsleiste angezeigt wird.
%\setbeamertemplate{navigation symbols}{}

% zusaetzlich ist das usepackage{beamerthemeshadow} eingebunden 
\usepackage{beamerthemeshadow}
\usetheme{CambridgeUS}

%  \beamersetuncovermixins{\opaqueness<1>{25}}{\opaqueness<2->{15}}
%  sorgt dafuer das die Elemente die erst noch (zukuenftig) kommen 
%  nur schwach angedeutet erscheinen 
\beamersetuncovermixins{\opaqueness<1>{25}}{\opaqueness<2->{15}}
% klappt auch bei Tabellen, wenn teTeX verwendet wird\ldots
\begin{document}


\begin{frame}
\titlepage
\centering 
    \scalebox{0.40}{\includegraphics{dhbw_ma_logo.eps}}
 
\end{frame} 



\section{Ganze Zahlen und ihre Restklassen}


\begin{frame}
\frametitle{Teiler und Teilbarkeit}

Wir betrachten positive ganze Zahlen $k$ und $n$. 

\begin{definition}  $k$ heißt Teiler von $n$, wenn es eine ganze Zahl $l$ gibt mit $n = k \cdot l$. 
\end{definition}

\pause 
\begin{beispiel} Die Zahl $2$ ist ein Teiler von $36$. \end{beispiel} 
\pause 
\begin{beispiel}
Für jede positive ganze Zahl $n$ sind $1$ und $n$ Teiler von $n$. Diese Teiler heißen die \textbf{trifivialen Teiler} 
von $n$. Alle anderen Teiler heißen \textbf{echte Teiler} von $n$. 
\end{beispiel}


\end{frame}

\begin{frame}
\frametitle{Primzahlen}

\begin{definition} Eine Zahl $p \in \mathbb N$, $p \geq 2$ heißt \textbf{Primzahl}, 
wenn $p$ keine echten Teiler hat.
\end{definition}

\pause 

\begin{beispiel} $2, 3, 5, 7$ und $11$ sind Primzahlen, $9$ ist keine.
\end{beispiel}

\pause 

\begin{notiz} Eine Zahl $p \in \mathbb N \setminus \{0, 1\}$ ist genau dann eine Primzahl, wenn 
für alle ganzen Zahlen $a, b \in \mathbb Z$ gilt
  	$$ p \vert \, a \cdot b \iff p \vert a  \,\, \textrm{ oder } \, p \vert b $$
also $p$ teilt genau dann ein Produkt von zwei Zahlen, wenn es schon eine der beiden Zahlen teilt.
\end{notiz}

\end{frame}

\begin{frame}
\frametitle{Primzahlen}

\begin{notiz} Jede ganze Zahle $z \in \mathbb Z \setminus \{ 0 \}$ schreibt sich in eindeutiger Weise als
  	$$ z = \varepsilon \cdot p_1^{n_1} \cdot p_2^{n_2} \cdots p_t^{n_t} $$
mit $\varepsilon \in \{-1,1\}$, Primzahlen $p_1 < p_2 < \ldots < p_t$ und positiven natürlichen Zahlen 
$n_1, \ldots , n_t$. (Dabei lassen wir den Spezialfall $t = 0$ für $z = \pm 1$ zu).
\end{notiz}

\pause 

\begin{beispiel}
Die Zahl $36$ hat die Primfaktorzerlegung
	$$ 36 = 2^2 \cdot 3^2 $$ 
Der Vorfaktor $\varepsilon = +1$ wird dabei in der Regel weggelassen. \pause 

Die Zahl $-6615$ hat die Primfaktorzerlegung 
	$$ -6615 = (-1) \cdot 3^3 \cdot 5 \cdot 2^7 $$
\end{beispiel}

\end{frame}

\begin{frame}
\frametitle{Teiler}

\begin{definition} Sind $m,n \in \mathbb N \setminus \{0 \}$, so heißt eine Zahl $g \in \mathbb N$ der 
\textbf{größte gemeinsame Teiler} von $m$ und $n$, wenn gilt:
\begin{itemize}
\item<2-> $g$ ist ein Teiler von $m$ und ein Teiler von $n$.
\item<3-> Ist $h$ ein weiterer Teiler von $m$ und $n$, so ist $h$ auch ein Teiler von $g$.
\end{itemize}

\pause \pause \pause 
Wir schreiben 
	$$ \mathrm{ggT}(m,n) := g $$ 
für den größten gemeinsame Teiler von $m$ und $n$.

\pause 
Zwei Zahlen $m,n \in \mathbb N \setminus \{0 \}$ heißen \index{teilerfremd}\textbf{teilerfremd}, wenn 
$\textrm{ggT}(m,n) = 1$, wenn sie also keinen echten gemeinsamen Teiler besitzen.
\end{definition}
\end{frame}

\begin{frame}
\frametitle{Teiler}

\begin{uebung}
Bestimmen Sie den größten gemeinsamen Teiler von $n = 5040$ und $m = 15288$.
\end{uebung}

\pause \pause 

\bigbreak

\begin{sol}
	$$ \mathrm{ggT}(m,n) = 2^3 \cdot 3 \cdot 7 = 168 $$
% $n=2^4 \cdot 3^2 \cdot 5 \cdot 7, \quad m = 2^3 \cdot 3 \cdot 7^2 \cdot 13
\end{sol}

\pause 

Die Bestimmung der Primfaktorzerlegung großer Zahlen ist ein sehr schwieriges Problem, 
für das kein effizienter Algorithmus bekannt ist. Daher ist diese Methode zur Bestimmung 
des größten gemeinsamen Teilers ineffizient.

\end{frame}

\section{euklidischer Algorithmus}

\begin{frame}
\frametitle{euklidischer Algorithmus}

\begin{itemize} 
\item \textbf{Vorbereitungsschritt:\,} Ordne $m$ und $n$ so, dass $m \geq n$. (Vertausche  
$m$ und $n$, falls n\"otig, denn $\textrm{ggT}(m,n) = \textrm{ggT}(n,m)$). Setzen $i = 0$ und 
$r_0 = m$, $r_1 = n$.
\item<2-> \textbf{Verarbeitungsschritt:\,} Wir dividieren $r_i$ durch $r_{i+1}$ mit Rest:
  $$ r_i = a \cdot r_{i+1} + b $$
mit einer nat\"urlichen Zahl $a$ und einem Rest $b \in \{0, 1, \ldots, r_{i+1} - 1 \}$.
\begin{itemize}
\item[-] Falls $b = 0$ (d.h. die Division geht ohne Rest auf) $\, \longrightarrow \textbf{ STOPP}$.
\item[-] Falls $b \neq 0$ setze $r_{i+2} = b$ und $i = i+1$. Wiederhole den Verarbeitungsschritt.
\end{itemize}
\item<3-> \textbf{Ergebnisschritt:} Nach endlich vielen Verarbeitungsschritten (h\"ochstens $m$ vielen) geht die 
Division erstmals ohne Rest auf, d.h.$ r_i = a \cdot r_{i+1} + 0 $
mit $r_{i+1} \neq 0$, Das STOPP--Kriterium wird also immer erreicht und $r_{i+1}$ ist der gr\"o{\ss}te gemeinsame 
Teiler von $m$ und $n$, $r_{i+1} = \textrm{ggT}(m,n)$.
\end{itemize}


\end{frame}

\begin{frame}
\frametitle{euklidischer Algorithmus}

\begin{beispiel}

Wir betrachten die Zahlen $m = 222$ und $n = 156$. Hier gilt bereits $m \geq n$, und wir setzen
$r_0 = 222$ und $r_1 = 156$.

\begin{itemize}
\item<2-> $i = 0$: $\quad 222 = 1 \cdot 156 + 66$. Wir setzen $r_2 = 66$.
\item<3-> $i = 1$: $\quad 156 = 2 \cdot 66 + 24$. Wir setzen $r_3 = 24$.
\item<4-> $i = 2$: $\quad 66 = 2 \cdot 24 + 18$. Wir setzen $r_4 = 18$.
\item<5-> $i = 3$: $\quad 24 = 1 \cdot 18 + 6$. Wir setzen $r_5 = 6$.
\item<6-> $i = 4$: $\quad 18 = 3 \cdot 6 + 0$. $\, \longrightarrow \textbf{ STOPP}$.
\end{itemize}

\bigbreak

\pause \pause \pause \pause \pause \pause 
\textbf{Ergebnis:} $\mathrm{ggT}(222, 156) = 6$.

\end{beispiel}
\end{frame}

\begin{frame}
\frametitle{euklidischer Algorithmus}

\begin{uebung}
Bestimmen Sie den größten gemeinsamen Teiler von $m = 239$ und $n=144$. 
\end{uebung}

\bigbreak
\pause \pause 

\begin{sol}
$\mathrm{ggT}(239, 144) = 1$.
\end{sol}

\end{frame}

\begin{frame}
\frametitle{euklidischer Algorithmus}

\begin{satz} Sind $m,n\in \mathbb N \setminus \{0 \}$ mit 
$\mathrm{ggT}(m,n) = g$, so gibt es ganze Zahlen $a$, $b$ mit 
  	$$ a \cdot m + b \cdot n = g $$
\end{satz}

\bigbreak

\pause
Das erhält man durch Rückwärtsrechnen aus dem euklidischen Algorithmus.


\end{frame}

\begin{frame}
\frametitle{euklidischer Algorithmus}

\begin{beispiel}
Wir wollen $6$ mit $156$ und $222$ darstellen.

\begin{enumerate}
\item<2-> Aus Schritt $i=3$ erhalte: $6 = 24 - 1 \cdot 18$.
\item<3-> Aus Schritt $i=2$ erhaltet $18 = 66 - 2 \cdot 24$.
Eingesetzt in  (1): 
  	$$ 6 = 24 - 1 \cdot  \left(66 - 2 \cdot 24\right) =3 \cdot 24 - 1 \cdot 66 $$
\item<4-> Aus Schritt $ i = 1$ erhalte zunächst $24 = 156 - 2 \cdot 66$.
Eingesetzt in  (2): 
  	$$ 6 = 3 \cdot \left( 156 - 2 \cdot 66 \right) - 1 \cdot 66 = 3 \cdot 156 - 7 \cdot 66 $$
\item<5-> aus Schritt $i = 0$ erhalte zunächst $66 = 222 - 1 \cdot 156$.
Eingesetzt in  (3): 
  	$$ 6 =  3 \cdot 156 - 7 \cdot \left( 222 - 1 \cdot 156 \right) = 10 \cdot 156 - 7 \cdot 222 $$
\end{enumerate}
\vspace{-0.6cm}
\pause \pause \pause \pause \pause %\pause 
Damit haben wir eine gewünschte Darstellung $ 6 = 10 \cdot 156 + (- 7) \cdot 222$.
\end{beispiel}

\end{frame}

\begin{frame}
\frametitle{euklidischer Algorithmus}

\begin{uebung}
Schreiben Sie $1 = \mathrm{ggT}(239, 144)$ in der Form 
	$$ 1 = a \cdot 239 + b \cdot 144 $$
mit ganzen Zahlen $a$ und $b$.
\end{uebung}

\bigbreak

\pause \pause 

\begin{sol}
Es gilt 
	$$ 1 = 47 \cdot 239 + (-78) \cdot 144 $$
\end{sol} 

\end{frame}

\section{Restklassenringe}
\begin{frame}
\frametitle{Die Ringe $\Z/n\Z$}

Die Ringe $\Z/n\Z$ haben wir schon kennengelernt. \pause Zur Vereinfachung der Notation identifizieren wir 
	$$ \Z / n\Z = \{ 0, 1, \ldots, n-1\} $$
(und schreiben also nicht mehr $[k]$ für die Äquivalenzklasse von $k$ sondern einfach $k$). 

\pause 

Dann gilt in $\Z/n\Z$: 

\begin{itemize} 
\item<3-> Ist $k + l = q \cdot n + r$, so gilt in $\Z/n \Z$: 
	$$ k+l = r $$
\item<4->  Ist $k \cdot l = q \cdot n + r$, so gilt in $\Z/n \Z$: 
	$$ k \cdot l = r $$
\end{itemize}
\end{frame}

\begin{frame}
\frametitle{Die Ringe $\Z/n\Z$}

\begin{beispiel}
Für den $R = \Z/31439 \cdot \Z$ gilt 

\begin{itemize}
\item<2-> $11812 + 7403 = 19215 = 0 \cdot 31439 + 19215$. Also gilt in $R$: 
	$$ 11812 + 7403 = 19215 $$
\item<3-> $11812 + 27403 = 39215 = 1 \cdot 31439 + 7786$. Also gilt in $R$: 
	$$ 11812 + 27403 = 7786 $$
\item<4-> $11812 \cdot 7403 = \num{87444236} = 2781 \cdot 31439 + 12377$. Also gilt in $R$: 
	$$ 11812 \cdot 7403 = 12377 $$
\item<5-> $117879 \cdot 10579 = \num{198663041} = 6319 \cdot 31439 + 0$. Also gilt in $R$: 
	$$ 11879 \cdot 10579 = 0 $$
\end{itemize}

%\pause \pause \pause \pause \pause
%Die letzte Multiplikation zeigt insbesondere auch, dass $\Z/31439 \cdot \Z$ kein Körper ist. 
\end{beispiel}

\end{frame}

\begin{frame}
\frametitle{Die Ringe $\Z/n\Z$}

\begin{uebung}
Berechnen Sie in $\Z/9379$ die Zahl
	$$ z = 1315 \cdot 6439 $$
\end{uebung}

\bigbreak

\pause \pause 

\begin{sol}
Es ist 
	$$  1315 \cdot 6439 = 902 \cdot 9379 + 7427 $$
und daher 
	$$ z = 7427 $$
\end{sol}
\end{frame}

\begin{frame}
\frametitle{Die Ringe $\Z/n\Z$}

\begin{satz} Genau dann ist $\Z/n \Z$ ein Körper, wenn $n$ eine Primzahl ist. 
\end{satz}

\bigbreak
\pause
\begin{bez} Ist $p$ eine Primzahl, so bezeichnen wir $\Z/p\Z$ mit $\F_p$ und nennen 
$\F_p$ den Körper mit $p$ Elementen
\end{bez}

\pause
\begin{beispiel} Der Ring $\Z/17 \Z$ ist ein Körper. \end{beispiel}

\pause
\begin{beispiel} Der Ring $\Z/15 \Z$ ist kein Körper. \end{beispiel}
\end{frame}

\begin{frame}
\frametitle{Die Körper $\F_p$}

Wir betrachten nun ein Primzahl $p$. Bei der Division in $\F_p$ hilft der euklidische Algorithmus: 

Ist $l \in \F_p \setminus \{0 \} = \{1, 2, \ldots, p-1\}$, so sind $l$ und $p$ teilerfremd, dh. 
	$$ 1 = \mathrm{ggT}(l,p) $$ \pause 
Berechnen wir nun $\mathrm{ggT}(l,p) $ mit den euklidischen Algorithmus und führen eine Rückwärtsrechnung durch, 
so erhalten wir daraus ein Darstelleung 
	$$ 1 = a \cdot l + b \cdot p $$ \pause 
Betrachten wir dann die Restklassen modulo $p$, so erhalten wir 
	$$ 1 = (a \cdot l + b \cdot p) \,\text{ mod } p = a \cdot l \, \text{ mod } p $$
(denn $b \cdot p = 0 \, \text{ mod } p$). Also gilt 
	$$ \frac {1}{l} = a \qquad \text{ in } \,\, \F_p $$   
\end{frame}

\begin{frame}
\frametitle{Die Körper $\F_p$}

\begin{beispiel}
Die Zahl $p = 239$ ist eine Primzahl. Wir haben schon gesehen, dass $\mathrm{ggT}(239,144) = 1$ und dass
	$$ 1 = 47 \cdot 239 + (-78) \cdot 144 $$ \pause 
Damit gilt modulo $p$
 	$$ 1 = (-78) \cdot 144 = (239-78) \cdot 144 = 161 \cdot 144 $$ \pause 
Also gilt in $\F_{239}$:
	$$ \frac {1}{144} = 161 $$ \pause
Daraus folgt
	$$ \frac {207}{144} = 207 \cdot \frac {1}{144} = 207 \cdot 161 = 106 $$
\end{beispiel}
\end{frame}

\begin{frame}
\frametitle{Die Körper $\F_p$}

\begin{uebung}
Berechnen Sie $\frac {13}{74}$ im Körper $\F_{83}$. 
\end{uebung}

\bigbreak 

\pause \pause 

\begin{sol}
Es ist 
	$$ 1 = 33 \cdot 83 + (-37) \cdot 74 $$
also ist 
	$$ \frac {1}{74} = -37 = 46 $$
und damit 
	$$ \frac {13}{74} = 13 \cdot 46 = 17 $$
\end{sol}
\end{frame}

\begin{frame}
\frametitle{Die Körper $\F_p$}

Für eine Primzahl $p$ bezeichnen wir mit $E(\F_p) = \F_p \setminus \{0 \}$ die 
\textbf{Einheitengruppe} von $\F_p$.

\pause 

\begin{satz}
Die Gruppe $E(\F_p)$ ist zyklisch von der Ordnung $p-1$, dh. es gibt ein Element $g \in \F_p \setminus \{0 \}$ mit 
	$$ \F_p \setminus \{0 \} = \langle g \rangle =  \{g,g^2, \ldots, g^{p-2}, g^{p-1} = 1 \} $$ 
\end{satz}

\pause 

\begin{notiz}
Beachten Sie, dass $E(\F_{p-1})$ die Ordnung $p-1$ hat, dass also für alle $a \in E(\F_p)$ gilt 
	$$ a^{p-1} = 1 $$
Damit gilt auch 
	$$ a^p = a $$
\end{notiz}
\end{frame}


\begin{frame}
\frametitle{Die Körper $\F_p$}

\begin{beispiel}
Im Körper $\F_{17}$ wird die Einheitengruppe von dem Element $3$ erzeugt. Das Element $5$ ist 
ebenfalls ein Erzeuger der Einheitengruppe, nicht aber das Element $4$, denn 
	$$ \langle 4 \rangle = \{ 4, 16, 13, 1\} $$
\end{beispiel}

\pause 

\begin{notiz}
Gilt für ein $g \in E(\F_p)$ für alle echten Teiler $n$ von $p-1$: 
	$$ g^n \neq 1 $$
so erzeugt $g$ die Gruppe $E(\F_p)$.
\end{notiz}

\end{frame}

\begin{frame}
\frametitle{Die Körper $\F_p$}

\begin{uebung}
Bestimmen Sie einen Erzeuger der Einheitengruppe von $\F_{19}$.
\end{uebung}

\bigbreak

\pause \pause 

\begin{sol}
Das Element $2$ ist ein Erzeuger von $E(\F_{19})$. 
\end{sol}
\end{frame}

\begin{frame}
\frametitle{Ausblick - Verschlüsselung} 

Rechnen mit ganzen Zahlen und Restklassen und der euklidische Algorithmus spielen 
eine wichtige Rolle in vielen Verfahren der sogenannenten \textbf{asymmetrischen Verschlüsselung}. 

\pause 

Bei diesen asymmetrischen Verfahren kommen zwei Schlüssel zur Anwendung,  ein öffentlicher 
Schlüssel (\textit{public key}), der zum Verschlüsseln benutzt wird, 
und ein geheimer Schlüssel (\textit{private key}), der zum Entschlüsseln verwendet wird.

\pause 

Der öffentliche Schlüssel ist allgemein bekannt. Die Sicherheit dieser Verfahren beruht also darauf, dass der 
private Schlüssel aus dem öffentlichen Schlüssel nicht berechnet werden kann.  

\pause 

\bigbreak

Eines der bekanntesten asymmetrischen Verfahren ist das \textbf{RSA--Verfahren}.

\end{frame}

\begin{frame}
\frametitle{Das RSA--Verfahren}

Der öffentliche Schlüssel des RSA--Verfahrens besteht aus einem Paar $(e,N)$ positiver ganzer Zahlen, 
der private Schlüssel besteht ebenfalls aus einem Paar $(d,N)$ positiver ganzer Zahlen, wobei diese 
Zahlen wie folgt zu konstruieren sind: 

\pause

\begin{itemize}
\item<2-> Wähle zwei große Primzahlen $p$ und $q$ mit $p \neq q$ und setze $N = p \cdot q$. 
\item<3-> Wähle eine zu $(p-1) \cdot (q-1)$ teilerfremde Zahl $e$. 
\item<4-> Bestimme $d$ mit $0 < d < N$ so, dass $d \cdot e \equiv 1 \, \text{ mod } (p-1) \cdot (q-1)$. 
\end{itemize}

\pause \pause \pause 
Dann kann $(e,N)$ als öffentlicher Schlüssel und $(d,N)$ als privater Schlüssel benutzt werden.  

\end{frame} 

\begin{frame}
\frametitle{Das RSA--Verfahren}

Die Wirkungsweise des RSA--Verfahrens beruht auf folgender Aussage (einer Folgerung aus der Aussage, 
dass die Ordnung eines Elements immer die Gruppenordnung teilt): 

\begin{satz} Für jede ganze Zahl $m$ mit $1 \leq m \leq N$ gilt 
	$$ \left(m^e\right)^d \equiv m \, \text{ mod } N $$
\end{satz}

\end{frame} 

\begin{frame}
\frametitle{Das RSA--Verfahren}

Damit funktioniert das RSA--Verfahren wie folgt: 

Alice will Bob eine Nachricht schicken, die durch eine Zahl $m$ mit $1 < m < N$ dargestellt wird. 

\begin{itemize}
\item<2-> Alice benutzt den öffentlichen Schlüssel und berechnet $b = m^e \, \text{ mod } N$. 
\item<3-> Alice schickt  $b$ über einen öffentlichen Kanal an Bob. 
\item<4-> Bob benutzt seinen privaten Schlüssel und berechnet $c = b^d \, \text{ mod } N$. 
\item<5-> Wegen $b \equiv m \, \text{ mod } N$ hat Bob die Nachricht entschlüsselt. 
\end{itemize}

\end{frame} 

\begin{frame}
\frametitle{Das RSA--Verfahren}

\begin{beispiel}

Wir wählen die Primzahlen $p=137$ und $q = 89$ und erhalten 
	$$ N = p \cdot q = \num{12193}$$ 
Ferner ist 
	$$ (p-1) \cdot (q-1) = 136 \cdot 88 = \num{11968} $$
Ferner wählen wir $e = 97$. Dann ist $e$ teilerfremd  zu $\num{11968}$ und für $d = 9377$ gilt 
	$$ d \cdot e \equiv 1 \quad \text{ mod } \, \num{11968} $$
Das Paar $(97, \num{12193})$ ist unser öffentlicher Schlüssel, das Paar $(\num{9377}, \num{12193})$ ist unser 
privater Schlüssel.
\end{beispiel}

\end{frame} 

\begin{frame}
\frametitle{Das RSA--Verfahren}

\begin{beispiel} 
Alice will Bob eine Nachricht $m$ senden, die durch die Zahl $m = 7842$ repäsentiert wird. 

\begin{itemize}
\item<2-> Alice berechnet $b = 7842^{97} \quad \text{ mod } 12193$. 
\item<3-> Sie erhält $b = 2853$ und schickt $b = 2853$ an Bob. 
\item<4-> Bob berechnet $m = 2853^{9377} = 7842 \quad \text{ mod } 12193$. 
\item<5-> Bob hat die Nachricht korrekt entschlüsselt. 
\end{itemize}

\end{beispiel}

\pause \pause \pause \pause \pause

\bigbreak

Die Sicherheit des Verfahrens beruht darauf, dass zwar Bob die Zahl $d$ über Euklid berechnen kann (da 
er $p$ und $q$ kennt), nicht jedoch ein Angreifer.  Es gibt nämlich kein bekanntes effizientes Verfahren, dass 
aus einer (großen) Zahl $N$ ihre Primfaktoren ermittelt. 

\end{frame}

\end{document}
