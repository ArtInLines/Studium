\documentclass[hyperref={pdfpagelabels=false}]{beamer}
\providecommand\thispdfpagelabel[1]{}
% Die Hyperref Option hyperref={pdfpagelabels=false} verhindert die Warnung:
% Package hyperref Warning: Option `pdfpagelabels' is turned off
% (hyperref)                because \thepage is undefined. 
% Hyperref stopped early 
%

\usepackage{lmodern}
% Das Paket lmodern erspart die folgenden Warnungen:
% LaTeX Font Warning: Font shape `OT1/cmss/m/n' in size <4> not available
% (Font)              size <5> substituted on input line 22.
% LaTeX Font Warning: Size substitutions with differences
% (Font)              up to 1.0pt have occurred.
%
\usepackage{epstopdf}
\usepackage{german}
\usepackage{amssymb}
\usepackage{amsfonts}
\usepackage{amsmath}
\usepackage{amsthm}
\usepackage{latexsym}
\usepackage{newalg}
\usepackage{ifthen}
\usepackage{fancybox}
\usepackage{fancyhdr}
%\usepackage[dvips, dvipdfm]{graphicx}
%\usepackage[pdftex]{color,graphicx}
\usepackage{eurofont}
\usepackage{aeguill}
\usepackage[T1]{fontenc}
\usepackage{textcomp}
\usepackage{wrapfig}
\usepackage{mathrsfs}
\usepackage{float}
\usepackage{color}
\usepackage{siunitx}

\usepackage[utf8]{inputenc}
%\usepackage{hyperref}

\theoremstyle{plain}% default
\newtheorem*{satz}{Satz} 
\newtheorem*{hilf}{Lemma} 
\newtheorem*{korollar}{Folgerung}
\newtheorem*{regel}{Regel} 

\theoremstyle{definition}
%\newtheorem*{definition}{Definition} 
\newtheorem*{aufgabe}{Aufgabe} 
\newtheorem*{uebung}{Übung}
\newtheorem*{sol}{Lösung:}

\theoremstyle{remark}
\newtheorem*{beispiel}{Beispiel} 
\newtheorem*{notiz}{Bemerkung} 
\newtheorem*{notation}{Notation}
%\newtheorem*{problem}{Problem} 

\def \ii{{\relax\ifmmode \mathrm{i} \else i \fi}}
\def \ee{{\relax\ifmmode \mathrm{e} \else e \fi}}
\def \dd{{\relax\ifmmode \mathrm{d} \else d \fi}}
\def \g{{\relax\ifmmode \mathrm{g} \else g \fi}}
\def \R{\mathbb R}
\def \C{\mathbb C}
\def \Q{\mathbb Q}
\def \N{\mathbb N}
\def \Z{\mathbb Z}
\def \F{\mathbb F}
\def \A{\mathbb A}

% tikz Anpassungen 

\usepackage{lscape}
\usepackage{gnuplot-lua-tikz}
\usepackage{pgf,tikz}
\usetikzlibrary{arrows,decorations.pathmorphing,backgrounds,positioning,fit,petri}

\usepackage{pgfplots}
\usepackage{pgfplotstable}
\pgfplotsset{compat=newest}

\usepackage{environ}
\makeatletter

\newsavebox{\measure@tikzpicture}
\NewEnviron{scaletikzpicturetowidth}[1]{%
  \def\tikz@width{#1}%
  \def\tikzscale{1}\begin{lrbox}{\measure@tikzpicture}%
  \BODY
  \end{lrbox}%
  \pgfmathparse{#1/\wd\measure@tikzpicture}%
  \edef\tikzscale{\pgfmathresult}%
  \BODY
}
\makeatother

\setcounter{secnumdepth}{-1}


% Wenn \titel{\ldots} \author{\ldots} erst nach \begin{document} kommen,
% kommt folgende Warnung:
% Package hyperref Warning: Option `pdfauthor' has already been used,
% (hyperref) ... 
% Daher steht es hier vor \begin{document}

\title{Lineare Algebra \\ Grundlagen der Logik}   
\author{ Reinhold Hübl} 
\date{Wintersemester 2020/21} 

% Dadurch wird verhindert, dass die Navigationsleiste angezeigt wird.
%\setbeamertemplate{navigation symbols}{}

% zusaetzlich ist das usepackage{beamerthemeshadow} eingebunden 
\usepackage{beamerthemeshadow}
\usetheme{CambridgeUS}

%  \beamersetuncovermixins{\opaqueness<1>{25}}{\opaqueness<2->{15}}
%  sorgt dafuer das die Elemente die erst noch (zukuenftig) kommen 
%  nur schwach angedeutet erscheinen 
\beamersetuncovermixins{\opaqueness<1>{25}}{\opaqueness<2->{15}}
% klappt auch bei Tabellen, wenn teTeX verwendet wird\ldots
\begin{document}


\begin{frame}
\titlepage
\centering 
    \scalebox{0.40}{\includegraphics{dhbw_ma_logo.eps}}
 
\end{frame} 

%\begin{frame}
%\frametitle{Inhaltsverzeichnis}
%\tableofcontents
%\end{frame} 



\section{Aussagenlogik}

\begin{frame}
\frametitle{Aussagen}
Die \textit{Logik} ist die \textit{Lehre vom Argumentieren} oder der \textit{Lehre vom Schlussfolgern} und 
beschäftigt sich damit, Sicherheit in die Regeln das Schließens und des Argumentierens zu bringen.

\pause 

\begin{definition} Eine \textbf{Aussage} $A$ ist ein Satz der (in einem gegebenen Kontext) 
eindeutig entweder wahr \textit{w} oder falsch \textit{f} ist.
\end{definition}

\pause

\begin{beispiel}
Beispiele für Aussagen sind etwa:
\begin{itemize}
\item $A_1:$ Berlin ist die Hauptstadt von Österreich.
\item $A_2:$ 3 ist größer als 7
\item $A_3:$ 5 ist eine Quadratzahl (in $\mathbb R$)
\end{itemize}
\end{beispiel}


\end{frame}

\begin{frame}
\frametitle{Negation}

Aussagen können mit \textbf{Junktoren} miteinander verknüpft werden, wodurch neue Aussagen entstehen, deren 
Wahrheitswert formal aus dem der eingehenden Aussagen ermittelt werden kann.

\pause

Die Negation einer Aussage $A$ ist eine Aussage, die mit $\neg A$ bezeichnet wird. Der Wahrheitswert von $\neg A$ 
kann aus dem von $A$ durch folgende Tabelle ermittelt werden: 

\pause 

	$$ \begin{array}{ r | | l }
	A &  \neg A \\
	\hline \hline
	w &  f \\
	f &  w \\
	\end{array} $$

\end{frame}

\begin{frame}
\frametitle{Konjunktion , Disjunktion und Kontravalenz}

Die \textbf{Konjunktion} oder UND--Verknüpfung von zwei Aussagen $A$ und $B$ ist eine Aussage, die mit 
$A \wedge B$ bezeichnet wird. 

\pause 

Die \textbf{Disjunktion} oder ODER--Verknüpfung von zwei Aussagen $A$ und $B$ ist eine Aussage, die 
mit $A \vee B$ bezeichnet wird. 

\pause

Die \textbf{Kontravalenz} oder ENTWEDER--ODER--Verknüpfung  von zwei Aussagen $A$ und $B$ ist eine Aussage, 
die  mit $A \oplus B$ bezeichnet wird. 

\pause 

Konjunktion, Disjunktion und Kontravalenz werden durch folgende Wahrheitstafel beschreiben:

	$$ \begin{array}{ r | c || c | c | c }
	A & B & A \wedge B & A \vee B & A \oplus B\\
	\hline \hline
	w & w &  w & w & f\\
	w & f &  f & w & w \\
	f & w &  f & w & w \\
	f & f &  f & f & f
	\end{array}$$
\end{frame}

\begin{frame}
\frametitle{Konjunktion , Disjunktion und Kontravalenz}

Die Konjunktion von zwei Aussagen ist also nur dann wahr, wenn beide Aussagen wahr sind, die Disjunktion von 
zwei Aussagen ist genau dann wahr, wenn eine von beiden Aussagen wahr ist (oder auch beide), und die 
Kontravalenz ist genau dann wahr, wenn genau eine der beiden Aussagen wahr und eine der beiden Aussagen falsch ist. 

\pause 

\begin{beispiel}
Wir betrachten die beiden Aussagen 
\begin{itemize}
\item $A$: \textit{Berlin ist die Hauptstadt von Deutschland.} 
\item $B$: \textit{Mannheim 
ist die Hauptstadt von Baden---Württemberg.}
\end{itemize} 
Dann gilt 
\begin{itemize}
\item<3-> $A \wedge B$ ist falsch. 
\item<4-> $A \vee B$ ist wahr. 
\item<5-> $A \oplus B$ ist wahr. 
\end{itemize}
\end{beispiel}

\end{frame}

\begin{frame}
\frametitle{Implikation und Äquivalenz}

Die \textbf{Implikation} oder WENN--DANN--Beziehung von $A$ nach $B$ ist eine neue Aussage, die mit 
$A \Longrightarrow B$ bezeichnet wird. 

\pause 

Die \textbf{Äquivalenz} oder GENAU--DANN--WENN--Beziehung von $A$ und $B$ ist eine neue Aussage, die mit 
$A \iff B$ bezeichnet wird. 

\pause 

Implikation und Äquivalenz werden durch folgende Wahrheitstafel beschreiben:

	$$ \begin{array}{ r | c || c | c}
	A & B & A \Longrightarrow B & A \iff B \\
	\hline \hline
	w & w &  w & w \\
	w & f &  f & f \\
	f & w &  w & f \\
	f & f &  w & w
	\end{array} $$

\end{frame}

\begin{frame}
\frametitle{Implikation und Äquivalenz}

Die Implikation von $A$ nach $B$ besagt, dass $A$ eine hinreichende Bedingung für das Auftreten von $B$ ist 
(wenn $A$ wahr ist, dann auch $B$), die Äquivalenz von $A$ und $B$ besagt, dass $A$ hnireichend und notwendig 
für $B$ ist ($A$ ist genau dann wahr, wenn auch $B$ wahr ist). 
\pause 

\begin{beispiel}
Wir betrachten die beiden Aussagen 
\begin{itemize}
\item $A$: \textit{Berlin ist die Hauptstadt von Deutschland.} 
\item $B$: \textit{Mannheim 
ist die Hauptstadt von Baden---Württemberg.}
\end{itemize} 
Dann gilt 
\begin{itemize}
\item<3-> $A \Longrightarrow B$ ist falsch. 
\item<4-> $A \iff B$ ist falsch. 
\item<5-> $B \Longrightarrow A$ ist wahr. 
\end{itemize}
\end{beispiel}

\end{frame}

\begin{frame}
\frametitle{Die Sprache der Aussagenlogik}

Formal besteht die Sprache einer Aussagenlogik aus 
\begin{itemize}
\item<2-> einer (abzählbaren) Menge von elementaren Aussagevariablen $\mathscr{A}_0, \mathscr{A}_1, \ldots$.
\item<3-> den Junktoren $\neg, \wedge, \vee, \rightarrow$ und $\leftrightarrow$.
\item<4-> den Gliederungssysmbolen $($ und $)$.
\end{itemize}
\pause \pause \pause \pause
und aus aussagenlogischen Formeln. Dabei sind alle Aussagevariablen Formeln, und alles, was mit Hilfe von Junktoren 
und Gliederungssysmbolen in endlich vielen Schritten aus Formeln gewonnen werden kann ist wieder eine Formel. 

\pause 

Die Menge aller Formeln wird mit $\mathcal{A}$ bezeichnet.
\end{frame}

\begin{frame}
\frametitle{Die Sprache der Aussagenlogik}

Um mit diesen Formeln arbeiten zu können, ist es notwendig, ihnen Wahrheitswerte zuzuweisen. Dazu sei 
$\mathbb B  = \{ \mathrm{wahr}, \mathrm{falsch} \}$ oder kurz $\mathbb B = \{w, f\}$ die Menge der 
Wahrheitswerte. 

\pause 

\begin{definition} Eine Abbildung $f: V \longrightarrow \mathbb B$, die jeder Aussagenvariable einen Wert 
zuordnet, heißt \textbf{Belegung der Aussagevariablen} 
\end{definition}

\pause 

Aus jeder Belegung der Aussagenvariablen lässt sich auch eine eindeutige Belegung aller Formeln, also eine 
Abbildung $f: \mathcal{A} \longrightarrow \mathbb B$ herleiten. Dazu verwenden wir die Wahrheitstafeln, die
wir oben betrachtet haben. Dadurch werden die Formeln zu Aussagen, 
die mit eindeutigen Wahrheitswerten belegt sind. 

\end{frame}

\begin{frame}
\frametitle{Die Sprache der Aussagenlogik}

\begin{definition} Wir betrachten eine Formel $\alpha \in \mathcal{A}$.

\begin{itemize}
\item $\alpha$ heißt \textbf{erfüllbar}, wenn es eine Belegung $f$ mit $f(\alpha) = w$ gibt.
\item $\alpha$ heißt \textbf{Tautologie}, wenn für jede Belegung $f$ gilt: $f(\alpha) = w$.
\item $\alpha$ heißt \textbf{Kontradiktion} oder \textbf{widersprüchlich}, wenn für jede 
Belegung $f$ gilt $f(\alpha) = f$.
\end{itemize}
\end{definition}

\pause 

\begin{beispiel} Die Aussage $\alpha \leftrightarrow \alpha$ ist eine Tautologie, die Aussage $\alpha \leftrightarrow 
\neg \alpha$ dagegen ist eine Kontradiktion. Das folgt sofort aus obigen Verknüpfungstabellen.
\end{beispiel}

\pause 

\begin{notiz} Die Verneinung einer Tautologie ist eine Kontradiktion, die Verneinung einer Kontradiktion ist 
eine Tautologie.
\end{notiz}


\end{frame}

\begin{frame}
\frametitle{Die Sprache der Aussagenlogik}

\begin{definition} 
Ist $\alpha \rightarrow \beta$ eine Tautologie, so 
schreiben wir hierfür $\alpha \Longrightarrow \beta$.

Ist $\alpha \leftrightarrow \beta$ eine Tautologie, so schreiben wir hierfür $\alpha \iff \beta$.
\end{definition}

\pause 

\begin{regel} 
%Einige wichtige Regeln der Aussagenlogik: 
\begin{itemize}
\item<3-> Es gelten die \textbf{Distributivgesetze}

\begin{itemize}
\item $(\alpha \wedge \beta) \vee \gamma \iff (\alpha \vee \gamma) \wedge (\beta \vee \gamma)$.
\item $(\alpha \vee \beta) \wedge \gamma \iff (\alpha \wedge \gamma) \vee (\beta \wedge \gamma)$.
\end{itemize}
\item<4-> Es gelten die \textbf{Absorptionsgesetze}

\begin{itemize}
\item $\alpha \vee (\alpha \wedge \beta) \iff \alpha $.
\item $\alpha \wedge (\alpha \vee \beta)  \iff \alpha $.
\end{itemize}

\item<5-> Es gelten die \textbf{Gesetze von de Morgan}

\begin{itemize}
\item $\neg(\alpha \wedge \beta)  \iff  \neg \alpha \vee \neg \beta $.
\item $\neg(\alpha \vee \beta)  \iff \neg \alpha \wedge \neg \beta $.
\end{itemize}
\end{itemize}
\end{regel}

\end{frame}

\begin{frame}
\frametitle{Die Sprache der Aussagenlogik}

\begin{uebung}
Zeigen Sie die folgenden Gesetze über Negationen

\begin{enumerate}
\item $\neg ( \neg \alpha) \iff \alpha$.
\item $\alpha \vee \neg \alpha$ ist eine Tautologie
\item $\alpha \wedge \neg \alpha $ ist eine Kontradiktion.
\end{enumerate}

\end{uebung}

\bigbreak

\bigbreak

\pause \pause 

\begin{sol}
Alle drei Aussagen sind wahr. 
\end{sol}

\end{frame}

\begin{frame}
\frametitle{Die Sprache der Aussagenlogik}

\begin{regel}
Es gelten die folgenden Regeln zum Schließen:
\begin{itemize}
\item<2-> Der \textbf{Kettenschluss}: Für Formeln $\alpha$ , $\beta$ und $\gamma$ gilt: 
  	$$ ( \alpha \rightarrow \beta) \wedge ( \beta \rightarrow \gamma) \Longrightarrow ( \alpha \rightarrow \gamma) $$
\item<3-> Die \textbf{Abtrennungsregel}: Für Formeln $\alpha$ und $\beta$ gilt: 
  	$$ \alpha \wedge ( \alpha \rightarrow \beta) \Longrightarrow \beta $$
\item<4-> Die \textbf{Aufhebungsregel}:  Für Formeln $\alpha$ und $\beta$ gilt: 
  	$$ \neg \alpha \wedge ( \beta \longrightarrow \alpha) \Longrightarrow \neg \beta $$
\item<5-> Der \textbf{Umkehrschluß}:  Für Formeln $\alpha$ und $\beta$ gilt: 
  	$$ (\alpha \longrightarrow \beta) \Longrightarrow ( \neg \beta \longrightarrow \neg \alpha ) $$
\end{itemize}
\end{regel}

\end{frame}

\begin{frame}
\frametitle{Die Sprache der Aussagenlogik}

\begin{beispiel}
Sind die beiden Aussagen: \\ \textit{Wenn Montag ist, besuche ich eine Mathematikvorlesung}  und 
\textit{Wenn ich eine Mathematikvorlesung besuche, dann lerne ich etwas Neues} wahr, so auch 

\textit{Wenn Montag ist lerne ich etwas Neues.}
\end{beispiel}

\pause 

\begin{beispiel}
Sind die beiden Aussagen: \\ \textit{Wenn Montag ist, besuche ich eine Mathematikvorlesung}  und 
\textit{Wenn ich eine Mathematikvorlesung besuche, dann lerne ich etwas Neues} wahr, \\ so auch 

\textit{Wenn ich nichts Neues lerne, dann ist nicht Montag.}

\bigbreak

\pause 
Nicht korrekt ist es aber, daraus abzuleiten:

\textit{ Wenn nicht Montag ist, lerne ich nichts Neues.}
\end{beispiel}
\end{frame}

\begin{frame}
\frametitle{Die Sprache der Aussagenlogik}

\begin{uebung}
Gegeben ist, dass die beiden Aussagen: \\ 
\textit{Wenn Montag ist, besuche ich eine Mathematikvorlesung}  und 
\textit{Ich besuche heute keine Mathematikvorlesung} wahr sind. 

Kann man daraus ableiten, dass die Aussage 

\medbreak

\textit{Es ist nicht Montag.}

\medbreak

wahr ist? 
\end{uebung}

\bigbreak \bigbreak
\pause \pause 

\begin{sol}
Die Aussage ist wahr nach der Aufhebungsregel. 
\end{sol}
\end{frame}

\section{Prädikatenlogik}

\begin{frame}
\frametitle{Die Prädikatenlogik}

Die Aussagenlogik befasst sich mit konkreten Aussagen wie etwa \glqq{}$\alpha: 1>0$\grqq{}, denen sich 
eine eindeutiger Wahrheitswert zuweisen lässt. In der Mathematik  wichtig sind aber oft Formeln 
wie 
	$$ \text{ \glqq{} } \beta(x): x >0 \,\,\text{\grqq{}} \quad \text{ oder } \,\, 
	\text{\glqq{}}\,\, \gamma(x,y): x^2-y^2 > 0\,\,  \text{\grqq{}}, $$ 
in denen 
(ein oder mehrere) Platzhalter oder Variablen auftreten. \pause Diese Formeln sind keine Aussagen, 
denn ihnen lässt sich kein Wahrheitswert zuweisen. Das kann erst erfolgen, wenn für die Platzhalter 
konkrete Werte eingesetzt werden. \pause So ist etwa $\beta(1)$ wahr, aber $\beta(-3)$ nicht.  \pause 
Entsprechend ist $\gamma(4,2)$ wahr aber $\gamma(2,4)$ nicht. 

\pause 

\bigbreak

Solche Ausdrücke mit Platzhaltern werden als \textbf{Prädikate} bezeichnet, und mit ihrer Behandlung 
beschäftigt sich die \textbf{Prädikatenlogik}
\end{frame}

\begin{frame}
\frametitle{Die Prädikatenlogik}

Der \textbf{Prädikatenlogik} liegen zwei \textbf{Alphabete} zugrunde, 
ein Alphabet der logischen Zeichen, bestehend aus Variablen, Konnektoren, Trennzeichen, 
Quantoren und logischen Atomen, und ein Alphabet der Theoriezeichen, bestehend aus einer Menge 
von Konstanten (also Objekten  aus einem vorgegebenen Individuenbereich, der 
betrachtet wird), einer Menge von Funktionszeichen und von Relationszeichen. 

\pause

Funktionszeichen $f$ sind dabei mathematische Funktionszeichen, etwa 
$f(x, y) = x \cdot y$, und Relationszeichen $r$ entsprechen mathematische Relationen wie 
$r(x,y) : x > y$ vorstellen

\pause

Variablen sind Platzhalter, die zu 
einem gegebenen Zeitpunkt genau einen Wert aus dem Individuenbereich annehmen können. 

\pause

Ein Term ist alles, was sich aus Konstanten, Variablen und Funktionszeichen in endlich 
vielen Schritten bilden lässt. Mit  Konstanten $\mathbb R$ etwa können wir den Term $t_1 = f(x,2) = 
2 \cdot x$ bilden, mit diesem Term wiederum den Term $f(x, t_1) = x \cdot t_1 = 2x^2$. Konstanten 
und Variablen selbst sind auch Terme.

\end{frame}


\begin{frame}
\frametitle{Die Prädikatenlogik}

Bei Prädikaten ist es nicht nur interessant, zu untersuchen, für welche Belegungen der Platzhalter eine wahre 
Aussage besteht, es ist auch von Bedeutung, festzustellen, ob es überhaupt Belegungen gibt, die zu wahren 
Aussagen führen, oder ob alle Belegungen wahre Aussagen ergeben. 

\pause 

Für diese Untersuchungen werden \textbf{Quantoren} verwendet, der \textbf{Allquantor} $\forall$ 
und der \textbf{Existenzquantor} $\exists$. Dadurch wird aus einem Prädikat (mit einem Platzhalter) 
eine Aussage

\pause 

\begin{itemize}
\item $\forall x \,\, \alpha(x) \quad$ ist die Aussage \glqq{}\textit{Für alle $x$ (aus dem
Individuenbereich) gilt $\alpha(x)$}\grqq{}. Sie ist genau dann wahr, wenn für \textbf{jede} 
zulässige Belegung von $x$ mit einem konkreten Wert $x_0$ die Aussage $\alpha(x_0)$ wahr ist.
\item $\exists x \,\, \alpha(x) \quad$ ist die Aussage \glqq{}\textit{Es gibt ein $x$ (aus dem
Individuenbereich) für das $\alpha(x)$ gilt}\grqq{}. Sie ist genau dann wahr, wenn für 
\textbf{mindestens eine} zulässige Belegung von $x$ mit einem konkreten Wert $x_0$ die 
Aussage $\alpha(x_0)$ wahr ist.
\end{itemize}

\end{frame}


\begin{frame}
\frametitle{Die Prädikatenlogik}

\begin{beispiel}
Mit der Sprache der Prädikatenlogik lassen sich mathematische Fragenstellungen knapp und präzise fromulieren 
und definieren. \pause Die Aussage

\centerline{ \textit{Die Zahlenfolge $a_n = \frac {2n+3}{3n}$ konvergiert gegen $\frac {2}{3}$ 
für $n \rightarrow \infty$.}}

formuliert sich dann etwa so \pause
  	$$ \forall \varepsilon > 0 \,\, \exists N \in N \,\, \forall n \geq N \, : \,  
	\left\vert a_n - \frac {2}{3} \right\vert < \varepsilon $$
\pause 
Dabei handelt es sich um eine wahre Aussage.

\end{beispiel}

\end{frame}


\begin{frame}
\frametitle{Die Prädikatenlogik}

\begin{beispiel}
Die Aussage
 
\centerline{\textit{Jede gerade Zahl, die größer als zwei ist, ist die Summe von zwei Primzahlen.}}

lässt sich wie folgt formulieren: \pause 

  	$$ \forall n \in \mathbb N \, \, \left( ( 2n > 2 ) \, \rightarrow \, (\exists x \in \mathbb P \, \exists y \in 
    	\mathbb P \, : \, 2n = x + y) \right) $$
wobei $\mathbb N$ die natürlichen Zahlen und $\mathbb P$ die Primzahlen bezeichnet. 

\pause

Bei dieser Aussage ist nicht bekannt, ob sie wahr oder falsch ist (\textit{Goldbach--Vermutung}). 
\end{beispiel}

\end{frame}


\begin{frame}
\frametitle{Die Prädikatenlogik}

\begin{uebung}
Formulieren Sie die Aussage 
	\centerline{\textit{Das Quadrat einer geraden ganzen Zahl ist eine gerade Zahl.}}
in der Sprache der  Prädikatenlogik
\end{uebung}

\pause \pause 

\bigbreak

\begin{sol}
Die Aussage kann entweder wie folgt formuliiert werden 
  	$$ \forall x \in \mathbb Z \, \left( 2 \vert x\, \Longrightarrow \, 2 \vert x^2 \right) $$
oder als 
  	$$ \forall x \in \mathbb Z \, \left( (\exists k \in \mathbb Z \,\, x = 2 \cdot k) \, \Longrightarrow \, 
	(\exists l \in \mathbb Z \,\,  x^2 = 2 l) \right) $$
\end{sol}

\end{frame}

\section{Beweistechniken}

\begin{frame}
\frametitle{mathematische Beweise}

In der Mathematik geht es oft darum, dass eine Voraussetzung gegeben ist, und dass dann daraus eine 
Behauptung abzuleiten ist. Das lässt sich häufig wie folgt formulieren: 

\centerline{\textit{ Für alle $x$, die eine Bedingung erfüllen, gilt eine bestimmte Behauptung}} 

\pause 

In der Sprache der Prädikatenlogik: 
	$$ \forall x \,\, \alpha(x) \Longrightarrow \beta(x) $$

\pause

Die Aussage ist genau dann bewiesen, wenn für jede (mögliche) Belegung von $x$ durch ein $x_0$ die 
resultierende  Aussage $\alpha(x_0) \Longrightarrow \beta(x_0) $ wahr ist. 
\end{frame}

\begin{frame}
\frametitle{direkte Beweise}

Von einem direkten Beweis sprechen wir, wenn aus der Voraussetzung unmittelbar durch logische
Umformungen die Behauptung abgeleitet werden kann. 

\pause 

Zum Beweis von 
	$$ \forall x \,\, \alpha(x) \Longrightarrow \beta(x) $$
ist dabei durch die Umformungsregeln der Logik zu zeigen, dass für alle $x$, für die $\alpha(x)$ wahr ist, 
auch $\beta(x)$ wahr ist. 

\pause 

\begin{beispiel}
Wir zeigen mit einem direkten Beweis, dass 
  	$$ \forall x \in \mathbb Z \, \left( 2 \vert x\, \Longrightarrow \, 2 \vert x^2 \right) $$
eine korrekte Aussage ist. 
\end{beispiel}

\end{frame}

\begin{frame}
\frametitle{direkte Beweise}

\begin{uebung}
Zeigen Sie die folgende Aussage mit einem direkten Beweis: 

Ist $x \in \mathbb Z$ eine positive ganze Zahl, für die $\sqrt{x}$ rational ist, so ist $\sqrt{x}$ schon eine 
ganze Zahl, dh.  zeigen Sie 
	$$ \forall x \in \mathbb Z, x > 0 \,\, \left(\sqrt{x} \in \mathbb Q \Longrightarrow \sqrt{x} \in \mathbb Z 
	\right) $$ 
\end{uebung}

\pause \pause 

\begin{sol}
Der entscheidende Punkt ist, dass wir $\sqrt{x} = \frac {m}{n}$ (mit $m,n \in \mathbb N$) schreiben und 
zeigen, dass $n \vert m$. Dazu nutzen wir aus, dass $x^2 = \frac {m^2}{n^2} \in \mathbb Z$.  
\end{sol}
\end{frame}

\begin{frame}
\frametitle{indirekte Beweise}
Von einem indirekten Beweis sprechen wir, wenn für den Beweis der Umkehrschluss benutzt wird. 

\pause 

Zum Beweis von 
	$$ \forall x \,\, \alpha(x) \Longrightarrow \beta(x) $$
ist dabei durch die Umformungsregeln der Logik zu zeigen, dass für alle $x$, für die $\beta(x)$ falsch ist, 
auch $\alpha(x)$ falsch ist. 

\pause 

\begin{beispiel}
Wir zeigen mit einem indirekten Beweis, dass 
  	$$ \forall x \in \mathbb Z \, \left( 2 \vert x\, \longrightarrow \, 2 \vert x^2 \right) $$
eine korrekte Aussage ist. 
\end{beispiel}


\end{frame}

\begin{frame}
\frametitle{indirekte Beweise}

\begin{uebung}
Zeigen Sie die folgende Aussage mit einem indirekten Beweis: 

Ist $x \in \mathbb Z$ eine positive ganze Zahl, für die $\sqrt{x}$ rational ist, so ist $\sqrt{x}$ schon eine 
ganze Zahl, dh.  zeigen Sie 
	$$ \forall x \in \mathbb Z, x > 0 \,\, \left( \sqrt{x} \in \mathbb Q \Longrightarrow \sqrt{x} \in \mathbb Z 
	\right) $$ 
\end{uebung}

\pause \pause 

\begin{sol}
Hier schreiben wir wieder $\sqrt{x} = \frac {m}{n}$ (mit $m,n \in \mathbb N$) und nehmen an, dass 
$m$ und $n$ teilerfremd sind und $n \neq 1$ (also das die Folgerung nicht stimmt). Daraus leiten wir 
über $x = \frac {m^2}{n^2}$ ab, dass $x$ nicht ganzzahlig ist. 
\end{sol}
\end{frame}

\begin{frame}
\frametitle{Induktionsbeweis}

In der Mathematik gibt es viele Formeln, die für alle natürlichen Zahlen formuliert werden können, z.B. 
	$$ A(n) : \,\, \sum\limits_{k = 1}^n \frac {1}{k \cdot (k+1)}  = \frac {n}{n+1} $$
\pause 
Wir haben also eine Folge von Aussagen $A(n)$ über deren Wahrheitswert entschieden werden soll. 

\pause 

In diesen Situationen ist es oft schwierig, einen (direkten oder indirekten) Beweis zu finden, der für alle 
$n \in \mathbb N$ gleichmäßig arbeitet. 

\pause 

Oft hilft hier aber das Prinzip der vollständigen Induktion.  
\end{frame}

\begin{frame}
\frametitle{Induktionsbeweis}

Wir betrachten Aussagen $A(n)$ für alle $n \in \mathbb N$. Gilt dann 
\begin{enumerate}
\item<2-> Induktionsanfang: $A(1)$ ist wahr.
\item<3-> Induktionsschluß: $A(n) \Longrightarrow A(n+1)$ für alle $n \geq 1$.
\end{enumerate}
\pause \pause \pause 

so ist $A(n)$ wahr für alle $n \geq 1$

\pause
Das Prinzip der vollständigen Induktion funktioniert auch mit einem beliebigen Startwert $n_0 \in \mathbb N$ 
(anstelle von $1$). Dann gilt $A(n)$ für alle $n \geq n_0$

\end{frame}

\begin{frame}
\frametitle{Induktionsbeweis}

\begin{beispiel}
Wir beweisen 
	$$ A(n) : \,\, \sum\limits_{k = 1}^n \frac {1}{k \cdot (k+1)}  = \frac {n}{n+1} $$
für alle $n \geq 1$ mit vollständiger Induktion.
\end{beispiel}

\end{frame}

\begin{frame}
\frametitle{Induktionsbeweis}

\begin{uebung}
Zeigen Sie mit vollständiger Induktion: 

Für alle $n \geq 0$ gilt 
	$$ A(n): \,\, \sum\limits_{k=0}^n \frac {1}{2^k} =2 \cdot \left( 1 - \frac {1}{2^n} \right) $$
\end{uebung}

\end{frame}

\begin{frame}
\frametitle{Induktionsbeweis}

\begin{uebung}
Zeigen Sie mit vollständiger Induktion: 

Für alle $n \geq 2$ gilt 
	$$ A(n): \,\, \prod\limits_{k=2}^n \left(1 - \frac {1}{k} \right) =\frac {1}{n}$$
\end{uebung}

\end{frame}


\end{document}
