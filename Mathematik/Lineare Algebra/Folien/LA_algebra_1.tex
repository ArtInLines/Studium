\documentclass[hyperref={pdfpagelabels=false}]{beamer}
\providecommand\thispdfpagelabel[1]{}
% Die Hyperref Option hyperref={pdfpagelabels=false} verhindert die Warnung:
% Package hyperref Warning: Option `pdfpagelabels' is turned off
% (hyperref)                because \thepage is undefined. 
% Hyperref stopped early 
%

\usepackage{lmodern}
% Das Paket lmodern erspart die folgenden Warnungen:
% LaTeX Font Warning: Font shape `OT1/cmss/m/n' in size <4> not available
% (Font)              size <5> substituted on input line 22.
% LaTeX Font Warning: Size substitutions with differences
% (Font)              up to 1.0pt have occurred.
%
\usepackage{epstopdf}
\usepackage{german}
\usepackage{amssymb}
\usepackage{amsfonts}
\usepackage{amsmath}
\usepackage{amsthm}
\usepackage{latexsym}
\usepackage{newalg}
\usepackage{ifthen}
\usepackage{fancybox}
\usepackage{fancyhdr}
%\usepackage[dvips, dvipdfm]{graphicx}
%\usepackage[pdftex]{color,graphicx}
\usepackage{eurofont}
\usepackage{aeguill}
\usepackage[T1]{fontenc}
\usepackage{textcomp}
\usepackage{wrapfig}
\usepackage{mathrsfs}
\usepackage{float}
\usepackage{color}
\usepackage{siunitx}

\usepackage[utf8]{inputenc}
%\usepackage{hyperref}

\theoremstyle{plain}% default
\newtheorem*{satz}{Satz} 
\newtheorem*{hilf}{Lemma} 
\newtheorem*{korollar}{Folgerung}
\newtheorem*{regel}{Regel} 

\theoremstyle{definition}
%\newtheorem*{definition}{Definition} 
\newtheorem*{aufgabe}{Aufgabe} 
\newtheorem*{uebung}{Übung}
\newtheorem*{sol}{Lösung:}

\theoremstyle{remark}
\newtheorem*{beispiel}{Beispiel} 
\newtheorem*{notiz}{Bemerkung} 
\newtheorem*{notation}{Notation}
%\newtheorem*{problem}{Problem} 

\def \ii{{\relax\ifmmode \mathrm{i} \else i \fi}}
\def \ee{{\relax\ifmmode \mathrm{e} \else e \fi}}
\def \dd{{\relax\ifmmode \mathrm{d} \else d \fi}}
\def \g{{\relax\ifmmode \mathrm{g} \else g \fi}}
\def \R{\mathbb R}
\def \C{\mathbb C}
\def \Q{\mathbb Q}
\def \N{\mathbb N}
\def \Z{\mathbb Z}
\def \F{\mathbb F}
\def \A{\mathbb A}

% tikz Anpassungen 

\usepackage{lscape}
\usepackage{gnuplot-lua-tikz}
\usepackage{pgf,tikz}
\usetikzlibrary{arrows,decorations.pathmorphing,backgrounds,positioning,fit,petri}

\usepackage{pgfplots}
\usepackage{pgfplotstable}
\pgfplotsset{compat=newest}

\usepackage{environ}
\makeatletter

\newsavebox{\measure@tikzpicture}
\NewEnviron{scaletikzpicturetowidth}[1]{%
  \def\tikz@width{#1}%
  \def\tikzscale{1}\begin{lrbox}{\measure@tikzpicture}%
  \BODY
  \end{lrbox}%
  \pgfmathparse{#1/\wd\measure@tikzpicture}%
  \edef\tikzscale{\pgfmathresult}%
  \BODY
}
\makeatother

\setcounter{secnumdepth}{-1}


% Wenn \titel{\ldots} \author{\ldots} erst nach \begin{document} kommen,
% kommt folgende Warnung:
% Package hyperref Warning: Option `pdfauthor' has already been used,
% (hyperref) ... 
% Daher steht es hier vor \begin{document}

\title{Lineare Algebra \\ algebraische Strukturen 1}   
\author{ Reinhold Hübl} 
\date{Wintersemester 2020/21} 

% Dadurch wird verhindert, dass die Navigationsleiste angezeigt wird.
%\setbeamertemplate{navigation symbols}{}

% zusaetzlich ist das usepackage{beamerthemeshadow} eingebunden 
\usepackage{beamerthemeshadow}
\usetheme{CambridgeUS}

%  \beamersetuncovermixins{\opaqueness<1>{25}}{\opaqueness<2->{15}}
%  sorgt dafuer das die Elemente die erst noch (zukuenftig) kommen 
%  nur schwach angedeutet erscheinen 
\beamersetuncovermixins{\opaqueness<1>{25}}{\opaqueness<2->{15}}
% klappt auch bei Tabellen, wenn teTeX verwendet wird\ldots
\begin{document}


\begin{frame}
\titlepage
\centering 
    \scalebox{0.40}{\includegraphics{dhbw_ma_logo.eps}}
 
\end{frame} 



\section{Mengen}

\begin{frame}
\frametitle{Mengen und Mengenalgebren}

\begin{definition} Eine \textit{Menge}  ist eine Zusammenfassung $M$ von bestimmten wohlunterschiedenen 
Objektem $m$, genannt die Elemente von $M$, unseres Anschauungsraums oder unseres Denkens zu einem Ganzen.
\end{definition}

\bigbreak

\pause

Ist $m$ ein Element  von $M$, so schreiben wir $m \in M$, andernfalls schreiben wir $m \notin M$. 
F\"ur jedes Objekt $m$ unserer Anschauung und jede Menge $M$ gilt also genau entweder $m \in M$  oder 
$m \notin M$, nicht aber beides. 

\medbreak

\pause 

\begin{definition} Zwei Mengen $N$ und $M$ hei{\ss}en gleich, geschrieben $M = N$, wenn ein Objekt 
$x$ genau dann Element von $N$ ist, wenn es Element von $M$ ist, d.h. wenn die Äquivalenz 
$x \in N \iff x \in M$ gilt.
\end{definition}
\end{frame}

\begin{frame}
\frametitle{Mengen und Mengenalgebren}

\begin{definition} Eine Menge $N$ hei{\ss}t \textbf{Teilmenge} eine Menge $M$, geschrieben 
$N \subseteq M$, wenn jedes Element von $N$ auch Element von $M$ ist, d.h. wenn die Implikation 
$x \in N \Longrightarrow x \in M$ gilt. In diesem Fall hei{\ss}t $M$ auch \textit{Obermenge} 
von $N$.

\pause 

$N$ hei{\ss}t echte Teilmenge von $M$, geschrieben $N \subset 
N$, wenn $N$ Teilmenge von $M$ mit $N \neq M$ ist.
\end{definition}

\medbreak

\pause

\begin{beispiel} Die Menge $\mathbb N$ der natürlichen Zahlen ist eine (echte) Teilmenge der 
ganzen Zahlen $\Z$, $\N \subseteq \Z$. 

\pause 

Jede natürliche Zahl ist eine ganze Zahl, es gibt aber ganze Zahlen (z.B. $-1$), die keine natürlichen 
Zahlen sind. 
\end{beispiel}

\end{frame}

\begin{frame}
\frametitle{Mengenoperationen}

\begin{definition} Die \textbf{Schnittmenge} von $A$ und $B$, geschrieben $A \cap B$ 
besteht aus den Elementen von $M$, die sowohl in $A$ als auch in $B$ sind:
  	$$ A \cap B = \{ x \in M \vert x \in A \textrm{ und } x \in B \} $$
Falls $A \cap B = \emptyset$, so nennen wir $A$ und $B$ \index{Menge!disjunkt}\textit{disjunkt}.
\end{definition}

\pause

\begin{definition} Die \textbf{Vereinigungsmenge} von $A$ und $B$, geschrieben $A \cup B$ 
besteht aus den Elementen von $M$, die entweder in $A$ oder in $B$ sind:
  	$$ A \cup B = \{ x \in M \vert \, x \in A \textrm{ oder } x \in B \} $$
\end{definition}

\end{frame}

\begin{frame}
\frametitle{Mengenoperationen - Schnitt- und Vereinigungsmengen}

\begin{satz} F\"ur Teilmengen $A, B, C \subseteq M$ einer Grundmenge $M$ gilt

\begin{tabular} {l l }
  	Kommutativgesetz & $A \cup B = B \cup A$ \\
  	& $A \cap B = B \cap A$ \\
  	Assoziativgesetz & $( A \cup B) \cup C = A \cup ( B \cup C)$ \\
  	& $(A \cap B) \cap C = A  \cap (B \cap C)$ \\
  	Distributivgesetz & $(A \cup B) \cap C = (A \cap C) \cup (B \cap C)$ \\
  	& $(A \cap B) \cup C = (A \cup C) \cap (B \cup C)$ \\
  	Verschmelzungsgesetz & $A \cap (A \cup B) = A$ \\
  	& $A \cup (A \cap B) = A$   
\end{tabular}
\end{satz}


\end{frame}

\begin{frame}
\frametitle{Mengenoperationen - Differenzmengen}

\begin{definition} Die \textbf{Differenzmenge} von $B$ und $A$, geschrieben $B \setminus A$ 
besteht aus den Elementen von $Ma$, die in $B$ aber nicht in $A$ sind:
  	$$ B \setminus A = \{ x \in M \vert \, x \in B \, \textrm{ und } x \notin A \} $$
Falls $A \subseteq B$ nennen wir $B \setminus A$ auch das \textbf{Komplement} von $A$ in $B$ und 
schreiben hierf\"ur $\overline{A}^B$. Falls $B = M$, schreiben wir hierf\"ur auch kurz $\overline{A}$ 
und nennen es das Komplement von $A$. 
\end{definition}

\end{frame}

\begin{frame}
\frametitle{Mengenoperationen - das Komplement}

\begin{regel}\label{mengen_regel_komplement} F\"ur eine Teilmenge $A \subseteq M$ gilt:
\begin{enumerate}
\item<2-> $\overline{\overline{A}} = A$
\item<3-> $\overline{A} \cup A = M$.
\item<4-> $\overline{A} \cap A = \emptyset$.
\item<5-> $\overline{\emptyset} = M$
\item<6-> $\overline{M} = \emptyset$.
\end{enumerate}
\end{regel}

\end{frame}

\begin{frame}
\frametitle{Mengenoperationen - das kartesische Produkt}

\begin{definition} Das \textbf{kartesische Produkt}\index{kartesisches Produkt} zweier Mengen $M$ und $N$ ist 
die Menge $M \times N$, deren Elemente die geordneten Paare $(m, n)$ sind, wobei $m \in M$ und $n \in N$, also
  	$$ M \times N = \{ (m, n) \vert \, m \in M \textrm{ und } n \in  N \} $$
\end{definition}

\pause 

\begin{beispiel}
Für $M = \{1,2,3\}$ und $N = \{r,l\}$ ist 
	$$ M \times N = \{(1,r), (1,l), (2,r), (2,l), (3,r), (3,l) \} $$
\end{beispiel}

\pause 

\begin{notiz}
Die Bildung des kartesischen Produkts kann auch iteriert werden: 

Für Mengen $L, M$ und $N$ gilt 
	$$ L \times M \times N = (L \times M) \times N = L \times (M \times N) $$
\end{notiz}

\end{frame}

\section{Relationen}

\begin{frame}
\frametitle{Relationen}

\begin{definition} Eine \textbf{Relation} $R$ zwischen zwei Mengen $M$ und $N$ ist eine Beziehung
zwischen Elementen von $M$ und $N$, dargestellt durch geordnete Paare $(m,n)$ mit $m \in M$ und 
$n \in N$. Wir schreiben hierfür $m \sim_R n$ oder $mRn$ und sagen \textit{$m$ steht in Relation mit 
$n$ (bezüglich $R$)}.

Ist $M = N$, so heißt $R$ auch Relation auf $M$. In diesem Fall nennen wir $R$ auch \textbf{homogen}.
\end{definition}

\pause

\begin{notiz} Eine Relation $R$ zwischen $M$ und $N$ ist ein Teilmenge $R \subseteq M \times N$ des kartesischen 
Produktes.
\end{notiz}

\pause 

\begin{notiz}
Gleichheit $=$ definiert eine Relation auf $\Z$, 
	$$ R = \{(z,z) \, \vert \, z \in \Z \} \subseteq \Z \times \Z $$
\end{notiz}

\end{frame}

\begin{frame}
\frametitle{Relationen}

Jede Abbildung $f: M \longrightarrow N$ ist eine Relation $R$ zwischen $M$ und $N$ mit der folgenden 
speziellen Eigenschaft 
	$$ \text{Für alle } \, m \in M \, \text{ existiert genau ein } \, n \in N \, \text{ mit } (m,n) \in R $$
\pause 
Eine solche Relation heißt auch \textbf{linkstotal} und \textbf{rechtseindeutig}

\pause 

\bigbreak

Umgekehrt definiert jede linkstotale und rechtseindeutige Relation eine Abbildung: 

Ist $m \in M$, so gibt es genau ein $n \in N$ mit $(m,n) \in R$, und wir definieren $f: M \longrightarrow N$ 
durch $f(m)  = n$.  

\end{frame}

\begin{frame}
\frametitle{Relationen}

\begin{beispiel}
Die Relation 
	$$ R = \{ (x, \ee^x) \, \vert \, x \in \R\} \subseteq \R \times \R_{>0} $$
definiert die Exponentialfunktion. \pause Die Exponentialfunktion wird auch definiert durch die 
folgende Beschreibung der Relation 
	$$ R = \{ \ln(x), x)\, \vert \, x \in \R_{>0}\} \subseteq \R \times \R_{>0} $$
\pause 
Der natürliche Logarithmus dagegen wird definiert durch die Relation 
	$$ R = \{ (x, \ln(x)) \, \vert \, x \in \R_{>0} \} \subseteq \R_{> 0} \times \R $$
\end{beispiel}

\end{frame}

\begin{frame}
\frametitle{Relationen}

\begin{uebung}
Überprüfen Sie, ob durch die Relation 
	$$ R = \{ (x^3, x^5) \, \vert \, \, x \in \mathbb R \} \subseteq \R \times \R $$
eine Funktion definiert wird. 
\end{uebung}

\pause \pause 

\begin{sol}
Diese Relation definiert eine Funktion $f: \R \longrightarrow \R$, die explizit auch durch 
$f(x) = \sqrt[3]{x^5}$ beschrieben werden kann. 
\end{sol}
\end{frame}

\begin{frame}
\frametitle{Relationen}

\begin{beispiel}\label{relation_geq} Ist $M = \R$ die Menge der reellen Zahlen, so definiert die 
Beziehung $R$: \textit{ist größer oder gleich} eine Relation auf $M$, die mit $\geq$ bezeichnet wird. 
Ein Zahlenpaar $(a,b)$ ist also genau dann in $R \subseteq \R \times R$, wenn $a \geq b$.
\end{beispiel}

\pause

\begin{beispiel}\label{relation_modn} Ist wieder $M = \mathbb Z$ die Menge der ganzen Zahlen und ist $n \in 
\mathbb Z$ eine vorgegebene Zahl, so definiert die Beziehung $R$: \textit{unterscheiden sich um ein 
Vielfaches von $n$} eine Relation of $\mathbb Z$. Ein Zahlenpaar $(a,b)$ ist also genau dann in $R$ wenn $a - b$ 
durch $n$ teilbar ist.
\end{beispiel}

\end{frame}

\begin{frame}
\frametitle{Relationen}

Wir interessieren uns vor allem für Relation auf einer Menge $M$.
\bigbreak

\pause 
\begin{definition} Betrachte eine Relation $R$ auf einer Menge $M$.

\begin{itemize} 
\item<3-> $R$ heißt \index{Relation!reflexiv}\textbf{reflexiv}, wenn für alle $m$ aus $M$ gilt: $m \sim_R m$.
\item<4-> $R$ heißt \index{Relation!transitiv}\textbf{transitiv}, wenn gilt: 
   	$$ m_1 \sim_R m_2 \, \textrm{ und } m_2 \sim_R m_3 \, \Longrightarrow \, m_1 \sim_R m_3 $$
\item<5-> $R$ heißt \index{Relation!symmetrisch}\textbf{symmetrisch}, wenn gilt:
   	$$ m_1 \sim_R m_2 \, \Longrightarrow \, m_2 \sim_R m_1 $$
\end{itemize}
\end{definition}


\end{frame}

\begin{frame}
\frametitle{Relationen}

\begin{definition} Betrachte eine Relation $R$ auf einer Menge $M$.

\begin{itemize} 
\item<2-> $R$ heißt \textbf{Äquivalenzrelation}, wenn $R$ reflexiv, 
transitiv und symmetrisch ist.
\item<3-> $R$ heißt \\textbf{antisymmetrisch}, wenn gilt:
\vspace{-0.3cm}
   	$$ m_1 \sim_R m_2 \, \textrm{ und } m_2 \sim_R m_1 \, \Longrightarrow \, m_1 = m_2 $$
\item<4-> $R$ heißt \textbf{asymmetrisch}, wenn gilt:
\vspace{-0.3cm}
   	$$ m_1 \sim_R m_2 \, \Longrightarrow \, \neg \left( m_2 \sim_R m_1 \right) $$
\vspace{-0.3cm}
\end{itemize}
\end{definition}
\bigbreak

\pause \pause \pause \pause 

Eine antisymmetrische Relation kann reflexiv sein (muss aber nicht), eine asymmetrische Relation 
ist niemals reflexiv (wenn $M \neq \emptyset$)

\end{frame}

\begin{frame}
\frametitle{Äquivalenzrelationen}

Jede Äquivalenzrelation $\sim$ auf $M$ liefert uns eine natürliche Zerlegung von $M$ in disjunkte Teilmengen:

\pause 

\begin{definition} Wir betrachten eine Äquivalenzrelation $\sim_R$ auf $M$.

\pause 

Zwei Elemente $m, n \in M$ hei{ss}en \textbf{äquivalent} (bezüglich $\sim_R$), wenn $m \sim_R n$.

\pause 
Eine Teilmenge $A \subseteq M$ heißt \index{Äquivalenzklasse}\textbf{Äquivalenzklasse}, wenn gilt

\begin{itemize}
\item<5-> Sind $m, n \in A$, so ist $m \sim_R n$.
\item<6-> Ist $m \in A$ und $n \in M$ mit $n \sim_R m$, so ist $n \in A$.
\end{itemize}

\pause \pause \pause
Ist $m \in M$, so heißt $[m]_R := \{ n \in M: n \sim_R m \}$ die \textbf{Äquivalenzklasse von $m$}.
\end{definition}
\end{frame}

\begin{frame}
\frametitle{Äquivalenzrelationen}

\begin{notiz} Ist $\sim_R$ eine Äuqivalenzrelation auf $M$ und sind $m, n \in M$, so gilt entweder  
$[m]_R = [n]_R$ oder $[m]_R$ und $[n]_R$ sind disjunkt. 

\pause 
Eine Äquivalenzrelation $\sim_R$ auf $M$ zerlegt also $M$ in disjunkte Teilmengen, die Äquivalenzklassen. 
Wir schreiben $M/\sim_R$ für die Menge der Äquivalenzklassen, also 
  	$$ M/\sim_R = \{ [m]_R\, \vert \, m \in M \} $$
\end{notiz}

\pause 
\begin{definition} Ist $R$ eine Äquivalenzrealtion auf $M$ und $A$ ein Äquivalenzklasse (bezüglich $R$), 
so heißt ein beliebiges Element $a \in A$ ein 
\textbf{Repräsentant} der Äquivalenzklasse $A$.

\pause
Ein \textbf{Repräsentantensystem} der Äquivalenzrelation $R$ ist eine Teilmenge $
N \subseteq M$ die genau einen Repräsentanten jeder Äquivlenzklasse enthält.
\end{definition}

\end{frame}

\begin{frame}
\frametitle{Äquivalenzrelationen}

\begin{beispiel}\label{rela_z_mod_n} 
Für die Äquivalenzrelation $\sim_R$ auf $\Z$ definiert durch 
\textit{unterscheiden sich um ein Vielfaches von $n$}  gilt:
\vspace{-0.1cm}
\begin{itemize}
\item<2-> Falls $n = 0$:
  	$$ \mathbb Z/\sim_R = \mathbb Z$$
\item<3-> Falls $n > 0$: 
  	$$ \mathbb Z/\sim_R = \{ [0]_R, [1]_R, \ldots , [n-1]_R \} $$
\item<4-> Falls $n < 0$:
  	$$ \mathbb Z/\sim_R = \{ [0]_R, [1]_R, \ldots , [-n-1]_R \} $$
\end{itemize}
\vspace{-0.5cm}
\pause \pause \pause \pause 
Für $n \neq 0$ sind die angegebenen Äquivalenzklassen paarweise disjunkt. Wir schreiben in diesem Fall 
auch $\mathbb Z_n$, $\mathbb Z/(n)$ oder $\mathbb Z / n \mathbb Z$ für $\mathbb Z/\sim_R$.

\pause
Die Menge $\{0, 1, \, \ldots , n-1\}$ bildet also ein Repräsentantensystem der Relation $R$. Es gibt aber 
noch viele weitere Repräsentantensysteme, etwa $\{1, 2, \ldots, n \}$, $\{n, n+1, \ldots, 2n-1\}$, 
$\{0, n+ 1, 2n+2, 3n+3 \ldots, n^2-1 \}$.
\end{beispiel}

\end{frame}

\begin{frame}
\frametitle{Äquivalenzrelationen}

\begin{uebung}

Wir betrachten eine Relation $R$ auf $\Z$ mit 
	$$ x \sim_R y \iff x^2 = y^2 $$
Zeigen Sie, dass $R$ eine Äquivalenzrelation ist und bestimmen Sie ein Repräsentantensystem von $R$.  
\end{uebung}

\bigbreak
\pause \pause 

\begin{sol}
Die Relation $R$ ist eine Äquivalenzrelation und ein Repräsentantensystem sind alle ganzen Zahlen $z \geq 0$ 
(oder alle ganzen Zahlen $z \leq 0$). 
\end{sol}

\end{frame}

\begin{frame}
\frametitle{Ordnungsrelationen}

Neben Äquvalenzrelation eine besondere Rolle spielen Vergleichsrelationen, wie etwa die Relationen $\geq$ 
oder $>$ auf den reellen Zahlen. 

\pause 

\begin{definition} Es sei $R$ eine Relation auf eine Menge $M$.

$R$ heißt \textbf{Ordnungsrelation} oder \textbf{Ordnung} auf $M$, wenn sie 
reflexiv, transitiv und antisymmetrisch ist.

\pause 

$R$ heißt \textbf{strikte Ordnungsrelation} oder \textbf{strikte Ordnung} 
auf $M$, wenn sie asymmetrisch und  transitiv ist.
\end{definition}

\pause 
\begin{beispiel} Die Relation $\geq$ auf $\R$ ist eine Ordnungsrelation aber keine 
strikte Ordnungsrelation.
\end{beispiel}

\pause 

\begin{beispiel} Die Relation $>$ auf $\R$ ist eine strikte Ordnungsrelation.
\end{beispiel}

\end{frame}

\begin{frame}
\frametitle{Ordnungsrelationen}

\begin{beispiel}
Die Relation $R$ auf $\Z^2$ mit 
	$$ (a,b) \sim_R (c,d) \iff a+b > c+d $$
ist eine strikte Ordnung auf $\Z^2$.
\end{beispiel}

\pause 

\begin{uebung} 
Überprüfen Sie, ob die Relation $R$ aus $\Z^2$ mit 
	$$ (a,b) \sim_R (c,s) \iff a+b \geq c+d $$
eine Ordnung auf $\Z^2$  definiert.
\end{uebung}

\pause \pause 

\begin{sol} $R$ definiert keine Ordnung auf $\Z^2$. 

%\pause 
%
%z.B. ist $(1,1) \sim_R (0,2)$ und $(0,2) \sim_R (1,1)$ aber $)1,1) \neq (0,2)$. 
\end{sol}

\end{frame}

\section{Gruppen}

\begin{frame}
\frametitle{innere Verknüpfung}

Wir betrachten eine (beliebige) Menge $M$. 

\pause

\begin{definition} Eine Abbildung 
  	$$ \circ : M \times M \longrightarrow M $$
also eine Abbildung mit Definitonsbereich $M \times M$ und Bildbereich $M$ heißt 
\textbf{(innere) Verknüpfung} von $M$. 

Wir schreiben in diesem Fall $m \circ n$ für $\circ(m,n)$.
\end{definition}

\pause 

\begin{beispiel} Ist $M = \mathbb Z$ und 
  	$$ \circ = '+': \mathbb Z \times \mathbb Z \longrightarrow \mathbb Z, \quad (a,b) \longmapsto a + b $$
die Addition ganzer Zahlen, so ist $'+'$ eine innere Verknüpfung auf $\mathbb Z$.
\end{beispiel}
\end{frame}

\begin{frame}
\frametitle{innere Verknüpfung}

\begin{beispiel} 
Es sei $M = \{f: \mathbb R \longrightarrow \mathbb R \}$ die Menge aller Abbildungen 
$f: \mathbb R \longrightarrow \mathbb R$ und es sei  
  	$$ \circ : M \times M \longrightarrow M, \quad (f, g) \longmapsto g \circ f $$
die Komposition von zwei Abbildungen. Dann ist $\circ$ eine innere Verknüpfung von $M$.
\end{beispiel}

\pause 

\begin{beispiel} Es sei $M$ eine beliebige Menge und es sei 
$\mathfrak{P}(M)$ ihre Potenzmenge. Dann wird durch
  	$$ \circ : \mathfrak{P}(M) \times \mathfrak{P}(M) \longrightarrow \mathfrak{P}(M), \quad
    	(A,B) \longmapsto A \cup B $$
also das Bilden der Vereinigungsmenge, eine innere Verknüpfung auf $\mathfrak{P}(M)$ definiert.
(Analoges gilt für die Durchschnittsbildung).
\end{beispiel}

\end{frame}

\begin{frame}
\frametitle{Halbgruppen}
Für eine Menge $M$ mit einer Verknüpfung $\circ$ schreiben wir kurz $(M, \circ)$.

\pause

\begin{definition} Eine nichtleere Menge $(M, \circ)$ mit einer Verknüpfung $ \circ$ heißt 
\index{Halbgruppe}\textbf{Halbgruppe}, 
wenn $\circ$ das Assoziativgesetz erfüllt, also wenn gilt:
  	$$ (n \circ m) \circ l = n \circ (m \circ l) \qquad \textrm{ für alle } \, l, m, n \in M $$ 
\end{definition}

\pause 
\begin{beispiel} $(\mathbb N, +)$ ist eine Halbgruppe. \end{beispiel}

\pause 
\begin{beispiel} $(\mathbb N \setminus \{ 0 \}, +)$ ist eine Halbgruppe. \end{beispiel}

\pause
\begin{beispiel} $(\mathbb N \setminus \{ 0 \}, \cdot)$ ist eine Halbgruppe. \end{beispiel}

\end{frame}

\begin{frame}
\frametitle{Halbgruppen}

\begin{uebung}
Auf der Menge $M = \mathbb R$ definieren wir die innere Verknüpfung
  	$$ \circ : M \times M \longrightarrow M $$
durch 
  	$$ \circ(a,b) = \left\{ \begin{array}{l c l} 1 & \quad & \textrm{ falls } a \geq b \\
	0 & & \textrm{ falls } a < b \end{array} \right. $$
Überprüfen Sie, ob $(m \circ)$ eine Halbgruppe ist.
\end{uebung}

\bigbreak

\pause \pause 

\begin{sol}
Dieses $(M, \circ )$ ist keine Halbgruppe, denn 
	$$ (4 \circ 3) \circ 2 = 1 \circ 2 = 0 $$
aber 
	$$ 4 \circ (3 \circ 2) = 4 \circ 1 = 1 $$ 
\end{sol}

\end{frame}

\begin{frame}
\frametitle{Monoid}

\begin{definition} Ein Element $e$ einer Halbgruppe $(M, \circ)$ heißt 
\textbf{neutrales Element} der Halbgruppe, wenn gilt 
  	$$ m \circ e = m, \quad e \circ m = m \qquad \textrm{ für alle }\, m \in M $$
Eine Halbgruppe $(M, \circ)$ mt neutralem Element $e$ heiß \index{Monoid}\textbf{Monoid}. Wir 
schreiben hierfür auch $(M, \circ, e)$
\end{definition}

\pause 

\begin{notiz} Das neutrale Element eines Monoids $(M,\circ)$ ist eindeutig. Sind nämlich $e$ und $e'$ zwei 
Elemente aus $M$ mit de Eigenschaft des neutralen Elements, so folgt aus ebendieser Eigenschaft
  	$$ e' = e \circ e' = e $$
\end{notiz}

\end{frame}

\begin{frame}
\frametitle{Monoid}

\begin{beispiel} $(\mathbb N, +)$ ist ein Monoid mit neutralem Element 0. \end{beispiel}

\pause
\begin{beispiel} $(\mathbb N \setminus \{ 0 \}, +)$ ist kein Monoid. \end{beispiel}

\pause
\begin{beispiel} $(\mathbb N \setminus \{ 0 \}, \cdot)$ ist ein Monoid mit neutralem Element 1.
\end{beispiel}

\pause
\begin{beispiel} $(\mathbb Z, +)$ ist ein Monoid mit neutralem Element 0. 
$(\mathbb Z, \cdot)$ ist ein Monoid mit neutralem Element $1$. \end{beispiel}

\end{frame}

\begin{frame}
\frametitle{Monoid}

\begin{beispiel} Ist $M = \{f: \mathbb R \longrightarrow \mathbb R \}$ die Menge aller Funktionen 
von $\R$ in sich mit der inneren Verknüpfung $\circ$, gegeben durch 
 	$$ (f \circ g)(x) = f(g(x)) $$
(Komposition von Abbildungen), so ist $(M \circ)$ ein Monoid mit neutralem Element 
 	$$ \mathrm{id}: \mathbb R \longrightarrow \mathbb R, \quad x \longmapsto x \, . $$
\end{beispiel}

\end{frame}

\begin{frame}
\frametitle{Monoid}

\begin{uebung}
Überprüfen Sie, ob die Menge 
	$$ M = 3 \cdot \N = \{0, 3, 6, 9, 12, \ldots \} =\{ 3 \cdot k \, \vert \,\, k \in \N \} $$
zusammen mit der Addition ganzer Zahlen ein Monoid ist. 
\end{uebung}

\pause \pause 

\bigbreak

\begin{sol} 
Die Menge $(M,+)$ ist ein Monoid mit neutralem Element $0$. 
\end{sol}

\end{frame}

\begin{frame}
\frametitle{Gruppen}

\begin{definition} Ist $(M, \circ)$ ein Monoid mit neutralem Element $e$ und ist $m \in M$, so heißt 
ein Element $n \in M$ \textbf{inverses Element} zu $m$ wenn gilt
  	$$ m \circ n  = e, \qquad n \circ m = e $$
In diesem Fall schreiben wir $m^{-1}$ für $n$.
\pause 

Ein Monoid $(G, \circ)$ heißt \textbf{Gruppe}, wenn es zu jedem Element $m \in G$ eine 
inverses Element in $G$ gibt.
\end{definition}

\pause 

\begin{beispiel}
Die Menge $\Z$ mit der Addition $+$ ist eine Gruppe. 

\pause 

Zu jeder ganzen Zahl $z$ gibt es eine ganze Zahl $-z$ mit 
	$$ z + (-z) = 0 $$
\end{beispiel}

\end{frame}

\begin{frame}
\frametitle{Gruppen}

\begin{beispiel} Das Monoid $(\mathbb N, +)$ ist keine Gruppe. So gibt es etwa keine natürliche Zahl $n$ mit
$1 + n = 0$. \end{beispiel}

\pause 

\begin{beispiel} Das Monoid $(\mathbb Z, \cdot)$ ist keine Gruppe. So gibt es etwa keine ganze Zahl $n$ mit
$2 \cdot  n = 1$. \end{beispiel}

\pause 

\begin{beispiel} Das Monoid $(\mathbb R, +)$ ist eine Gruppe. Zu jeder reellen Zahl $r$ gibt es 
eine reelle Zahl $-r$ mit $ r + (-r) = 0 $. \end{beispiel}

\pause 


\begin{beispiel} Das Monoid $(\mathbb R, \cdot)$ ist keine Gruppe, denn es gibt kein $r \in \R$ mit
$0 \cdot  r = 1$.

\pause 
Das Monoid $(\R \setminus \{ 0 \}, \cdot)$ ist eine Gruppe. %Zu jeder reellen Zahl $r \neq 0$ gibt es nämlich 
%eine reelle Zahl $r' = \frac {1}{r}$ mit $r \cdot r' = 1$.
\end{beispiel}
\end{frame}

\begin{frame}
\frametitle{Gruppen}

\begin{beispiel} Ist $M = \{f: \mathbb R \longrightarrow \mathbb R \}$ die Menge aller Funktionen 
von $\R$ in sich mit der Komposition $\circ$, gegeben durch 
 	$$ (f \circ g)(x) = f(g(x)) $$
als innerer Verknüpfung, so ist $(M, \circ)$ keine Gruppe, denn zur Nullabbildung $0$ gibt es keine 
Funktion $f$ mit $f\circ 0 = \mathrm{id}$. 

\pause 

Ist allerdings $M' \subseteq M$ die Teilmenge aller bijektiven Funktionen $f: \R \longrightarrow R$, so 
definiert $\circ$ eine innere Verknüpfung auf $M'$ und $(M', \circ)$ ist eine Gruppe. 

\pause 

Ist $f$ aus $M'$ und ist $f^{-1}$ die zu $f$ inverse Abbildung, so gilt hierfür 
	$$ (f \circ f^{-1})(x) = x = \mathrm{id}(x) = x = (f^{-1} \circ f)(x) $$
und damit ist $f^{-1}$ das zu $f$ inverse Element.
\end{beispiel}

\end{frame}

\begin{frame}
\frametitle{Permutationsgruppe}

\begin{beispiel}  
Ist $M_n = \{1, 2, \, \ldots , n \}$ die Menge der Zahlen $1, 2, \ldots, n$, und ist $S_n$ die Menge der 
bijektiven Abbildungen auf $M_n$, so heißt 
$S_n$ \textbf{Permutationsgruppe} der Zahlen $1,  
\ldots, n$ und ihre Elemente heißen Permutationen von $1, \ldots, n$. 

\pause
Ein Element $\sigma \in S_n$ lässt sich am besten tabellarisch darstellen

\medbreak

	$$ \begin{array}{ | c | c | c | c | }
	\hline
	1 & 2 & \quad \ldots \quad & n  \\
	\hline
	\sigma(1) & \sigma(2) & \quad \ldots  \quad & \sigma(n)  \\
	\hline
	\end{array} $$

\pause
\medbreak

Hierfür schreiben wir auch 
  	$$ \sigma = \left( \begin{matrix} 1 & 2 & \ldots & n \\ \sigma(1) & \sigma(2) & \ldots & \sigma(n) 
   	\end{matrix} \right) $$
\end{beispiel}
\end{frame}

\begin{frame}
\frametitle{Permutationsgruppe}

Eine Permuation $\tau$ heißt \textbf{Transposition} wenn sie nur zwei Zahlen $i$ und $j$ vertauscht, aber alle 
andern festlässt. \pause Hierfür schreiben wir $\tau =\tau_{i,j} = <\! i \,\, j\! >$. 
\pause

\begin{beispiel} Die Permuation
  	$$ \sigma = \left( \begin{matrix} 1 & 2 & 3 & 4\\ 3 & 2 & 1 & 4 
   	\end{matrix} \right) $$
ist die Transposition $<\!1 \,\,\, 3\!>$ der Zahlen $1$ und $3$
\end{beispiel}
\pause 

\begin{regel} $<\! i \,\, j\!> \circ <\! i \,\, j\! > = \mathrm{id}$. \end{regel}

\pause 

\begin{regel}
$ \vert S_n \vert = n! $.
\end{regel}
\end{frame}

\begin{frame}
\frametitle{Permutationsgruppe}

Ein Paar $i, j \in \{ 1, \ldots , n\}$ eine Fehlstand von $\sigma$, wenn $i < j$ aber
$\sigma(i) > \sigma(j)$. \pause Wir definieren die \textbf{Signatur} 
$\mathrm{sign}(\sigma)$ von $\sigma$ durch
  	$$ \mathrm{sign}(\sigma) = \left\{ \begin{array} {l c l}
	+1 & \quad &\textrm{ falls die Anzahl der Fehlstände gerade ist} \\
	-1 & \quad &\textrm{ falls die Anzahl der Fehlstände ungerade ist} 
	\end{array} \right. $$
\pause
Eine Permuation $\sigma$ heißt \textbf{gerade}, wenn $\mathrm{sign}(\sigma) = +1$ und \textbf{ungerade}, 
wenn $\mathrm{sign}(\sigma) = -1$.

\pause 
Die Signatur hat auch folgende Beschreibung
  	$$ \mathrm{sign}(\sigma) = \prod_{i<j} \frac {\sigma(j) - \sigma(i)}{j - i} $$
\pause 
Es ist $\mathrm{sign}(<\!i\,\,j\!>) = -1$. 
\end{frame}

\begin{frame}
\frametitle{kommutative Gruppen}

\begin{definition} Eine Gruppe $(G, \circ)$ heißt \textbf{kommutativ} oder
\textbf{abelsch} wenn für je zwei Elemente $g, h \in G$ gilt:
  	$$ g \circ h = h \circ g $$
\end{definition}

\pause 
\begin{beispiel} $(\mathbb Z, +), (\mathbb R, +)$ und $(\mathbb Q, +)$ sind kommutative Gruppen.

$(\mathbb R \setminus \{0\}, \cdot)$ und $(\mathbb Q  \setminus \{0\}, \cdot)$ sind kommutative Gruppen
\end{beispiel}
\pause 

\begin{beispiel} Ist $G$ die Gruppe der bijektiven Abbildungen von $\mathbb R$ in sich, 
so ist $G$ nicht kommutativ.
\end{beispiel}

\end{frame}

\begin{frame}
\frametitle{kommutative Gruppen}

\begin{uebung} Überprüfen Sie, ob die Gruppe $S_3$ der Permutationen der Zahlen $1, 2$ und $3$ 
kommutativ ist. 
\end{uebung}

\bigbreak

\pause \pause 

\begin{sol} 
Dei Gruppe $S_3$ ist nicht kommutativ. Generell ist 
für $n \geq 3$ die Gruppe $S_n$ der Permuationen  nicht 
kommutativ. 

\pause 
So ist etwa $\langle 1 \,\,\, 2 \rangle \circ \langle 1 \,\,\, 3 \rangle 
\neq \langle 1 \,\,\, 3 \rangle \circ \langle 1 \,\,\, 2 \rangle$.
\end{sol}

\end{frame}
\end{document}
