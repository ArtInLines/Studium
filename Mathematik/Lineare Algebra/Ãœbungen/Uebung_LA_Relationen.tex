\documentclass[a4paper]{article}
\usepackage{ifthen}
\usepackage{german}
\usepackage[T1]{fontenc}
\usepackage[utf8]{inputenc}

\begin{document}
Übungen zur linearen Algebra - Mengen und Relationen
\begin{enumerate}
\item Gegeben sei die Menge $M = \{ 1, 2, \cdots , 10 \}$ sowie die Teilmengen $A = \{ x \in M : x $ ist Primzahl $\} $ sowie $B = \{ x \in M : x $ ist Quadratzahl $\}$. Bestimmen Sie : 
\begin{enumerate}
\item $\overline{ A \cup B }$ 
\item $A  \setminus \overline{B} $
\end{enumerate}
\item Welche Eigenschaft hat die Relation auf der Menge der Menschen die beschreibt, ob sie verheiratet sind.
\item Bestimme die Signatur der Permutation $<53142>$
\item Die Äquivalenzrelation auf $X = \{ x = 10a + b : a, b \in \{ 0, \cdots , 9 \} \}$ sei beschrieben durch  $R = \{ (x1, x2) \in X^2  : a1 + b1 = a2 + b2 \}$. Geben Sie ein Repräsentantensystem an und bestimmen Sie die Äquivalenzklasse $X_{24}$, d.h. die Klasse die $x = 24$ enthält an.



\end{enumerate}

\end{document}
