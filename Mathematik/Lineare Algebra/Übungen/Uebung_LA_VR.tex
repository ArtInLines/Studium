\documentclass[a4paper]{article}
\usepackage{ifthen}
\usepackage{german}
\usepackage[T1]{fontenc}
\usepackage[utf8]{inputenc}
 \usepackage{amsfonts} 
\usepackage{amsmath}

\begin{document}
Übungen zur linearen Algebra - Vektorräume
\begin{enumerate}
\item Ist der Polynomring $\mathbb R[X]$ mit der Addition ein $\mathbb R$-Vektorraum ?
\item Gegeben seien die folgenden Vektoren in $\mathbb C^2 \\
\vec{v}_1 = \begin{pmatrix} 1 \\ 0 \end{pmatrix}, \quad \vec{v}_2 = \begin{pmatrix} i \\ 0 \end{pmatrix}, \quad \vec{v}_3 = \begin{pmatrix} 0 \\ 1 \end{pmatrix}, \quad \vec{v}_4 = \begin{pmatrix} 0 \\ i \end{pmatrix}$ \\
Zeigen Sie, daß die Vektoren über $K = \mathbb  C$ linear abhängig sind aber über $K = \mathbb R$ linear unabhängig.
\item Ist $U$ ein Untervektoraum von $V$ ? Warum oder warum nicht.
\begin{enumerate}
\item $ U = \{ \begin{pmatrix} x \\ y \\ z \end{pmatrix} \in \mathbb R^3 | x + 2 y- z = 0 \} , \quad V = \mathbb R^3 $
\item $ U = \{ \begin{pmatrix} x \\ y \end{pmatrix} \in \mathbb R^2 | x + y^2 = 0 \} , \quad V = \mathbb R^2 $
\item $  
        U = \{ (a_n)_{n \in \mathbb N} \in V | a_i \neq 0, \textrm{für höchstens endlich viele}\quad  i \in \mathbb N  \}   \\
        V =  \{ (a_n)_{n \in \mathbb N} | a_i \in \mathbb R, \forall i \in \mathbb N \} $
\item $ U = \{ f : \mathbb R \rightarrow \mathbb R | f(x) = f(-x), \quad\forall x \in \mathbb R \} , \quad V = Abb( \mathbb R, \mathbb R)$
        
\end{enumerate}
\item Gegeben sind die Vektoren in $V = \mathbb R^4$ ( Achtung geändert !!!)\\
$\vec{v}_1 = \begin{pmatrix} 2 \\ 1 \\ 1 \\ 3 \end{pmatrix}, \quad
\vec{v}_2 = \begin{pmatrix} 0 \\ 1 \\ 2 \\ 3 \end{pmatrix}, \quad
\vec{v}_3 = \begin{pmatrix} 1 \\ -1 \\ 0 \\ 0 \end{pmatrix}, \quad
\vec{v}_4 = \begin{pmatrix} 0 \\ -1 \\ 1 \\ 0 \end{pmatrix} $ \\
Zeigen Sie , daß $\{ \vec{v}_1, \vec{v}_2, \vec{v}_3, \vec{v}_4 \}$ eine Basis von $V$ ist. \\
Stellen Sie den Vektor $\vec{a} = \begin{pmatrix} 0 \\ 5 \\ 2 \\ 6\end{pmatrix}$ in dieser Basis dar. 
\end{enumerate}

\end{document}



