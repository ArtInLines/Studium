\documentclass[a4paper]{article}
\usepackage{ifthen}
\usepackage{german}
\usepackage[T1]{fontenc}
\usepackage[utf8]{inputenc}

\begin{document}
Übungen zur linearen Algebra - Logik
\begin{enumerate}
\item (Aufgabe der Mathematik-Olympiade, Landesrunde, 6. Klasse)
Rubin, Sarah, Omar und Viola malen im Kunstunterricht eine
Wand mit gelber Farbe an. Plötzlich wird der Farbeimer (von
einem der vier) umgestoßen und die Farbe breitet sich im ganzen
Kunstraum aus. Wer war es nun?
\begin{itemize}
\item Rubin sagt: ”Sarah hat die Farbe verschüttet. Ich war es nicht!“
\item Daraufhin sagt Sarah: ”Omar hat es getan; Rubin war es
wirklich nicht.“
\item Omar meint: ”Sarah war es nicht; ich habe die Farbe
umgestoßen.“
\item Viola sagt: ”Omar war es nicht; Rubin hat die Farbe
umgekippt.“
\end{itemize}
Bei jedem Schüler ist eine der Aussagen wahr und eine falsch.
Wer war es denn nun?
\item Alle Spielmarken eines Spiels haben auf der einen Seite einen
Buchstaben, auf der anderen Seite eine Ziffer. „Wenn auf der
einen Seite ein Konsonant ist, dann steht auf der anderen Seite
eine gerade Ziffer “. \\
\textcircled{E} \textcircled{L} \textcircled{4} \textcircled{3}\\
Welche der vier Spielmarken muss man umdrehen, um die oben
stehende Regel zu überprüfen? Was muss dann auf der anderen
Seite stehen? Weshalb muss man die anderen Spielmarken nicht
umdrehen? 
\item Beweisen Sie die Distributivgesetze durch erstellen einer Wahrheitstafel : 
\begin{itemize}
\item $(\alpha \wedge \beta) \vee \gamma \iff (\alpha \vee \gamma) \wedge (\beta \vee \gamma)$.
\item $(\alpha \vee \beta) \wedge \gamma \iff (\alpha \wedge \gamma) \vee (\beta \wedge \gamma)$.
\end{itemize}
\item Sei M die Menge der Mathematikstudenten in diesem Kurs. Wir definieren die Aussagen : 
\begin{itemize}
\item $L(x)$ : der Student $x$ liest während der Vorlesung e-mails
\item $K(x)$ : der Student $x$ kann sich gut konzentrieren 
\item $Z(x,y)$ : die Studenten $x$ und $y$ können gut zusammen arbeiten
\end{itemize}
Die folgenden Aussagen sollen 
\begin{enumerate}
\item mit Quantoren dargestellt werden
\item die Quantorenaussagen negiert werden
\item wieder umgangssprachlich formuliert werden
\end{enumerate}
Die Aussagen lauten : 
\begin{itemize}
\item Alle Studenten lesen während der Vorlesung e-mails oder können sich nicht konzentrieren
\item Mindestens ein Student kann sich nicht konzentrieren, wenn er während der Vorlesung e-mails liest
\item Es gibt einen Studenten, der mit keinem anderen zusammen arbeiten kann
\end{itemize}
\item Es gilt $n^3 - n$ ist durch 3 teilbar für natürliches $n$. 
\begin{enumerate}
 \item Beweisen Sie dies direkt durch Fallunterscheidung
\item Beweisen Sie dies durch vollständige Induktion 
\end{enumerate}

\end{enumerate}

\end{document}
