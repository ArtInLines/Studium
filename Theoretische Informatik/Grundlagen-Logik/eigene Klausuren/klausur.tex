\documentclass{exam}
\author{Valentin Richter}

\usepackage[ngerman]{babel}
\usepackage{amsmath}

\newcommand{\Punkte}[1]{\begin{flushright}(#1 Punkte)\end{flushright}}

\begin{document}
	
	\section{Aufgabe 1}
		
	\Punkte{6}
	
	Stelle die folgenden Aussagen als logische Formeln dar:
	
	\begin{itemize}
		
		\item ...
		\item ...
		\item ...
		\item ...
		
	\end{itemize}
	
	\clearpage
	\section{Aufgabe 2}
	
	\Punkte{5}
	
	Nehme eine Signatur an mit den Symbol-Sorten 's1' und 's2',
	
	einer 3-gliedrigen Funktion f(s1,s2,s1) -> s2,
	
	einer 2-gliedrigen Funktion g(s2,s1) -> s1,
	
	einem 2-gliedrigen Prädikat P(s1,s2),
	
	einem 3-gliedrigen Prädikat Q(s2,s1,s2),
	
	und den Variablen x, y, z der Sorte s1 und a, b, c der Sorte s2.
	
	Welche der folgenden Ausdrücke und Formeln in der Prädikatenlogik? Falls nicht, gebe den Grund an warum!
	
	\begin{parts}
		\part ...
		\vspace{\stretch{1}}
		
		\part ...
		\vspace{\stretch{1}}
		
		\part ...
		\vspace{\stretch{1}}
		
		\part ...
		\vspace{\stretch{1}}
		
		\part ...
		\vspace{\stretch{1}}
	\end{parts}

	\clearpage
	\section{Aufgabe 3}
		
	\Punkte{6}
	
	Nehme die folgenden Formelmengen an:\\
	$$ X = \{ R \land (S \rightarrow Q), \lnot R \rightarrow P \lor \lnot S, (\lnot Q \lor S) \land \lnot R \} $$\\
	$$ Y = \{ R \land (S \rightarrow Q), \lnot R \rightarrow P \lor \lnot S, (\lnot Q \lor S) \land \lnot R \} $$\\
	$$ Z = \{ R \land (S \rightarrow Q), \lnot R \rightarrow P \lor \lnot S, (\lnot Q \lor S) \land \lnot R \} $$\\
	Bestimme für jede Menge alle Belegungen, die es wahr machen. Nenne alle logischen Folgerungen, die zwischen den Formelmengen möglich sind.
	
	\clearpage
	\section{Aufgabe 4}
	
	\Punkte{3 + 3 + 4}
	
	Transformiere die folgenden Formeln in die konjunktive Normalform und notiere alle resultierenden Gentzen-Formeln.\\
	
	\begin{parts}
		\part ...
		\vspace{\stretch{1}}
		
		\part ...
		\vspace{\stretch{1}}
		
		\part ...
		\vspace{\stretch{1}}
	\end{parts}
	
	
	
\end{document}