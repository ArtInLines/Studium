\documentclass{exam}
\author{Valentin Richter}

\usepackage[ngerman]{babel}
\usepackage{amsmath}

\newcommand{\Punkte}[1]{\begin{flushright}(#1 Punkte)\end{flushright}}

\begin{document}
	
	
	\section{Aufgabe 1}
	
	\Punkte{6}
	
	Stelle die folgenden Aussagen als logische Formeln dar:
	
	\begin{itemize}
		
		\item Zwei ungerade, natürlichen Zahlen zusammen addiert ergeben einen gerade, natürliche Zahl.
		\item Jede natürliche Zahl ist auch eine reele Zahl.
		\item ...
		\item ...
		
		
		\item $$ odd(x) \land odd(y) \implies even(x+y) $$
		\item $$ \forall x (nat(x)) \implies re(x) $$
		\item $$  $$
		\item $$  $$
		
		% - The predecessor of an odd natural number is an even natural number
		% - The root of a natural square number is a natural number or it is negative
		% - The sum of the negative root and the positive root of a natural square number is zero
		% - The sum of the predecessor and the successor of a natural square numbers is even
	\end{itemize}
	
	
	\clearpage
	\section{Aufgabe 2}
	
	\Punkte{5}
	
	Nehme eine Signatur an mit den Symbol-Sorten 's1' und 's2',
	
	einer 3-gliedrigen Funktion f(s1,s2,s1) -> s2,
	
	einer 2-gliedrigen Funktion g(s2,s1) -> s1,
	
	einem 2-gliedrigen Prädikat P(s1,s2),
	
	einem 3-gliedrigen Prädikat Q(s2,s1,s2),
	
	und den Variablen x, y, z der Sorte s1 und a, b, c der Sorte s2.
	
	Welche der folgenden Ausdrücke und Formeln in der Prädikatenlogik? Falls nicht, gebe den Grund an warum!
	
	\begin{parts}
		\part ...
		\vspace{\stretch{1}}
		
		\part ...
		\vspace{\stretch{1}}
		
		\part ...
		\vspace{\stretch{1}}
		
		\part ...
		\vspace{\stretch{1}}
		
		\part ...
		\vspace{\stretch{1}}
	\end{parts}
	
	\clearpage
	\section{Aufgabe 3}
	
	\Punkte{6}
	
	Nehme die folgenden Formelmengen an:\\
	$$ X = \{  \} $$\\
	$$ Y = \{ (R \iff !S) \lor Q, \lnot (R v Q \rightarrow S), (Q \land P) \lor (S \land \lnot P) \} $$\\
	$$ Z = \{ R \land (S \rightarrow Q), \lnot R \rightarrow P \lor \lnot S, (\lnot Q \lor S) \land \lnot R \} $$\\
	Bestimme für jede Menge alle Belegungen, die es wahr machen. Nenne alle logischen Folgerungen, die zwischen den Formelmengen möglich sind.
	
	
	$$ Y = \{ P,Q, \} $$ 
	
	\clearpage
	\section{Aufgabe 4}
	
	\Punkte{3 + 3 + 4}
	
	Transformiere die folgenden Formeln in die konjunktive Normalform und notiere alle resultierenden Gentzen-Formeln.\\
	
	\begin{parts}
		\part ...
		\vspace{\stretch{1}}
		
		\part ...
		\vspace{\stretch{1}}
		
		\part ...
		\vspace{\stretch{1}}
	\end{parts}
	
	
	\clearpage
	\section{Aufgabe 5}
	
	\Punkte{4 + 4}
	
	Nehme die folgenden Formeln an. Transformiere sie erst in die Pränexe Normalform und skolemisiere sie zuletzt.
	
	\begin{parts}
		\part ...
		\vspace{\stretch{1}}
		
		\part ...
		\vspace{\stretch{1}}
	\end{parts}
	
	
	\clearpage
	\section{Aufgabe 6}
	\Punkte{8}
	
	Beweise die Korrektheit der folgenden Spezifikation:
	
	...
	
	
	\clearpage
	\section{Aufgabe 7}
	\Punkte{6}
	
	
	Nehme die folgenden Ausdrücke mit Nachbedingung an.
	
	...
	
	Schließe auf die schwächste Vorbedingung.
	
	
	\clearpage
	\section{Aufgabe 8}
	\Punkte{8}
	
	Sind die folgenden Paare von Literalen unifizierbar? Falls ja, gib die  Belegungen an, die as der Unifizierung folgen.
	
	\begin{parts}
		\part ...
		\vspace{\stretch{1}}
		
		\part ...
		\vspace{\stretch{1}}
		
		\part ...
		\vspace{\stretch{1}}
		
		\part ...
		\vspace{\stretch{1}}
		
		\part ...
		\vspace{\stretch{1}}
	\end{parts}
	
	
	\clearpage
	\section{Aufgabe 9}
	\Punkte{5+3}
	
	\begin{parts}
		\part Definiere eine ... Funktion analog zu der angegebenen ... Funktion:
		
		...
		\vspace{\stretch{1}}
		
		\part Determine for the Lisp-Funktion given in a) the number of executed test operations, if the function is activated with
		a list containing n (n>0) positive numbers. Start the cost calculation with a recursion equation and solve it finally / ...
		\vspace{\stretch{1}}
	\end{parts}
	
	
	\clearpage
	\section{Aufgabe 10}
	\Punkte{6}
	
	Definiere eine Prolog Funktion ...
	
	
	\clearpage
	\section{Aufgabe 11}
	\Punkte{1+2+2+3}
	
	Nehme an L sei eine Liste mit mindestens zwei nicht-leeren Sublisten mit jeweils min. 2 Zahlen. Gib einen Lisp-Ausdruck an, der...
	
	\begin{parts}
		\part ...
		\vspace{\stretch{1}}
		
		\part ...
		\vspace{\stretch{1}}
		
		\part ...
		\vspace{\stretch{1}}
		
		\part ...
		\vspace{\stretch{1}}
	\end{parts}
	
	
	\clearpage
	\section{Aufgabe 12}
	\Punkte{6}
	
	Definiere eine Lisp Funktion ...
	
	
	
\end{document}